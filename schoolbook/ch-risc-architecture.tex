\section{Архитектура RISC-процессора \MyProc}
\label{ch::risc}


Преимущества RISC-архитектур в быстроте освоения програмистом, а недостатки заключаются в необходимости реализовать относительно простые операции нетривиальными подпрограммами или макроподстановками. Особенности работы с RISC исчерпывающе изложены в \cite{bib:warren:algTriks}.

\subsection{Память и адресные пространства \MyProc}

В процессоре \MyProc:
\begin{itemize}
    \item имеется 8 байтовых (8-и битных) регистров \Machine{r0},\Machine{r1},\ldots,\Machine{r7};
    \item доступна память данных из 256 8-битных ячеек с адресами от 0 до 255;
    \item оступен единственный порт ввода 8-и битного числа (см. описание команды \Opcode{in});
    \item доступен единственный порт вывода 8-и битного числа, (см. описание команды \Opcode{out});
    \item обращение к ячейке памяти данных возможно двумя способами:
    \begin{itemize}
        \item непосредственная адресация, например, <<\Machine{[0]}>> --- обращение к нулевой ячейке памяти;
        \item регистровая адресация, например, <<\Machine{[r1]}>> --- обращение к ячейке памяти с адресом, значение которого берется из регистра \Machine{r1}. Использовать для обращения к памяти можно любой из 8-и регистров.
    \end{itemize}
\end{itemize}


\subsection{Система команд \MyProc}

Архитектура {\MyProc} позволяет пользователю выбрать набор базисных команд из полного набора. Это на практике позволяет предоставить каждому студенту индивидуальный набор базисных команд. Принципы построения базисного набора инструкций изложены ниже. Полный набор команд следующий:

\begin{itemize}
    \item <<\CmdOneAddr{in}{приемник}>>. Команда ввода байта из порта ввода в \Operand{приемник}. При выполнении этой команды в программной модели {\MyProc} пользователю предлагается ввести число с экрана, которое затем заносится в \Operand{приемник}.
    
    \item <<\CmdOneAddr{out}{источник}>>. Команда записи в порт вывода. При выполнении этой команды в программной модели {\MyProc} значение байта \Operand{источник} выводится на экран.
    
    \item <<\CmdThreeAddr{ror}{источник}{количество-разрядов}{приемник}>>. Команда циклического сдвига вправо на заданное количество разрядов. \Operand{источник} циклически сдвигается на \Operand{количество-разрядов} вправо и результат сдвига записывается в \Operand{приемник}.
    
    \item <<\CmdThreeAddr{rol}{источник}{количество-разрядов}{приемник}>>. Команда циклического сдвига влево на заданное количество разрядов. \Operand{источник} циклически сдвигается на \Operand{количество-разрядов} влево и результат сдвига записывается в \Operand{приемник}.
    
    \item <<\CmdTwoAddr{not}{источник}{приемник}>>. Поразрядное логическое НЕ. Все разряды байта \Operand{источник} инвертиурются и результат записыается в \Operand{приемник}.
    
    \item <<\CmdThreeAddr{or}{источник1}{источник2}{приемник}>>. Поразрядное логическое ИЛИ. 
        \[
            \Operand{приемник} \gets \Operand{источник1} \lor \Operand{источник2}.
        \]
    
    \item <<\CmdThreeAddr{and}{источник1}{источник2}{приемник}>>. Поразрядное логическое И.
    \item <<\CmdThreeAddr{nor}{источник1}{источник2}{приемник}>>.  Поразрядное логическое ИЛИ-НЕ.
    \item <<\CmdThreeAddr{nand}{источник1}{источник2}{приемник}>>. Поразрядное логическое И-НЕ.
    \item <<\CmdThreeAddr{xor}{источник1}{источник2}{приемник}>>. Поразрядное логическое XOR.
    \item <<\CmdThreeAddr{add}{источник1}{источник2}{приемник}>>. Арифметическое сложение (с потерей переноса из старшего разряда).
    \item <<\CmdThreeAddr{sub}{источник1}{источник2}{приемник}>>. Арифметическое вычитание, соответствующее следующему сложению с потерей переноса из старшего разряда:
        \[
            \Operand{приемник} \gets \Operand{источник1} + \overline{\Operand{источник2}} + 1.
        \]
        
    \item <<\CmdTwoAddr{jz}{источник}{имя-метки}>>. Переход на метку с именем \Operand{имя-метки}, если все разряды байта \Operand{источник} равны нулю.
    \item <<\CmdTwoAddr{jo}{источник}{имя-метки}>>. Переход на метку с именем \Operand{имя-метки}, если все разряды байта \Operand{источник} равны единице.
\end{itemize}

В общем случае:
\begin{itemize}
    \item \Machine{приемник} --- это регистр или ячейка памяти, например: \Machine{r0}, \Machine{[0]}, \Machine{[r0]}, но не константа;
    \item \Machine{источник}, \Machine{количество-разрядов} --- это константа, регистр или ячейка памяти;
\end{itemize}

Все возможные базисные наборы команд определяются декартовым произведением следующих подмножеств их полного множества:
\begin{itemize}
    \item базисы команд сдвига:
        \[\{\Opcode{rol},\Opcode{ror}\};\]
    \item базисы логических команд:
        \[\{ \{\Opcode{and},\Opcode{not}\}, \{\Opcode{or},\Opcode{not}\}, \{\Opcode{and},\Opcode{xor}\}, \{\Opcode{or},\Opcode{xor}\}, \Opcode{nand},\Opcode{nor} \};\]
    \item базисы арифметических команд:
        \[\{ \Opcode{add},\Opcode{sub} \};\]
    \item базисы команд перехода:
        \[\{ \Opcode{jz},\Opcode{jo} \};\]
\end{itemize}

Для упрощения создания рабочих прототипов программ предлагается единственный избыточный набор команд $\MyProc^{(0)}$:
        \[\{ \Opcode{rol},\Opcode{ror}, \Opcode{and}, \Opcode{or}, \Opcode{not}, \Opcode{xor}, \Opcode{add},\Opcode{sub}, \Opcode{jz}, \Opcode{jo} \}, \]
от которого затем можно будет перейти к заданному набору команд.


\subsection{Язык ассемблера \MyProc}

\begin{itemize}
    \item Программа на языке ассемблера {\MyProc} оформляется в виде текстового файла, содержащего текстовые мнемоники команд \MyProc, комментарии, литералы целочисленных констант и метки.
    \item Литералы чисел могут представляться в десятичной, шестнадцатеричной (префикс <<\Machine{0x}>>), восьмеричной (префикс <<\Machine{0o}>> или <<\Machine{0}>>) и двоичной (префикс <<\Machine{0b}>>) системах счисления. Например: \Machine{10}, \Machine{0xA}, \Machine{012}, \Machine{0b1010}.
    \item Разделителем команд является перевод строки.
    \item Количество пробелов-разделителей не имеет значения.
    \item Пустые строки игнорируются.
    \item В непустой строке текстового файла может быть команда, метка или комментарий.
    \item Метка задается как <<\Machine{имя-метки:}>>.
    \item Имена меток (без двоеточия) используются в командах перехода.
    \item В программе не может двух и более меток с одинаковыми именами.
    \item Комментарий может следовать после команды или метки, признаком начала комментария всегда является символ <<\Machine{;}>>.
    \item Признаком конца комментария является перевод строки.
\end{itemize}


\subsection{Выполнение программы \MyProc}
\label{ch:risc:time}

Время, затрачиваемое на выполнение команды складывается из времени выполнения операции и времени доступа к ячейкам памяти\footnote{На практике RISC-процессор в среднем выполняет команду за один машинный такт. Это достигается за счет того, что команды просты, их размер в памяти команд одинаков, их выборка конвееризируеся и активно используется кэширование команд и данных. {\MyProc} не удовлетворяет данным требованиям. Это сделано для того, чтобы сделать очевидными критерии низкоуровневой оптимизации программ}.

Любая операция выполняется за один такт. Доступ к регистру осуществляется за один такт. Доступ к ячейке памяти осуществляется за 8 тактов. На обращение к константе (и метке в командах перехода) времени не требуется. Например, oбщее время выполнения (19 тактов) команды 
\[
    \CmdThreeAddr{add}{[r1]}{r2}{[5]}
\]
складывается как
\begin{enumerate}
    \item доступ к регистру $t(\Machine{r1})=1$;
    \item доступ к памяти $t(\Machine{[r1]})=8$;
    \item доступ к регистру $t(\Machine{r2})=1$;
    \item выполнение операции $t(\Opcode{add})=1$;
    \item доступ к памяти $t(\Machine{[5]})=8$.
\end{enumerate}

Если при выполнении команд перехода (\Opcode{jz}, \Opcode{jo}) происходит переход по метке, то на выборку новой команды процессор затрачивает 8 тактов.
