\section{Задания на лабораторные работы}


Студент получает номер индивидуального набора команд \MyProc. С помощью команд полученного набора студенту необходимо реализовать алгоритмы основного задания. В основном задании определяется разрядность исходных данных, порядок байт и реализуемый алгоритм. 

Операнды должны быть считаны в заданном порядке из порта ввода. Результат  должен быть выведен в порт вывода, также в заданном порядке.

\subsection{Задания на первую лабораторную работу}

В порядке ознакомления с программной моделью {\MyProc} студенту предлагается реализовать следующие алгоритмы.

\begin{itemize}
    \item Сброс $n$-го бита в 8-разрядном числе.
    \item Перевод 8-разрядного числа из прямого кода в дополнительный.
    \item Сложение двух 16-разрядных чисел в дополнительном коде. Порядок байт --- little-endian.
    \item Подсчет единичных бит в 16-разрядном числе.
    \item Сравнение 16-разрядных чисел $A$ и $B$, представленных в дополнительном коде. Порядок байт --- big-endian. Вывести:
    \[
        \begin{cases}
            -1, &\text{если $A<B$};\\
             0, &\text{если $A=B$};\\
            +1, &\text{если $A>B$}.
        \end{cases}
    \]
    \item Сдвиг 16-разрядного дополнительного кода числа на один разряд вправо. Порядок байт --- little-endian.
    \item Сдвиг 16-разрядного обратного кода числа на один разряд влево. Порядок байт --- big-endian.
\end{itemize}


\subsection{Задания на вторую лабораторную работу}

Вторая лабораторная работа предназначена для закрепления теоретического материала по арифметическим основам ЭВМ. 
На выбор предлагаются следующие группы заданий:
\begin{itemize}
    \item Умножение чисел в формате с фиксированной точкой.
    \item Сложение чисел в формате с плавающей точкой.
    \item Умножение чисел в формате с плавающей точкой.
\end{itemize}

Студенту необходимо получить номер варианта в рамках выбранной группы заданий и реализовать соответствующий алгоритм.

Варианты заданий для умножения чисел в формате с фиксированной точкой:
\begin{enumerate}
    \item Алгоритм умножения 8-разрядных беззнаковых чисел I-м способом.
    \item Алгоритм умножения 8-разрядных беззнаковых чисел II-м способом.
    \item Алгоритм умножения 8-разрядных беззнаковых чисел III-м способом.
    \item Алгоритм умножения 8-разрядных беззнаковых чисел IV-м способом.
    \item Алгоритм умножения 8-разрядных беззнаковых чисел с ускорением второго порядка I-м способом.
    \item Алгоритм умножения 8-разрядных беззнаковых чисел с ускорением второго порядка II-м способом.
    \item Алгоритм умножения 8-разрядных беззнаковых чисел с ускорением второго порядка III-м способом.
    \item Алгоритм умножения 8-разрядных беззнаковых чисел с ускорением второго порядка IV-м способом.
    \item Алгоритм умножения 8-разрядных чисел в дополнительном коде с автоматической коррекцией I-м способом.
    \item Алгоритм умножения 8-разрядных чисел в дополнительном коде с автоматической коррекцией II-м способом.
    \item Алгоритм умножения 8-разрядных чисел в дополнительном коде с автоматической коррекцией III-м способом.
    \item Алгоритм умножения 8-разрядных чисел в дополнительном коде с автоматической коррекцией IV-м способом.
    \item Алгоритм умножения 8-разрядных чисел в дополнительном коде с простой коррекцией I-м способом.
    \item Алгоритм умножения 8-разрядных чисел в дополнительном коде с простой коррекцией II-м способом.
    \item Алгоритм умножения 8-разрядных чисел в дополнительном коде с простой коррекцией III-м способом.
    \item Алгоритм умножения 8-разрядных чисел в дополнительном коде с простой коррекцией IV-м способом.
    \item Алгоритм умножения 16-разрядных беззнаковых чисел I-м способом.
    \item Алгоритм умножения 16-разрядных беззнаковых чисел II-м способом.
    \item Алгоритм умножения 16-разрядных беззнаковых чисел III-м способом.
    \item Алгоритм умножения 16-разрядных беззнаковых чисел IV-м способом.
    \item Алгоритм умножения 16-разрядных беззнаковых чисел с ускорением второго порядка I-м способом.
    \item Алгоритм умножения 16-разрядных беззнаковых чисел с ускорением второго порядка II-м способом.
    \item Алгоритм умножения 16-разрядных беззнаковых чисел с ускорением второго порядка III-м способом.
    \item Алгоритм умножения 16-разрядных беззнаковых чисел с ускорением второго порядка IV-м способом.
    \item Алгоритм умножения 16-разрядных чисел в дополнительном коде с автоматической коррекцией I-м способом.
    \item Алгоритм умножения 16-разрядных чисел в дополнительном коде с автоматической коррекцией II-м способом.
    \item Алгоритм умножения 16-разрядных чисел в дополнительном коде с автоматической коррекцией III-м способом.
    \item Алгоритм умножения 16-разрядных чисел в дополнительном коде с автоматической коррекцией IV-м способом.
    \item Алгоритм умножения 16-разрядных чисел в дополнительном коде с простой коррекцией I-м способом.
    \item Алгоритм умножения 16-разрядных чисел в дополнительном коде с простой коррекцией II-м способом.
    \item Алгоритм умножения 16-разрядных чисел в дополнительном коде с простой коррекцией III-м способом.
    \item Алгоритм умножения 16-разрядных чисел в дополнительном коде с простой коррекцией IV-м способом.
\end{enumerate}

Варианты заданий для сложения чисел в формате с плавающей точкой:
\begin{enumerate}
    \item Алгоритм сложения чисел в 16-разрядном формате с плавающей точкой. Форматом используется характеристика. Для представления мантиссы используется прямой код. Остальные особенности формата и соглашение о порядке следования байт (little-endian, big-endian) разрабатываются студентом самостоятельно.
    \item Алгоритм сложения чисел в 16-разрядном формате с плавающей точкой. Форматом используется характеристика. Для представления мантиссы используется дополнительный код. Остальные особенности формата и соглашение о порядке следования байт (little-endian, big-endian) разрабатываются студентом самостоятельно.
    \item Алгоритм сложения чисел в 16-разрядном формате с плавающей точкой. Форматом используется характеристика. Для представления мантиссы используется обратный код. Остальные особенности формата и соглашение о порядке следования байт (little-endian, big-endian) разрабатываются студентом самостоятельно.
    \item Алгоритм сложения чисел в 16-разрядном формате с плавающей точкой. Форматом используется порядок. Для представления мантиссы используется прямой код. Остальные особенности формата и соглашение о порядке следования байт (little-endian, big-endian) разрабатываются студентом самостоятельно.
    \item Алгоритм сложения чисел в 16-разрядном формате с плавающей точкой. Форматом используется порядок. Для представления мантиссы используется дополнительный код. Остальные особенности формата и соглашение о порядке следования байт (little-endian, big-endian) разрабатываются студентом самостоятельно.
    \item Алгоритм сложения чисел в 16-разрядном формате с плавающей точкой. Форматом используется порядок. Для представления мантиссы используется обратный код. Остальные особенности формата и соглашение о порядке следования байт (little-endian, big-endian) разрабатываются студентом самостоятельно.
    \item Алгоритм сложения чисел в 24-разрядном формате с плавающей точкой. Форматом используется характеристика. Для представления мантиссы используется прямой код. Остальные особенности формата и соглашение о порядке следования байт (little-endian, big-endian) разрабатываются студентом самостоятельно.
    \item Алгоритм сложения чисел в 24-разрядном формате с плавающей точкой. Форматом используется характеристика. Для представления мантиссы используется дополнительный код. Остальные особенности формата и соглашение о порядке следования байт (little-endian, big-endian) разрабатываются студентом самостоятельно.
    \item Алгоритм сложения чисел в 24-разрядном формате с плавающей точкой. Форматом используется характеристика. Для представления мантиссы используется обратный код. Остальные особенности формата и соглашение о порядке следования байт (little-endian, big-endian) разрабатываются студентом самостоятельно.
    \item Алгоритм сложения чисел в 24-разрядном формате с плавающей точкой. Форматом используется порядок. Для представления мантиссы используется прямой код. Остальные особенности формата и соглашение о порядке следования байт (little-endian, big-endian) разрабатываются студентом самостоятельно.
    \item Алгоритм сложения чисел в 24-разрядном формате с плавающей точкой. Форматом используется порядок. Для представления мантиссы используется дополнительный код. Остальные особенности формата и соглашение о порядке следования байт (little-endian, big-endian) разрабатываются студентом самостоятельно.
    \item Алгоритм сложения чисел в 24-разрядном формате с плавающей точкой. Форматом используется порядок. Для представления мантиссы используется обратный код. Остальные особенности формата и соглашение о порядке следования байт (little-endian, big-endian) разрабатываются студентом самостоятельно.
\end{enumerate}

Варианты заданий для умножения чисел в формате с плавающей точкой:
\begin{enumerate}
    \item Алгоритм умножения чисел в 16-разрядном формате с плавающей точкой. Форматом используется характеристика. Для представления мантиссы используется прямой код. Перемножение мантисс выполняется I способом умножения. Остальные особенности формата и соглашение о порядке следования байт (little-endian, big-endian) разрабатываются студентом самостоятельно.
    \item Алгоритм умножения чисел в 16-разрядном формате с плавающей точкой. Форматом используется характеристика. Для представления мантиссы используется прямой код. Перемножение мантисс выполняется II способом умножения. Остальные особенности формата и соглашение о порядке следования байт (little-endian, big-endian) разрабатываются студентом самостоятельно.
    \item Алгоритм умножения чисел в 16-разрядном формате с плавающей точкой. Форматом используется характеристика. Для представления мантиссы используется прямой код. Перемножение мантисс выполняется III способом умножения. Остальные особенности формата и соглашение о порядке следования байт (little-endian, big-endian) разрабатываются студентом самостоятельно.
    \item Алгоритм умножения чисел в 16-разрядном формате с плавающей точкой. Форматом используется характеристика. Для представления мантиссы используется прямой код. Перемножение мантисс выполняется IV способом умножения. Остальные особенности формата и соглашение о порядке следования байт (little-endian, big-endian) разрабатываются студентом самостоятельно.
    \item Алгоритм умножения чисел в 16-разрядном формате с плавающей точкой. Форматом используется характеристика. Для представления мантиссы используется дополнительный код. Перемножение мантисс выполняется I способом умножения. Остальные особенности формата и соглашение о порядке следования байт (little-endian, big-endian) разрабатываются студентом самостоятельно.
    \item Алгоритм умножения чисел в 16-разрядном формате с плавающей точкой. Форматом используется характеристика. Для представления мантиссы используется дополнительный код. Перемножение мантисс выполняется II способом умножения. Остальные особенности формата и соглашение о порядке следования байт (little-endian, big-endian) разрабатываются студентом самостоятельно.
    \item Алгоритм умножения чисел в 16-разрядном формате с плавающей точкой. Форматом используется характеристика. Для представления мантиссы используется дополнительный код. Перемножение мантисс выполняется III способом умножения. Остальные особенности формата и соглашение о порядке следования байт (little-endian, big-endian) разрабатываются студентом самостоятельно.
    \item Алгоритм умножения чисел в 16-разрядном формате с плавающей точкой. Форматом используется характеристика. Для представления мантиссы используется дополнительный код. Перемножение мантисс выполняется IV способом умножения. Остальные особенности формата и соглашение о порядке следования байт (little-endian, big-endian) разрабатываются студентом самостоятельно.
    \item Алгоритм умножения чисел в 16-разрядном формате с плавающей точкой. Форматом используется порядок. Для представления мантиссы используется прямой код. Перемножение мантисс выполняется I способом умножения. Остальные особенности формата и соглашение о порядке следования байт (little-endian, big-endian) разрабатываются студентом самостоятельно.
    \item Алгоритм умножения чисел в 16-разрядном формате с плавающей точкой. Форматом используется порядок. Для представления мантиссы используется прямой код. Перемножение мантисс выполняется II способом умножения. Остальные особенности формата и соглашение о порядке следования байт (little-endian, big-endian) разрабатываются студентом самостоятельно.
    \item Алгоритм умножения чисел в 16-разрядном формате с плавающей точкой. Форматом используется порядок. Для представления мантиссы используется прямой код. Перемножение мантисс выполняется III способом умножения. Остальные особенности формата и соглашение о порядке следования байт (little-endian, big-endian) разрабатываются студентом самостоятельно.
    \item Алгоритм умножения чисел в 16-разрядном формате с плавающей точкой. Форматом используется порядок. Для представления мантиссы используется прямой код. Перемножение мантисс выполняется IV способом умножения. Остальные особенности формата и соглашение о порядке следования байт (little-endian, big-endian) разрабатываются студентом самостоятельно.
    \item Алгоритм умножения чисел в 16-разрядном формате с плавающей точкой. Форматом используется порядок. Для представления мантиссы используется дополнительный код. Перемножение мантисс выполняется I способом умножения. Остальные особенности формата и соглашение о порядке следования байт (little-endian, big-endian) разрабатываются студентом самостоятельно.
    \item Алгоритм умножения чисел в 16-разрядном формате с плавающей точкой. Форматом используется порядок. Для представления мантиссы используется дополнительный код. Перемножение мантисс выполняется II способом умножения. Остальные особенности формата и соглашение о порядке следования байт (little-endian, big-endian) разрабатываются студентом самостоятельно.
    \item Алгоритм умножения чисел в 16-разрядном формате с плавающей точкой. Форматом используется порядок. Для представления мантиссы используется дополнительный код. Перемножение мантисс выполняется III способом умножения. Остальные особенности формата и соглашение о порядке следования байт (little-endian, big-endian) разрабатываются студентом самостоятельно.
    \item Алгоритм умножения чисел в 16-разрядном формате с плавающей точкой. Форматом используется порядок. Для представления мантиссы используется дополнительный код. Перемножение мантисс выполняется IV способом умножения. Остальные особенности формата и соглашение о порядке следования байт (little-endian, big-endian) разрабатываются студентом самостоятельно.
\end{enumerate}
