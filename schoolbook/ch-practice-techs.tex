\section{Технические решения в модели Microcode}

Используются следующие обозначения.
\begin{itemize}
	\item $S$ --- студент.
	\item $T$ --- учитель.
	\item $task$ --- выданное задание. Задание уникально, в частности, для каждого задания генерируется функция перестановки управляющих и осведомительных сигналов.
	\item $result$ --- результаты работы студента (промежуточные и окончательные).
	\item $test$ --- результаты автоматического тестирования работы студента
	\item $id_S$ --- идентификационные данные студента, такие как фамилия, имя, отчество, группа.
	\item $pwd_T$ --- секрет преподавателя.
	\item $H(m)$ --- результат вычисления значения криптографической функции хеширования $H:M\to h$ от аргумента произвольной длины $m\in M$. Результат вычисления $H(m)$ --- последовательность фиксированной длины.
	\item $\{M_1,M_2,\ldots\}$ --- конкатенация $M_1,M_2,\ldots$
\end{itemize}

\subsubsection{Получение задания}

% <<Сегодня меня очень истанцевали!>> --- сказала маленькая девочка, выходя из ДК <<Космос>>.

Управляющие и осведомительне сигналы перемешиваются в каждом задании.

Секрет вводится преподавателем в момент генерации задания и затем уничтожается программой. Защита целостности и принадлежности задания от модификации осуществляется подписью преподавателя:
\[task_{sign} = H(\{id_S, task, pwd_T\}).\]

Структура файла после получения задания:

\[\{id_S, task, task_{sign}\}.\]


\subsubsection{Решение задания}


Защита целостности результатов работы осуществляется подписью:
\[result_{sign} = H(\{result, task_{sign}\}).\]

Структура файла в процессе работы над заданием:

\[\{id_S, task, result, task_{sign}, result_{sign}\}.\]


\subsubsection{Защита результатов работы}

Студент проходит тестирование, когда считает, что навыки получены, а знания усвоены. Тестирование осуществляетя в автоматическом режиме. В ходе тестирования проверяются ошибки работы с шиной, корректность результатов, выполняются проверки выдачи несовместимых сигналов. При этом также варируются задержки поступления задания и освобождения шины.

Программа формирует результаты тестирования $test$. Верификация результатов тестирования осуществляется при участии преподавателя, который проверив и обсудив работу, вновь вводит секрет. Программа сверяет подпись задания и, если проверка успешна, вычисляет подпись результатов работы студента и тестов:

\[test_{sign} = H(\{id_S, task, result, test, pwd_T\}).\]

Окончательная структура файла. 

\[\{id_S, task, result, test, task_{sign}, result_{sign}, test_{sign}\}.\]
