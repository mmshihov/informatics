\chapter{Умножение двоичных чисел}
\label{ch:binmul}

Пусть для положительных чисел $A$ и $B$ имеются дробно-масштабированные представления в двоичной системе счисления. Пусть
\[
    A\equiv(0,a_{1}\cdots a_{n})_2.
\]

Тогда результат произведения дробных $A\times B$ будет определяться по формуле:
\[
    A\times B = B\cdot\sum_{i=1}^{n} a_{i}\cdot 2^{-i} = \sum_{i=1}^{n} (B\cdot 2^{-i})\cdot a_i = \sum_{i=1}^{n} (B \gg i)\cdot a_i.
\]

Результат произведения --- сумма, в которой $i$-е слагаемое есть двоичное представление множимого $B$, сдвинутое на $i$ разрядов вправо\footnote{Сдвиг вправо множимого $B$ здесь обозначен как $(B \gg i)$. Напомним, что умножение на $2$ в двоичной системе счисления эквивалентно сдвигу на один разряд влево (или, что то же самое, переносу запятой на один разряд вправо). Делению на $2$ соответствует сдвиг на один разряд вправо или перенос запятой на разряд влево.}. Причем $i$-е слагаемое входит в сумму только когда $a_i=1$.

Например, требуется найти произведение $A\times B$, где $A=23$ и $B=25$. Дробные представления (с масштабирующим множителем $M=2^5$) будут:
\begin{align*}
    A\equiv\Number{,10111},\\
    B\equiv\Number{,11001}.
\end{align*}

Процесс умножения\footnote{Здесь и в последующих примерах вместо точки в представлении числа можно ставить нуль. Таким образом авторы постарались повысить наглядность. Целая часть числа от дробной части будет отделяться запятой.}:
\begin{center}
    \begin{tabular}{c|c|c}
                                                                   \hline\hline
        $i$ & Разряды $a_i$    & $(B\cdot 2^{-i}) \cdot a_i$  \\ \hline\hline
        1   & \Number{,1....} & \Number{,.1100 1....} \\
        2   & \Number{,.0...} & \Number{,..... .....} \\
        3   & \Number{,..1..} & \Number{,...11 001..} \\
        4   & \Number{,...1.} & \Number{,....1 1001.} \\
        5   & \Number{,....1} & \Number{,..... 11001} \\ \hline
        \multicolumn{2}{r}{Результат:} 
                               & \Number{0,10001 11111} \\
    \end{tabular}
\end{center}

Сумму сдвигов множимого, полученную на некотором шаге алогритма умножения, будем назвать \emph{суммой частичных произведений} (СЧП).

В схемной реализации АЛУ для умножения удобно использовать сдвиговые регистры. При этом разработчику решать:
\begin{itemize}
    \item в какую сторону сдвигать регистр множителя (или, что то же самое: какой разряд множителя, старший или младший, анализировать);
    \item сдвигать множимое или вместо этого сдвигать накопленную сумму частичных произведений.
\end{itemize}

Ответы на эти вопросы рождают четыре основных способа\footnote{Можно сказать, что и четыре основные аппаратные схемы} умножения, представленных на рисунке \ref{fig:binmul:baseschemes}.
\begin{figure}[!ht]
    \begin{tabular}{c||c||c}
            & Сдвиг СЧП
                & Сдвиг множимого\\
        \hline\hline
        \rotatebox{90}{Сдвиг мн-ля вправо}
            & \includegraphics[width=.45\textwidth]{fig/mult-methods.1}
                & \includegraphics[width=.45\textwidth]{fig/mult-methods.2}\\
        \hline\hline
        \rotatebox{90}{Сдвиг мн-ля влево}
            & \includegraphics[width=.45\textwidth]{fig/mult-methods.3}
                & \includegraphics[width=.45\textwidth]{fig/mult-methods.4}\\
    \end{tabular}
    \caption{Основные схемы методов умножения на сдвиговых регистрах}\label{fig:binmul:baseschemes}
\end{figure}

Кратко опишем способы и их особенности.
\begin{itemize}
    \item I-й способ. Анализ младшего разряда множителя со сдвигом СЧП вправо. Множимое прибавляется к старшей половине СЧП (см. рисунок \ref{fig:binmul:baseschemes}). В процессе умножения может возникнуть единица переноса из старшего (на рисунке 1-го) разрядя регистра СЧП. Эта единица не должна быть потеряна. При очередном сдвиге СЧП вправо она должна заноситься в старший разряд СЧП. Также обязательно выполняется сдвиг СЧП на последнем шаге основного цикла умножения.
        
    \item II-й способ. Анализ младшего разляда множителя со сдвигом множимого влево. Множимое заносится в младшие $n$ разрядов соответствующего $2n$ разрядного регистра. 
    
    \item III-й способ. Анализ старшего разряда множителя со сдвигом СЧП влево. Множимое прибавляется к младшим $n$ разрядам СЧП, а к старшим прибавляется нули. На последнем шаге умножения сдвиг СЧП не выполняется.
    
    \item IV-й способ. Анализ старшего разляда множителя со сдвигом множимого влево. Множимое заносится в младшие $n$ разрядов соответствующего $2n$ разрядного регистра. На первом шаге до анализа старшего разряда множителя выполняется сдвиг множимого на один разряд влево.
\end{itemize}

Суммарную разрядность регистров и сумматоров нетрудно посчитать:
\begin{itemize}
    \item I-й способ: регистров --- $4n$, сумматор --- $n$;
    \item II-й способ: регистров --- $5n$, сумматор --- $2n$;
    \item III-й способ: регистров --- $4n$, сумматор --- $2n$;
    \item IV-й способ: регистров --- $5n$, сумматор --- $2n$.
\end{itemize}

Отдельно стоит отметить экономичность I-го способа умножения, который позволяет сократить суммарную разрядность регистров до $3n$ при одном $n$-разрядном сумматоре. При этом младшие биты регистра СЧП заносятся в старшие разряды регистра множителя при их одновременном сдвиге. Старшие $n$ бит результата при этом получаются в регистре СЧП, а младшие в регистре множителя (рис. \ref{fig:binmul:Ioptimized}).
\begin{figure}[!ht]
    \centering
    \includegraphics{fig/mult-methods.5}
    \caption{Первый способ умножения (экономная версия)}\label{fig:binmul:Ioptimized}
\end{figure}


\section{Целочисленное умножение}
\label{ch::div}

Далее рассматриваются основные алгоритмы умножения беззнаковых целых чисел и целых чисел со знаком (в дополнительном коде). Следует отметить, что для удобства рассуждений используется \emph{дробное} масштабирование с масштабом $M=2^n$, где $n$ --- разрядность. 

\subsection{Беззнаковое умножение}
\label{ch:mult:sct:pc}

Используется уже рассмотренный выше базовый алгоритм умножения <<столбиком>>.
\begin{Example}\label{ex:binmul:pcI}
    Множитель $25=(11001)_2$, множимое $23=(10111)_2$. Перемножить числа I-м способом.
\end{Example}
\begin{Solve}
    Используется дробное масштабирование $M=2^5$:
    \begin{align*}
        25=\Number{,11001},\\
        23=\Number{,10111}.
    \end{align*}

    Умножение чисел приведено на рисунке \ref{fig:binmul:pcI}. Результат: 
    \begin{align*}
        (.1000111111)_2\cdot 2^{10}=575.
    \end{align*}
    
    Обратите внимание на временное ПРС --- перенос в разряд целой части, возникший на пятом шаге умножения. Чтобы его не терять, разрядность СЧП была увеличена на один разряд ($2\cdot5+1$), что было бы необязательно, если к СЧП в основном цикле умножения прибавлять не множимое, а половину (сдвинутое на разряд вправо) множимое. При таком подходе ПРС никогда не возникает, достаточно ровно $2n$ разрядов СЧП, и на последнем шаге сдвиг СЧП не выполняется.
\end{Solve}

\begin{figure}[!ht]
    \centering
    \begin{tabular}{c|r|l}
                                                                   \hline\hline
        множитель $\rightarrow$ & 
                                \multicolumn{1}{|c|}{СЧП $\rightarrow$}       
                                                        & примечание \\ \hline\hline
        \NumberLo{,1100}{1} & \Addition{.,00000 00000}
                                       {.,10111 .....}
                                       {0,10111 00000} & приб. мн-е; сдвиг\\ \hline
        \NumberLo{,.110}{0} &   \Number{.,01011 10000} & сдвиг\\ \hline
        \NumberLo{,..11}{0} &   \Number{.,00101 11000} & сдвиг\\ \hline
        \NumberLo{,...1}{1} & \Addition{.,00010 11100}
                                       {.,10111 .....}
                                       {0,11001 11100} & приб. мн-е; сдвиг\\ \hline
        \NumberLo{,....}{1} & \Addition{.,01100 11110}
                                       {.,10111 .....}
                                 {\fbox{1},00011 11110} & приб. мн-е; сдвиг\\ \hline
        \NumberLo{,....}{.} &   \Number{.,10001 11111} & модуль р-та!\\
    \end{tabular}
    \caption{Беззнаковое умножение $25\times 23$ (I сп., к примеру \ref{ex:binmul:pcI})}
    \label{fig:binmul:pcI}
\end{figure}


\begin{Example}\label{ex:binmul:pcII}
    Множитель $25=(11001)_2$, множимое $23=(10111)_2$. Перемножить числа II-м способом.
\end{Example}
\begin{Solve}
    Используется дробное масштабирование $M=2^5$:
    \begin{align*}
        25=\Number{,11001},\\
        23=\Number{,10111}.
    \end{align*}

    Умножение чисел приведено на рисунке \ref{fig:binmul:pcII}. Результат: 
    \begin{align*}
        (.1000111111)_2\cdot 2^{10}=575.
    \end{align*}
\end{Solve}

\begin{figure}[!ht]
    \centering
    \begin{tabular}{c|r|r|l}
                                                                   \hline\hline
        множитель $\rightarrow$ 
                            & \multicolumn{1}{|c|}{множимое $\leftarrow$}       
                                                     & \multicolumn{1}{|c|}{СЧП}       
                                                                                  & примечание \\ \hline\hline
        \NumberLo{,1100}{1} & \Number{,..... 10111} & \Addition {,00000 00000} 
                                                                {,..... 10111}
                                                                {,00000 10111} & приб. мн-е; сдвиг\\ \hline
        \NumberLo{,.110}{0} & \Number{,....1 0111.} &                           & сдвиг\\ \hline
        \NumberLo{,..11}{0} & \Number{,...10 111..} &                           & сдвиг\\ \hline
        \NumberLo{,...1}{1} & \Number{,..101 11...} & \Addition {,00000 10111} 
                                                                {,..101 11...}
                                                                {,00110 01111} & приб. мн-е; сдвиг\\ \hline
        \NumberLo{,....}{1} & \Number{,.1011 1....} & \Addition {,00110 01111} 
                                                                {,.1011 1....}
                                                                {,10001 11111} & модуль р-та!\\
    \end{tabular}
    \caption{Беззнаковое умножение $25\times 23$ (II сп., к примеру \ref{ex:binmul:pcII})}
    \label{fig:binmul:pcII}
\end{figure}


\begin{Example}\label{ex:binmul:pcIII}
    Множитель $25=(11001)_2$, множимое $23=(10111)_2$. Перемножить числа III-м способом.
\end{Example}
\begin{Solve}
    Используется дробное масштабирование $M=2^5$:
    \begin{align*}
        25=\Number{,11001},\\
        23=\Number{,10111}.
    \end{align*}

    Умножение чисел приведено на рисунке \ref{fig:binmul:pcIII}. Результат: 
    \begin{align*}
        (.1000111111)_2\cdot 2^{10}=575.
    \end{align*}
\end{Solve}

\begin{figure}[!ht]
    \centering
    \begin{tabular}{c|r|l}
                                                                   \hline\hline
        множитель $\leftarrow$ 
                                & \multicolumn{1}{|c|}{СЧП $\leftarrow$}       
                                                           & примечание\\ \hline\hline
        \NumberMid{,}{1}{1001} & \Addition{,00000 00000}
                                          {,..... 10111}
                                          {,00000 10111} & приб. мн-е; сдвиг\\ \hline
        \NumberMid{,}{1}{001.} & \Addition{,00001 0111.}
                                          {,..... 10111}
                                          {,00010 00101} & приб. мн-е; сдвиг\\ \hline
        \NumberMid{,}{0}{01..} &   \Number{,00100 0101.} & сдвиг\\ \hline
        \NumberMid{,}{0}{1...} &   \Number{,01000 101..} & сдвиг\\ \hline
        \NumberMid{,}{1}{....} & \Addition{,10001 01...}
                                          {,..... 10111}
                                          {,10001 11111} & модуль р-та!\\ 
    \end{tabular}
    \caption{Беззнаковое умножение $25\times 23$ (III сп., к примеру \ref{ex:binmul:pcIII})}
    \label{fig:binmul:pcIII}
\end{figure}


\begin{Example}\label{ex:binmul:pcIV}
    Множитель $25=(11001)_2$, множимое $23=(10111)_2$. Перемножить числа IV-м способом.
\end{Example}
\begin{Solve}
    Используется дробное масштабирование с множителем $2^5$:
    \begin{align*}
        25=\Number{,11001},\\
        23=\Number{,10111}.
    \end{align*}

    Умножение чисел приведено на рисунке \ref{fig:binmul:pcIV}. Результат: 
    \begin{align*}
        (.1000111111)_2\cdot 2^{10}=575.
    \end{align*}
    
    Следует обратить внимание на множимое, которое на первом шаге уже сдвинуто на один разряд вправо.
\end{Solve}

\begin{figure}[!ht]
    \centering
    \begin{tabular}{c|r|r|l}
                                                                   \hline\hline
        множитель $\leftarrow$ 
                               & \multicolumn{1}{|c|}{множимое $\rightarrow$}       
                                                       & \multicolumn{1}{|c|}{СЧП}       
                                                                                 & примечание \\ \hline\hline
        \NumberMid{,}{1}{1001} & \Number{,.1011 1....} & \Addition{,00000 00000} 
                                                                  {,.1011 1....}
                                                                  {,01011 10000} & приб. мн-е; сдвиг;\\ \hline
        \NumberMid{,}{1}{001.} & \Number{,..101 11...} & \Addition{,01011 10000} 
                                                                  {,..101 11...}
                                                                  {,10001 01000} & приб. мн-е; сдвиг\\ \hline
        \NumberMid{,}{0}{01..} & \Number{,...10 111..} &                         & сдвиг\\ \hline
        \NumberMid{,}{0}{1...} & \Number{,....1 0111.} &                         & сдвиг\\ \hline
        \NumberMid{,}{1}{....} & \Number{,..... 10111} & \Addition{,10001 01000} 
                                                                  {,..... 10111}
                                                                  {,10001 11111} & модуль р-та!\\
    \end{tabular}
    \caption{Беззнаковое умножение $25\times 23$ (IV сп., к примеру \ref{ex:binmul:pcIV})}
    \label{fig:binmul:pcIV}
\end{figure}

\subsection{Беззнаковое умножение с ускорением 2-го порядка}
\label{ss::binmul:fast:double}

В данном методе ускорения работают с четверичными цифрами. Разряды двоичного числа группируются по два и сдвиги множителя (а также множимого или суммы частичных произведений) выполняются сразу на два двоичных разряда. Количество разрядов двоичной сетки выбирается кратным двум. Такой подход сокращает количество шагов умножения вдвое.

На $i$-м шаге умножения при анализе пары двоичных разрядов $(a_{2i+1},a_{2i})$ множителя $A$ должны выполняться следующие действия:
\[
    \begin{tabular}{cc|l}
        \hline\hline
        $a_{2i+1}$ & $a_{2i}$ & Действие над СЧП\\
        \hline\hline
        0 & 0 & $+0$, нет действий\\
        0 & 1 & $+M$, прибавить множимое $M$\\
        1 & 0 & $+2M$, прибавить $M$, сдвинутое на 1-н разряд влево\\
        1 & 1 & $+3M$, прибавить утроенное множимое\\
        \hline
    \end{tabular}
\]

В процессе умножения легко получить удвоенное множимое <<на лету>>, с помощью сдвига. Утроенное же множимое потребует предварительных вычислений. Но этого можно избежать: $3=4-1$, что в двоичном представлении
\[(\Number{11})_2=(\Number{1 00})_2-\Number{1}.\]

\begin{Rule}\label{rule:binbul:double:fastTripple}
    На текущем шаге умножиния вместо сложения с утроенным множимым можно выполнить вычитание множимого $(-M)$ и учесть единицу переноса в старшую пару на \emph{следующем шаге}.
\end{Rule}
Таким образом, находить утроенное множимое не потребуется.

В целях наглядности используем следующие обозначения для переносов $p$:
\[
    p=
    \begin{cases}
        \text{\Number{(+0)}} & \text{при отсутствии переноса ($p=0$)},\\
        \text{\Number{(+1)}} & \text{при единице переноса ($p=1$)}.
    \end{cases}
\]

Для способов, в которых выполняется анализ младших разрядов множителя (I,~II~сп.), правило \ref{rule:binbul:double:fastTripple} работает без поправок. Для I и II способов умножения комбинации разрядов $(a_{2i+1},a_{2i})$ и переноса $p_i$ текущего $i$-го шага и соответвтвующие им действие над СЧП и перенос $p_{i+1}$ приведены в таблице 
\[
    \begin{tabular}{ccc||l|c}
        \hline\hline
        $a_{2i+1}$ & $a_{2i}$   & $p_i$      & СЧП   & $p_{i+1}$\\
        \hline\hline
        \Number{0} & \Number{0} & \Number{(+0)} & $+0$  & \Number{(+0)}\\
        \Number{0} & \Number{1} & \Number{(+0)} & $+M$  & \Number{(+0)}\\
        \Number{1} & \Number{0} & \Number{(+0)} & $+2M$ & \Number{(+0)}\\
        \Number{1} & \Number{1} & \Number{(+0)} & $-M$  & \Number{(+1)}\\
        
        \Number{0} & \Number{0} & \Number{(+1)} & $+M$  & \Number{(+0)}\\
        \Number{0} & \Number{1} & \Number{(+1)} & $+2M$ & \Number{(+0)}\\
        \Number{1} & \Number{0} & \Number{(+1)} & $-M$  & \Number{(+1)}\\
        \Number{1} & \Number{1} & \Number{(+1)} & $+0$  & \Number{(+1)}\\
        \hline
    \end{tabular}
\]

Для способов, в которых выполняется анализ старших разрядов множителя (III,~IV~сп.), правило \ref{rule:binbul:double:fastTripple} требует поправок в силу того, что распространять перенос попросту \emph{некуда}. Но можно обойтись \emph{без распространения переноса}, анализируя разряды младшей пары, находящейся справа от текущей:
\[\cdots \underbrace{(a_{2i+1}a_{2i})}_\textit{текущая} \underbrace{(a_{2i-1}a_{2i-2})}_\textit{младшая}\cdots\]
При этом существует неопределенность возникновения переноса из младшей пары $(a_{2i-1},a_{2i-2})$:
\[
    \begin{tabular}{cc||l}
        \hline\hline
        $(a_{2i-1}$ & $a_{2i-2})$ & Перенос\\
        \hline\hline
        \Number{0} & \Number{0} & \Number{(+0)}\\
        \Number{0} & \Number{1} & \Number{(+0)}\\
        \Number{1} & \Number{0} & ? \emph{Возможно\ldots}\\
        \Number{1} & \Number{1} & \Number{(+1)}\\
        \hline
    \end{tabular}
\]

Действительно, если младшая пара \Number{10} и \emph{в неё} не будет переноса, то \emph{из неё} переноса также не будет. Но если в неё перенос будет, то, уходя от получившейся комбинации \Number{11} по правилу \ref{rule:binbul:double:fastTripple}, из неё будет сгенерирован перенос. Добьёмся определённости с переносом --- будем уходить от комбинации \Number{10} по правилу:
\[
    (\Number{10})_2=(\Number{1 00})_2-(\Number{10})_2,
\]
и тогда перенос будет \emph{всегда}, если старший бит младшей пары равен \Number{1}! 

\begin{Rule}[дополнение к правилу \ref{rule:binbul:double:fastTripple}]\label{rule:binbul:double:fastBin}
    На текущем шаге умножения вместо сложения с удвоенным множимым (комбинация \Number{10}), следует выполнить вычитание удвоенного множимого $(-2M)$. 
\end{Rule}
При этом, в случае, если перенос в пару \Number{10} будет, то с учётом правила \ref{rule:binbul:double:fastBin} нужно: $(4-2+1)$ --- вычитать множимое. Таким образом, с учётом правил \ref{rule:binbul:double:fastTripple} и \ref{rule:binbul:double:fastBin} общий алгоритм умножения заметно упрощается. Нужно анализировать разряды текущей пары и только старший разряд $a_{2i-1}$ младшей пары.

Для III и IV способов комбинации разрядов $(a_{2i+1},a_{2i})(a_{2i-1},\ldots)$ текущего $i$-го шага и соответвтвующее им действие приведены в таблице
\[
    \begin{tabular}{cc|c||l}
        \hline\hline
        $(a_{2i+1}$ & $a_{2i})$   & $(a_{2i-1}$ & СЧП \\
        \hline\hline
        \Number{0} & \Number{0} & \Number{0} & $+0$ \\
        \Number{0} & \Number{1} & \Number{0} & $+M$ \\
        \Number{1} & \Number{0} & \Number{0} & $-2M$\\
        \Number{1} & \Number{1} & \Number{0} & $-M$ \\
        
        \Number{0} & \Number{0} & \Number{1} & $+M$ \\
        \Number{0} & \Number{1} & \Number{1} & $+2M$\\
        \Number{1} & \Number{0} & \Number{1} & $-M$ \\
        \Number{1} & \Number{1} & \Number{1} & $+0$ \\
        \hline
    \end{tabular}
\]


\begin{Example}\label{ex:binmul:fast:Icrp}
    Множитель $109=(1101101)_2$, множимое $233=(11101001)_2$. Перемножить числа I-м способом с ускорением второго порядка.
\end{Example}
\begin{proof}[Решение]
    Используем дробное масштабирование с множителем $2^8$. Тогда 
    \begin{align*}
        109=\Number{,01101101},\\
        233=\Number{,11101001}.
    \end{align*}

    Умножение чисел приведено на рисунке \ref{fig:binmul:fast:Icrp}. Результат: 
    \[
        (.01100011 00110101)_2\cdot 2^{16}=25397.
    \]
\end{proof}

\begin{figure}[!ht]
    \centering
    \begin{tabular}{c|r|l}
                                                                   \hline\hline
        множитель $\rightarrow$ & 
                                \multicolumn{1}{|c|}{СЧП $\rightarrow$}       
                                                        & примечание \\ \hline\hline
        \NumberLo{00,011011}{01|(+0)} & \Addition{00,00000000 00000000}  
                                              {00,11101001 ........}
                                              {00,11101001 00000000} & $+M$; \Number{(+0)}; сдвиг\\ \hline
        \NumberLo{..,000110}{11|(+0)} & \Addition{00,00111010 01000000}
                                              {11,00010111 ........}
                                              {11,01010001 01000000} & $-M$; \Number{(+1)}; сдвиг\\ \hline
        \NumberLo{..,..0001}{10|(+1)} & \Addition{11,11010100 01010000}
                                              {11,00010111 ........}
                                              {10,11101011 01010000} & $-M$; \Number{(+1)}; сдвиг\\ \hline
        \NumberLo{..,....00}{01|(+1)} & \Addition{11,10111010 11010100}
                                              {01,11010010 ........}
                                              {01,10001100 11010100} & $+2M$; \Number{(+0)}; сдвиг\\ \hline
        \NumberLo{..,......}{00|(+0)} &   \Number{00,01100011 00110101} & $+0$; результат!\\ \hline
    \end{tabular}
    \caption{Беззнаковое умножение $109\times 233$ с ускорением второго порядка (I сп., к примеру \ref{ex:binmul:fast:Icrp})}
    \label{fig:binmul:fast:Icrp}
\end{figure}


\begin{Example}\label{ex:binmul:fast:IIcrp}
    Множитель $183=(10110111)_2$, множимое $242=(11110010)_2$. Перемножить числа II-м способом с ускорением второго порядка.
\end{Example}
\begin{proof}[Решение]
    Используем дробное масштабирование с множителем $2^8$. Тогда 
    \begin{align*}
        183=\Number{,10110111},\\
        242=\Number{,11110010}.
    \end{align*}

    Умножение чисел приведено на рисунке \ref{fig:binmul:fast:pcII}. Результат: 
    \[
        (.10101100 11111110)_2\cdot 2^{16}=44286.
    \]
\end{proof}

\begin{figure}[!ht]
    \centering
    \begin{tabular}{c|r|l}
        \hline\hline
        множитель $\rightarrow$ 
            & \multicolumn{1}{|c|}{множимое $\leftarrow$; СЧП}
            & примечание \\ 
        \hline\hline
        \NumberLo{00,101101}{11|(+0)}  
            & \Stack{
                \Register{МН-Е}{00,00000000 11110010}}{
                \Addition{00,00000000 00000000}
                         {11,11111111 00001110}
                         {11,11111111 00001110}}  
            & $-M$; \Number{(+1)}; сдвиг\\ \hline
        \NumberLo{..,001011}{01|(+1)}  
            & \Stack{
                \Register{МН-Е}{00,00000011 110010..}}{
                \Addition{11,11111111 00001110}
                         {00,00000111 10010...}
                         {00,00000110 10011110}}  
            & $+2M$; \Number{(+0)}; сдвиг\\ \hline
        \NumberLo{..,..0010}{11|(+0)}  
            & \Stack{\Register{МН-Е}{00,00001111 0010....}}{
                \Addition{00,00000110 10011110}
                         {11,11110000 1110....}
                         {11,11110111 01111110}}  
            & $-M$; \Number{(+1)}; сдвиг\\ \hline
        \NumberLo{..,....00}{10|(+1)}  
            & \Stack{
                \Register{МН-Е}{00,00111100 10......}}{
                \Addition{11,11110111 01111110}
                         {11,11000011 10......}
                         {11,10111010 11111110}}  
            & $-M$; \Number{(+1)}; сдвиг\\ \hline
        \NumberLo{..,......}{00|(+1)}  
            & \Stack{
                \Register{МН-Е}{00,11110010 ........}}{
                \Addition{11,10111010 11111110}
                         {00,11110010 ........}
                         {00,10101100 11111110}}  
            & $+M$; результат!\\ \hline
    \end{tabular}
    \caption{Беззнаковое умножение $183\times 242$ с ускорением второго порядка (II сп., к примеру \ref{ex:binmul:fast:IIcrp})}
    \label{fig:binmul:fast:pcII}
\end{figure}


\begin{Example}\label{ex:binmul:fast:III}
    Множитель $167=(10100111)_2$, множимое $218=(11011010)_2$. Перемножить числа III-м способом с ускорением второго порядка.
\end{Example}
\begin{proof}[Решение]
    Используем дробное масштабирование с множителем $2^8$. Тогда 
    \begin{align*}
        167=\Number{10100111},\\
        218=\Number{11011010}.
    \end{align*}

    Умножение чисел приведено на рисунке \ref{fig:binmul:fast:III}. Результат: 
    \[
        (.10001110 00110110)_2\cdot 2^{16}=36406.
    \]
\end{proof}

\begin{figure}[!ht]
    \centering
    \begin{tabular}{c|r|l}
        \hline\hline
        множитель $\leftarrow$ 
            & \multicolumn{1}{|c|}{СЧП $\leftarrow$}       
            & примечание \\ 
        \hline\hline
        \NumberHi{00,1}{0100111} 
            & \Addition{00,00000000 00000000}
                       {00,00000000 11011010}
                       {00,00000000 11011010} 
            & $+M$; сдвиг\\ \hline
        \NumberHi{10,1}{00111..} 
            & \Addition{00,00000011 01101000}
                       {11,11111111 00100110}
                       {00,00000010 10001110} 
            & $-M$; сдвиг\\ \hline
        \NumberHi{10,0}{111....} 
            & \Addition{00,00001010 00111000}
                       {11,11111110 01001100}
                       {00,00001000 10000100} 
            & $-2M$; сдвиг\\ \hline
        \NumberHi{01,1}{1......} 
            & \Addition{00,00100010 00010000}
                       {00,00000001 10110100}
                       {00,00100011 11000100} 
            & $+2M$; сдвиг\\ \hline
        \NumberHi{11,.}{.......} 
            & \Addition{00,10001111 00010000}
                       {11,11111111 00100110}
                       {00,10001110 00110110} 
            & $-M$; результат!\\ \hline
    \end{tabular}
    \caption{Беззнаковое умножение $167\times 218$ с ускорением второго порядка (III сп., к примеру \ref{ex:binmul:fast:III})}
    \label{fig:binmul:fast:III}
\end{figure}


\begin{Example}\label{ex:binmul:fast:IV}
    Множитель $222=(11011110)_2$, множимое $227=(11100011)_2$. Перемножить числа IV-м способом с ускорением второго порядка.
\end{Example}
\begin{proof}[Решение]
    Используем дробное масштабирование с множителем $2^8$. Тогда 
    \begin{align*}
        222=\Number{,11011110},\\
        227=\Number{,11100011}.
    \end{align*}

    Умножение чисел приведено на рисунке \ref{fig:binmul:fast:IV}. Результат: 
    \[
        (.11000100 11011010)_2\cdot 2^{16}=50394.
    \]
\end{proof}

\begin{figure}[!ht]
    \centering
    \begin{tabular}{c|r|l}
        \hline\hline
        множитель $\leftarrow$ 
            & \multicolumn{1}{|c|}{множимое $\rightarrow$; СЧП}
            & примечание \\ 
        \hline\hline
        \NumberHi{00,1}{1011110} 
            & \Stack{
                \Register{МН-Е}{00,11100011 ........}}{
                \Addition{00,00000000 00000000}
                         {00,11100011 ........}
                         {00,11100011 00000000}}  
            & $+M$; сдвиг\\ \hline
        \NumberHi{11,0}{11110..} 
            & \Stack{
                \Register{МН-Е}{00,00111000 11......}}{
                \Addition{00,11100011 00000000}
                         {11,11000111 01......}
                         {00,10101010 01000000}} 
            & $-M$; сдвиг\\ \hline
        \NumberHi{01,1}{110....} 
            & \Stack{
                \Register{МН-Е}{00,00001110 0011....}}{
                \Addition{00,10101010 01000000}
                         {00,00011100 0110....}
                         {00,11000110 10100000}}  
            & $+2M$; сдвиг\\ \hline
        \NumberHi{11,1}{0......}  
            & \Stack{
                \Register{МН-Е}{00,00000011 100011..}}{
                \Number{00,11000110 10100000}}  
            & $+0$; сдвиг\\ \hline
        \NumberHi{10,.}{.......} 
            & \Stack{
                \Register{МН-Е}{00,00000000 11100011}}{
                \Addition{00,11000110 10100000}
                         {11,11111110 00111010}
                         {00,11000100 11011010}}  
            & $-2M$; результат!\\ \hline
    \end{tabular}
    \caption{Беззнаковое умножение $222\times 227$ с ускорением второго порядка (IV сп., к примеру \ref{ex:binmul:fast:IV})}
    \label{fig:binmul:fast:IV}
\end{figure}
        
Методику для способов умножения с анализом старших разрядов множителя (III, IV) можно перенести на способы с анализом младших разрядов множителя (I, II) без каких-либо поправок (см. пример \ref{ex:binmul:fast:Itriplet}).
\begin{Example}\label{ex:binmul:fast:Itriplet}
    Множитель $155=(10011011)_2$, множимое $233=(11101001)_2$. Перемножить числа I-м способом с ускорением второго порядка.
\end{Example}
\begin{proof}[Решение]
    Используем дробное масштабирование с множителем $2^8$. Тогда 
    \begin{align*}
        222=\Number{,10011011},\\
        227=\Number{,11101001}.
    \end{align*}

    Умножение чисел приведено на рисунке \ref{fig:binmul:fast:Itriplet}. Результат: 
    \[
        (.10001101 00010011)_2\cdot 2^{16}=36115.
    \]
\end{proof}

\begin{figure}[!ht]
    \centering
    \begin{tabular}{c|r|l}
        \hline\hline
        множитель $\rightarrow$ 
            & \multicolumn{1}{|c|}{СЧП $\rightarrow$}
            & примечание \\ 
        \hline\hline
        \NumberLo{00,100110}{11|.}  
            &  \Addition{00,00000000 00000000}
                        {11,00010111 00000000}
                        {11,00010111 00000000}  
            & $-M$; сдвиг\\ \hline
        \NumberLo{..,001001}{10|1}  
            &  \Addition{11,11000101 11000000}
                        {11,00010111 00000000}
                        {10,11011100 11000000}  
            & $-M$; сдвиг\\ \hline
        \NumberLo{..,..0010}{01|1}  
            &  \Addition{11,10110111 00110000}
                        {01,11010010 00000000}
                        {01,10001001 00110000}  
            & $+2M$; сдвиг\\ \hline
        \NumberLo{..,....00}{10|0}  
            &  \Addition{00,01100010 01001100}
                        {10,00101110 00000000}
                        {10,10010000 01001100}  
            & $-2M$; сдвиг\\ \hline
        \NumberLo{..,......}{00|1}  
            &  \Addition{11,10100100 00010011}
                        {00,11101001 00000000}
                        {00,10001101 00010011}  
            & $+M$; результат\\ \hline
    \end{tabular}
    \caption{Беззнаковое умножение $155\times 233$ с ускорением второго порядка (I сп., к примеру \ref{ex:binmul:fast:Itriplet})}
    \label{fig:binmul:fast:Itriplet}
\end{figure}


\subsection{Беззнаковое умножение с ускорением 3-го порядка}

Подход аналогичен умножению с ускорением второго порядка (см. подраздел \ref{ss::binmul:fast:double}).

В данном методе ускорения работают с восьмиричными цифрами. Разряды двоичного числа группируются по три и сдвиги множителя (а также множимого или суммы частичных произведений, в зависимости от способа) выполняются сразу на три двоичных разряда (одну восьмиричную цифру). Количество разрядов двоичной сетки выбирается кратным трем. Такой подход сокращает количество шагов умножения втрое.

На $i$-м шаге умножения при анализе текущей двоично-кодированной восьмиричной цифры 
\[
    (a_{3i+2},a_{3i+1},a_{3i})
\]
множителя $A$ должны выполняться следующие действия:
\[
    \begin{tabular}{ccc|l}
        \hline\hline
        $a_{3i+2}$ & $a_{3i+1}$ & $a_{3i}$ & Действие над СЧП\\
        \hline\hline
        0 & 0 & 0 & $+0$\\
        0 & 0 & 1 & $+M$\\
        0 & 1 & 0 & $+2M$\\
        0 & 1 & 1 & $+3M$\\
        1 & 0 & 0 & $+4M$\\
        1 & 0 & 1 & $+5M$\\
        1 & 1 & 0 & $+6M$\\
        1 & 1 & 1 & $+7M$\\
        \hline
    \end{tabular}
\]

$M,2M,4M$ легко получаются <<на лету>> сдвигом множимого на соответствующее количество разрядов. Проблему в виде необходимости предварительных вычислений представляют $3M,5M,6M,7M$. Увы, совсем без предварительных вычислений не обойтись: обычно заранее вычисляют утроенное множимое $3M$. 

Для способов со сдвигом множителя впрво (т.е. с анализом младших разрядов множителя; I,II способы умножения) от проблемных слагаемых можно уйти следующим образом:
\begin{itemize}
    \item $3M$ вычисляется;
    \item $5M$ представляется как $8M-3M$, т.е. на текущем шаге выполняется вычитание $3M$, а к анализируемой восьмиричной цифре следующего шага прибавляется единица (переноса);
    \item $6M$ можно получить сдвигом влево $3M$ или, соблюдая равенство $(+8M-2M)$, выполнить вычитание удвоенного множимого, и прибавить единицу переноса к анализируемой восьмиричной цифре на следующем шаге.
    \item $7M$ представляется как $(+8M-M)$.
\end{itemize}

Для способов со сдвигом множителя влево (т.е. III, IV cпособы умножения) следует учесть возникновение переноса из младшей тройки заранее.
\[
    \cdots\underbrace{(a_{3i+2},a_{3i+1},a_{3i})}_\textit{текущая $i$-я}\underbrace{(a_{3i-1},a_{3i-2},a_{3i-3})}_\textit{младшая $(i-1)$-я}\cdots
\]

Проанализируем возможность возникновения переноса из младшей тройки:
\[
    \begin{tabular}{c|ccc|l}
        \hline\hline
          &$a_{3i-1}$ & $a_{3i-2}$ & $a_{3i-3}$ & Сгенерируется перенос\\
        \hline\hline
        0 & 0 & 0 & 0 & нет\\
        1 & 0 & 0 & 1 & нет\\
        2 & 0 & 1 & 0 & нет\\
        3 & 0 & 1 & 1 & нет (т.к. $3M$ вычисляется отдельно)\\
        4 & 1 & 0 & 0 & ? \emph{Возможно\ldots}\\
        5 & 1 & 0 & 1 & ? \emph{Возможно\ldots}\\
        6 & 1 & 1 & 0 & ? \emph{Возможно\ldots}\\
        7 & 1 & 1 & 1 & да\\
        \hline
    \end{tabular}
\]

Следует добиться однозначности в спорных моментах.
\begin{itemize}
    \item $(100)$ требует либо $+4M$, либо $+5M=(+8M-3M)$. Чтобы перенос был всегда, вместо $+4M$ используем $(+8M-4M)$.
    \item $(101)$ требует либо $+5M=(+8M-3M)$, либо $+6M=+2\cdot3M$. Чтобы перенос был всегда, вместо $+6M$ используем $(+8M-2M)$.
    \item $(110)$ требует либо $+6M=+2\cdot 3M$, либо $+7M=(+8M-M)$. Чтобы перенос был всегда, вместо $+6M$ используем $+8M-2M$.
\end{itemize}

Обобщенные результаты можно свести в таблицы:
\[
    \begin{tabular}{cc}
        способы I,II 
            & способы III,IV, а также и I,II\\
        \begin{tabular}{ccc|c||l|c}
            \hline\hline
            $(a_{3i+2}$ & $a_{3i+1}$ & $a_{3i})$ & $p_i$ & СЧП & $p_{i+1}$\\
            \hline\hline
            0 & 0 & 0 & \Number{(+0)} & $+0$  & \Number{(+0)}\\
            0 & 0 & 1 & \Number{(+0)} & $+M$  & \Number{(+0)}\\
            0 & 1 & 0 & \Number{(+0)} & $+2M$ & \Number{(+0)}\\
            0 & 1 & 1 & \Number{(+0)} & $+3M$ & \Number{(+0)}\\
            1 & 0 & 0 & \Number{(+0)} & $+4M$ & \Number{(+0)}\\
            1 & 0 & 1 & \Number{(+0)} & $-3M$ & \Number{(+1)}\\ \hline
            1 & 1 & 0 & \Number{(+0)} & $-2M$ & \Number{(+1)}\\
            1 & 1 & 0 & \Number{(+0)} & $+2\cdot 3M$ & \Number{(+0)}\\ \hline
            1 & 1 & 1 & \Number{(+0)} & $-M$  & \Number{(+1)}\\ \hline
            
            0 & 0 & 0 & \Number{(+1)} & $+M$  & \Number{(+0)}\\
            0 & 0 & 1 & \Number{(+1)} & $+2M$ & \Number{(+0)}\\
            0 & 1 & 0 & \Number{(+1)} & $+3M$ & \Number{(+0)}\\
            0 & 1 & 1 & \Number{(+1)} & $+4M$ & \Number{(+0)}\\
            1 & 0 & 0 & \Number{(+1)} & $-3M$ & \Number{(+1)}\\ \hline
            1 & 0 & 1 & \Number{(+1)} & $-2M$ & \Number{(+1)}\\
            1 & 0 & 1 & \Number{(+1)} & $+2\cdot 3M$ & \Number{(+0)}\\ \hline
            1 & 1 & 0 & \Number{(+1)} & $-M$  & \Number{(+1)}\\
            1 & 1 & 1 & \Number{(+1)} & $+0$  & \Number{(+1)}\\
            \hline
        \end{tabular}
            &
            \begin{tabular}{ccc|c||l}
                \hline\hline
                $(a_{3i+2}$ & $a_{3i+1}$ & $a_{3i})$ & $(a_{3i-1}$ & СЧП\\
                \hline\hline
                0 & 0 & 0 & 0 & $+0$\\
                0 & 0 & 1 & 0 & $+M$\\
                0 & 1 & 0 & 0 & $+2M$\\
                0 & 1 & 1 & 0 & $+3M$\\
                1 & 0 & 0 & 0 & $-4M$\\
                1 & 0 & 1 & 0 & $-3M$\\
                1 & 1 & 0 & 0 & $-2M$\\
                1 & 1 & 1 & 0 & $-M$\\ \hline
                0 & 0 & 0 & 1 & $+M$\\
                0 & 0 & 1 & 1 & $+2M$\\
                0 & 1 & 0 & 1 & $+3M$\\
                0 & 1 & 1 & 1 & $+4M$\\
                1 & 0 & 0 & 1 & $-3M$\\
                1 & 0 & 1 & 1 & $-2M$\\
                1 & 1 & 0 & 1 & $-M$\\
                1 & 1 & 1 & 1 & $+0$\\
                \hline
            \end{tabular}
    \end{tabular}
\]

\begin{Example}\label{ex:binmul:fast:IIIx3Nibble}
    Множитель $3306=(6352)_8=(110011101010)_2$, множимое $2973=(5635)_8=(101110011101)_2$. Перемножить числа III-м способом с ускорением третьего порядка.
\end{Example}
\begin{proof}[Решение]
    Представим числа в двоично-кодированной восьмиричной системе, выбрав масштабирующий множитель $8^4=2^{12}$. Тогда 
    \begin{align*}
        3306=\Number{,110011101010},\\
        2973=\Number{,101110011101}.
    \end{align*}

    Утроенное множимое $(M+2M)$ формируется отдельно:
    \[
        \Addition{.,............ 101110011101}
                 {.,...........1 01110011101.}
                 {0,000000000010 001011010111}
    \]

    Дополнительный код для вычитания утроенного множимого $(-3M)$:
    \[
        \Number{1,111111111101 110100101001}
    \]
        
    Умножение чисел приведено на рисунке \ref{fig:binmul:fast:IIIx3Nibble}. Результат: 
    \[
        (.100101011111 100110000010)_2\cdot 2^{24}=9828738.
    \]
\end{proof}

\begin{figure}[!ht]
    \centering
    \begin{tabular}{c|r|l}
        \hline\hline
        множитель $\rightarrow$ 
            & \multicolumn{1}{|c|}{СЧП $\rightarrow$}
            & прим. \\ 
        \hline\hline
        \NumberHi{...,1}{10011101010}  
            & \Addition{0,000000000000 000000000000}
                       {.,............ 101110011101}
                       {0,000000000000 101110011101}  
            & $+M; \ll$\\ \hline
        \NumberHi{110,0}{11101010...}  
            & \Addition{0,000000000101 110011101...}
                       {1,111111111110 10001100011.}
                       {0,000000000100 010110101110}  
            & $-2M; \ll$\\ \hline
        \NumberHi{011,1}{01010......}  
            & \Addition{0,000000100010 110101110...}
                       {.,..........10 1110011101..}
                       {0,000000100101 101111100100}  
            & $+4M; \ll$\\ \hline
        \NumberHi{101,0}{10.........}  
            & \Addition{0,000100101101 111100100...}
                       {1,111111111101 110100101001}
                       {0,000100101011 110001001001}  
            & $-3M; \ll$\\ \hline
        \NumberHi{010,.}{...........}  
            & \Addition{0,100101011110 001001001...}
                       {.,...........1 01110011101.}
                       {0,100101011111 100110000010}  
            & $+2M; End!$\\ \hline  
    \end{tabular}
    \caption{Беззнаковое умножение $3306\times 2973$ с ускорением третьего порядка (III сп., к примеру \ref{ex:binmul:fast:IIIx3Nibble})}
    \label{fig:binmul:fast:IIIx3Nibble}
\end{figure}

\subsection{Умножение в ДК с ручной коррекцией}

Если использовать дробное масштабирование и считать старший ($-1$-й) разряд дробной части знаковым, то дополнительный код дробного числа $A$:
\begin{equation}
    \DC{A} = 
    \begin{cases}
        |A|,      & \text{если $A$ положительно},\\
        1-|A|,    & \text{если $A$ отрицательно}.\\
    \end{cases}
    \label{eq:ch:mult:sct:dcmanual:dc}
\end{equation}

Если перемножить дополнительные коды чисел, то в зависимости от знаков операндов, может потребоваться коррекция результата.

Возможны следующие варианты сочетаний знаков сомножимых.
\begin{itemize}
    \item \emph{Оба сомножителя положительны}. Поправок не требуется.
    
    \item \emph{Один из сомножителей отрицателен}. Пусть $A$ --- отрицательно, $B$ --- положительно. Произведению дополнительных кодов будет соответствовать:
    \[
        (1-|A|)\cdot|B|=|B|-|A|\cdot|B|.
    \] 
    
    Чтобы получить корректный дополнительный код результата: $(1-|AB|)$, следует скорректировать полученный результат на величину $(1-|B|)$. То есть прибавить к результату $\DC{-B}=(1-|B|)$.
    
    \item \emph{Оба сомножителя отрицательны}. Результат произведения дополнительных кодов: 
    \[
        (1-|A|)(1-|B|)=1-|A|-|B|+|AB|
    \] 
    требует коррекции. Прибавив поправку $(|A|+|B|)$, получим $(1+|AB|)$, который вследствие переноса единицы в целую часть эквивалентен правильному $|AB|$.
\end{itemize}

\begin{Note}[Упрощенные правила коррекции]
    Правила ручной коррекции можно упростить: достаточно проверить знак каждого аргумента и, если этот аргумент отрицателен, то из суммы частичных произведений \emph{вычитается парный} отрицательному аргумент.
\end{Note}

В дополнительном коде знаковые разряды не приходится обрабатывать отдельно. Коррекцию, если она необходима, можно выполнить как до, так и после основного цикла умножения.

Умножение дополнительных кодов выполняется по правилам перемножения положительных чисел, так как, согласно формуле \eqref{eq:ch:mult:sct:dcmanual:dc} получающийся дополнительный код можно рассматривать как положительное дробное число. Поэтому основной цикл умножения можно выполнить по правилам перемножения модулей в прямом коде (см. раздел \ref{ch:mult:sct:pc}).

\begin{Example}\label{ex:binmul:dcmanualIoneof}
    Множитель $25=(11001)_2$, множимое $-23=(-10111)_2$. Перемножить числа в дополнительном коде с ручной коррекцией. Использовать I способ умножения.
\end{Example}
\begin{proof}[Решение]
    Чтобы правильно выбрать масштаб, достаточно вспомнить диапазон представления целых чисел в $n$-разрядном коде:
    \[
        [-2^{n-1},+(2^{n-1}-1)].
    \]

    Видно, что для наших чисел нужно $n=6$. Используем дробное масштабирование с $M=2^6$. При этом знаковым разрядом будет считаться старший разряд \emph{дробной} части. Тогда 
    \begin{align*}
        \DC{25\cdot 2^{-6}} =\Number{,011001},\\
        \DC{-23\cdot 2^{-6}}=1-0.010111=\underbrace{0.101001}_{\text{ДК } \equiv \text{ положит. число!}}=\Number{,101001}.
    \end{align*}

    Умножение дополнительных кодов как положительных чисел и последующая коррекция приводятся на рисунке \ref{fig:binmul:dcmanualIoneof}.
    
    Проверка результата: 
    \[\Number{,110111 000001}\Rightarrow(-.001000111111)_2\cdot 2^{12}\Rightarrow(-1000111111)_2=-575.\]
\end{proof}

\begin{figure}[!ht]
    \centering
    \begin{tabular}{c|r|l}
                                                                   \hline\hline
        мн-ль $\rightarrow$ 
                              & \multicolumn{1}{|c|}{СЧП $\rightarrow$}       
                                                           & прим. \\ \hline\hline
        \NumberLo{,01100}{1} & \Addition{.,000000 000000}
                                        {.,101001 ......}
                                        {.,101001 000000} & +мн-е ($\DC{-23}$); сдвиг\\ \hline
        \NumberLo{,.0110}{0} &   \Number{.,.10100 100000} & сдвиг\\ \hline
        \NumberLo{,..011}{0} &   \Number{.,..1010 010000} & сдвиг\\ \hline
        \NumberLo{,...01}{1} & \Addition{.,...101 001000}
                                        {.,101001 ......}
                                        {.,101110 001000} & +мн-е; сдвиг\\ \hline
        \NumberLo{,....0}{1} & \Addition{.,.10111 000100}
                                        {.,101001 ......}
                                        {1,000000 000100} & +мн-е; сдвиг\\ \hline
        \NumberLo{,....}{0} &    \Number{.,100000 000010} & сдвиг; \\ \hline
        \NumberLo{,....}{.} &    \Number{.,.10000 000001} & Рез-т на коррекцию!\\ \hline\hline
                             & \Addition{0,010000 000001}
                                        {.,100111 ......}
                                        {.,110111 000001} & поправка $+\DC{-0.011001}$\\ \hline
                             &   \Number{.,110111 000001} & Рез-т в ДК!\\
    \end{tabular}
    \caption{Умножение $\DC{25}\times \DC{-23}$ (I сп., к примеру \ref{ex:binmul:dcmanualIoneof})}
    \label{fig:binmul:dcmanualIoneof}
\end{figure}


\begin{Example}\label{ex:binmul:dcmanualIboth}
    Множитель $-25=(-11001)_2$, множимое $-23=(-10111)_2$. Перемножить числа в дополнительном коде с ручной коррекцией. Использовать I способ умножения.
\end{Example}
\begin{proof}[Решение]
    Используем дробное масштабирование с множителем $2^6$. Тогда 
    \begin{align*}
        \DC{-25\cdot 2^{-6}}=\Number{,100111},\\
        \DC{-23\cdot 2^{-6}}=\Number{,101001}.
    \end{align*}

    Умножение дополнительных кодов (как беззнаковых целых) и коррекция приведены на рисунке \ref{fig:binmul:dcmanualIboth}.
    
    Проверка результата: 
    \[\Number{,001000 111111}\Rightarrow(0.001000111111)_2\cdot 2^{12}\Rightarrow(1000111111)_2=575\]
\end{proof}

\begin{figure}[!ht]
    \centering
    \begin{tabular}{c|r|l}
        \hline\hline
        мн-ль $\rightarrow$ 
                             & \multicolumn{1}{|c|}{СЧП $\rightarrow$}       
                                                        & прим. \\ \hline\hline
        \NumberLo{.,10011}{1} & \Addition{.,000000 000000}
                                         {.,101001 ......}
                                         {.,101001 000000} & +мн-е; сдвиг\\ \hline
        \NumberLo{.,.1001}{1} & \Addition{.,.10100 100000}
                                         {.,101001 ......}
                                         {.,111101 100000} & +мн-е; сдвиг\\ \hline
        \NumberLo{.,..100}{1} & \Addition{.,.11110 110000}
                                         {.,101001 ......}
                                         {1,000111 110000} & +мн-е; сдвиг\\ \hline
        \NumberLo{.,...10}{0} &   \Number{.,100011 111000} & сдвиг\\ \hline
        \NumberLo{.,....1}{0} &   \Number{.,.10001 111100} & сдвиг\\ \hline
        \NumberLo{.,.....}{1} & \Addition{.,..1000 111110}
                                         {.,101001 ......}
                                         {.,110001 111110} & +мн-е; сдвиг\\ \hline
        \NumberLo{.,.....}{.} &   \Number{.,.11000 111111} & Р-т на коррекцию!\\ \hline\hline
                              & \Addition{.,.11000 111111}
                                         {.,011001 ......}
                                         {.,110001 111111} & поправка $+\DC{0.011001}$\\ \hline
                              & \Addition{.,110001 111111}
                                         {.,010111 ......}
                                         {1,001000 111111} & поправка $+\DC{0.010111}$\\ \hline
                              &   \Number{.,001000 111111} & Рез-т в ДК!\\
    \end{tabular}
    \caption{Умножение $\DC{-25}\times \DC{-23}$ (I сп., к примеру \ref{ex:binmul:dcmanualIboth})}
    \label{fig:binmul:dcmanualIboth}
\end{figure}

С технической точки зрения, каждый способ умножения(см. рис. \ref{fig:binmul:baseschemes}) накладывает свои ограничения на коррекцию.
\begin{itemize}
    \item I-й способ. Так как регистр СЧП сдвиговый и модифицируется только его старшая часть, то верные поправки возможно выполнить только в конце цикла умножения. Вклад поправок, сделанных в начале цикла, с каждым сдвигом будет уменьшаться вдвое. Так как множитель в течение цикла сдвигается, то его значение нужно сохранить, чтобы в конце цикла использвоать для коррекции.
    
    Так как на каждом шаге умножения нужно прибавлять множимое, деленное на 2 (сдвинутое на 1 разряд вправо), а коррекцию в ряде случаев выполнять множимым, то можно на каждом шаге умножения прибавлять не сдвинутое множимое, но делать сдвиг всей суммы на последнем шаге.
    
    \item II-й способ. Так как регистр СЧП не сдвиговый, то поправку множителем лучше сделать в начале цикла. Если множитель сохранить, то поправку можно выполнить в любое время. Поправку множимым можно сделать только в конце цикла, когда, после серии сдвигов, разряды регистра множимого будут содержать правильное значение.

    \item III-й с способ. Поправку множимым лучше всего сделать в начале цикла умножения, так как в этом случае исходный вклад 
    \[
        \texttt{множимое}\cdot 2^{-n}
    \]
    в сумму частичных произведений к концу цикла будет правильным (до первого шаго умножения обязательно выполняется сдвиг СЧП). Поправку множителем можно сделать либо в начале цикла, сдвинутым значением 
    \[
        \texttt{множитель}\cdot 2^{-n},
    \]
    либо в конце цикла (позаботившись о сохранении множителя).
    
    \item IV-й способ. Поправку множимим логично сделать в начале цикла умножения, до его первого сдвига. Поправку множителем можно сделать в начале или в любой другой момент, позаботившись о его сохранении.
\end{itemize}  
\subsection{Умножение в ДК с автоматической коррекцией} 

Будем рассматривать положительное двоичное число со структурной точки зрения: как последовательность непрерывных групп единиц, разделённых непрерывными группами нулей. Минимальная длина группы при этом будет равна одному символу. Рассмотрим группу единиц отдельно:
\[
    \Number{...0001111100...}
\]

Соответствующее ей число можно получить как разность:
\[
    \Subtraction{...0010000000...}
                {...0000000100...}
                {...0001111100...}
\]

Видно, что в представление числа $A$ ненулевой вклад вносят только \emph{переходы из группы в группу}: \Number{01} и \Number{10}. Обозначим пару разрядов, образующих переход $(a_i,a_{i-1})$. Тогда переход $(a_i,a_{i-1})=\Number{01}$ вносит положительный вклад $2^i$, а переход $(a_i,a_{i-1})=\Number{10}$ вносит отрицательный вклад ${-2^i}$.

То есть можно получать число, представленное в двоичной системе счисления, суммируя только вклады переходов!

Разберёмся, что получится, если мы попытаемся найти представление отрицательного числа. При этом не забудем, что после самого младшего в представлении значащего разряда домысливается ноль. Чтобы получить отрицательное число, достаточно изменить знаки вкладов. 

Например, представление положительного числа $124$ 
\[
    \Number{...0001111100}
\] 
вносит вклады $(2^7) + (-2^2)$. Тогда представление $-124$ должно вносить вклады $(-2^7) + (2^2)$, то есть иметь переход \Number{10} в седьмом разряде и \Number{01} во втором. Но! Вопреки ожиданиям, построенное инверсное представление
\[
    \Number{...1110000011}
\]
не является правильным, так как имеет неявный третий переход:
\[
    \Number{...1110000011|0}
\]
который дополнительно вносит вклад $-2^0$. Выполнить коррекцию просто: достаточно прибавить к младшему разряду единицу:
\[
    \Number{...1110000100}
\]

Проверяя, убеждаемся, что $-124 = (-2^7) + (2^3) + (-2^2)$. Кстати, мы получили ни что иное, как \emph{дополнительный код}! Читателю осталось  самостоятельно разобраться, например, с представлением отрицания числа $15$, чтобы убедиться окончательно: 
\begin{Theorem}
    Чтобы получить число, представленное в дополнительном коде, достаточно просуммировать вклады переходов между группами нулей и единиц кода!
\end{Theorem}

При таком подходе становится возможным выполнять умножение непосредственно в дополнительном коде.  Анализируя в $i$-м такте умножения \emph{пары} разрядов множителя $A$ будем либо прибавлять к регистру частных сумм сдвинутое на соответствующее количество разрядов множимое $B$, либо вычитать:
\begin{itemize}
    \item если $(a_i, a_{i-1})=\Number{01}$, то к общей сумме нужно \emph{прибавить} $B\cdot 2^i$;
    \item если $(a_i, a_{i-1})=\Number{10}$, то из общей суммы нужно \emph{вычесть} $B\cdot 2^i$;
    \item для остальных комбинаций никаких действий не требуется.
\end{itemize}

Масштаб выбирается таким образом, чтобы <<знаковым>> разрядом был старший разряд дробной части. Результатом перемножения $n$-разрядных дополнительных кодов является $2n$ разрядный дополнительный код.

\begin{Example}\label{ex:binmul:autoI}
    Множитель $25=(11001)_2$, множимое $22=(10110)_2$. Перемножить числа в дополнительном коде с автоматической коррекцией. Использовать I способ умножения.
\end{Example}
\begin{proof}[Решение]
    Используем дробное масштабирование с множителем $2^6$, при этом знаковым разрядом будет старший разряд дробной части. Тогда 
    \begin{align*}
        \DC{25}=.011001,\\
        \DC{22}=.010110.
    \end{align*}

    Умножение дробных чисел приведено на рисунке \ref{fig:binmul:autoI}. Получен результат: 
    \begin{align*}
        (.001000100110)_2\cdot 2^{12}=550.
    \end{align*}
\end{proof}

\begin{figure}[!ht]
    \centering
    \begin{tabular}{c|r|l}
                                                                   \hline\hline
        мн-ль $\rightarrow$ & 
                                \multicolumn{1}{|c|}{СЧП $\rightarrow$}       
                                                          & прим. \\ \hline\hline
        \NumberLo{,01100}{1|0} & \Addition{,000000 000000}
                                          {,110101 0.....}
                                          {,110101 000000} & $-M$; сдвиг;\\ \hline
        \NumberLo{,.0110}{0|1} & \Addition{,111010 100000}
                                          {,.01011 0.....}
                                          {,000101 100000} & $+M$; сдвиг;\\ \hline
        \NumberLo{,..011}{0|0} &   \Number{,000010 110000} & сдвиг\\ \hline
        \NumberLo{,...01}{1|0} & \Addition{,000001 011000}
                                          {,110101 0.....}
                                          {,110110 011000} & $-M$; сдвиг;\\ \hline
        \NumberLo{,....0}{1|1} &   \Number{,111011 001100} & сдвиг;\\ \hline
        \NumberLo{,.....}{0|1} & \Addition{,111101 100110}
                                          {,.01011 0.....}
                                          {,001000 100110} & $+M$; Рез-т!\\ 
    \end{tabular}
    \caption{Умножение $25\times 22$ в дополнительном коде с автоматической коррекцией (I сп., к примеру \ref{ex:binmul:autoI})}
    \label{fig:binmul:autoI}
\end{figure}


\begin{Example}\label{ex:binmul:autoII}
    Множитель $25=(11001)_2$, множимое $-22=(-10110)_2$. Перемножить числа в дополнительном коде с автоматической коррекцией. Использовать II способ умножения.
\end{Example}
\begin{proof}[Решение]
    Используем дробное масштабирование с множителем $2^6$. Тогда 
    \begin{align*}
        \DC{25} =.011001,\\
        \DC{-22}=.101010. 
    \end{align*}
    
    Процесс умножения приведён на рисунке \ref{fig:binmul:autoII}. Результат в ДК: 
    \begin{align*}
        \Number{,110111011010} = \DC{-.001000100110};\\
        (-.001000100110)_2\cdot 2^{12} = -550.
    \end{align*}
\end{proof}

\begin{figure}[!ht]
    \centering
    \begin{tabular}{c|r|r|l}
                                                                   \hline\hline
        мн-ль $\rightarrow$ 
                               & \multicolumn{1}{|c|}{мн-е $\leftarrow$}       
                                                         & \multicolumn{1}{|c|}{СЧП}
                                                                                    & прим. \\ \hline\hline
        \NumberLo{,01100}{1|0} & \Number{,111111 101010} & \Addition {,000000 000000} 
                                                                     {,000000 010110}
                                                                     {,000000 010110} & $-M$; сдвиг;\\ \hline
        \NumberLo{,.0110}{0|1} & \Number{,111111 01010.} & \Addition {,000000 010110} 
                                                                     {,111111 01010.}
                                                                     {,111111 101010} & $+M$; сдвиг;\\ \hline
        \NumberLo{,..011}{0|0} & \Number{,111110 1010..} &                            & сдвиг;\\ \hline
        \NumberLo{,...01}{1|0} & \Number{,111101 010...} & \Addition {,111111 101010} 
                                                                     {,000010 110...}
                                                                     {,000010 011010} & $-M$; сдвиг;\\ \hline
        \NumberLo{,....0}{1|1} & \Number{,111010 10....} &                            & сдвиг;\\ \hline
        \NumberLo{,.....}{0|1} & \Number{,110101 0.....} & \Addition {,000010 011010} 
                                                                     {,110101 0.....}
                                                                     {,110111 011010} & $+M$; Рез-т!\\ 
    \end{tabular}
    \caption{Умножение $25\times -22$ в дополнительном коде с автоматической коррекцией (II сп., к примеру \ref{ex:binmul:autoII})}
    \label{fig:binmul:autoII}
\end{figure}


\begin{Example}\label{ex:binmul:autoIII}
    Множитель $-22=(-10110)_2$, множимое $-25=(-11001)_2$. Перемножить числа в дополнительном коде с автоматической коррекцией. Использовать III способ умножения.
\end{Example}
\begin{Solve}
    Используем дробное масштабирование с множителем $2^6$. Тогда
    \begin{align*}
        \DC{-22}=.101010,\\
        \DC{-25}=.100111. 
    \end{align*}
    
    Процесс умноженя приведён на рисунке\ref{fig:binmul:autoIII}. Получен результат: 
    \begin{align*}
        (0.001000 100110)_2 \cdot 2^{12} = 550.
    \end{align*}
\end{Solve}

\begin{figure}[!ht]
    \centering
    \begin{tabular}{c|r|l}
                                                                   \hline\hline
        мн-ль $\leftarrow$ & 
                                \multicolumn{1}{|c|}{СЧП $\leftarrow$}       
                                                        & прим.\\ \hline\hline
        \NumberHi{,10}{1010} & \Addition{,000000 000000}
                                        {,...... 011001}
                                        {,000000 011001} & $-M$; сдвиг;\\ \hline
        \NumberHi{,01}{010.} & \Addition{,000000 11001.}
                                        {,111111 100111}
                                        {,000000 011001} & $+M$; сдвиг;\\ \hline
        \NumberHi{,10}{10..} & \Addition{,000000 11001.}
                                        {,...... 011001}
                                        {,000001 001011} & $-M$; сдвиг;\\ \hline
        \NumberHi{,01}{0...} & \Addition{,000010 01011.}
                                        {,111111 100111}
                                        {,000001 111101} & $+M$; сдвиг;\\ \hline
        \NumberHi{,10}{....} & \Addition{,000011 11101.}
                                        {,...... 011001}
                                        {,000100 010011} & $-M$; сдвиг;\\ \hline
        \NumberHi{,0.}{....} &   \Number{,001000 100110} & Рез-т!\\ 
    \end{tabular}
    \caption{Умножение $-22\times-25$ в дополнительном коде с автоматической коррекцией (III сп., к примеру \ref{ex:binmul:autoIII})}
    \label{fig:binmul:autoIII}
\end{figure}


\begin{Example}\label{ex:binmul:autoIV}
    Множитель $-25=(-11001)_2$, множимое $22=(10110)_2$. Перемножить числа в дополнительном коде с автоматической коррекцией. Использовать IV способ умножения.
\end{Example}
\begin{proof}[Решение]
    Используем дробное масштабирование с множителем $2^6$. Тогда
    \begin{align*}
        \DC{-25}=.100111,\\
        \DC{22} =.010110. 
    \end{align*}
    
    Процесс умножения приведён на рисунке \ref{fig:binmul:autoIV}. 
    
    Результат:
    \begin{align*}
        \Number{,110111011010} = \DC{-.001000100110};\\
        (-.001000100110)_2\cdot 2^{12} = -550.
    \end{align*}
\end{proof}

\begin{figure}[!ht]
    \centering
    \begin{tabular}{c|r|r|l}
                                                                   \hline\hline
        мн-ль $\leftarrow$ 
                             & \multicolumn{1}{|c|}{мн-е $\rightarrow$}       
                                                      & \multicolumn{1}{|c|}{СЧП}       
                                                                                  & прим. \\ \hline\hline
        \NumberHi{,10}{0111} & \Number{,.01011 0.....} & \Addition {,000000 000000} 
                                                                   {,110101 0.....}
                                                                   {,110101 000000} & $-M$; сдвиг;\\ \hline
        \NumberHi{,00}{111.} & \Number{,..0101 10....} &                            & сдвиг\\ \hline
        \NumberHi{,01}{11..} & \Number{,...010 110...} & \Addition {,110101 000000} 
                                                                   {,...010 110...}
                                                                   {,110111 110000} & $+M$; сдвиг;\\ \hline
        \NumberHi{,11}{1...} & \Number{,....01 0110..} &                            & сдвиг;\\ \hline
        \NumberHi{,11}{....} & \Number{,.....0 10110.} &                            & сдвиг;\\ \hline
        \NumberHi{,1.}{....} & \Number{,...... 010110} & \Addition {,110111 110000} 
                                                                   {,111111 101010}
                                                                   {,110111 011010} & $-M$; Рез-т!\\ 
    \end{tabular}
    \caption{Умножение $-25\times 22$ в дополнительном коде с автоматической коррекцией (IV сп., к примеру \ref{ex:binmul:autoIV})}
    \label{fig:binmul:autoIV}
\end{figure}


\section{Умножение чисел в формате с плавающей точкой}
\label{ch::float}

Для примеров будет использоваться формат с плавающей точкой, определенный в примере \ref{ch:digitFormat:16char} на странице \pageref{ch:digitFormat:16char}.

Результат произведения чисел, представленных в формате с плавающей точкой, $A=\FloatExpression{A}{2}$ и $B=\FloatExpression{B}{2}$ определяется как
\[
   A\cdot B= (\MantissOf{A}\cdot\MantissOf{B})\cdot 2^{(\OrderOf{A} + \OrderOf{B})}.
\]

Видно, что мантисса результата есть произведение соответствующих мантисс, а порядок --- сумма порядков операндов. В общем случае мантиса результата $(\MantissOf{A}\cdot\MantissOf{B})$ может оказаться денормализованной, и в процессе нормализации порядок результата будет скорректирован.

Алгоритм умножения (без возможных оптимизаций) чисел состоит из следующийх шагов:
\begin{enumerate}
   \item определяется результат перемножения мантисс по правилам умножения чисел с фиксированной точкой;
   \item определяется порядок результата суммированием порядков операндов по правилам сложения чисел с фиксированной точкой;
   \item выполняется нормализация мантиссы результата с соответствующими поправками порядка результата.
\end{enumerate}

Рассмотрим несколько примеров на формате с плавающей точкой (см. пример \ref{ch:digitFormat:16char} на странице \pageref{ch:digitFormat:16char}).

\begin{Example}
    Перемножить числа $A=-57$ и $B=11$ в формате с плавающей точкой.
\end{Example}
\begin{Solve}
    $-57 = (-111001)_2$, порядок $6$, характеристика $6+32=38=(100110)_2$. 
    
    $11 = (1011)_2$, порядок $4$, характеристика $36=(100100)_2$

    $A=-57$:
    \[
        \FloatMyCharHex{1}{111001000}{100110}
    \]
    
    $B=11$:
    \[
        \FloatMyCharHex{0}{101100000}{100100}
    \]

    Перемножая мантиссы (с масштабом $2^0$) получим дробный результат (также с масштабом $2^0$):
    \[
        \Number{,111001000}\cdot\Number{,101100000} = \Number{,100111001 100000000}.
    \]

    Как и положено, результат без потери точности будет представляться в $2n$ разрядной сетке, из которой в качестве результата придется взять $n$ старших разрядов результата\footnote{Здесь мы используем округление <<отсечением>>. О том как правильно следует округлять результат, рассказано в разделе \ref{ch::float::ss::round} на странице \pageref{ch::float::ss::round}}.

    Характеристику получим на основе следующих соображений. Характеристика $\Char$ из порядка $\Order$ получается следующим образом:
    \[
        \Char = \DC{\Order} + \Delta
    \]

    Тогда, складывая характеристики:
    \[
        \CharOf{A} + \CharOf{B} = 
            \DC{\OrderOf{A}} + \Delta + 
            \DC{\OrderOf{B}} + \Delta = 
                \underbrace{\DC{\OrderOf{A}} + \DC{\OrderOf{B}}}_{\DC{\OrderOf{A} + \OrderOf{B}}} + 2\Delta.
    \]

    Чтобы получить характеристику результата, очевидно из суммы характеристик нужно вычесть $\Delta$.
    
    Характеристика --- это положительное число. Увеличим разрядность для представления характеристики на два разряда, добавив их слева занеся в них нули (знак). После этого появляется возможность выполнять все поправки в дополнительном коде, и как только в добавочных разрядах окончательного результата возникнет комбинация, отличная от \Number{00} --- имеем ПРС. Фактически, в данном случае используется модифицированный дополнительный код (см. раздел \ref{ch:binadd}).

    Увеличиваем разрядность и находим сумму характеристик:
    \[
        \Addition{00100110}
                 {00100100}
                 {01001010}
    \]
    
    Выполняем поправку $(-\Delta=-32)$, получая характеристику результата:
    \[
        \Addition{01001010}
                 {11100000}
                 {00101010}
    \]

    В добавочных разрядах получается комбинация \Number{00} --- следовательно ПРС не возникло. Отбрасываем их и формируем результат:
    \[
        \FloatMyCharHex{1}{100111001}{101010}
    \]

    В формате представлено число $(-0.100111001)_2\cdot 2^{10} = (-1001110010)_2 = -626$. Правильный результат $-627$. Видно, что произошла (на практике совершенно неизбежная) потеря точности.
\end{Solve}

\begin{Example}
    Перемножить числа $A=0.5$ и $B=0.25$.
\end{Example}
\begin{Solve}

    $A=0.5$:
    \[
        \FloatMyCharHex{0}{100000000}{100000}
    \]
    
    $B=0.25$:
    \[
        \FloatMyCharHex{0}{100000000}{011111}
    \]
    
    После перемножения мантисс, результат получается ненормализованным: 
    \[
        \Number{,010000000 000000000}
    \]
    
    Нормализуем и отбрасываем младщую половину $2n$ разрядного результата: $\Number{,100000000}$. Мантисса была сдвинута влево, следовательно порядок (характеристику), нужно уменьшить на единицу.

    Увеличиваем разрядность и находим сумму характеристик:
    \[
        \Addition{00100000}
                 {00011111}
                 {00111111}
    \]
    
    Выполняем поправку $(-\Delta=-32)$, получая характеристику результата:
    \[
        \Addition{00111111}
                 {11100000}
                 {00011111}
    \]
    
    Уменьшаем характеристику на единицу (из-за ненормализованной мантиссы):
    \[
        \Addition{00011111}
                 {11111111}
                 {00011110}
    \]
    
    ПРС не возникло. Результат:
    \[
        \FloatMyCharHex{0}{100000000}{011110}
    \]
    
    $(0.100000000)_2\cdot 2^{-2} = 0.125$. Результат в данном случае получен без потерь в точности.
\end{Solve}

\begin{Example} 
    Перемножить числа
    \[
        \FloatMyCharHex{0}{100000000}{111100}\times
        \FloatMyCharHex{0}{100000000}{100100}
    \]
\end{Example}
\begin{Solve}
    Требуется перемножить числа $(0.5\cdot 2^{28})\cdot(0.5\cdot 2^{4})$.
    
    Мантисса результата ($2n$): $\Number{,010000000 000000000}$.
    
    Характеристика:
    \[
        \Addition{00111100}
                 {00100100}
                 {01100000}
    \]
    
    После поправки:
    \[
        \Addition{01100000}
                 {11100000}
                 {01000000}
    \]
    
    Получено ПРС характеристик! Однако в ходе нормализации мантиссы потребуется уменшение характеристики на единицу:
    \[
        \Addition{01000000}
                 {11111111}
                 {00111111}
    \]
    
    ПРС отсутствует! Результат верен:
    \[
        \FloatMyCharHex{0}{100000000}{111111}
    \]
    
    Балансируя на границе диапазона представления чисел, получили корректный результат $0.5\cdot 2^{31}$.
\end{Solve}

\begin{Example} 
    Перемножить числа
    \[
        \FloatMyCharHex{0}{100000000}{111000} \times
        \FloatMyCharHex{0}{100000000}{101111}
    \]
\end{Example}
\begin{Solve}
    Требуется перемножить числа $(0.5\cdot 2^{24})\cdot(0.5\cdot 2^{15})$.
    
    Мантисса результата ($2n$): $\Number{,010000000 000000000}$.
    
    Характеристика:
    \[
        \Addition{00111000}
                 {00101111}
                 {01100111}
    \]
    
    После поправки:
    \[
        \Addition{01100111}
                 {11100000}
                 {01000111}
    \]
    
    Получено ПРС характеристики! В ходе нормализации характеристика результата будет уменшена на единицу:
    \[
        \Addition{01000111}
                 {11111111}
                 {01000110}
    \]
    
    Но ПРС остается --- комбинация \Number{01} в добавочных разрядах. Полученный дополнительный код результата (с двумя добавочными разрядами) свидетельствует о том, что он положителен. В данном случае фиксируется ошибка вычислений ---  характеристика вышла за допустимую верхнюю границу представления.
\end{Solve}

\begin{Example} 
    Перемножить числа
    \[
        \FloatMyCharHex{0}{100000000}{001110} \times
        \FloatMyCharHex{0}{100000000}{001111}
    \]
\end{Example}
\begin{Solve}
    Требуется перемножить числа $(0.5\cdot 2^{-18})\cdot(0.5\cdot 2^{-17})$.
    
    Мантисса результата ($2n$): $\Number{,010000000 000000000}$.
    
    Характеристика:
    \[
        \Addition{00001110}
                 {00001111}
                 {00011101}
    \]
    
    После поправки:
    \[
        \Addition{00011101}
                 {11100000}
                 {11111101}
    \]
    
    Получено ПРС характеристики (в добавочных разрядах \Number{11})! Возможно, что ситуация изменится после вычитания единицы (из-за ненормализованной мантиссы):
    \[
        \Addition{11111101}
                 {11111111}
                 {11111100}
    \]
    
    Увы, ПРС остается --- в добавочном разряде единица. В данном случае имеем отрицательный результат. Это означает, что характеристика вышла за нижнюю границу представления. В данном случае ошибка вычислений не фиксируется. Получено очень малелнькое по модулю число, вместо которого можно вернуть ноль:
    
    \[
        \FloatMyCharHex{0}{000000000}{000000} \times
    \]
    
    В вычислительных устройствах такая ситуация отмечается, например установкой флага, который называется ПМР - потеря младших разрядов.
\end{Solve}


\section{Особенности округления чисел при умножении}
\label{ch::float::ss::round}

Допустим, что в некотором формате с плавающей запятой под мантиссу отводится $n$ разрядов. При перемножении мантисс, результат, как известно, получается разрядностью $2n$, а чтобы <<упаковать>> результат в формат, придётся отбросить половину\footnote{В случае дробной нормализации будет отброшена младшая половина $2n$-разрядной мантиссы результата. Что и предполагается в дальнейшем изложении} значащих разрядов, мирясь с неизбежными потерями в точности.

Один из самых простых вариантов --- округление <<отсечением>>: младшая половина разрядов отбрасывается, а старшая никак не изменяется.

С точки зрения увеличения точности, более корректным будет округление с учетом значений отбрасываемой половины. При перемножении чисел в прямых кодах особенных трудностей не возникает:
\begin{enumerate}
    \item результат нормализуется;
    \item младшая половина разрядов отбрасывается;
    \item если в старшем разряде отбрасываемой половины находится единица, то старшая <<оставшаяся>> половина результата инкрементируется;
    \item при необходимости выполняется нормализация.
\end{enumerate}

Для примеров возьмем все тот же формат.

\begin{Example}
    Перемножить числа $258$ и $381$.
\end{Example}
\begin{Solve}
    $258$:
    \[
        \FloatMyCharHex{0}{100000010}{101001}
    \]
    $381$:
    \[
        \FloatMyCharHex{0}{101111101}{101001}
    \]
    
    Результатом произведения мантисс будет
    \[
        \Number{,010111111 111111010}
    \]
    
    Полученный результат нормализуется (характеристика результата должна быть уменьшена на единицу):
    \[
        \Number{,101111111 111110100}
    \]
    
    И округляется:
    \[
        \Number{,110000000}
    \]

    Результат с корректным округлением $98304$:
    \[
        \FloatMyCharHex{0}{110000000}{110001}
    \]
    
    Результат без потерь в точности должен быть: $98298$. 
    
    Если использовать округление <<отсечением>>, то получившийся рузультат $98048$:
    \[
        \FloatMyCharHex{0}{101111111}{110001}
    \]
    будет куда менее точным.
\end{Solve}

Округление чисел, представленных в дополнительном коде\footnote{Ранее были рассмотрены методы умножения в дополнительных кодах}, имеет ряд особенностей. Если число положительное, то округление осуществляется так же, как в прямом коде. Если число отрицательное, то следует проанализировать несколько вариантов.

Допустим, что отрицательное число в дополнительном коде разделено на две части
\[
    \underbrace{1.xx\cdots xx}_A\underbrace{xx\cdots xx}_B,
\]
где $x$ --- обозначение произвольного значения двоичного разряда; часть $A$ --- старшие разряды дробного результата (целая часть представляет знак), а $B$ --- отбрасываемая часть.

\begin{itemize}
    \item Пусть $B=(10\cdots 0)$. Т.е. в отбрасываемой части старший разряд --- единица, а остальные --- нули. При извлечении модуля числа из дополнительного кода всего числа, проинвертируются все разряды части $A$, а в старшем разряде части $B$ останется единица. Т.е. модуль округленного результата должен быть инкрементирован: $|A| = \overline{A} + 1$. Каким он и будет являться, если часть $B$ просто отбросить.
    
    \item Пусть $B=(0\cdots 1\cdots)$. Т.е. в отбрасываемой части старший разряд --- ноль, а в остальных встречается хотя бы одна единица. Ситуация аналогична предыдущей: при извлечении модуля числа из дополнительного кода всего числа, проинвертируются все разряды части $A$, а в старшем разряде части $B$ возникнет единица. Т.е. модуль округленного результата должен быть инкрементирован: $|A| = \overline{A} + 1$. Каким он и будет являться, если часть $B$ просто отбросить.
    
    \item Пусть $B=(1\cdots 1\cdots)$. Т.е. в отбрасываемой части старший разряд --- единица, а в остальных встречается хотя бы одна единица. При извлечении модуля числа из дополнительлного кода всего числа, проинвертируются все разряды части $A$, а в старшем разряде части $B$ возникнет ноль. Инкрементировать модуль по правилам округления нет необходимости. Модуль округленного результата должен представлять собой $|A| = \overline{A}$, а если отбросить часть $B$, то он будет равен $(\overline{A} + 1)$. Следовательно, в этом случае часть $A$ нужно инкрементировать и отбросить часть $B$. Это следует из равенства $\overline{A}=\overline{(A + 1)} + 1$.
\end{itemize}



