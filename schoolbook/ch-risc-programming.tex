\section{Программирование на ассемблере \MyProc}
\label{ch::programming}

В данном разделе приводятся сведения, как можно реализовать некоторые операторы языков высокого уровня на языке ассемблера {\MyProc}. Эти сведения носят справочный характер и не означают того, что, программируя на ассемблере, следует мыслить конструкциями ЯВУ.


\subsection{Условные переходы}

Логика работы команды условного перехода проста: если оговоренное условие истинно, то выполняется переход, иначе выполняется следующая за командой перехода команда. Например, для команды \Opcode{jz} (<<если 0>>):
\begin{algorithmic}[1]
    \STATE{\CmdTwoAddr{jz}{r1}{LabelR1isZ}}
    \STATE{}\COMMENT{если $\Operand{r1}\neq 0$, то выполнятся эти команды}
    \STATE{\Operand{LabelR1isZ:}} \COMMENT{сюда выполняется переход, если $\Operand{r1}=0$}
\end{algorithmic}

Если нужен переход по инверсному условию, (в данном примере \Opcode{jnz} --- <<если не 0>>), то можно это сделать на основе имеющейся команды:
\begin{algorithmic}[1]
    \STATE{\CmdTwoAddr{jz}{r1}{LabelR1isZ}}
    \STATE{\CmdTwoAddr{jz}{0}{LabelR1isNotZ}} \COMMENT{безусловный переход!}
    \STATE{\Operand{LabelR1isZ:}} 
    \STATE{}\COMMENT{делаем что-то, при $\Operand{r1}=0$}
    \STATE{\Operand{LabelR1isNotZ:}} \COMMENT{инверсный переход! Выполняется, если $\Operand{r1}\neq 0$}
\end{algorithmic}


\subsection{Аналоги условных операторов ЯВУ}

Условный оператор на языке высокого уровня
\begin{algorithmic}[1]
    \IF {$\Operand{r1}=0$}
        \STATE{} \COMMENT{делаем что-то}
    \ELSE
        \STATE{} \COMMENT{делаем нечто}
    \ENDIF
\end{algorithmic}

на ассемблере {\MyProc} может быть реализован с помощью следующих команд:
\begin{algorithmic}[1]
    \STATE{\CmdTwoAddr{jz}{r1}{LabelFoo}}
    \STATE{}\COMMENT{делаем нечто}
    \STATE{\CmdTwoAddr{jz}{0}{LabelEndIf}}
    \STATE{\Operand{LabelFoo:}} 
    \STATE{}\COMMENT{делаем что-то}
    \STATE{\Operand{LabelEndIf:}} 
\end{algorithmic}


\subsection{Аналоги циклов ЯВУ}

\subsubsection{Цикл while}

На языке высокого уровня обычно выглядит следующим образом:
\begin{algorithmic}[1]
    \WHILE {$\Operand{r1}=0$}
        \STATE{} \COMMENT{делаем нечто}
    \ENDWHILE
\end{algorithmic}

На ассемблере он может быть реализован следующими командами:
\begin{algorithmic}[1]
    \STATE{\Operand{LabelWhileStart:}} 
    \STATE{\CmdTwoAddr{jz}{r1}{LabelWhileBody}}
    \STATE{\CmdTwoAddr{jz}{0}{LabelWhileEnd}}
    \STATE{\Operand{LabelWhileBody:}} 
    \STATE{}\COMMENT{делаем нечто}
    \STATE{\CmdTwoAddr{jz}{0}{LabelWhileStart}}
    \STATE{\Operand{LabelWhileEnd:}} 
\end{algorithmic}


\subsubsection{Цикл repeat}

\begin{algorithmic}[1]
    \REPEAT
        \STATE{} \COMMENT{делаем нечто}
    \UNTIL{$\Operand{r1}=0$}
\end{algorithmic}

на ассемблере:
\begin{algorithmic}[1]
    \STATE{\Operand{LabelRepeatStart:}} 
    \STATE{}\COMMENT{делаем нечто}
    \STATE{\CmdTwoAddr{jz}{r1}{LabelRepeatEnd}}
    \STATE{\CmdTwoAddr{jz}{0}{LabelRepeatStart}}
    \STATE{\Operand{LabelRepeatEnd:}} 
\end{algorithmic}


\subsubsection{Цикл for}

\begin{algorithmic}[1]
    \FOR{$\Operand{r1}\gets 0$ \Opcode{to} $n$}
        \STATE{} \COMMENT{делаем нечто}
    \ENDFOR
\end{algorithmic}
где $n$ --- число, на ассемблере может быть реализован следующим набором команд:

\begin{algorithmic}[1]
    \STATE{\CmdThreeAddr{add}{0}{0}{r1}}
    \STATE{\Operand{LabelForStat:}} 
    \STATE{\CmdThreeAddr{xor}{r1}{$n$}{r1}} \COMMENT{трюк: $\Operand{r1}\oplus n=0$, если $\Operand{r1}=n$}
    \STATE{\CmdTwoAddr{jz}{r1}{LabelForEnd}}
    \STATE{\CmdThreeAddr{xor}{r1}{$n$}{r1}} \COMMENT{трюк: восстанавливаем $\Operand{r1}=\Operand{r1}\oplus n\oplus n=\Operand{r1}\oplus 0$}
    \STATE{}\COMMENT{делаем нечто}
    \STATE{\CmdThreeAddr{add}{r1}{1}{r1}}
    \STATE{\CmdTwoAddr{jz}{0}{LabelForStat}} 
    \STATE{\Operand{LabelForEnd:}} 
    \STATE{\CmdThreeAddr{xor}{r1}{$n$}{r1}} \COMMENT{если $\Operand{r1}=n$ нужен далее}
\end{algorithmic}


\subsection{Вычисление формул: распространение переноса}
\label{ch:risc:p}

При сложении беззнаковых целых чисел в $n$-разрядной сетке может возникнуть ситуация ПРС, когда теряется перенос из старшего разряда. Нужно уметь определять ситуацию ПРС, так как специальных средств в {\MyProc} для этого не предусмотрено.

Пусть складываются числа $A$ и $B$ и перенос $c$:
\[
    A + B + c,
\]
где значение переноса $c$ либо 1, либо 0.

Перенос возникнет, если оба старших разряда чисел $A,B$ равны 1. Также он возникает, если старшие разряды $A,B$ различны, а старший разряд суммы $A+B+c$ получается нулевым. После вычисления выражения
\[
    (A\& B)\lor ((A \lor B) \& \lnot(A + B + c))
\]
в старшем разряде результата получается значение бита переноса.

Распределить при их нехватке регистры (или ячейки памяти) для хранения промежуточных результатов не такая уж и простая задача, даже если для всех операций находится подходящая команда:
\[
    \underbrace{
        (\underbrace{
            \overbrace{A}^\Operand{r1}
            \& 
            \overbrace{B}^\Operand{r2}
        }_\Operand{r5 (6)})
        \lor 
        \underbrace{(
            (\underbrace{
                \overbrace{A}^\Operand{r1} 
                \lor 
                \overbrace{B}^\Operand{r2}
            }_\Operand{r5 (4)})
            \& 
            \underbrace{
                \lnot
                (\underbrace{
                    \underbrace{
                        \overbrace{A}^\Operand{r1}
                        + 
                        \overbrace{B}^\Operand{r2}
                    }_\Operand{r4 (1)}
                    + 
                    \overbrace{c}^\Operand{r3}
                }_\Operand{r4 (2)})
            }_\Operand{r6 (3)}
        )}_\Operand{r6 (5)}
    }_\Operand{r5 (7)},
\]
где в подписях под фигурными скобками --- регистр, сохраняющий результат промежуточных вычислений, и (в скобках) номер команды в приведенной ниже программе, вычисляющей результат сложения и бит переноса.

\begin{algorithmic}[1]
    \REQUIRE{\Operand{r1} --- $A$, \Operand{r2} --- $B$, \Operand{r3} --- $c\in\{0,1\}$}
    \ENSURE{\Operand{r4} --- байт результата без переноса, \Operand{r5} --- перенос}
    
    \COMMENT{\Operand{r6} --- для промежуточных результатов}
    \STATE{\CmdThreeAddr{add}{r1}{r2}{r4}} \COMMENT{$\Operand{r4}\gets A+B$}
    \STATE{\CmdThreeAddr{add}{r3}{r4}{r4}} \COMMENT{$\Operand{r4}\gets A+B+c$, байт результата без переноса}
    \STATE{\CmdTwoAddr{not}{r4}{r6}} \COMMENT{$\Operand{r6}\gets \lnot(A+B+c)$}
    \STATE{\CmdThreeAddr{or}{r1}{r2}{r5}} \COMMENT{$\Operand{r5}\gets A\lor B$}
    \STATE{\CmdThreeAddr{and}{r5}{r6}{r6}} \COMMENT{$\Operand{r6}\gets (A \lor B) \& \lnot(A + B + c)$}
    \STATE{\CmdThreeAddr{and}{r1}{r2}{r5}} \COMMENT{$\Operand{r5}\gets A \& B$}
    \STATE{\CmdThreeAddr{or}{r5}{r6}{r5}} \COMMENT{$\Operand{r5}\gets (A\& B)\lor ((A \lor B) \& \lnot(A + B + c))$}
    \STATE{\CmdThreeAddr{rol}{r5}{1}{r5}} \COMMENT{признак переноса в 7-м бите r5}
    \STATE{\CmdThreeAddr{and}{r5}{1}{r5}} \COMMENT{в r5 перенос: 0 или 1}
\end{algorithmic}
