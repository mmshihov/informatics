\chapter{Сложение двоично-десятичных чисел}
\label{ch:bcd}


\newcommand{\NaturalLabel}{\text{``\texttt{8421}''}}
\newcommand{\Natural}[1]{\NaturalLabel(#1)}

\newcommand{\PlusThreeLabel}{\text{``\texttt{8421+3}''}}
\newcommand{\PlusThree}[1]{\PlusThreeLabel(#1)}

\newcommand{\AikenLabel}{\text{``\texttt{2421}''}}
\newcommand{\Aiken}[1]{\AikenLabel(#1)}

\newcommand{\PentaLabel}{\text{``\texttt{3a+2}''}}
\newcommand{\Penta}[1]{\PentaLabel(#1)}


Двоично-десятичный код используется для представления десятичных цифр. Сложение же выполняется по правилам десятичной арифметики, только вместо сложения десятичных цифр выполняется сложение кодов. Далее рассматривается сложение в десятичной системе счисления.

Обратный код в десятичной системе счисления получается следующим образом:
\[
    \OC{X} = 
    \begin{cases}
        \overline{|X|}, &\text{ если $X<0$,}\\
        |X|,            &\text{ если $X\ge 0$,}
    \end{cases}
\]
где $\overline{|X|}$ --- порязрядное дополнение цифр десятичного числа $X$ до $9$, то есть разряд $x_i$ числа находится как $(9-x_i)$.    

Восстановление числа со знаком из обратного кода выполняется следующим образом.
\[
    X = 
    \begin{cases}
        -(\overline{\OC{X}}), &\text{ если $msb(\OC{X})=9$,}\\
         \OC{X},              &\text{ если $msb(\OC{X})=0$,}
    \end{cases}
\]
где $msb(x)$ --- функция, возвращающая старший значащий бит последовательности $x$.    

При сложении единицу переноса из старшего разрядя следует прибавить к младшему разряду.

\begin{Example}
    Сложить числа $527$ и ${-365}$ в 4-х рязрядной десятичной сетке.
\end{Example}
\begin{Solve}
    Выполним дробное масштабирование $M=10^4$ и переведем числа в обратный код:
    \begin{align*}
        \OC{527} &=\Number{,0527},\\
        \OC{-365}&=\Number{,9634}.
    \end{align*}
    
    Складываем обратные коды:
    \[
        \Addition{,0527}
                 {,9634}
                {1,0161}
    \]
    
    Поправка единицей переноса:
    \[
        \Addition{,0161}
                 {,0001}
                 {,0162}
    \]
    
    Результат: $\Number{,0162}\Rightarrow(.0162)_{10}\cdot 10^{4}=162$.
\end{Solve}

\section{Четырехбитные коды}

Для представления десятичной цифры требуется минимум 4-е двоичных разряда --- тетрада. Такое представление избыточно --- так как фактически остается еще шесть лишних значений тетрад. Следует отметить, что инверсия разрядов тетрады соответствует арифметическому дополнению значения тетрады до 15:
\[
    \overline{(\Number{xxxx})}_2 \Leftrightarrow \left((\Number{1111})_2-(\Number{xxxx})_2\right) \Leftrightarrow (15-(\Number{xxxx})_2).
\]

Например $(\Number{1001})_2=9$:
\[
    \overline{(\Number{1001})}_2=(\Number{0110})_2=6=(15-9).
\]


\subsection{Код \NaturalLabel}

Код \NaturalLabel --- код с естественными весами: $\Natural{a}=a$. То есть значение тетрады кода совпадает с десятичной цифрой. Кодирование приведено в таблице \ref{t:bcd:Natural}.

\begin{table}[!ht]
    \caption{Код \NaturalLabel}
    \label{t:bcd:Natural}
    \centering
    \begin{tabular}{|c|c|c|}
        \hline\hline
        $a$ в 10СС  & $\Natural{a}$         & $\Natural{9-a}$\\
        \hline\hline
        $0$         & $\Number{0000}$       & $\Number{1001}$ \\
        $1$         & $\Number{0001}$       & $\Number{1000}$ \\
        $2$         & $\Number{0010}$       & $\Number{0111}$ \\
        $3$         & $\Number{0011}$       & $\Number{0110}$ \\
        $4$         & $\Number{0100}$       & $\Number{0101}$ \\
        $5$         & $\Number{0101}$       & $\Number{0100}$ \\
        $6$         & $\Number{0110}$       & $\Number{0011}$ \\
        $7$         & $\Number{0111}$       & $\Number{0010}$ \\
        $8$         & $\Number{1000}$       & $\Number{0001}$ \\
        $9$         & $\Number{1001}$       & $\Number{0000}$ \\
        \hline
    \end{tabular}
\end{table}

$\Natural{9-a}=9-a=(15-a)-6=\overline{\Natural{a}}-6=\overline{\Natural{a}}+\Number{1010}.$

Рассмотрим процесс сложения $S=A+B$, где $A=(a_{n-1}\cdots a_0)$ и $B=(b_{n-1}\cdots b_0)$. При этом $k$-й разряд суммы в 10СС получается по формуле:

\begin{equation}
    \label{eq:bcd:decAddition}
    s_k=a_k+b_k+c_k,
\end{equation}
где $c_k$ --- перенос в $k$-й разряд, а $s_k,a_k,b_k$ --- десятичные цифры.

Рассмотрим, что произойдет, если вместо десятичных цифр сложить их коды.
\begin{enumerate}
    \item $(a_k+b_k+c_k)<10$; $c_{k+1}=0$. Код $(a_k+b_k+c_k)$ корректен.
    
    \item $10\leq (a_k+b_k+c_k)\leq 15$; $c_{k+1}=0$. Неверно! Перенос в 10СС должен быть, но в 16СС его не случилось. Правильная 10СС цифра $(a_k+b_k+c_k-10)$, и перенос: $(a_k+b_k+c_k-10)+16=(a_k+b_k+c_k+6)$. Поправка: $+6=\Number{0110}$.
    
    \item $(a_k+b_k+c_k)\geq 16$; $c_{k+1}=1$. Неверно! Перенос корректен, а полученная цифра $(a_k+b_k+c_k-16)$ неправильна. Правильный код $(a_k+b_k+c_k-10)$. Поправка: $+6=\Number{0110}$. При такой поправке переноса не будет.
\end{enumerate}

Выделим из предыдущих рассуждений условия поправок и попробуем их упростить.
\begin{enumerate}
    \item\label{en:bcd:fixup} Если $0\leq(a_k+b_k+c_k)\leq 9$, то поправок не надо.
    \item Если $10\leq(a_k+b_k+c_k)\leq 19$, то нужна поправка $(+6)=\Number{0110}$. Переносы из тетрады в тетраду при этом возникают автоматически.
\end{enumerate}

Поэтому поправку $(+6)=\Number{0110}$ можно прибавить к обратному коду одного из операндов заранее. И тогда, если переноса из тетрады не будет (только в случае условия из п. \ref{en:bcd:fixup}) поправку нужно вычесть из тетрады.

Алгоритм сложения в коде {\NaturalLabel} следующий.

\begin{enumerate}
    \item Слагаемые переводятся в обратный $\NaturalLabel$-код. Каждая тетрада модуля отрицательного слагаемого инвертируются и к результату прибавляется код $(-6)=\Number{1010}$. Переносы между тетрадами при этом не распространяются.
    \item К каждой тетраде одного из слагаемых прибавляется поправка $(+6)=\Number{0110}$. Переносов между тетрадами при этом не возникает\footnote{Если одно из слагаемых отрицательно, то поправки перевода в ОК (-6) и поправка данного шага (+6) друг друга компенсируют!}.
    \item Выполняется сложение по правилам двоичной арифметики. Переносы распространяются.
    \item Корректируются тетрады, \emph{из которых} не было переносов. К каждой такой тетраде прибавляется $-6$, т.е. тетрада $\Number{1010}$. Переносы не распространяются.
    \item Результат получен в обратном \NaturalLabel-коде.
\end{enumerate}

\begin{Example}
    Cложить $637.54$ и $-48.54$.
\end{Example}
\begin{Solve}
    Используем целое масштабирование $M=10^{-2}$, $5$-и разрядную сетку и добавим слева два двоичных разряда под знак (МОК).
    \begin{enumerate}
        \item Перевод в ОК:
        \begin{align*}
            \OC{63754}=\Number{00 0110 0011 0111 0101 0100}\\
            -4854\Rightarrow-\Number{0000 0100 1000 0101 0100}\Rightarrow\\
            \Rightarrow\Addition{1111 1011 0111 1010 1011}
                                {1010 1010 1010 1010 1010}
                                {1001 0101 0001 0100 0101}\\
            \OC{-4854}=\Number{11 1001 0101 0001 0100 0101}\\
        \end{align*}
        
        \item К каждой тетраде $\OC{-4854}$ прибавлена тетрда $\Number{0110}$.
        \[
            \Addition{11 1001 0101 0001 0100 0101}
                     {.. 0110 0110 0110 0110 0110}
                     {11 1111 1011 0111 1010 1011}
        \]

        Обратите внимание, что в этом случае (когда хотя бы один из операндов отрицателен), текущая поправка $+6$ и поправка $-6$, которая прибавлялась к инверсии тетрад отрицательного числа, компенсируют друг друга. В данном случае были сделаны лишние действия, достаточно было найти инверсию тетрад $-4854$. В случае, когда оба слагаемых положительны этот шаг необходим.
        
        \item Выполняется сложение полученного числа с $\OC{63754}$. 
        \[
            \Addition{11 1111 1011 0111 1010 1011}
                     {00 0110 0011 0111 0101 0100}
                   {1*00*0101 1110 1110 1111 1111}
        \]
        Коррекция обратного кода переносом из знакового разряда:
        \[
            \Addition{00*0101 1110 1110 1111 1111}
                     {.. .... .... .... .... ...1}
                     {00*0101 1110 1111*0000*0000}
        \]

        \item Корректируются тетрады \emph{из которых не было переносов} в процессе сложения на предыдущем шаге. Переносы между тетрадами не распространяются.
        \[
            \Addition{00*0101 1110 1111*0000*0000}
                     {.. .... 1010 1010 .... ....}
                     {00 0101 1000 1001 0000 0000}
        \]
    \end{enumerate}

    ПРС не возникло, \[\OC{S}=\Number{00 0101 1000 1001 0000 0000},\] $S=58900\cdot 10^{-2}=589$.
\end{Solve}


\subsection{Код \PlusThreeLabel}

Код \PlusThreeLabel --- это код с избытком 3: $\PlusThree{a}=(a+3)$. Соответствие кодов десятичным цифрам представлено в таблице \ref{t:bcd:PlusThree}. 
    
\begin{table}[!ht]
    \caption{Код \PlusThreeLabel}
    \label{t:bcd:PlusThree}
    \centering
    \begin{tabular}{|c|c|c|}
        \hline\hline
        $a$ в 10СС  & $\PlusThree{a}$  & $\PlusThree{9-a}$\\
        \hline\hline
        $0$         & $\Number{0011}$  & $\Number{1100}$ \\
        $1$         & $\Number{0100}$  & $\Number{1011}$ \\
        $2$         & $\Number{0101}$  & $\Number{1010}$ \\
        $3$         & $\Number{0110}$  & $\Number{1001}$ \\
        $4$         & $\Number{0111}$  & $\Number{1000}$ \\
        $5$         & $\Number{1000}$  & $\Number{0111}$ \\
        $6$         & $\Number{1001}$  & $\Number{0110}$ \\
        $7$         & $\Number{1010}$  & $\Number{0101}$ \\
        $8$         & $\Number{1011}$  & $\Number{0100}$ \\
        $9$         & $\Number{1100}$  & $\Number{0011}$ \\
        \hline
    \end{tabular}
\end{table}

Двоично-десятичный код называется самодополняемым, когда в результате инверсии тетрады кода получается код дополнения исходной десятичной цифры до девяти. Код {\PlusThreeLabel} самодополняемый:
\[
    \overline{\PlusThree{a}}=15-(a+3)=12-a=(9-a)+3=\PlusThree{9-a}.
\]

При сложении кодов чисел возможны следующие случаи.

\begin{enumerate}
    \item $s_k<10$; При сложении кодов: \[(a_k+3)+(b_k+3)+c_k=(a_k+b_k+c_k)+6=(s_k+6).\] 

    Так как $(s_k+6)\leq 15$, то переноса не возникает. Правильная тетрада должна быть $(s_k+3)$, следовательно, нужна поправка $-3=\Number{1101}$. Перенос игнорируется.
    
    \item $s_k\geq 10$; При сложении кодов возникнет перенос:
    \[(a_k+3)+(b_k+3)+c_k-16=(a_k+b_k+c_k)-10=s_k-10.\] 

    Для получения правильного: $(s_k-10)+3$, нужна поправка $+3=\Number{0011}$. Перенос игнорируется.
\end{enumerate}

Алгоритм сложения по сравнению с кодов {\NaturalLabel} упрощается
    
\begin{enumerate}
    \item Слагаемые переводятся в обратный $\PlusThreeLabel$-код. Каждая тетрада модуля отрицательного числа инвертируется.
    \item Выполняется сложение полученных операднов по правилам двоичной арифметики (с распространением переносов).
    \item К тетрадам, из которых не было переноса, прибавляется $\Number{1101}$, а к остальным прибавляется $\Number{0011}$. Переносы игнорируются.
    \item Результат получен в обратном \PlusThreeLabel-коде.
\end{enumerate}

\begin{Example}
    Cложить $637.54$ и $-48.54$.
\end{Example}
\begin{Solve}
    Используем целое масштабирование в 5-и разрядной сетке: $M=10^{-2}$. 
    \begin{enumerate}
        \item Перевод в ОК (добавлены два знаковых двоичных разряда МОК):
        \begin{align*}
            -4854\Rightarrow-\Number{0011 0111 1011 1000 0111}\\
            \OC{-4854}=\Number{11 1100 1000 0100 0111 1000}\\
            \OC{63754}=\Number{00 1001 0110 1010 1000 0111}
        \end{align*}
        
        \item Выполняется сложение обратных кодов. 
        \[
            \Addition{11 1100 1000 0100 0111 1000}
                     {00 1001 0110 1010 1000 0111}
                   {1 00*0101 1110 1110 1111 1111}
        \]
        Коррекция переносом:
        \[
            \Addition{00*0101 1110 1110 1111 1111}
                     {.. .... .... .... .... ...1}
                     {00*0101 1110 1111*0000*0000}
        \]

        \item Выполняется коррекция. Переносы между тетрадами не распространяются.
        \[
            \Addition{00*0101 1110 1111*0000*0000}
                     {.. 0011 1101 1101 0011 0011}
                     {00 1000 1011 1100 0011 0011}
        \]
    \end{enumerate}
    
    ПРС не возникло, \[\OC{S}=\Number{00 1000 1011 1100 0011 0011}\] $S=58900\cdot 10^{-2}=589$.
\end{Solve}


\subsection{Код \AikenLabel}

Этот код также называется кодом Айкена, а в исходном названии отражены веса разрядов. Если в коде с естественными весами {\NaturalLabel} вклады разрядов действительно \emph{естественные}, то в коде Айкена третий разряд имеет вес $2$: 
\begin{align*}
    \Aiken{a}\equiv t_{3}t_{2}t_{1}t_{0},\\ 
    a=2t_{3}+4t_{2}+2t_{1}+1t_{0}
\end{align*}

Соответствие кодов десятичным цифрам представлено в таблице \ref{t:bcd:Aiken}. Следует обратить внимание, что одна и та же цифра может быть представлена различными кодами, например, $\Aiken{2}$ это и $\Number{0010}$, и $\Number{1000}$.
    
\begin{table}[!ht]
    \caption{Код \AikenLabel}
    \label{t:bcd:Aiken}
    \centering
    \begin{tabular}{|c|c|c|}
        \hline\hline
        $a$ в 10СС  & $\Aiken{a}$                                   & $\Aiken{9-a}$\\
        \hline\hline
        $0$         & $\Number{0000}$                               & $\Number{1111}$ \\
        $1$         & $\Number{0001}$                               & $\Number{1110}$ \\
        %                  
        $2$         & $\Number{0010}\leftrightarrow\Number{1000}$   & $\Number{1101}\leftrightarrow\Number{0111}$ \\
        $3$         & $\Number{0011}\leftrightarrow\Number{1001}$   & $\Number{1100}\leftrightarrow\Number{0110}$ \\
        $4$         & $\Number{0100}\leftrightarrow\Number{1010}$   & $\Number{1011}\leftrightarrow\Number{0101}$ \\
        $5$         & $\Number{0101}\leftrightarrow\Number{1011}$   & $\Number{1010}\leftrightarrow\Number{0100}$ \\
        $6$         & $\Number{0110}\leftrightarrow\Number{1100}$   & $\Number{1001}\leftrightarrow\Number{0011}$ \\
        $7$         & $\Number{0111}\leftrightarrow\Number{1101}$   & $\Number{1000}\leftrightarrow\Number{0010}$ \\
        %                  
        $8$         & $\Number{1110}$                               & $\Number{0001}$ \\
        $9$         & $\Number{1111}$                               & $\Number{0000}$ \\
        \hline
    \end{tabular}
\end{table}

Код Айкена самодополняем:    
\begin{align*}
    &\overline{T}=\overline{t_{3}t_{2}t_{1}t_{0}}=(2-2t_3) + (4-4t_3) + (2-2t_2) + (1-1t_0)=9-T,\\
    &\overline{\Aiken{a}} = \Aiken{9-a}.
\end{align*}

Обобщая таблицу \ref{t:bcd:Aiken}:
\begin{enumerate}
    \item если $0\leq a \leq 1$, то $\Aiken{a}=a$;
    \item если $2\leq a \leq 7$, то $\Aiken{a}=a$ или $\Aiken{a}=(a+6)$;
    \item если $8\leq a \leq 9$, то $\Aiken{a}=(a+6)$.
\end{enumerate}
    
Преследуя цель <<пусть будет>>: <<однозначность>>, <<перенос>> и <<самодополняемость>>, формулируются новые правила:
\begin{enumerate}
    \item\label{bcd:aiken:ruleI} если $0\leq a \leq 4$, то $\Aiken{a}=a$:
    \[\overline{\Aiken{a}}=(15-a)=\underbrace{(9-a)+6}_{\text{см. п. \ref{bcd:aiken:ruleII}}}=\Aiken{9-a};\]
    
    \item\label{bcd:aiken:ruleII} если $5\leq a \leq 9$, то $\Aiken{a}=(a+6)$:
    \[\overline{\Aiken{a}}=15-(a+6)=\underbrace{(9-a)}_{\text{см. п. \ref{bcd:aiken:ruleI}}}=\Aiken{9-a}.\]
\end{enumerate}

Полученное однозначное кодирование отражено в таблице \ref{t:bcd:AikenClean}.

\begin{table}[!ht]
    \caption{Код \AikenLabel}
    \label{t:bcd:AikenClean}
    \centering
    \begin{tabular}{|c|c|c|}
        \hline\hline
        $a$ в 10СС  & $\Aiken{a}$       & $\Aiken{9-a}$\\
        \hline\hline                   
        $0$         & $\Number{0000}$   & $\Number{1111}$ \\
        $1$         & $\Number{0001}$   & $\Number{1110}$ \\
        $2$         & $\Number{0010}$   & $\Number{1101}$ \\
        $3$         & $\Number{0011}$   & $\Number{1100}$ \\
        $4$         & $\Number{0100}$   & $\Number{1011}$ \\
        $5$         & $\Number{1011}$   & $\Number{0100}$ \\
        $6$         & $\Number{1100}$   & $\Number{0011}$ \\
        $7$         & $\Number{1101}$   & $\Number{0010}$ \\
        $8$         & $\Number{1110}$   & $\Number{0001}$ \\
        $9$         & $\Number{1111}$   & $\Number{0000}$ \\
        \hline
    \end{tabular}
\end{table}

Код по-прежнему самодополняем: $\overline{\Aiken{a}} = \Aiken{9-a}$.

Рассмотрим все возможные случаи, которые могут возникнуть при сложении кодов:
\begin{enumerate}
    \item Если $0\leq a_{k},b_{k}\leq 4$, то сложение кодов $(a_{k}+b_{k}+c_{k})$:
    \begin{enumerate}
        \item если $0\leq (a_{k}+b_{k}+c_{k}) \leq 4$, то поправок не нужно;
        \item если $5\leq(a_{k}+b_{k}+c_{k}) \leq 9$, то неверно! Должно быть: $(a_{k}+b_{k}+c_{k})+6$. Поправка: $+6=\Number{0110}$.
    \end{enumerate}

    \item\label{bcd:aiken:diff} Если $0\leq a_{k}\leq 4$ и $5\leq b_{k}\leq 9$, то код $(a_{k}+b_{k}+c_{k}+6)$:
    \begin{enumerate}
        \item если $5\leq (a_{k}+b_{k}+c_{k}) \leq 9$, то код верен;
        \item если $10\leq(a_{k}+b_{k}+c_{k}) \leq 14$, то формируется перенос и $(a_{k}+b_{k}+c_{k}+6)-16$. Код $(a_{k}+b_{k}+c_{k}-10)$ верен.
    \end{enumerate}
    
    \item Если $5\leq a_{k}\leq 9$ и $0\leq b_{k}\leq 4$ код верен (аналогично п.\ref{bcd:aiken:diff}).

    \item Если $5\leq a_{k},b_{k}\leq 9$, то код $(a_{k}+b_{k}+c_{k}-10)+6$:
    \begin{enumerate}
        \item если $0\leq (a_{k}+b_{k}+c_{k}-10) \leq 4$, то неверно! Должно быть: $(a_{k}+b_{k}+c_{k} - 10)$. Поправка: $-6=\Number{1010}$.
        \item если $5\leq(a_{k}+b_{k}+c_{k}-10) \leq 9$, то код верен!
    \end{enumerate}
\end{enumerate}

Выделим запрещенные комбинации кода:
\[
    \AikenLabel\notin\{\Number{0101},\Number{0110},\Number{0111},\Number{1000},\Number{1001},\Number{1010}\}.
\]
    
Сформулируем правила коррекции.
\begin{enumerate}
    \item Если $0\leq a_{k},b_{k}\leq 4$ и $5\leq(a_{k}+b_{k}+c_{k})\leq 9$, то поправка: $+6=\Number{0110}$. 
    При этом $msb(\Aiken{a_k})=msb(\Aiken{b_k})=0$ и в результате получается одна из запрещенных комбинаций.
    
    \item Если $5\leq a_{k},b_{k}\leq 9$ и $0\leq(a_{k}+b_{k}+c_{k}-10)\leq 4$, то поправка: $-6=\Number{1010}$.
    При этом $msb(\Aiken{a_k})=msb(\Aiken{b_k})=1$ и в результате получается одна из запрещенных комбинаций.
\end{enumerate}

Алгоритм сложения.
    
\begin{enumerate}
    \item Перевести слагаемые в обратный $\AikenLabel$-код. Каждая тетрада модуля отрицательного числа инвертируется.
    \item Выполняется сложение полученных операднов по правилам двоичной арифметики.
    \item Выполняется коррекция результата, в ходе которой переносы между тетрадами не распространяются. 
    Если старшие биты слагаемых тетрад, в результате сложения которых получилась запрещенная тетрада результата, имеют одинаковые значения:
    \begin{enumerate}
        \item $0$, то поправка: $+6=\Number{0110}$;
        \item $1$, то поправка: $-6=\Number{1010}$.
    \end{enumerate}
    \item Результат получен в обратном $\AikenLabel$-коде.
\end{enumerate}

\begin{Example}
    Cложить $-467.08$ и $-125.31$.
\end{Example}
\begin{Solve}
    Используем целое масштабирование в 5-и разрядной сетке: $M=10^{-2}$. 
    \begin{enumerate}
        \item Перевод в ОК (добавлены два знаковых двоичных разряда МОК):
        \begin{align*}
            -46708\Rightarrow-\Number{0100 1100 1101 0000 1110}\\
            \OC{-46708}=\Number{11 1011 0011 0010 1111 0001}\\
            -12531\Rightarrow-\Number{0001 0010 1011 0011 0001}\\
            \OC{-12531}=\Number{11 1110 1101 0100 1100 1110}\\
        \end{align*}
        
        \item Выполняется сложение обратных кодов. 
        \[
            \Addition{11 1011 0011 0010 1111 0001}
                     {11 1110 1101 0100 1100 1110}
                   {1 11 1010 0000 0111 1011 1111}
        \]
        Коррекция переносом:
        \[
            \Addition{11 1010 0000 0111 1011 1111}
                     {.. .... .... .... .... ...1}
                     {11 1010 0000 0111 1100 0000}
        \]

        \item Выполняется коррекция тетрад результата. Переносы между тетрадами не распространяются.
        \begin{align*}
            \OC{-46708}=\Number{11 1011 0011 0010 1111 0001}\\
            \OC{-12531}=\Number{11 1110 1101 0100 1100 1110}\\
                      \Addition{11 1010 0000 0111 1100 0000}
                               {.. 1010 .... 0110 .... ....}
                               {11 0100 0000 1101 1100 0000}
        \end{align*}
    \end{enumerate}
    
    ПРС не возникло, восстанавливаем результат из обратного кода.
    \begin{align*}
        \OC{S}=\Number{11 0100 0000 1101 1100 0000}\\
        \OC{|S|}=\Number{00 1011 1111 0010 0011 1111}
    \end{align*}
    $S=-59239\cdot 10^{-2}=-592.39$.
\end{Solve}


\section{Пентадный код \PentaLabel}

Пятибитный код десятичной цифры $a$ получается по следующей формуле: $\Penta{a}=(3\cdot a + 2)$. Соответствие десятичных цифр пентадам представлено в таблице \ref{t:bcd:Penta}. 
    
\begin{table}[!ht]
    \caption{Код \PentaLabel}
    \label{t:bcd:Penta}
    \centering
    \begin{tabular}{|c|c|c|}
        \hline\hline
        $a$ в 10СС  & $\Penta{a}$       & $\Penta{9-a}$\\
        \hline\hline
        $0$         & $\Number{00010}$  & $\Number{11101}$ \\
        $1$         & $\Number{00101}$  & $\Number{11010}$ \\
        $2$         & $\Number{01000}$  & $\Number{10111}$ \\
        $3$         & $\Number{01011}$  & $\Number{10100}$ \\
        $4$         & $\Number{01110}$  & $\Number{10001}$ \\
        $5$         & $\Number{10001}$  & $\Number{01110}$ \\
        $6$         & $\Number{10100}$  & $\Number{01011}$ \\
        $7$         & $\Number{10111}$  & $\Number{01000}$ \\
        $8$         & $\Number{11010}$  & $\Number{00101}$ \\
        $9$         & $\Number{11101}$  & $\Number{00010}$ \\
        \hline
    \end{tabular}
\end{table}

Пентадный код избыточен:    
\[
    \overline{\Penta{a}}=31-(3a+2)=3(9-a)+2=\Penta{9-a}
\]

Пусть $a_k$, $b_k$ --- $k$-я десятичные цифры операндов, $c_k$ --- перенос в $k$-й разряд из предыдущего.
    
\begin{enumerate}
    \item Если $0\leq a_{k}+b_{k}+c_{k}\leq 9$:
    \begin{enumerate}
        \item если $c_{k}=0$, то код $3(a_{k}+b_{k}) + 4$. Верный код: $3(a_{k}+b_{k})+2$. Поправка: $-2=\Number{11110}$:
        \item если $c_{k}=1$, то код $(3a_k+2)+(3b_{k}+2)+1=3(a_{k}+b_{k}+1)+2$. Код верен!
    \end{enumerate}

    \item Если $a_{k}+b_{k}+c_{k}\geq 10$:
    \begin{enumerate}
        \item если $c_{k}=0$, то код $(3a_{k}+2)+(3b_{k}+2)-32=3(a_{k}+ b_{k}-10)+2$. Код верен!
        \item если $c_{k}=1$, то код $(3a_{k}+2)+(3b_{k}+2)+1-32$, т.е. $3((a_{k}+b_{k}+1)-10)$. Верный код:
        $3((a_{k}+b_{k}+1)-10)+2$. Поправка: $+2=\Number{00010}$.
    \end{enumerate}
\end{enumerate}

Алгоритм сложения:

\begin{enumerate}
    \item Перевести слагаемые в обратный $\PentaLabel$-код. Каждая тетрада модуля отрицательного числа инвертируется.
    \item Выполняется сложение полученных операднов по правилам двоичной арифметики.
    \item Выполняется коррекция. Прибавляется код $11110_{2}$ к пентадам, в которые и из которых не формировались единицы переноса. Прибавляется код $00010_{2}$ к пентадам, в которые и из которых формировались единицы переноса. В процессе коррекции переносы из пентады в пентаду не распространяются.
    \item Результат получен в обратном $\PentaLabel$-коде.
\end{enumerate}

\begin{Example}
    Cложить $684.38$ и $-248.78$.
\end{Example}
\begin{Solve}
    Используем целое масштабирование в 5-и разрядной сетке: $M=10^{-2}$. 
    \begin{enumerate}
        \item Перевод в ОК (добавлены два знаковых двоичных разряда МОК):
        \begin{align*}
            \OC{68438}=\Number{00 10100 11010 01110 01011 11010}\\
            -24878\Rightarrow-\Number{01000 01110 11010 10111 11010}\\
            \OC{-12531}=\Number{11 10111 10001 00101 01000 00101}
        \end{align*}
        
        \item Выполняется сложение обратных кодов. 
        \[
            \Addition{00 10100 11010 01110 01011 11010}
                     {11 10111 10001 00101 01000 00101}
                   {1 00*01100*01011 10011 10011 11111}
        \]
        Коррекция переносом:
        \[
            \Addition{00*01100*01011 10011 10011 11111 }
                     {.. ..... ..... ..... ..... ....1 }
                     {00*01100*01011 10011 10100*00000*}
        \]

        \item Выполняется коррекция пентад результата. К пентадам в которые и из которых был перенос --- прибавляется код \Number{00010}. К пентадам в которые и из которых переноса не было --- прибавляется код \Number{11110}. К остальным пентадам поправок нет. Переносы между пентадами не распространяются.
        \begin{align*}
                      \Addition{00*01100*01011 10011 10100*00000*}
                               {.. 00010 ..... 11110 ..... 00010 }
                               {00 01110 01011 10001 10100 00010 }
        \end{align*}
    \end{enumerate}
    
    ПРС не возникло, восстанавливаем результат из обратного кода.
    \begin{align*}
        \OC{S}=\Number{00 01110 01011 10001 10100 00010}\\
    \end{align*}
    $S=43560\cdot 10^{-2}=435.6$.
\end{Solve}
