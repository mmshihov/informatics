\chapter{Сложение двоичных чисел}
\label{ch:binadd}


\section{Сложение чисел с фиксированной точкой}

В формате с фиксированной точкой масштаб результата сложения такой же, что и масштаб операндов. Поэтому операция сложения особенностей не имеет и выполняется по правилам сложения дополнительных кодов.

\begin{Example}
    Выполнить сложение чисел $({-30.625}+{-5.75})$ в формате с фиксированной точкой 16-разрядной сетке с масштабом $M=2^{-3}$.
\end{Example}
\begin{Solve}
    ${-30.625} = (-11110.101)_2, {-5.75} = (101.11)_2$.
    \begin{align*}
        -30.625=\UnsignedTwoBytes{1111111100001011}\\
        -5.75  =\UnsignedTwoBytes{1111111111010010}\\
    \end{align*}
    
    Сложение в МДК:
    \[
        \Addition{1 11111111 00001011}
                 {1 11111111 11010010}
                 {1 11111110 11011101}
    \]
    
    ПРС отсуствует. Результат с масштабом $2^{-3}$:
    \[
        \UnsignedTwoBytes{1111111011011101}
    \]
    
    $\Number{1111111011011,101}=(-100100,011)_2=-36.375$
\end{Solve}



\section{Сложение чисел с плавающей точкой}


Пусть $A=\FloatExpression{A}{2}$, $B=\FloatExpression{B}{2}$, где $\Mantiss$ --- \emph{нормализованная} мантисса, $\Order$ --- порядок.

При сложении над мантиссами и порядками операндов выполняются различные действия, для которых в составе вычислительных машин предусмотрены различные устройства. Сложение выполняется в несколько этапов.
\begin{enumerate}
    \item Выравниваются порядки слагаемых. Меньший порядок увеличивается до большего, а соответствующая мантисса сдвигается влево на разность порядков --- (\emph{денормализуется}). 
    \[
        \begin{cases}
            \text{Если $(\OrderOf{A}-\OrderOf{B}) = 0$,} & \text{то порядки выравнены},\\
            \text{Если $(\OrderOf{A}-\OrderOf{B}) < 0$,} & \text{то $\MantissOf{A} \gets (\MantissOf{A} \gg |\OrderOf{A}-\OrderOf{B}|), \OrderOf{A}\gets\OrderOf{B}$,}\\
            \text{Если $(\OrderOf{A}-\OrderOf{B}) > 0$,} & \text{то $\MantissOf{B} \gets (\MantissOf{B} \gg |\OrderOf{A}-\OrderOf{B}|), \OrderOf{B}\gets\OrderOf{A}$.}
        \end{cases}
    \]
    
    \item Мантисса результата получается сложением мантисс операндов, получившихся после выравнивания порядков и денормализации одной из мантисс.
    
    \item Мантисса результата нормализуется. При каждом сдвиге мантиссы на один разряд влево, порядок результата должен быть уменьшен на единицу, а при сдвиге вправо --- на единицу увеличиен.
\end{enumerate}

\begin{Example}
    Выполнить сложение чисел $30+5$.
\end{Example}
\begin{Solve}
    Представления чисел (формат см. пример \ref{ch:digitFormat:16order}) будут следующими. 
    \begin{align*}
        30 = \FloatMyHex{0}{111100000}{0}{00101}\\
        5 = \FloatMyHex{0}{101000000}{0}{00011}
    \end{align*}
    
    Выравниваются порядки:
    \begin{align*}
        \FloatMyHex{0}{111100000}{0}{00101}\\
        \FloatMyHex{0}{001010000}{0}{00101}
    \end{align*}

    Выполняется сложение получившихся мантисс в модифицированном дополнительном коде:
    \[
        \Addition{00,111100000}
                 {00,001010000}
                 {01,000110000}
    \]

    Имеется временное ПРС мантисс, которое устраняется в процессе нормализации (при этом порядок увеличивается на единицу):
    \[
        \FloatMyHex{0}{100011000}{0}{00110}
    \]
    
    Результат: $\Number{,100011000}\cdot 2^{6} = 35$.
\end{Solve}

Если в представлении числа используется характеристика, то в силу того, что 
\[
    \Char = \Order + \Delta,
\]
то разность характеристик равна разности порядков:
\[
    (\CharOf{A}-\CharOf{B}) = (\OrderOf{A} + \Delta - (\OrderOf{B} + \Delta)) = (\OrderOf{A}-\OrderOf{B}).
\]

Таким образом, с характеристиками при сложении работают так же, как и с порядками.

\begin{Example}
    Выполнить сложение чисел $-34+19$.
\end{Example}
\begin{Solve}
    Представления чисел (формат см. пример \ref{ch:digitFormat:16char}) будут следующими. 
    
    \begin{align*}
        -34=\FloatMyCharHex{1}{100010000}{100110}\\
         19=\FloatMyCharHex{0}{100110000}{100101}
    \end{align*}
    
    В дополнительном коде выполняется вычитание характеристик $(\CharOf{A}-\CharOf{B})$. Используется семиразрядная сетка (добавлен знаковый разряд):
    \[
        \Addition{0100110}
                 {1011011}
                 {0000001}
    \]
    
    Результат положительный --- денормализуется число $B$:
    \[
        \FloatMyCharHex{0}{010011000}{100110}
    \]
    
    Выполняется сложение получившихся мантисс в МДК:
    \[
        \Addition{11,011110000}
                 {00,010011000}
                 {11,110001000}
    \]

    Результат получается денормализованным:
    \[
        \FloatMyCharHex{1}{001111000}{100110}
    \]
    
    Нормализация мантиссы заключается в сдвге влево на два разряда. Характеристику результата следует уменьшить на два:
    \[
        \FloatMyCharHex{1}{111100000}{100100}
    \]
    
    Результат: $-\Number{,11110000}\cdot 2^{4} = -15$.
\end{Solve}

Ситуация ПРС в формате с плавающей точкой возникает только при переполнении в ходе обработки порядков (или характеристик). Если порядок результата выходит за предел представления положительных чисел, то фиксируется ошибка вычислений. Если порядок результата выходит за предел представления отрицательных числел --- результатом является машинный ноль (ситуация ПМР --- потеря младших разрядов).
