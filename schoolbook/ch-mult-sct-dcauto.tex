\subsection{Умножение в ДК с автоматической коррекцией} 

Будем рассматривать положительное двоичное число со структурной точки зрения: как последовательность непрерывных групп единиц, разделённых непрерывными группами нулей. Минимальная длина группы при этом будет равна одному символу. Рассмотрим группу единиц отдельно:
\[
    \Number{...0001111100...}
\]

Соответствующее ей число можно получить как разность:
\[
    \Subtraction{...0010000000...}
                {...0000000100...}
                {...0001111100...}
\]

Видно, что в представление числа $A$ ненулевой вклад вносят только \emph{переходы из группы в группу}: \Number{01} и \Number{10}. Обозначим пару разрядов, образующих переход $(a_i,a_{i-1})$. Тогда переход $(a_i,a_{i-1})=\Number{01}$ вносит положительный вклад $2^i$, а переход $(a_i,a_{i-1})=\Number{10}$ вносит отрицательный вклад ${-2^i}$.

То есть можно получать число, представленное в двоичной системе счисления, суммируя только вклады переходов!

Разберёмся, что получится, если мы попытаемся найти представление отрицательного числа. При этом не забудем, что после самого младшего в представлении значащего разряда домысливается ноль. Чтобы получить отрицательное число, достаточно изменить знаки вкладов. 

Например, представление положительного числа $124$ 
\[
    \Number{...0001111100}
\] 
вносит вклады $(2^7) + (-2^2)$. Тогда представление $-124$ должно вносить вклады $(-2^7) + (2^2)$, то есть иметь переход \Number{10} в седьмом разряде и \Number{01} во втором. Но! Вопреки ожиданиям, построенное инверсное представление
\[
    \Number{...1110000011}
\]
не является правильным, так как имеет неявный третий переход:
\[
    \Number{...1110000011|0}
\]
который дополнительно вносит вклад $-2^0$. Выполнить коррекцию просто: достаточно прибавить к младшему разряду единицу:
\[
    \Number{...1110000100}
\]

Проверяя, убеждаемся, что $-124 = (-2^7) + (2^3) + (-2^2)$. Кстати, мы получили ни что иное, как \emph{дополнительный код}! Читателю осталось  самостоятельно разобраться, например, с представлением отрицания числа $15$, чтобы убедиться окончательно: 
\begin{Theorem}
    Чтобы получить число, представленное в дополнительном коде, достаточно просуммировать вклады переходов между группами нулей и единиц кода!
\end{Theorem}

При таком подходе становится возможным выполнять умножение непосредственно в дополнительном коде.  Анализируя в $i$-м такте умножения \emph{пары} разрядов множителя $A$ будем либо прибавлять к регистру частных сумм сдвинутое на соответствующее количество разрядов множимое $B$, либо вычитать:
\begin{itemize}
    \item если $(a_i, a_{i-1})=\Number{01}$, то к общей сумме нужно \emph{прибавить} $B\cdot 2^i$;
    \item если $(a_i, a_{i-1})=\Number{10}$, то из общей суммы нужно \emph{вычесть} $B\cdot 2^i$;
    \item для остальных комбинаций никаких действий не требуется.
\end{itemize}

Масштаб выбирается таким образом, чтобы <<знаковым>> разрядом был старший разряд дробной части. Результатом перемножения $n$-разрядных дополнительных кодов является $2n$ разрядный дополнительный код.

\begin{Example}\label{ex:binmul:autoI}
    Множитель $25=(11001)_2$, множимое $22=(10110)_2$. Перемножить числа в дополнительном коде с автоматической коррекцией. Использовать I способ умножения.
\end{Example}
\begin{proof}[Решение]
    Используем дробное масштабирование с множителем $2^6$, при этом знаковым разрядом будет старший разряд дробной части. Тогда 
    \begin{align*}
        \DC{25}=.011001,\\
        \DC{22}=.010110.
    \end{align*}

    Умножение дробных чисел приведено на рисунке \ref{fig:binmul:autoI}. Получен результат: 
    \begin{align*}
        (.001000100110)_2\cdot 2^{12}=550.
    \end{align*}
\end{proof}

\begin{figure}[!ht]
    \centering
    \begin{tabular}{c|r|l}
                                                                   \hline\hline
        мн-ль $\rightarrow$ & 
                                \multicolumn{1}{|c|}{СЧП $\rightarrow$}       
                                                          & прим. \\ \hline\hline
        \NumberLo{,01100}{1|0} & \Addition{,000000 000000}
                                          {,110101 0.....}
                                          {,110101 000000} & $-M$; сдвиг;\\ \hline
        \NumberLo{,.0110}{0|1} & \Addition{,111010 100000}
                                          {,.01011 0.....}
                                          {,000101 100000} & $+M$; сдвиг;\\ \hline
        \NumberLo{,..011}{0|0} &   \Number{,000010 110000} & сдвиг\\ \hline
        \NumberLo{,...01}{1|0} & \Addition{,000001 011000}
                                          {,110101 0.....}
                                          {,110110 011000} & $-M$; сдвиг;\\ \hline
        \NumberLo{,....0}{1|1} &   \Number{,111011 001100} & сдвиг;\\ \hline
        \NumberLo{,.....}{0|1} & \Addition{,111101 100110}
                                          {,.01011 0.....}
                                          {,001000 100110} & $+M$; Рез-т!\\ 
    \end{tabular}
    \caption{Умножение $25\times 22$ в дополнительном коде с автоматической коррекцией (I сп., к примеру \ref{ex:binmul:autoI})}
    \label{fig:binmul:autoI}
\end{figure}


\begin{Example}\label{ex:binmul:autoII}
    Множитель $25=(11001)_2$, множимое $-22=(-10110)_2$. Перемножить числа в дополнительном коде с автоматической коррекцией. Использовать II способ умножения.
\end{Example}
\begin{proof}[Решение]
    Используем дробное масштабирование с множителем $2^6$. Тогда 
    \begin{align*}
        \DC{25} =.011001,\\
        \DC{-22}=.101010. 
    \end{align*}
    
    Процесс умножения приведён на рисунке \ref{fig:binmul:autoII}. Результат в ДК: 
    \begin{align*}
        \Number{,110111011010} = \DC{-.001000100110};\\
        (-.001000100110)_2\cdot 2^{12} = -550.
    \end{align*}
\end{proof}

\begin{figure}[!ht]
    \centering
    \begin{tabular}{c|r|r|l}
                                                                   \hline\hline
        мн-ль $\rightarrow$ 
                               & \multicolumn{1}{|c|}{мн-е $\leftarrow$}       
                                                         & \multicolumn{1}{|c|}{СЧП}
                                                                                    & прим. \\ \hline\hline
        \NumberLo{,01100}{1|0} & \Number{,111111 101010} & \Addition {,000000 000000} 
                                                                     {,000000 010110}
                                                                     {,000000 010110} & $-M$; сдвиг;\\ \hline
        \NumberLo{,.0110}{0|1} & \Number{,111111 01010.} & \Addition {,000000 010110} 
                                                                     {,111111 01010.}
                                                                     {,111111 101010} & $+M$; сдвиг;\\ \hline
        \NumberLo{,..011}{0|0} & \Number{,111110 1010..} &                            & сдвиг;\\ \hline
        \NumberLo{,...01}{1|0} & \Number{,111101 010...} & \Addition {,111111 101010} 
                                                                     {,000010 110...}
                                                                     {,000010 011010} & $-M$; сдвиг;\\ \hline
        \NumberLo{,....0}{1|1} & \Number{,111010 10....} &                            & сдвиг;\\ \hline
        \NumberLo{,.....}{0|1} & \Number{,110101 0.....} & \Addition {,000010 011010} 
                                                                     {,110101 0.....}
                                                                     {,110111 011010} & $+M$; Рез-т!\\ 
    \end{tabular}
    \caption{Умножение $25\times -22$ в дополнительном коде с автоматической коррекцией (II сп., к примеру \ref{ex:binmul:autoII})}
    \label{fig:binmul:autoII}
\end{figure}


\begin{Example}\label{ex:binmul:autoIII}
    Множитель $-22=(-10110)_2$, множимое $-25=(-11001)_2$. Перемножить числа в дополнительном коде с автоматической коррекцией. Использовать III способ умножения.
\end{Example}
\begin{Solve}
    Используем дробное масштабирование с множителем $2^6$. Тогда
    \begin{align*}
        \DC{-22}=.101010,\\
        \DC{-25}=.100111. 
    \end{align*}
    
    Процесс умноженя приведён на рисунке\ref{fig:binmul:autoIII}. Получен результат: 
    \begin{align*}
        (0.001000 100110)_2 \cdot 2^{12} = 550.
    \end{align*}
\end{Solve}

\begin{figure}[!ht]
    \centering
    \begin{tabular}{c|r|l}
                                                                   \hline\hline
        мн-ль $\leftarrow$ & 
                                \multicolumn{1}{|c|}{СЧП $\leftarrow$}       
                                                        & прим.\\ \hline\hline
        \NumberHi{,10}{1010} & \Addition{,000000 000000}
                                        {,...... 011001}
                                        {,000000 011001} & $-M$; сдвиг;\\ \hline
        \NumberHi{,01}{010.} & \Addition{,000000 11001.}
                                        {,111111 100111}
                                        {,000000 011001} & $+M$; сдвиг;\\ \hline
        \NumberHi{,10}{10..} & \Addition{,000000 11001.}
                                        {,...... 011001}
                                        {,000001 001011} & $-M$; сдвиг;\\ \hline
        \NumberHi{,01}{0...} & \Addition{,000010 01011.}
                                        {,111111 100111}
                                        {,000001 111101} & $+M$; сдвиг;\\ \hline
        \NumberHi{,10}{....} & \Addition{,000011 11101.}
                                        {,...... 011001}
                                        {,000100 010011} & $-M$; сдвиг;\\ \hline
        \NumberHi{,0.}{....} &   \Number{,001000 100110} & Рез-т!\\ 
    \end{tabular}
    \caption{Умножение $-22\times-25$ в дополнительном коде с автоматической коррекцией (III сп., к примеру \ref{ex:binmul:autoIII})}
    \label{fig:binmul:autoIII}
\end{figure}


\begin{Example}\label{ex:binmul:autoIV}
    Множитель $-25=(-11001)_2$, множимое $22=(10110)_2$. Перемножить числа в дополнительном коде с автоматической коррекцией. Использовать IV способ умножения.
\end{Example}
\begin{proof}[Решение]
    Используем дробное масштабирование с множителем $2^6$. Тогда
    \begin{align*}
        \DC{-25}=.100111,\\
        \DC{22} =.010110. 
    \end{align*}
    
    Процесс умножения приведён на рисунке \ref{fig:binmul:autoIV}. 
    
    Результат:
    \begin{align*}
        \Number{,110111011010} = \DC{-.001000100110};\\
        (-.001000100110)_2\cdot 2^{12} = -550.
    \end{align*}
\end{proof}

\begin{figure}[!ht]
    \centering
    \begin{tabular}{c|r|r|l}
                                                                   \hline\hline
        мн-ль $\leftarrow$ 
                             & \multicolumn{1}{|c|}{мн-е $\rightarrow$}       
                                                      & \multicolumn{1}{|c|}{СЧП}       
                                                                                  & прим. \\ \hline\hline
        \NumberHi{,10}{0111} & \Number{,.01011 0.....} & \Addition {,000000 000000} 
                                                                   {,110101 0.....}
                                                                   {,110101 000000} & $-M$; сдвиг;\\ \hline
        \NumberHi{,00}{111.} & \Number{,..0101 10....} &                            & сдвиг\\ \hline
        \NumberHi{,01}{11..} & \Number{,...010 110...} & \Addition {,110101 000000} 
                                                                   {,...010 110...}
                                                                   {,110111 110000} & $+M$; сдвиг;\\ \hline
        \NumberHi{,11}{1...} & \Number{,....01 0110..} &                            & сдвиг;\\ \hline
        \NumberHi{,11}{....} & \Number{,.....0 10110.} &                            & сдвиг;\\ \hline
        \NumberHi{,1.}{....} & \Number{,...... 010110} & \Addition {,110111 110000} 
                                                                   {,111111 101010}
                                                                   {,110111 011010} & $-M$; Рез-т!\\ 
    \end{tabular}
    \caption{Умножение $-25\times 22$ в дополнительном коде с автоматической коррекцией (IV сп., к примеру \ref{ex:binmul:autoIV})}
    \label{fig:binmul:autoIV}
\end{figure}
