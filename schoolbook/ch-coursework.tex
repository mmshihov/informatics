\chapter{Курсовая работа}
\label{ch::courcework}

\section{Задание}

В техническом задании на курсовую работу заданы значения четырёх чисел, обозначенных далее как $A$, $B$, $C$, $D$.

Числа $A$ и $B$ --- смешанные десятичные числа, содержащие три значащих цифры в целой части и две значащих цифры в дробной части, причем одно число взято из интервала $[260;500]$, а второе --- из интервала $[600;900]$.

Числа $C$ и $D$ – целые двухразрядные десятичные числа из интервала $[20;90]$, числа $C$ и $D$ не должны быть кратны друг другу.

Задания:
\begin{enumerate}
    \item \emph{Перевод чисел. Форматы.}
    
    Выполнить перевод чисел $A$ и $B$ из одной позиционной системы в другую с использованием промежуточных систем счисления и изобразить их в различных форматах.
    \begin{enumerate}
        \item Числа $A$ и $B$ перевести из 10СС в 2СС, используя 8СС и 16СС в качестве промежуточных. Оценить для заданных чисел получившиеся абсолютную и относительную погрешности представления (перевода $\text{10СС}\to\text{10СС}$).
        \begin{align*}
            A: & \text{10СС}\to\text{8СС}\to\text{2СС}\to\text{16СС}\to\text{10СС}\\
            B: & \text{10СС}\to\text{16СС}\to\text{2СС}\to\text{8СС}\to\text{10СС}.
        \end{align*}
        
        \item Пусть $A>0$, $B<0$. Изобразить каждое число в форме с фиксированной запятой (ФЗ) в дополнительном коде, в 32-разрядной сетке ЦВМ, указав масштаб операндов. Масштаб обоснованно выбирается одинаковый для всех чисел. Следует использовать \emph{дробное} масштабирование.
        
        \item Пусть $A<0$, $B>0$. Изобразить каждое число в 32-разрядной сетке, в форматах с плавающей запятой (ПЗ): ПЭВМ и ЕС ЭВМ.
    \end{enumerate}

    \item \emph{Сложение двоичных чисел.}
    
    Выполнить сложение чисел $A$ и $B$, изменяя их знаки, форму представления и используя различные коды.
    \begin{enumerate}
        \item Знаки операндов: $A>0$, $B<0$. Сложить числа с ФЗ в обратном коде. Проверить результат операции.
        
        \item Знаки операндов: $A<0$, $B>0$. Сложить числа с ФЗ в дополнительном коде. Проверить результат операции.
        
        \item Знаки операндов: $A<0$, $B<0$. Сложить числа в форме с ФЗ в одном из модифицированных кодов --- МОК или МДК. При возникновении ситуации ПРС выполнить корректирующие действия и проверить результат.
        
        \item Оба операнда положительные. Сложить числа в форме с ПЗ, изобразив  исходные операнды в разрядной сетке условной машины. Ориентируясь на разрядность чисел $A$ и $B$, определить для условной машины необходимое количество разрядов для изображения нормализованной мантиссы со знаком и порядка со знаком. Сумму изобразить в разрядной сетке той же условной машины и проверить результат. 
    \end{enumerate}
        
    \item \emph{Умножение двоичных чисел.}
    
    Числа $C$ и $D$ перевести в 2СС и перемножить ($C$ --- множитель, $D$ --- множимое), изменяя их знаки и форму представления, используя различные алгоритмы и способы умножения.
    \begin{enumerate} 
        \item $C>0$, $D<0$. Умножить числа с ФЗ в прямом коде, используя первый способ умножения. Выполнить проверку результата.
        
        \item $C<0$, $D<0$.  Представить их в форме с ФЗ в дополнительном коде и перемножить, используя третий способ умножения и алгоритм с простой коррекцией. Выполнить проверку результата.

        \item $C<0$, $D>0$. Перемножить числа с ФЗ в дополнительном коде, используя второй способ умножения и алгоритм с автоматической коррекцией. Выполнить проверку результата.
        
        \item $C>0$, $D<0$. Умножить числа с ФЗ в прямом коде с ускорением второго порядка, используя четвертый способ умножения. Выполнить проверку результата.
        
        \item $C>0$, $D>0$. Представить числа в форме с ПЗ, изобразив исходные операнды в разрядной сетке условной машины (с по-рядками). При умножении мантисс использовать четвёртый способ умножения.  Изобразить результат в разрядной сетке выбранной условной машины и выполнить проверку результата.
    \end{enumerate} 

    \item \emph{Деление двоичных чисел.}
    Числа $C$ и $D$ перевести в 2СС и представить в формате с плавающей запятой. Разрядность мантиссы выбрать самостоятельно, не менее 8 разрядов. Выполнить деление мантисс различными способами, меняя знаки операндов и коды для представления мантисс.
    \begin{enumerate}
        \item $C>0$, $D<0$; $C$ --- делимое. Представить мантиссы в прямом коде, выполнить деление первым способом, применив алгоритм деления с восстановлением остатков. Проверить результат операции, оценить погрешность округления.
        
        \item $C<0$, $D<0$; $D$ --- делимое. Представить мантиссы в прямом коде, выполнить деление вторым способом, применив алгоритм деления без восстановления остатков. Проверить результат операции, оценить погрешность округления.
        
        \item $C<0$, $D>0$; $D$ --- делимое. Представить мантиссы в дополнительном коде, выполнить деление вторым способом в соответствии с алгоритмом деления в ДК без восстановления остатков. Проверить результат операции, оценить погрешность округления.
    \end{enumerate}
    
    \item \emph{Сложение двоично-десятичных чисел.}
 	
    Сложить числа $A$ и $B$ в четырёх двоично-десятичных кодах: 
    \begin{enumerate}
        \item \texttt{8-4-2-1}; $A<0$, $B>0$;
        \item \texttt{8-4-2-1+3}; $A<0$, $B<0$;
        \item \texttt{2-4-2-1}; $A>0$, $B<0$;
        \item \texttt{3а+2}; $A>0$, $B>0$;
    \end{enumerate}
    Проверить результат. 	
\end{enumerate}


\section{Сроки сдачи}

Курсовая работа должна быть защищена не позднее, чем за две недели до начала сессии.

\section{Требования к оформлению}

Оформление курсовой работы выполняется в соответствии с требованими ЕСКД.
