\chapter*{Список сокращений}
\addcontentsline{toc}{chapter}{Список использованных сокращений}

\begin{itemize}
    \item АЛУ --- арифметико-логическое устройство
    \item ПМР --- потеря младших рязрядов. В формате с плавающей точкой: ситуация, когда получается переполнение порядков или характеристик за нижнюю границу диапазона представления (т.е. получается очень близкое к нулю число). При этом в качестве результата фиксируется машинный ноль
    \item ОЧ --- операционная часть, устройство, аппаратно выполняющее элементарные операции
    \item ПРС --- переполнение разрядной сетки
    \item СС --- система счисления
    \item СЧП --- сумма частичных произведений. В процессе умножения двоичных чисел: накапливаемая сумма сдвигов множимого, которая на последнем шаге умножения соответствует результату
    \item УЧ --- управляющая часть, управляющее устройство
    \item ЦУУ --- центральное устрйойство управления
    \item CISC --- Complex instruction set computing, вычисления на основе процессора с избыточным набором инструкций
    \item RISC --- Reduced instruction set computing, вычисления на основе процессора с сокращенным набором инструкций
\end{itemize}
