\subsection{Беззнаковое умножение с ускорением 2-го порядка}
\label{ss::binmul:fast:double}

В данном методе ускорения работают с четверичными цифрами. Разряды двоичного числа группируются по два и сдвиги множителя (а также множимого или суммы частичных произведений) выполняются сразу на два двоичных разряда. Количество разрядов двоичной сетки выбирается кратным двум. Такой подход сокращает количество шагов умножения вдвое.

На $i$-м шаге умножения при анализе пары двоичных разрядов $(a_{2i+1},a_{2i})$ множителя $A$ должны выполняться следующие действия:
\[
    \begin{tabular}{cc|l}
        \hline\hline
        $a_{2i+1}$ & $a_{2i}$ & Действие над СЧП\\
        \hline\hline
        0 & 0 & $+0$, нет действий\\
        0 & 1 & $+M$, прибавить множимое $M$\\
        1 & 0 & $+2M$, прибавить $M$, сдвинутое на 1-н разряд влево\\
        1 & 1 & $+3M$, прибавить утроенное множимое\\
        \hline
    \end{tabular}
\]

В процессе умножения легко получить удвоенное множимое <<на лету>>, с помощью сдвига. Утроенное же множимое потребует предварительных вычислений. Но этого можно избежать: $3=4-1$, что в двоичном представлении
\[(\Number{11})_2=(\Number{1 00})_2-\Number{1}.\]

\begin{Rule}\label{rule:binbul:double:fastTripple}
    На текущем шаге умножиния вместо сложения с утроенным множимым можно выполнить вычитание множимого $(-M)$ и учесть единицу переноса в старшую пару на \emph{следующем шаге}.
\end{Rule}
Таким образом, находить утроенное множимое не потребуется.

В целях наглядности используем следующие обозначения для переносов $p$:
\[
    p=
    \begin{cases}
        \text{\Number{(+0)}} & \text{при отсутствии переноса ($p=0$)},\\
        \text{\Number{(+1)}} & \text{при единице переноса ($p=1$)}.
    \end{cases}
\]

Для способов, в которых выполняется анализ младших разрядов множителя (I,~II~сп.), правило \ref{rule:binbul:double:fastTripple} работает без поправок. Для I и II способов умножения комбинации разрядов $(a_{2i+1},a_{2i})$ и переноса $p_i$ текущего $i$-го шага и соответвтвующие им действие над СЧП и перенос $p_{i+1}$ приведены в таблице 
\[
    \begin{tabular}{ccc||l|c}
        \hline\hline
        $a_{2i+1}$ & $a_{2i}$   & $p_i$      & СЧП   & $p_{i+1}$\\
        \hline\hline
        \Number{0} & \Number{0} & \Number{(+0)} & $+0$  & \Number{(+0)}\\
        \Number{0} & \Number{1} & \Number{(+0)} & $+M$  & \Number{(+0)}\\
        \Number{1} & \Number{0} & \Number{(+0)} & $+2M$ & \Number{(+0)}\\
        \Number{1} & \Number{1} & \Number{(+0)} & $-M$  & \Number{(+1)}\\
        
        \Number{0} & \Number{0} & \Number{(+1)} & $+M$  & \Number{(+0)}\\
        \Number{0} & \Number{1} & \Number{(+1)} & $+2M$ & \Number{(+0)}\\
        \Number{1} & \Number{0} & \Number{(+1)} & $-M$  & \Number{(+1)}\\
        \Number{1} & \Number{1} & \Number{(+1)} & $+0$  & \Number{(+1)}\\
        \hline
    \end{tabular}
\]

Для способов, в которых выполняется анализ старших разрядов множителя (III,~IV~сп.), правило \ref{rule:binbul:double:fastTripple} требует поправок в силу того, что распространять перенос попросту \emph{некуда}. Но можно обойтись \emph{без распространения переноса}, анализируя разряды младшей пары, находящейся справа от текущей:
\[\cdots \underbrace{(a_{2i+1}a_{2i})}_\textit{текущая} \underbrace{(a_{2i-1}a_{2i-2})}_\textit{младшая}\cdots\]
При этом существует неопределенность возникновения переноса из младшей пары $(a_{2i-1},a_{2i-2})$:
\[
    \begin{tabular}{cc||l}
        \hline\hline
        $(a_{2i-1}$ & $a_{2i-2})$ & Перенос\\
        \hline\hline
        \Number{0} & \Number{0} & \Number{(+0)}\\
        \Number{0} & \Number{1} & \Number{(+0)}\\
        \Number{1} & \Number{0} & ? \emph{Возможно\ldots}\\
        \Number{1} & \Number{1} & \Number{(+1)}\\
        \hline
    \end{tabular}
\]

Действительно, если младшая пара \Number{10} и \emph{в неё} не будет переноса, то \emph{из неё} переноса также не будет. Но если в неё перенос будет, то, уходя от получившейся комбинации \Number{11} по правилу \ref{rule:binbul:double:fastTripple}, из неё будет сгенерирован перенос. Добьёмся определённости с переносом --- будем уходить от комбинации \Number{10} по правилу:
\[
    (\Number{10})_2=(\Number{1 00})_2-(\Number{10})_2,
\]
и тогда перенос будет \emph{всегда}, если старший бит младшей пары равен \Number{1}! 

\begin{Rule}[дополнение к правилу \ref{rule:binbul:double:fastTripple}]\label{rule:binbul:double:fastBin}
    На текущем шаге умножения вместо сложения с удвоенным множимым (комбинация \Number{10}), следует выполнить вычитание удвоенного множимого $(-2M)$. 
\end{Rule}
При этом, в случае, если перенос в пару \Number{10} будет, то с учётом правила \ref{rule:binbul:double:fastBin} нужно: $(4-2+1)$ --- вычитать множимое. Таким образом, с учётом правил \ref{rule:binbul:double:fastTripple} и \ref{rule:binbul:double:fastBin} общий алгоритм умножения заметно упрощается. Нужно анализировать разряды текущей пары и только старший разряд $a_{2i-1}$ младшей пары.

Для III и IV способов комбинации разрядов $(a_{2i+1},a_{2i})(a_{2i-1},\ldots)$ текущего $i$-го шага и соответвтвующее им действие приведены в таблице
\[
    \begin{tabular}{cc|c||l}
        \hline\hline
        $(a_{2i+1}$ & $a_{2i})$   & $(a_{2i-1}$ & СЧП \\
        \hline\hline
        \Number{0} & \Number{0} & \Number{0} & $+0$ \\
        \Number{0} & \Number{1} & \Number{0} & $+M$ \\
        \Number{1} & \Number{0} & \Number{0} & $-2M$\\
        \Number{1} & \Number{1} & \Number{0} & $-M$ \\
        
        \Number{0} & \Number{0} & \Number{1} & $+M$ \\
        \Number{0} & \Number{1} & \Number{1} & $+2M$\\
        \Number{1} & \Number{0} & \Number{1} & $-M$ \\
        \Number{1} & \Number{1} & \Number{1} & $+0$ \\
        \hline
    \end{tabular}
\]


\begin{Example}\label{ex:binmul:fast:Icrp}
    Множитель $109=(1101101)_2$, множимое $233=(11101001)_2$. Перемножить числа I-м способом с ускорением второго порядка.
\end{Example}
\begin{proof}[Решение]
    Используем дробное масштабирование с множителем $2^8$. Тогда 
    \begin{align*}
        109=\Number{,01101101},\\
        233=\Number{,11101001}.
    \end{align*}

    Умножение чисел приведено на рисунке \ref{fig:binmul:fast:Icrp}. Результат: 
    \[
        (.01100011 00110101)_2\cdot 2^{16}=25397.
    \]
\end{proof}

\begin{figure}[!ht]
    \centering
    \begin{tabular}{c|r|l}
                                                                   \hline\hline
        множитель $\rightarrow$ & 
                                \multicolumn{1}{|c|}{СЧП $\rightarrow$}       
                                                        & примечание \\ \hline\hline
        \NumberLo{00,011011}{01|(+0)} & \Addition{00,00000000 00000000}  
                                              {00,11101001 ........}
                                              {00,11101001 00000000} & $+M$; \Number{(+0)}; сдвиг\\ \hline
        \NumberLo{..,000110}{11|(+0)} & \Addition{00,00111010 01000000}
                                              {11,00010111 ........}
                                              {11,01010001 01000000} & $-M$; \Number{(+1)}; сдвиг\\ \hline
        \NumberLo{..,..0001}{10|(+1)} & \Addition{11,11010100 01010000}
                                              {11,00010111 ........}
                                              {10,11101011 01010000} & $-M$; \Number{(+1)}; сдвиг\\ \hline
        \NumberLo{..,....00}{01|(+1)} & \Addition{11,10111010 11010100}
                                              {01,11010010 ........}
                                              {01,10001100 11010100} & $+2M$; \Number{(+0)}; сдвиг\\ \hline
        \NumberLo{..,......}{00|(+0)} &   \Number{00,01100011 00110101} & $+0$; результат!\\ \hline
    \end{tabular}
    \caption{Беззнаковое умножение $109\times 233$ с ускорением второго порядка (I сп., к примеру \ref{ex:binmul:fast:Icrp})}
    \label{fig:binmul:fast:Icrp}
\end{figure}


\begin{Example}\label{ex:binmul:fast:IIcrp}
    Множитель $183=(10110111)_2$, множимое $242=(11110010)_2$. Перемножить числа II-м способом с ускорением второго порядка.
\end{Example}
\begin{proof}[Решение]
    Используем дробное масштабирование с множителем $2^8$. Тогда 
    \begin{align*}
        183=\Number{,10110111},\\
        242=\Number{,11110010}.
    \end{align*}

    Умножение чисел приведено на рисунке \ref{fig:binmul:fast:pcII}. Результат: 
    \[
        (.10101100 11111110)_2\cdot 2^{16}=44286.
    \]
\end{proof}

\begin{figure}[!ht]
    \centering
    \begin{tabular}{c|r|l}
        \hline\hline
        множитель $\rightarrow$ 
            & \multicolumn{1}{|c|}{множимое $\leftarrow$; СЧП}
            & примечание \\ 
        \hline\hline
        \NumberLo{00,101101}{11|(+0)}  
            & \Stack{
                \Register{МН-Е}{00,00000000 11110010}}{
                \Addition{00,00000000 00000000}
                         {11,11111111 00001110}
                         {11,11111111 00001110}}  
            & $-M$; \Number{(+1)}; сдвиг\\ \hline
        \NumberLo{..,001011}{01|(+1)}  
            & \Stack{
                \Register{МН-Е}{00,00000011 110010..}}{
                \Addition{11,11111111 00001110}
                         {00,00000111 10010...}
                         {00,00000110 10011110}}  
            & $+2M$; \Number{(+0)}; сдвиг\\ \hline
        \NumberLo{..,..0010}{11|(+0)}  
            & \Stack{\Register{МН-Е}{00,00001111 0010....}}{
                \Addition{00,00000110 10011110}
                         {11,11110000 1110....}
                         {11,11110111 01111110}}  
            & $-M$; \Number{(+1)}; сдвиг\\ \hline
        \NumberLo{..,....00}{10|(+1)}  
            & \Stack{
                \Register{МН-Е}{00,00111100 10......}}{
                \Addition{11,11110111 01111110}
                         {11,11000011 10......}
                         {11,10111010 11111110}}  
            & $-M$; \Number{(+1)}; сдвиг\\ \hline
        \NumberLo{..,......}{00|(+1)}  
            & \Stack{
                \Register{МН-Е}{00,11110010 ........}}{
                \Addition{11,10111010 11111110}
                         {00,11110010 ........}
                         {00,10101100 11111110}}  
            & $+M$; результат!\\ \hline
    \end{tabular}
    \caption{Беззнаковое умножение $183\times 242$ с ускорением второго порядка (II сп., к примеру \ref{ex:binmul:fast:IIcrp})}
    \label{fig:binmul:fast:pcII}
\end{figure}


\begin{Example}\label{ex:binmul:fast:III}
    Множитель $167=(10100111)_2$, множимое $218=(11011010)_2$. Перемножить числа III-м способом с ускорением второго порядка.
\end{Example}
\begin{proof}[Решение]
    Используем дробное масштабирование с множителем $2^8$. Тогда 
    \begin{align*}
        167=\Number{10100111},\\
        218=\Number{11011010}.
    \end{align*}

    Умножение чисел приведено на рисунке \ref{fig:binmul:fast:III}. Результат: 
    \[
        (.10001110 00110110)_2\cdot 2^{16}=36406.
    \]
\end{proof}

\begin{figure}[!ht]
    \centering
    \begin{tabular}{c|r|l}
        \hline\hline
        множитель $\leftarrow$ 
            & \multicolumn{1}{|c|}{СЧП $\leftarrow$}       
            & примечание \\ 
        \hline\hline
        \NumberHi{00,1}{0100111} 
            & \Addition{00,00000000 00000000}
                       {00,00000000 11011010}
                       {00,00000000 11011010} 
            & $+M$; сдвиг\\ \hline
        \NumberHi{10,1}{00111..} 
            & \Addition{00,00000011 01101000}
                       {11,11111111 00100110}
                       {00,00000010 10001110} 
            & $-M$; сдвиг\\ \hline
        \NumberHi{10,0}{111....} 
            & \Addition{00,00001010 00111000}
                       {11,11111110 01001100}
                       {00,00001000 10000100} 
            & $-2M$; сдвиг\\ \hline
        \NumberHi{01,1}{1......} 
            & \Addition{00,00100010 00010000}
                       {00,00000001 10110100}
                       {00,00100011 11000100} 
            & $+2M$; сдвиг\\ \hline
        \NumberHi{11,.}{.......} 
            & \Addition{00,10001111 00010000}
                       {11,11111111 00100110}
                       {00,10001110 00110110} 
            & $-M$; результат!\\ \hline
    \end{tabular}
    \caption{Беззнаковое умножение $167\times 218$ с ускорением второго порядка (III сп., к примеру \ref{ex:binmul:fast:III})}
    \label{fig:binmul:fast:III}
\end{figure}


\begin{Example}\label{ex:binmul:fast:IV}
    Множитель $222=(11011110)_2$, множимое $227=(11100011)_2$. Перемножить числа IV-м способом с ускорением второго порядка.
\end{Example}
\begin{proof}[Решение]
    Используем дробное масштабирование с множителем $2^8$. Тогда 
    \begin{align*}
        222=\Number{,11011110},\\
        227=\Number{,11100011}.
    \end{align*}

    Умножение чисел приведено на рисунке \ref{fig:binmul:fast:IV}. Результат: 
    \[
        (.11000100 11011010)_2\cdot 2^{16}=50394.
    \]
\end{proof}

\begin{figure}[!ht]
    \centering
    \begin{tabular}{c|r|l}
        \hline\hline
        множитель $\leftarrow$ 
            & \multicolumn{1}{|c|}{множимое $\rightarrow$; СЧП}
            & примечание \\ 
        \hline\hline
        \NumberHi{00,1}{1011110} 
            & \Stack{
                \Register{МН-Е}{00,11100011 ........}}{
                \Addition{00,00000000 00000000}
                         {00,11100011 ........}
                         {00,11100011 00000000}}  
            & $+M$; сдвиг\\ \hline
        \NumberHi{11,0}{11110..} 
            & \Stack{
                \Register{МН-Е}{00,00111000 11......}}{
                \Addition{00,11100011 00000000}
                         {11,11000111 01......}
                         {00,10101010 01000000}} 
            & $-M$; сдвиг\\ \hline
        \NumberHi{01,1}{110....} 
            & \Stack{
                \Register{МН-Е}{00,00001110 0011....}}{
                \Addition{00,10101010 01000000}
                         {00,00011100 0110....}
                         {00,11000110 10100000}}  
            & $+2M$; сдвиг\\ \hline
        \NumberHi{11,1}{0......}  
            & \Stack{
                \Register{МН-Е}{00,00000011 100011..}}{
                \Number{00,11000110 10100000}}  
            & $+0$; сдвиг\\ \hline
        \NumberHi{10,.}{.......} 
            & \Stack{
                \Register{МН-Е}{00,00000000 11100011}}{
                \Addition{00,11000110 10100000}
                         {11,11111110 00111010}
                         {00,11000100 11011010}}  
            & $-2M$; результат!\\ \hline
    \end{tabular}
    \caption{Беззнаковое умножение $222\times 227$ с ускорением второго порядка (IV сп., к примеру \ref{ex:binmul:fast:IV})}
    \label{fig:binmul:fast:IV}
\end{figure}
        
Методику для способов умножения с анализом старших разрядов множителя (III, IV) можно перенести на способы с анализом младших разрядов множителя (I, II) без каких-либо поправок (см. пример \ref{ex:binmul:fast:Itriplet}).
\begin{Example}\label{ex:binmul:fast:Itriplet}
    Множитель $155=(10011011)_2$, множимое $233=(11101001)_2$. Перемножить числа I-м способом с ускорением второго порядка.
\end{Example}
\begin{proof}[Решение]
    Используем дробное масштабирование с множителем $2^8$. Тогда 
    \begin{align*}
        222=\Number{,10011011},\\
        227=\Number{,11101001}.
    \end{align*}

    Умножение чисел приведено на рисунке \ref{fig:binmul:fast:Itriplet}. Результат: 
    \[
        (.10001101 00010011)_2\cdot 2^{16}=36115.
    \]
\end{proof}

\begin{figure}[!ht]
    \centering
    \begin{tabular}{c|r|l}
        \hline\hline
        множитель $\rightarrow$ 
            & \multicolumn{1}{|c|}{СЧП $\rightarrow$}
            & примечание \\ 
        \hline\hline
        \NumberLo{00,100110}{11|.}  
            &  \Addition{00,00000000 00000000}
                        {11,00010111 00000000}
                        {11,00010111 00000000}  
            & $-M$; сдвиг\\ \hline
        \NumberLo{..,001001}{10|1}  
            &  \Addition{11,11000101 11000000}
                        {11,00010111 00000000}
                        {10,11011100 11000000}  
            & $-M$; сдвиг\\ \hline
        \NumberLo{..,..0010}{01|1}  
            &  \Addition{11,10110111 00110000}
                        {01,11010010 00000000}
                        {01,10001001 00110000}  
            & $+2M$; сдвиг\\ \hline
        \NumberLo{..,....00}{10|0}  
            &  \Addition{00,01100010 01001100}
                        {10,00101110 00000000}
                        {10,10010000 01001100}  
            & $-2M$; сдвиг\\ \hline
        \NumberLo{..,......}{00|1}  
            &  \Addition{11,10100100 00010011}
                        {00,11101001 00000000}
                        {00,10001101 00010011}  
            & $+M$; результат\\ \hline
    \end{tabular}
    \caption{Беззнаковое умножение $155\times 233$ с ускорением второго порядка (I сп., к примеру \ref{ex:binmul:fast:Itriplet})}
    \label{fig:binmul:fast:Itriplet}
\end{figure}


\subsection{Беззнаковое умножение с ускорением 3-го порядка}

Подход аналогичен умножению с ускорением второго порядка (см. подраздел \ref{ss::binmul:fast:double}).

В данном методе ускорения работают с восьмиричными цифрами. Разряды двоичного числа группируются по три и сдвиги множителя (а также множимого или суммы частичных произведений, в зависимости от способа) выполняются сразу на три двоичных разряда (одну восьмиричную цифру). Количество разрядов двоичной сетки выбирается кратным трем. Такой подход сокращает количество шагов умножения втрое.

На $i$-м шаге умножения при анализе текущей двоично-кодированной восьмиричной цифры 
\[
    (a_{3i+2},a_{3i+1},a_{3i})
\]
множителя $A$ должны выполняться следующие действия:
\[
    \begin{tabular}{ccc|l}
        \hline\hline
        $a_{3i+2}$ & $a_{3i+1}$ & $a_{3i}$ & Действие над СЧП\\
        \hline\hline
        0 & 0 & 0 & $+0$\\
        0 & 0 & 1 & $+M$\\
        0 & 1 & 0 & $+2M$\\
        0 & 1 & 1 & $+3M$\\
        1 & 0 & 0 & $+4M$\\
        1 & 0 & 1 & $+5M$\\
        1 & 1 & 0 & $+6M$\\
        1 & 1 & 1 & $+7M$\\
        \hline
    \end{tabular}
\]

$M,2M,4M$ легко получаются <<на лету>> сдвигом множимого на соответствующее количество разрядов. Проблему в виде необходимости предварительных вычислений представляют $3M,5M,6M,7M$. Увы, совсем без предварительных вычислений не обойтись: обычно заранее вычисляют утроенное множимое $3M$. 

Для способов со сдвигом множителя впрво (т.е. с анализом младших разрядов множителя; I,II способы умножения) от проблемных слагаемых можно уйти следующим образом:
\begin{itemize}
    \item $3M$ вычисляется;
    \item $5M$ представляется как $8M-3M$, т.е. на текущем шаге выполняется вычитание $3M$, а к анализируемой восьмиричной цифре следующего шага прибавляется единица (переноса);
    \item $6M$ можно получить сдвигом влево $3M$ или, соблюдая равенство $(+8M-2M)$, выполнить вычитание удвоенного множимого, и прибавить единицу переноса к анализируемой восьмиричной цифре на следующем шаге.
    \item $7M$ представляется как $(+8M-M)$.
\end{itemize}

Для способов со сдвигом множителя влево (т.е. III, IV cпособы умножения) следует учесть возникновение переноса из младшей тройки заранее.
\[
    \cdots\underbrace{(a_{3i+2},a_{3i+1},a_{3i})}_\textit{текущая $i$-я}\underbrace{(a_{3i-1},a_{3i-2},a_{3i-3})}_\textit{младшая $(i-1)$-я}\cdots
\]

Проанализируем возможность возникновения переноса из младшей тройки:
\[
    \begin{tabular}{c|ccc|l}
        \hline\hline
          &$a_{3i-1}$ & $a_{3i-2}$ & $a_{3i-3}$ & Сгенерируется перенос\\
        \hline\hline
        0 & 0 & 0 & 0 & нет\\
        1 & 0 & 0 & 1 & нет\\
        2 & 0 & 1 & 0 & нет\\
        3 & 0 & 1 & 1 & нет (т.к. $3M$ вычисляется отдельно)\\
        4 & 1 & 0 & 0 & ? \emph{Возможно\ldots}\\
        5 & 1 & 0 & 1 & ? \emph{Возможно\ldots}\\
        6 & 1 & 1 & 0 & ? \emph{Возможно\ldots}\\
        7 & 1 & 1 & 1 & да\\
        \hline
    \end{tabular}
\]

Следует добиться однозначности в спорных моментах.
\begin{itemize}
    \item $(100)$ требует либо $+4M$, либо $+5M=(+8M-3M)$. Чтобы перенос был всегда, вместо $+4M$ используем $(+8M-4M)$.
    \item $(101)$ требует либо $+5M=(+8M-3M)$, либо $+6M=+2\cdot3M$. Чтобы перенос был всегда, вместо $+6M$ используем $(+8M-2M)$.
    \item $(110)$ требует либо $+6M=+2\cdot 3M$, либо $+7M=(+8M-M)$. Чтобы перенос был всегда, вместо $+6M$ используем $+8M-2M$.
\end{itemize}

Обобщенные результаты можно свести в таблицы:
\[
    \begin{tabular}{cc}
        способы I,II 
            & способы III,IV, а также и I,II\\
        \begin{tabular}{ccc|c||l|c}
            \hline\hline
            $(a_{3i+2}$ & $a_{3i+1}$ & $a_{3i})$ & $p_i$ & СЧП & $p_{i+1}$\\
            \hline\hline
            0 & 0 & 0 & \Number{(+0)} & $+0$  & \Number{(+0)}\\
            0 & 0 & 1 & \Number{(+0)} & $+M$  & \Number{(+0)}\\
            0 & 1 & 0 & \Number{(+0)} & $+2M$ & \Number{(+0)}\\
            0 & 1 & 1 & \Number{(+0)} & $+3M$ & \Number{(+0)}\\
            1 & 0 & 0 & \Number{(+0)} & $+4M$ & \Number{(+0)}\\
            1 & 0 & 1 & \Number{(+0)} & $-3M$ & \Number{(+1)}\\ \hline
            1 & 1 & 0 & \Number{(+0)} & $-2M$ & \Number{(+1)}\\
            1 & 1 & 0 & \Number{(+0)} & $+2\cdot 3M$ & \Number{(+0)}\\ \hline
            1 & 1 & 1 & \Number{(+0)} & $-M$  & \Number{(+1)}\\ \hline
            
            0 & 0 & 0 & \Number{(+1)} & $+M$  & \Number{(+0)}\\
            0 & 0 & 1 & \Number{(+1)} & $+2M$ & \Number{(+0)}\\
            0 & 1 & 0 & \Number{(+1)} & $+3M$ & \Number{(+0)}\\
            0 & 1 & 1 & \Number{(+1)} & $+4M$ & \Number{(+0)}\\
            1 & 0 & 0 & \Number{(+1)} & $-3M$ & \Number{(+1)}\\ \hline
            1 & 0 & 1 & \Number{(+1)} & $-2M$ & \Number{(+1)}\\
            1 & 0 & 1 & \Number{(+1)} & $+2\cdot 3M$ & \Number{(+0)}\\ \hline
            1 & 1 & 0 & \Number{(+1)} & $-M$  & \Number{(+1)}\\
            1 & 1 & 1 & \Number{(+1)} & $+0$  & \Number{(+1)}\\
            \hline
        \end{tabular}
            &
            \begin{tabular}{ccc|c||l}
                \hline\hline
                $(a_{3i+2}$ & $a_{3i+1}$ & $a_{3i})$ & $(a_{3i-1}$ & СЧП\\
                \hline\hline
                0 & 0 & 0 & 0 & $+0$\\
                0 & 0 & 1 & 0 & $+M$\\
                0 & 1 & 0 & 0 & $+2M$\\
                0 & 1 & 1 & 0 & $+3M$\\
                1 & 0 & 0 & 0 & $-4M$\\
                1 & 0 & 1 & 0 & $-3M$\\
                1 & 1 & 0 & 0 & $-2M$\\
                1 & 1 & 1 & 0 & $-M$\\ \hline
                0 & 0 & 0 & 1 & $+M$\\
                0 & 0 & 1 & 1 & $+2M$\\
                0 & 1 & 0 & 1 & $+3M$\\
                0 & 1 & 1 & 1 & $+4M$\\
                1 & 0 & 0 & 1 & $-3M$\\
                1 & 0 & 1 & 1 & $-2M$\\
                1 & 1 & 0 & 1 & $-M$\\
                1 & 1 & 1 & 1 & $+0$\\
                \hline
            \end{tabular}
    \end{tabular}
\]

\begin{Example}\label{ex:binmul:fast:IIIx3Nibble}
    Множитель $3306=(6352)_8=(110011101010)_2$, множимое $2973=(5635)_8=(101110011101)_2$. Перемножить числа III-м способом с ускорением третьего порядка.
\end{Example}
\begin{proof}[Решение]
    Представим числа в двоично-кодированной восьмиричной системе, выбрав масштабирующий множитель $8^4=2^{12}$. Тогда 
    \begin{align*}
        3306=\Number{,110011101010},\\
        2973=\Number{,101110011101}.
    \end{align*}

    Утроенное множимое $(M+2M)$ формируется отдельно:
    \[
        \Addition{.,............ 101110011101}
                 {.,...........1 01110011101.}
                 {0,000000000010 001011010111}
    \]

    Дополнительный код для вычитания утроенного множимого $(-3M)$:
    \[
        \Number{1,111111111101 110100101001}
    \]
        
    Умножение чисел приведено на рисунке \ref{fig:binmul:fast:IIIx3Nibble}. Результат: 
    \[
        (.100101011111 100110000010)_2\cdot 2^{24}=9828738.
    \]
\end{proof}

\begin{figure}[!ht]
    \centering
    \begin{tabular}{c|r|l}
        \hline\hline
        множитель $\rightarrow$ 
            & \multicolumn{1}{|c|}{СЧП $\rightarrow$}
            & прим. \\ 
        \hline\hline
        \NumberHi{...,1}{10011101010}  
            & \Addition{0,000000000000 000000000000}
                       {.,............ 101110011101}
                       {0,000000000000 101110011101}  
            & $+M; \ll$\\ \hline
        \NumberHi{110,0}{11101010...}  
            & \Addition{0,000000000101 110011101...}
                       {1,111111111110 10001100011.}
                       {0,000000000100 010110101110}  
            & $-2M; \ll$\\ \hline
        \NumberHi{011,1}{01010......}  
            & \Addition{0,000000100010 110101110...}
                       {.,..........10 1110011101..}
                       {0,000000100101 101111100100}  
            & $+4M; \ll$\\ \hline
        \NumberHi{101,0}{10.........}  
            & \Addition{0,000100101101 111100100...}
                       {1,111111111101 110100101001}
                       {0,000100101011 110001001001}  
            & $-3M; \ll$\\ \hline
        \NumberHi{010,.}{...........}  
            & \Addition{0,100101011110 001001001...}
                       {.,...........1 01110011101.}
                       {0,100101011111 100110000010}  
            & $+2M; End!$\\ \hline  
    \end{tabular}
    \caption{Беззнаковое умножение $3306\times 2973$ с ускорением третьего порядка (III сп., к примеру \ref{ex:binmul:fast:IIIx3Nibble})}
    \label{fig:binmul:fast:IIIx3Nibble}
\end{figure}
