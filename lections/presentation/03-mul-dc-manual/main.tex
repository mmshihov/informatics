%include part: see main.beamer.tex and main.article.tex
%include common packages and settings
\usepackage{etex} %эта магическая херь избавляет от переполнения регистров TeX а!!!

\mode<article>{\usepackage{fullpage}}
\mode<presentation>{
    \usetheme{Madrid} %%Boadilla,Madrid,AnnArbor,CambridgeUS,Malmoe,Singapore,Berlin
    \useoutertheme{shadow}
} 

\usepackage[utf8]{inputenc}
\usepackage[russian]{babel}
\usepackage{indentfirst}
\usepackage{graphicx}

\usepackage{amsmath}
\usepackage{amsfonts}
\usepackage{amsthm}
\usepackage{algorithm}
\usepackage{algorithmic}

\usepackage[all]{xy}

\date{Лекция по дисциплине <<информатика>>\\(\today)}
\author[М.~М.~Шихов]{Михаил Шихов \\ \texttt{\underline{m.m.shihov@gmail.com}}}

%для рисования графов пакетом xy-pic
\entrymodifiers={++[o][F-]}

%для псевдокода алгоритмов (algorithm,algorithmic)
\renewcommand{\algorithmicrequire}{\textbf{Вход:}}
\renewcommand{\algorithmicensure}{\textbf{Выход:}}
\renewcommand{\algorithmiccomment}[1]{// #1}
\floatname{algorithm}{Псевдокод}

%%определённые мной команды логической разметки
\newcommand{\Signs}[2]{\fbox{{\small{\underbar{#1}}}{#2}}}
\newcommand{\Sign}[1]{\fbox{#1}}
\newcommand{\DC}[1]{\text{ДК}(#1)}
\newcommand{\OC}[1]{\text{ОК}(#1)}
\newcommand{\PC}[1]{\text{ПК}(#1)}

\newcommand{\MyProc}{\text{$\mathbf{R}_8$}}

\newboolean{IsNeedAnswer}
\setboolean{IsNeedAnswer}{false} %true/false

\newcommand{\Machine}[1]{\texttt{#1}}
\newcommand{\Opcode}[1]{\texttt{\bf{#1}}}
\newcommand{\Operand}[1]{\texttt{#1}}
\newcommand{\CmdOneAddr}[2]{\text{\Opcode{#1} \Operand{#2}}}
\newcommand{\CmdTwoAddr}[3]{\text{\Opcode{#1} \Operand{#2}, \Operand{#3}}}
\newcommand{\CmdThreeAddr}[4]{\text{\Opcode{#1} {\Operand{#2}}, \Operand{#3}, \Operand{#4}}}

\newcommand{\ProofAnswer}[1]{
    \ifthenelse{\boolean{IsNeedAnswer}}{
        \begin{proof}[Ответ]
            #1
        \end{proof}
    }{}
}

\newcommand{\LabeledAnswer}[1]{
    \ifthenelse{\boolean{IsNeedAnswer}}{
        \emph{Отв.}: #1 \qed
    }{}
}

\newcommand{\PlainAnswer}[1]{
    \ifthenelse{\boolean{IsNeedAnswer}}{#1}{}
}

\newcommand{\UnsignedAny}[2]{\text{\upshape
    \begin{tabular}{lr}
        \tiny{#1} & \tiny{0}\\ 
        \hline
        \multicolumn{2}{|c|}{\texttt{#2}} \\ 
        \hline
    \end{tabular}
}}

\newcommand{\UnsignedByte}[1]{\UnsignedAny{7}{#1}}

\newcommand{\UnsignedTwoBytes}[1]{\UnsignedAny{15}{#1}}

\newcommand{\FixedAny}[4]{\text{\upshape
    \begin{tabular}{clr}
        \tiny{#1}  &\tiny{#2} & \tiny{0}\\ 
        \hline
        \multicolumn{1}{|c|}{\texttt{#3}} & \multicolumn{2}{|c|}{\texttt{#4}} \\ 
        \hline
    \end{tabular}
}}

\newcommand{\FixedByte}[2]{\FixedAny{7}{6}{#1}{#2}}

\newcommand{\FixedTwoBytes}[2]{\FixedAny{15}{14}{#1}{#2}}

\newcommand{\SignedAny}[4]{\text{\upshape
    \begin{tabular}{clr}
        \tiny{#1}  &\tiny{#2} & \tiny{0}\\ 
        \hline
        \multicolumn{1}{|c|}{\Machine{#3}} & \multicolumn{2}{|c|}{\Machine{#4}} \\ 
        \hline
    \end{tabular}
}}

\newcommand{\SignedNibble}[2]{\SignedAny{3}{2}{#1}{#2}}

\newcommand{\SignedByte}[2]{\SignedAny{7}{6}{#1}{#2}}

\newcommand{\SignedTwoBytes}[2]{\SignedAny{15}{14}{#1}{#2}}

\newcommand{\FloatMyHex}[4]{\text{\upshape
    \begin{tabular}{clrclr}
        \tiny{15}  &\tiny{14} & \tiny{6} & \tiny{5} & \tiny{4} & \tiny{0}\\ 
        \hline
        \multicolumn{1}{|c|}{\texttt{#1}} 
            & \multicolumn{2}{|c|}{\texttt{#2}} 
                & \multicolumn{1}{|c|}{\texttt{#3}} 
                    & \multicolumn{2}{|c|}{\texttt{#4}} \\ 
        \hline
    \end{tabular}
}}

\newcommand{\FloatMyCharHex}[3]{\text{\upshape
    \begin{tabular}{clrlr}
        \tiny{15}  &\tiny{14} & \tiny{6} & \tiny{5} & \tiny{0}\\ 
        \hline
        \multicolumn{1}{|c|}{\texttt{#1}} 
            & \multicolumn{2}{|c|}{\texttt{#2}} 
                & \multicolumn{2}{|c|}{\texttt{#3}} \\ 
        \hline
    \end{tabular}
}}

\newcommand{\FloatMyDcMantCharHex}[2]{\text{\upshape
    \begin{tabular}{lrlr}
        \tiny{15} & \tiny{6} & \tiny{5} & \tiny{0}\\ 
        \hline
        \multicolumn{2}{|c|}{\texttt{#1}} 
            & \multicolumn{2}{|c|}{\texttt{#2}} \\ 
        \hline
    \end{tabular}
}}

\newcommand{\FloatESShort}[3]{\text{\upshape
    \begin{tabular}{clrlr}
        \tiny{31}  &\tiny{30} & \tiny{24} & \tiny{23} & \tiny{0}\\ 
        \hline
        \multicolumn{1}{|c|}{\texttt{#1}} 
            & \multicolumn{2}{|c|}{\texttt{#2}} 
                & \multicolumn{2}{|c|}{\texttt{#3}} 
                    \\ 
        \hline
    \end{tabular}
}}

\newcommand{\FloatPCShort}[3]{\text{\upshape
    \begin{tabular}{clrlr}
        \tiny{31}  &\tiny{30} & \tiny{23} & \tiny{22} & \tiny{0}\\ 
        \hline
        \multicolumn{1}{|c|}{\texttt{#1}} 
            & \multicolumn{2}{|c|}{\texttt{#2}} 
                & \multicolumn{2}{|c|}{\texttt{#3}} 
                    \\ 
        \hline
    \end{tabular}
}}

\newcommand{\FloatMyOrderX}[4]{\text{\upshape
    \begin{tabular}{clrclr}
        \tiny{9}  &\tiny{8} & \tiny{4} & \tiny{3} & \tiny{2} & \tiny{0}\\ 
        \hline
        \multicolumn{1}{|c|}{\texttt{#1}} 
            & \multicolumn{2}{|c|}{\texttt{#2}} 
                & \multicolumn{1}{|c|}{\texttt{#3}} 
                    & \multicolumn{2}{|c|}{\texttt{#4}} \\ 
        \hline
    \end{tabular}
}}

\newcommand{\FloatMyCharX}[3]{\text{\upshape
    \begin{tabular}{clrlr}
        \tiny{9}  &\tiny{8} & \tiny{4} & \tiny{3} & \tiny{0}\\ 
        \hline
        \multicolumn{1}{|c|}{\texttt{#1}} 
            & \multicolumn{2}{|c|}{\texttt{#2}} 
                & \multicolumn{2}{|c|}{\texttt{#3}} \\ 
        \hline
    \end{tabular}
}}

\newcommand{\FloatMyDcOrderX}[3]{\text{\upshape
    \begin{tabular}{lrclr}
        \tiny{9} & \tiny{4} & \tiny{3} & \tiny{2} & \tiny{0}\\ 
        \hline
        \multicolumn{2}{|c|}{\texttt{#1}} 
            & \multicolumn{1}{|c|}{\texttt{#2}} 
                & \multicolumn{2}{|c|}{\texttt{#3}} \\ 
        \hline
    \end{tabular}
}}

\newcommand{\FloatMyDcCharX}[2]{\text{\upshape
    \begin{tabular}{lrlr}
        \tiny{9} & \tiny{4} & \tiny{3} & \tiny{0}\\ 
        \hline
        \multicolumn{2}{|c|}{\texttt{#1}} 
            & \multicolumn{2}{|c|}{\texttt{#2}} \\ 
        \hline
    \end{tabular}
}}


%--- СПЕЦИФИЧНЫЕ ДЛЯ УМНОЖЕНИЯ КОМАНДЫ ---------------------------------------------------------------------------------------------


\newcommand{\Number}[1]{
    \texttt{#1}
}

\newcommand{\NumberHi}[2]{
    \underline{\underline{\texttt{#1}}}\texttt{#2}
}

\newcommand{\NumberMid}[3]{
    \texttt{#1}\underline{\underline{\texttt{#2}}}\texttt{#3}
}

\newcommand{\NumberLo}[2]{
    \texttt{#1}\underline{\underline{\texttt{#2}}}
}

\newcommand{\Stack}[2]{
    \begin{tabular}[t]{@{}r@{}}
        {#1}\\ \hline
        {#2}\\ 
    \end{tabular}
}

\newcommand{\StackThree}[3]{
    \begin{tabular}[t]{@{}r@{}}
        {#1}\\ \hline
        {#2}\\ \hline
        {#3}\\
    \end{tabular}
}

\newcommand{\StackFour}[4]{
    \begin{tabular}[t]{@{}r@{}}
        {#1}\\ \hline
        {#2}\\ \hline
        {#3}\\ \hline
        {#4}\\
    \end{tabular}
}

\newcommand{\Operation}[4]{
    \begin{tabular}[t]{@{}r@{}}
        \texttt{#4}
        \begin{tabular}{@{}r@{}}
            \Number{#1}\\
            \Number{#2}\\ \hline
        \end{tabular} \\ 
        \Number{#3}\\
    \end{tabular}
}

\newcommand{\Addition}[3]{\Operation{#1}{#2}{#3}{+}}

\newcommand{\Subtraction}[3]{\Operation{#1}{#2}{#3}{-}}

\newcommand{\Register}[2]{\Number{#1:#2}}

\newcommand{\Mantiss}{m}
\newcommand{\Order}{p}
\newcommand{\Char}{c}

\newcommand{\MantissOf}[1]{\Mantiss_{#1}}
\newcommand{\OrderOf}[1]{\Order_{#1}}
\newcommand{\CharOf}[1]{\Char_{#1}}

\newcommand{\FloatExpression}[2]{\MantissOf{#1}\cdot {#2}^{\OrderOf{#1}}}

\newcommand{\DivAnswer}[2]{(\texttt{$#1$ rem $#2$})}

\newenvironment{Solve}[1]%
    {\begin{proof}[Решение]#1}
    {\end{proof}}
    
    
%определённые мной команды логической разметки
\newcommand{\Signs}[2]{\fbox{{\small{\underbar{#1}}}{#2}}}
\newcommand{\Sign}[1]{\fbox{#1}}
\newcommand{\DC}[1]{\text{ДК}(#1)}
\newcommand{\OC}[1]{\text{ОК}(#1)}
\newcommand{\PC}[1]{\text{ПК}(#1)}

\newcommand{\MyProc}{\text{$\mathbf{R}_8$}}

\newboolean{IsNeedAnswer}
\setboolean{IsNeedAnswer}{false} %true/false

\newcommand{\Machine}[1]{\texttt{#1}}
\newcommand{\Opcode}[1]{\texttt{\bf{#1}}}
\newcommand{\Operand}[1]{\texttt{#1}}
\newcommand{\CmdOneAddr}[2]{\text{\Opcode{#1} \Operand{#2}}}
\newcommand{\CmdTwoAddr}[3]{\text{\Opcode{#1} \Operand{#2}, \Operand{#3}}}
\newcommand{\CmdThreeAddr}[4]{\text{\Opcode{#1} {\Operand{#2}}, \Operand{#3}, \Operand{#4}}}

\newcommand{\ProofAnswer}[1]{
    \ifthenelse{\boolean{IsNeedAnswer}}{
        \begin{proof}[Ответ]
            #1
        \end{proof}
    }{}
}

\newcommand{\LabeledAnswer}[1]{
    \ifthenelse{\boolean{IsNeedAnswer}}{
        \emph{Отв.}: #1 \qed
    }{}
}

\newcommand{\PlainAnswer}[1]{
    \ifthenelse{\boolean{IsNeedAnswer}}{#1}{}
}

\newcommand{\UnsignedAny}[2]{\text{\upshape
    \begin{tabular}{lr}
        \tiny{#1} & \tiny{0}\\ 
        \hline
        \multicolumn{2}{|c|}{\texttt{#2}} \\ 
        \hline
    \end{tabular}
}}

\newcommand{\UnsignedByte}[1]{\UnsignedAny{7}{#1}}

\newcommand{\UnsignedTwoBytes}[1]{\UnsignedAny{15}{#1}}

\newcommand{\FixedAny}[4]{\text{\upshape
    \begin{tabular}{clr}
        \tiny{#1}  &\tiny{#2} & \tiny{0}\\ 
        \hline
        \multicolumn{1}{|c|}{\texttt{#3}} & \multicolumn{2}{|c|}{\texttt{#4}} \\ 
        \hline
    \end{tabular}
}}

\newcommand{\FixedByte}[2]{\FixedAny{7}{6}{#1}{#2}}

\newcommand{\FixedTwoBytes}[2]{\FixedAny{15}{14}{#1}{#2}}

\newcommand{\SignedAny}[4]{\text{\upshape
    \begin{tabular}{clr}
        \tiny{#1}  &\tiny{#2} & \tiny{0}\\ 
        \hline
        \multicolumn{1}{|c|}{\Machine{#3}} & \multicolumn{2}{|c|}{\Machine{#4}} \\ 
        \hline
    \end{tabular}
}}

\newcommand{\SignedNibble}[2]{\SignedAny{3}{2}{#1}{#2}}

\newcommand{\SignedByte}[2]{\SignedAny{7}{6}{#1}{#2}}

\newcommand{\SignedTwoBytes}[2]{\SignedAny{15}{14}{#1}{#2}}

\newcommand{\FloatMyHex}[4]{\text{\upshape
    \begin{tabular}{clrclr}
        \tiny{15}  &\tiny{14} & \tiny{6} & \tiny{5} & \tiny{4} & \tiny{0}\\ 
        \hline
        \multicolumn{1}{|c|}{\texttt{#1}} 
            & \multicolumn{2}{|c|}{\texttt{#2}} 
                & \multicolumn{1}{|c|}{\texttt{#3}} 
                    & \multicolumn{2}{|c|}{\texttt{#4}} \\ 
        \hline
    \end{tabular}
}}

\newcommand{\FloatMyCharHex}[3]{\text{\upshape
    \begin{tabular}{clrlr}
        \tiny{15}  &\tiny{14} & \tiny{6} & \tiny{5} & \tiny{0}\\ 
        \hline
        \multicolumn{1}{|c|}{\texttt{#1}} 
            & \multicolumn{2}{|c|}{\texttt{#2}} 
                & \multicolumn{2}{|c|}{\texttt{#3}} \\ 
        \hline
    \end{tabular}
}}

\newcommand{\FloatMyDcMantCharHex}[2]{\text{\upshape
    \begin{tabular}{lrlr}
        \tiny{15} & \tiny{6} & \tiny{5} & \tiny{0}\\ 
        \hline
        \multicolumn{2}{|c|}{\texttt{#1}} 
            & \multicolumn{2}{|c|}{\texttt{#2}} \\ 
        \hline
    \end{tabular}
}}

\newcommand{\FloatESShort}[3]{\text{\upshape
    \begin{tabular}{clrlr}
        \tiny{31}  &\tiny{30} & \tiny{24} & \tiny{23} & \tiny{0}\\ 
        \hline
        \multicolumn{1}{|c|}{\texttt{#1}} 
            & \multicolumn{2}{|c|}{\texttt{#2}} 
                & \multicolumn{2}{|c|}{\texttt{#3}} 
                    \\ 
        \hline
    \end{tabular}
}}

\newcommand{\FloatPCShort}[3]{\text{\upshape
    \begin{tabular}{clrlr}
        \tiny{31}  &\tiny{30} & \tiny{23} & \tiny{22} & \tiny{0}\\ 
        \hline
        \multicolumn{1}{|c|}{\texttt{#1}} 
            & \multicolumn{2}{|c|}{\texttt{#2}} 
                & \multicolumn{2}{|c|}{\texttt{#3}} 
                    \\ 
        \hline
    \end{tabular}
}}

\newcommand{\FloatMyOrderX}[4]{\text{\upshape
    \begin{tabular}{clrclr}
        \tiny{9}  &\tiny{8} & \tiny{4} & \tiny{3} & \tiny{2} & \tiny{0}\\ 
        \hline
        \multicolumn{1}{|c|}{\texttt{#1}} 
            & \multicolumn{2}{|c|}{\texttt{#2}} 
                & \multicolumn{1}{|c|}{\texttt{#3}} 
                    & \multicolumn{2}{|c|}{\texttt{#4}} \\ 
        \hline
    \end{tabular}
}}

\newcommand{\FloatMyCharX}[3]{\text{\upshape
    \begin{tabular}{clrlr}
        \tiny{9}  &\tiny{8} & \tiny{4} & \tiny{3} & \tiny{0}\\ 
        \hline
        \multicolumn{1}{|c|}{\texttt{#1}} 
            & \multicolumn{2}{|c|}{\texttt{#2}} 
                & \multicolumn{2}{|c|}{\texttt{#3}} \\ 
        \hline
    \end{tabular}
}}

\newcommand{\FloatMyDcOrderX}[3]{\text{\upshape
    \begin{tabular}{lrclr}
        \tiny{9} & \tiny{4} & \tiny{3} & \tiny{2} & \tiny{0}\\ 
        \hline
        \multicolumn{2}{|c|}{\texttt{#1}} 
            & \multicolumn{1}{|c|}{\texttt{#2}} 
                & \multicolumn{2}{|c|}{\texttt{#3}} \\ 
        \hline
    \end{tabular}
}}

\newcommand{\FloatMyDcCharX}[2]{\text{\upshape
    \begin{tabular}{lrlr}
        \tiny{9} & \tiny{4} & \tiny{3} & \tiny{0}\\ 
        \hline
        \multicolumn{2}{|c|}{\texttt{#1}} 
            & \multicolumn{2}{|c|}{\texttt{#2}} \\ 
        \hline
    \end{tabular}
}}


%--- СПЕЦИФИЧНЫЕ ДЛЯ УМНОЖЕНИЯ КОМАНДЫ ---------------------------------------------------------------------------------------------


\newcommand{\Number}[1]{
    \texttt{#1}
}

\newcommand{\NumberHi}[2]{
    \underline{\underline{\texttt{#1}}}\texttt{#2}
}

\newcommand{\NumberMid}[3]{
    \texttt{#1}\underline{\underline{\texttt{#2}}}\texttt{#3}
}

\newcommand{\NumberLo}[2]{
    \texttt{#1}\underline{\underline{\texttt{#2}}}
}

\newcommand{\Stack}[2]{
    \begin{tabular}[t]{@{}r@{}}
        {#1}\\ \hline
        {#2}\\ 
    \end{tabular}
}

\newcommand{\StackThree}[3]{
    \begin{tabular}[t]{@{}r@{}}
        {#1}\\ \hline
        {#2}\\ \hline
        {#3}\\
    \end{tabular}
}

\newcommand{\StackFour}[4]{
    \begin{tabular}[t]{@{}r@{}}
        {#1}\\ \hline
        {#2}\\ \hline
        {#3}\\ \hline
        {#4}\\
    \end{tabular}
}

\newcommand{\Operation}[4]{
    \begin{tabular}[t]{@{}r@{}}
        \texttt{#4}
        \begin{tabular}{@{}r@{}}
            \Number{#1}\\
            \Number{#2}\\ \hline
        \end{tabular} \\ 
        \Number{#3}\\
    \end{tabular}
}

\newcommand{\Addition}[3]{\Operation{#1}{#2}{#3}{+}}

\newcommand{\Subtraction}[3]{\Operation{#1}{#2}{#3}{-}}

\newcommand{\Register}[2]{\Number{#1:#2}}

\newcommand{\Mantiss}{m}
\newcommand{\Order}{p}
\newcommand{\Char}{c}

\newcommand{\MantissOf}[1]{\Mantiss_{#1}}
\newcommand{\OrderOf}[1]{\Order_{#1}}
\newcommand{\CharOf}[1]{\Char_{#1}}

\newcommand{\FloatExpression}[2]{\MantissOf{#1}\cdot {#2}^{\OrderOf{#1}}}

\newcommand{\DivAnswer}[2]{(\texttt{$#1$ rem $#2$})}

\newenvironment{Solve}[1]%
    {\begin{proof}[Решение]#1}
    {\end{proof}}
    
    

\title[Умножение в ДК (РК)]{Умножение в дополнительном коде\\(с ручной коррекцией)}

\newcounter{TaskSimpleCtr}
\setcounter{TaskSimpleCtr}{1}
\newcommand{\TaskSimpleNumber}{ \arabic{TaskSimpleCtr}) \addtocounter{TaskSimpleCtr}{1} }

%вставка изображений из metapost (post script)
\DeclareGraphicsRule{*}{mps}{*}{}

\begin{document}

\mode<article>{\maketitle\tableofcontents}
\frame<presentation>{\titlepage}
\begin{frame}<presentation>
    \frametitle{Содержание}
    \tableofcontents
\end{frame}


\section{Обоснование корректности}


\subsection{Точка зрения на дополнительный код}


\begin{frame}
    \frametitle{Точка зрения на дополнительный код}

    С помощью дополнительного кода в $n$-разрядной сетке можно представить целые числа из отрезка
    \[
        X \in [-2^{n-1},+(2^{n-1}-1)].
    \]

    В этом случае:
    \[
        \DC{X} = 
        \begin{cases}
            |X|,      & \text{если $X\geq 0$},\\
            2^n-|X|,  & \text{если $X < 0$}.\\
        \end{cases}
    \]
\end{frame}

\begin{frame}
    \frametitle{Масштабированный дополнительный код}

    Если выполнить масштабирование с масштабом $M=2^n$:
    \[
        X = x \cdot 2^n.
    \]
     
    Тогда:
    \[
        \DC{X} = 
        \begin{cases}
            |x|\cdot 2^n,      & \text{если $X\geq 0$},\\
            (1-|x|)\cdot 2^n,    & \text{если $X < 0$}.\\
        \end{cases}
    \]
    
    Для дробных представлений $x$ справедливо:
    \begin{equation}
        \DC{x} = 
        \begin{cases}
            |x|,      & \text{если $x\geq 0$},\\
            1-|x|,    & \text{если $x < 0$}.\\
        \end{cases}
        \label{eq:ch:mult:sct:dcmanual:dc}
    \end{equation}
\end{frame}

\begin{frame}
    \frametitle{Дополнительный код --- положительное число?!}

    Согласно формуле \eqref{eq:ch:mult:sct:dcmanual:dc} дополнительный код после масштабирования можно рассматривать как 
    
    \begin{block}{}
        \begin{center}
            \emph{положительное} дробное число.
        \end{center}
    \end{block}
    
    Так как $X \in [-2^{n-1},+(2^{n-1}-1)]$, то $x \in [-2^{-1}, \leq +(2^{-1} - 2^{-n})]$, следовательно 
    \[(1-|x|) > 0.\] 
    
    Поэтому основной цикл умножения масштабированных представлений дополнительных кодов 
    \begin{block}{}
        \begin{center}
            нужно выполнять по правилам перемножения \emph{беззнаковых} чисел.
        \end{center}
    \end{block}
\end{frame}

\begin{frame}
    Пусть
    \begin{align*}
        A = a \cdot 2^n,\\
        B = b \cdot 2^n,
    \end{align*}
    далее выполняются операции с дробными $a,b$.
\end{frame}

\subsection{Нужна коррекция}

\begin{frame}
    \frametitle{Коррекция псевдопроизведения $\DC{a}\cdot\DC{b}$}

    \begin{itemize}
        \item \emph{Оба сомножителя положительны}. Поправок не требуется.
        
        \item \emph{Один из сомножителей отрицателен}. Пусть $a<0$, $b\geq 0$, тогда правильный код результата: $\DC{ab}=(1-|ab|)$. Псевдопроизведение:
        \[
            \DC{a}\cdot\DC{b} = (1-|a|)\cdot|b|=|b|-|a|\cdot|b|.
        \] 
        
        Нужна поправка: $(1-|b|)=\DC{-b}$. 
        
        \item \emph{Оба сомножителя отрицательны}. Правильный код результата: $\DC{ab}=|ab|$. Псевдопроизведение:
        \[
            \DC{a}\cdot\DC{b} = (1-|a|)(1-|b|)=1-|a|-|b|+|ab|
        \] 
        Прибавив поправку $(|a|+|b|)$, получим $(1+|ab|)$, который, вследствие переноса единицы в целую часть, эквивалентен правильному $|ab|$.
    \end{itemize}
\end{frame}

\begin{frame}
    \frametitle{Резюме: $\DC{ab}=\ldots$}

    \begin{itemize}
        \item $a\geq 0, b\geq 0$:       
            $\DC{ab}=\DC{a}\cdot\DC{b}$.
        
        \item $a\geq 0, b<0$:
            $\DC{ab}=\DC{a}\cdot\DC{b} + \DC{-a}$.

        \item $a<0, b\geq 0$:
            $\DC{ab}=\DC{a}\cdot\DC{b} + \DC{-b}$.
        
        \item $a<0, b<0$:
            $\DC{ab} = \DC{a}\cdot\DC{b}+\DC{-a}+\DC{-b}$.
    \end{itemize}
    \begin{block}{Упрощенное правило ручной коррекции}
        Достаточно проверить знак каждого аргумента и, если этот аргумент отрицателен, то из псевдопроизведения \emph{вычитается парный} отрицательному аргумент:
        \begin{columns}
            \column{.3\textwidth}
            \column{.7\textwidth}
            \begin{algorithmic}[1]
                \IF{$a<0$}
                    \STATE{$\Machine{СЧП}:=\Machine{СЧП}-b$;}
                \ENDIF
                \IF{$b<0$}
                    \STATE{$\Machine{СЧП}:=\Machine{СЧП}-a$;}
                \ENDIF
            \end{algorithmic}
        \end{columns}
    \end{block}
\end{frame}


\section{Коррекция вовремя}


\subsection{Технические ограничения}

\begin{frame}
    \frametitle{Основные способы умножения}
    \begin{center}
        \begin{tabular}{c||c||c}
                & Сдвиг СЧП
                    & Сдвиг множимого\\
            \hline\hline
            \rotatebox{90}{\texttt{shr}(Мн-ль)}
                & \includegraphics[width=.43\textwidth]{fig/mult-methods.1}
                    & \includegraphics[width=.43\textwidth]{fig/mult-methods.2}\\
            \hline\hline
            \rotatebox{90}{\texttt{shl}(Мн-ль)}
                & \includegraphics[width=.43\textwidth]{fig/mult-methods.3}
                    & \includegraphics[width=.43\textwidth]{fig/mult-methods.4}\\
        \end{tabular}
    \end{center}
\end{frame}

\begin{frame}
    \begin{block}{}
        \begin{center}
            Коррекции подлежит \emph{старшая} половина $2n$ разрядного псевдопроизведения\footnote{$n$-разрядные множимое и множитель вычитаются из \emph{старшей} половины псевдопроизведения}.
        \end{center}
    \end{block}
\end{frame}

\begin{frame}
    \frametitle{I-й способ: технические ограничения}
    
    \begin{columns}
        \column{.4\textwidth}
            \begin{block}{}
                \includegraphics[width=\columnwidth]{fig/mult-methods.1}            
            \end{block}
        \column{.55\textwidth}
            \begin{block}{Особенности I-го способа}
                \begin{itemize}
                    \item СЧП сдвигается вправо; 
                    \item Множимое прибавляется к старшей половине СЧП;
                    \item Множимое не сдвигается.
                \end{itemize}
            \end{block}
    \end{columns}
    
    \begin{itemize}
        \item Коррекция выполняется \emph{только в конце} цикла умножения. В противном случае все поправки <<уедут>> в младшие разряды СЧП.
        \item Исходное значение множителя нужно \emph{сохранять}.
    \end{itemize}
    Так как в цикле умножения к СЧП прибавляется половина множимого, а при коррекции нужно вычесть \emph{целое} множимое, то можно изменить алгоритм: в цикле прибавлять к СЧП \emph{целое} множимое (при этом СЧП в конце цикла окажется \emph{вдвое больше} правильного результата) и сдвинуть всю СЧП перед корреркцией.
\end{frame}    

\begin{frame}
    \frametitle{II-й способ: технические ограничения}
    
    \begin{columns}
        \column{.4\textwidth}
            \begin{block}{}
                \includegraphics[width=\columnwidth]{fig/mult-methods.2}
            \end{block}
        \column{.55\textwidth}
            \begin{block}{Особенности II-го способа}
                \begin{itemize}
                    \item СЧП не сдвигается; 
                    \item Множимое заносится в младшую часть $2n$-разрадного регистра.
                    \item Множимое сдвигается влево;
                \end{itemize}
            \end{block}
    \end{columns}
    
    \begin{itemize}
        \item Поправка множителем может выполняться как в начале (проще всего), так и в конце цикла умножения. В конце цикла сложнее --- нужно сохранять множитель.
        \item Поправка множимым без дополнительных затрат выполняется в конце цикла, когда после серии сдвигов множимое выходит в старшую часть $2n$-разрядного регистра.
    \end{itemize}
\end{frame}    

\begin{frame}
    \frametitle{III-й способ: технические ограничения}
    
    \begin{columns}
        \column{.4\textwidth}
            \begin{block}{}
                \includegraphics[width=\columnwidth]{fig/mult-methods.3}
            \end{block}
        \column{.55\textwidth}
            \begin{block}{Особенности III-го способа}
                \begin{itemize}
                    \item СЧП сдвигается влево; 
                    \item Множимое прибавляется к младшей половине $2n$-разрядной СЧП.
                    \item Множимое сдвигается влево;
                \end{itemize}
            \end{block}
    \end{columns}
    
    \begin{itemize}
        \item Поправка множителем выполняется либо к младшей части СЧП в начале цикла умножения, либо в конце цикла --- к старшей части (при этом множитель нужно сохранять).
        \item Поправка множимым без дополнительных затрат выполняется в начале цикла умножения. В конце цикла, после серии сдвигов СЧП, она станет правильной.
    \end{itemize}
\end{frame}    
    
\begin{frame}
    \frametitle{IV-й способ: технические ограничения}
    
    \begin{columns}
        \column{.4\textwidth}
            \begin{block}{}
                \includegraphics[width=\columnwidth]{fig/mult-methods.4}
            \end{block}
        \column{.55\textwidth}
            \begin{block}{Особенности IV-го способа}
                \begin{itemize}
                    \item СЧП не сдвигается; 
                    \item Множимое заносится в старшую часть $2n$-разрадного регистра.
                    \item Множимое сдвигается вправо;
                \end{itemize}
            \end{block}
    \end{columns}
    
    \begin{itemize}
        \item Поправка множителем к старшей половине СЧП может выполняеться либо в начале цикла умножения, либо в конце (позаботившись о спасении исходного значения множителя от <<сдвиговой>> смерти).

        \item Поправка множимим без дополнительных затрат выполняется до цикла умножения. После поправки выполняется сдвиг регистра множимого и цикл выполняется как обычно. 
    \end{itemize}
\end{frame}  


\subsection{Примеры}

\begin{frame}
    \frametitle{Операнды для примеров}

    В качестве примера будем перемножать числа $9$ и $11$ с различными комбинациями знаков. 
    
    Выбрав масштаб $M=2^5$, получим следующие представления:
    
    \begin{align*}
        \DC{9}   & = \Number{,01001},\\
        \DC{-9}  & = \Number{,10111},\\
        \DC{11}  & = \Number{,01011},\\
        \DC{-11} & = \Number{,10101}.
    \end{align*}
\end{frame}

\begin{frame}
    \frametitle{I-способ: $-9\cdot 11$. $\DC{-99}=\Number{,11100 11101}$}

    \resizebox{!}{0.8\height}{
        \begin{tabular}{c|r|l}
            \hline\hline
            мн-ль $\rightarrow$ & 
                                    \multicolumn{1}{|c|}{СЧП $\rightarrow$}       
                                                            & прим. \\ 
            \hline\hline
            \NumberLo{,1011}{1} & \Addition{,00000 00000}
                                           {,.0101 1....}
                                           {,00101 10000} & +мн-е/2; сдвиг \\ \hline
            \NumberLo{,.101}{1} & \Addition{,.0010 11000}                
                                           {,.0101 1....}                
                                           {,01000 01000} & +мн-е/2; сдвиг \\ \hline
            \NumberLo{,..10}{1} & \Addition{,.0100 00100}                
                                           {,.0101 1....}                
                                           {,01001 10100} & +мн-е/2; сдвиг \\ \hline
            \NumberLo{,...1}{0} &   \Number{,.0100 11010} & сдвиг        \\ \hline
            \NumberLo{,....}{1} & \Addition{,..010 01101}
                                           {,.0101 1....}
                                           {,00111 11101} & +мн-е/2; Рез-т?\\ \hline\hline
                                & \Addition{,00111 11101}
                                           {,10101 .....}
                                           {,11100 11101} & +\DC{-11}=\DC{-мн-е}; Рез-т!
        \end{tabular}
    }
\end{frame}

\begin{frame}
    \frametitle{I-способ (модиф.): $-9\cdot -11$. $\DC{99}=\Number{,00011 00011}$}

    \resizebox{!}{.7\height}{
        \begin{tabular}{c|r|l}
            \hline\hline
            мн-ль $\rightarrow$ & 
                                    \multicolumn{1}{|c|}{СЧП $\rightarrow$}       
                                                            & прим. \\ 
            \hline\hline
            \NumberLo{,1011}{1} & \Addition{0,00000 00000}
                                           {.,10101 .....}
                                           {0,10101 00000} & +мн-е; сдвиг \\ \hline
            \NumberLo{,.101}{1} & \Addition{.,01010 10000}                
                                           {.,10101 .....}                
                                           {0,11111 10000} & +мн-е; сдвиг \\ \hline
            \NumberLo{,..10}{1} & \Addition{.,01111 11000}                
                                           {.,10101 .....}                
                                           {1,00100 11000} & +мн-е; сдвиг \\ \hline
            \NumberLo{,...1}{0} &   \Number{.,10010 01100} & сдвиг        \\ \hline
            \NumberLo{,....}{1} & \Addition{.,.1001 00110}
                                           {.,10101 .....}
                                           {0,11110 00110} & +мн-е; сдвиг;\\ \hline
                                &   \Number{.,01111 00011} & Рез-т?       \\ \hline\hline
                                & \Addition{.,01111 00011}
                                           {.,01011 .....}
                                           {0,11010 00011} & +\DC{11}=\DC{-мн-е};    \\ \hline
                                & \Addition{0,11010 00011}
                                           {.,01001 .....}
                                           {1,00011 00011} & +\DC{9}=\DC{-мн-ль}; Рез-т!
        \end{tabular}
    }
\end{frame}

\begin{frame}
    \frametitle{II-способ: $-9\cdot -11$. $\DC{99}=\Number{,00011 00011}$}

    \resizebox{!}{.7\height}{
        \begin{tabular}{c|r|r|l}
                                                                       \hline\hline
            мн-ль $\rightarrow$ 
                                & \multicolumn{1}{|c|}{мн-е $\leftarrow$}       
                                                        & \multicolumn{1}{|c|}{СЧП}       
                                                                                      & прим. \\ \hline\hline
            \Number{,10111}     &                       & \Addition {,00000 00000} 
                                                                    {,01001 .....}
                                                                    {,01001 00000} & +\DC{9}=\DC{-мн-ль};\\ \hline\hline
            \NumberLo{,1011}{1} & \Number{,..... 10101} & \Addition {,01001 00000} 
                                                                    {,..... 10101}
                                                                    {,01001 10101} & +мн-е; сдвиг\\ \hline
            \NumberLo{,.101}{1} & \Number{,....1 0101.} & \Addition {,01001 10101} 
                                                                    {,....1 0101.}
                                                                    {,01010 11111} & +мн-е; сдвиг\\ \hline
            \NumberLo{,..10}{1} & \Number{,...10 101..} & \Addition {,01010 11111} 
                                                                    {,...10 101..}
                                                                    {,01101 10011} & +мн-е; сдвиг\\ \hline
            \NumberLo{,...1}{0} & \Number{,..101 01...} &                          & сдвиг\\ \hline
            \NumberLo{,....}{1} & \Number{,.1010 1....} & \Addition {,01101 10011} 
                                                                    {,.1010 1....}
                                                                    {,11000 00011} & +мн-е; Рез-т?\\ \hline\hline
                                & \Number{,10101 .....} & \Addition {,11000 00011} 
                                                                    {,01011 .....}
                                                                    {,00011 00011} & +\DC{11}=\DC{-мн-е}; Рез-т!
        \end{tabular}
    }
\end{frame}

\begin{frame}
    \frametitle{III-способ: $-11\cdot -9$. $\DC{99}=\Number{,00011 00011}$}

    \resizebox{!}{.7\height}{
        \begin{tabular}{c|r|l}
                                                                       \hline\hline
            мн-ль $\leftarrow$ 
                                   & \multicolumn{1}{|c|}{СЧП $\leftarrow$}       
                                                               & прим.      \\ \hline\hline
                                   & \Addition{,00000 00000}
                                              {,..... 01001}
                                              {,00000 01001} & +\DC{9}=\DC{-мн-е};\\ \hline
                                   & \Addition{,00000 01001}
                                              {,..... 01011}
                                              {,00000 10100} & +\DC{11}=\DC{-мн-ль};\\ \hline
                                   &   \Number{,00001 0100.} & сдвиг        \\ \hline\hline
            \NumberMid{,}{1}{0101} & \Addition{,00001 0100.}
                                              {,..... 10111}
                                              {,00001 11111} & +мн-е; сдвиг \\ \hline
            \NumberMid{,}{0}{101.} &   \Number{,00011 1111.} & сдвиг        \\ \hline
            \NumberMid{,}{1}{01..} & \Addition{,00111 111..}
                                              {,..... 10111}
                                              {,01000 10011} & +мн-е; сдвиг \\ \hline
            \NumberMid{,}{0}{1...} &   \Number{,10001 0011.} & сдвиг        \\ \hline
            \NumberMid{,}{1}{....} & \Addition{,00010 011..}
                                              {,..... 10111}
                                              {,00011 00011} & Рез-т!
        \end{tabular}
    }
\end{frame}

\begin{frame}
    \frametitle{IV-способ:: $-11\cdot -9$. $\DC{99}=\Number{,00011 00011}$}

    \resizebox{!}{.7\height}{
        \begin{tabular}{c|r|r|l}
                                                                       \hline\hline
            мн-ль $\leftarrow$ 
                                   & \multicolumn{1}{|c|}{мн-е $\rightarrow$}       
                                                           & \multicolumn{1}{|c|}{СЧП}       
                                                                                     & прим. \\ \hline\hline
                                                                                     
            \Number{,10101}        &                       & \Addition{,00000 00000} 
                                                                      {,01011 .....}
                                                                      {,01011 00000} & +\DC{11}=\DC{-мн-ль};\\ \hline
                                   & \Number{,10111 .....} & \Addition{,01011 00000} 
                                                                      {,01001 .....}
                                                                      {,10100 00000} & +\DC{9}=\DC{-мн-е};сдвиг\\ \hline\hline
            \NumberMid{,}{1}{0101} & \Number{,.1011 1....} & \Addition{,10100 00000} 
                                                                      {,.1011 1....}
                                                                      {,11111 10000} & +мн-е; сдвиг;\\ \hline
            \NumberMid{,}{0}{101.} & \Number{,..101 11...} &                         & сдвиг\\ \hline
            \NumberMid{,}{1}{01..} & \Number{,...10 111..} & \Addition{,11111 10000} 
                                                                      {,...10 111..}
                                                                      {,00010 01100} & +мн-е; сдвиг\\ \hline
            \NumberMid{,}{0}{1...} & \Number{,....1 0111.} &                         & сдвиг\\ \hline
            \NumberMid{,}{1}{....} & \Number{,..... 10111} & \Addition{,00010 01100} 
                                                                      {,..... 10111}
                                                                      {,00011 00011} & +мн-е; Рез-т\\ \hline
        \end{tabular}
    }
\end{frame}


\appendix


\section{Задания на практику}


\subsection{Проходное}

\begin{frame}
    \frametitle{\TaskSimpleNumber}
    Какая разрядность результата должна получиться, если дополнительные коды операндов занимают $n$ бит?
\end{frame}

\begin{frame}
    \frametitle{\TaskSimpleNumber}

    Перемножить числа:

    \begin{enumerate}
        \item $26$ и $-13$ I-м способом; 
        \item $-26$ и $13$ II-м способом; 
        \item $-26$ и $-13$ III-м способом; 
        \item $-13$ и $-26$ IV-м способом.
    \end{enumerate}
    
    Обосновать выбор масштаба.
\end{frame}

\begin{frame}
    \frametitle{\TaskSimpleNumber}
    Прорешать одним из методов <<краевые>> случаи в $n$-разрядной сетке:
    \begin{itemize}
        \item $-2^n\cdot -2^n$;
        \item $-2^n\cdot x$, где $x>0$;
        \item $(2^n-1)\cdot(2^n-1)$.
    \end{itemize}
\end{frame}

\begin{frame}
    \frametitle{\TaskSimpleNumber}
    
    Модифицируйте схему умножения первым способом с учетом работы в ДК (можно использовать условный блок <<получение ДК>> и мультиплексор):
    \begin{center}
        \includegraphics{fig/mult-methods.5}
    \end{center}
\end{frame}

\section{Самообучение}

\begin{frame}
    \frametitle{Советы самоучке}
    
    Рекомендуется почитать разделы посвященные работе с битами в \cite{bib:warren:algTriks}.
\end{frame}

\begin{frame}[allowframebreaks]{Библиография}
    \bibliographystyle{gost780u}
    \bibliography{./../../../bibliobase}
\end{frame}

\end{document}