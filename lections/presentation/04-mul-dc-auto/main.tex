%include part: see main.beamer.tex and main.article.tex
%include common packages and settings
\usepackage{etex} %эта магическая херь избавляет от переполнения регистров TeX а!!!

\mode<article>{\usepackage{fullpage}}
\mode<presentation>{
    \usetheme{Madrid} %%Boadilla,Madrid,AnnArbor,CambridgeUS,Malmoe,Singapore,Berlin
    \useoutertheme{shadow}
} 

\usepackage[utf8]{inputenc}
\usepackage[russian]{babel}
\usepackage{indentfirst}
\usepackage{graphicx}

\usepackage{amsmath}
\usepackage{amsfonts}
\usepackage{amsthm}
\usepackage{algorithm}
\usepackage{algorithmic}

\usepackage[all]{xy}

\date{Лекция по дисциплине <<информатика>>\\(\today)}
\author[М.~М.~Шихов]{Михаил Шихов \\ \texttt{\underline{m.m.shihov@gmail.com}}}

%для рисования графов пакетом xy-pic
\entrymodifiers={++[o][F-]}

%для псевдокода алгоритмов (algorithm,algorithmic)
\renewcommand{\algorithmicrequire}{\textbf{Вход:}}
\renewcommand{\algorithmicensure}{\textbf{Выход:}}
\renewcommand{\algorithmiccomment}[1]{// #1}
\floatname{algorithm}{Псевдокод}

%%определённые мной команды логической разметки
\newcommand{\Signs}[2]{\fbox{{\small{\underbar{#1}}}{#2}}}
\newcommand{\Sign}[1]{\fbox{#1}}
\newcommand{\DC}[1]{\text{ДК}(#1)}
\newcommand{\OC}[1]{\text{ОК}(#1)}
\newcommand{\PC}[1]{\text{ПК}(#1)}

\newcommand{\MyProc}{\text{$\mathbf{R}_8$}}

\newboolean{IsNeedAnswer}
\setboolean{IsNeedAnswer}{false} %true/false

\newcommand{\Machine}[1]{\texttt{#1}}
\newcommand{\Opcode}[1]{\texttt{\bf{#1}}}
\newcommand{\Operand}[1]{\texttt{#1}}
\newcommand{\CmdOneAddr}[2]{\text{\Opcode{#1} \Operand{#2}}}
\newcommand{\CmdTwoAddr}[3]{\text{\Opcode{#1} \Operand{#2}, \Operand{#3}}}
\newcommand{\CmdThreeAddr}[4]{\text{\Opcode{#1} {\Operand{#2}}, \Operand{#3}, \Operand{#4}}}

\newcommand{\ProofAnswer}[1]{
    \ifthenelse{\boolean{IsNeedAnswer}}{
        \begin{proof}[Ответ]
            #1
        \end{proof}
    }{}
}

\newcommand{\LabeledAnswer}[1]{
    \ifthenelse{\boolean{IsNeedAnswer}}{
        \emph{Отв.}: #1 \qed
    }{}
}

\newcommand{\PlainAnswer}[1]{
    \ifthenelse{\boolean{IsNeedAnswer}}{#1}{}
}

\newcommand{\UnsignedAny}[2]{\text{\upshape
    \begin{tabular}{lr}
        \tiny{#1} & \tiny{0}\\ 
        \hline
        \multicolumn{2}{|c|}{\texttt{#2}} \\ 
        \hline
    \end{tabular}
}}

\newcommand{\UnsignedByte}[1]{\UnsignedAny{7}{#1}}

\newcommand{\UnsignedTwoBytes}[1]{\UnsignedAny{15}{#1}}

\newcommand{\FixedAny}[4]{\text{\upshape
    \begin{tabular}{clr}
        \tiny{#1}  &\tiny{#2} & \tiny{0}\\ 
        \hline
        \multicolumn{1}{|c|}{\texttt{#3}} & \multicolumn{2}{|c|}{\texttt{#4}} \\ 
        \hline
    \end{tabular}
}}

\newcommand{\FixedByte}[2]{\FixedAny{7}{6}{#1}{#2}}

\newcommand{\FixedTwoBytes}[2]{\FixedAny{15}{14}{#1}{#2}}

\newcommand{\SignedAny}[4]{\text{\upshape
    \begin{tabular}{clr}
        \tiny{#1}  &\tiny{#2} & \tiny{0}\\ 
        \hline
        \multicolumn{1}{|c|}{\Machine{#3}} & \multicolumn{2}{|c|}{\Machine{#4}} \\ 
        \hline
    \end{tabular}
}}

\newcommand{\SignedNibble}[2]{\SignedAny{3}{2}{#1}{#2}}

\newcommand{\SignedByte}[2]{\SignedAny{7}{6}{#1}{#2}}

\newcommand{\SignedTwoBytes}[2]{\SignedAny{15}{14}{#1}{#2}}

\newcommand{\FloatMyHex}[4]{\text{\upshape
    \begin{tabular}{clrclr}
        \tiny{15}  &\tiny{14} & \tiny{6} & \tiny{5} & \tiny{4} & \tiny{0}\\ 
        \hline
        \multicolumn{1}{|c|}{\texttt{#1}} 
            & \multicolumn{2}{|c|}{\texttt{#2}} 
                & \multicolumn{1}{|c|}{\texttt{#3}} 
                    & \multicolumn{2}{|c|}{\texttt{#4}} \\ 
        \hline
    \end{tabular}
}}

\newcommand{\FloatMyCharHex}[3]{\text{\upshape
    \begin{tabular}{clrlr}
        \tiny{15}  &\tiny{14} & \tiny{6} & \tiny{5} & \tiny{0}\\ 
        \hline
        \multicolumn{1}{|c|}{\texttt{#1}} 
            & \multicolumn{2}{|c|}{\texttt{#2}} 
                & \multicolumn{2}{|c|}{\texttt{#3}} \\ 
        \hline
    \end{tabular}
}}

\newcommand{\FloatMyDcMantCharHex}[2]{\text{\upshape
    \begin{tabular}{lrlr}
        \tiny{15} & \tiny{6} & \tiny{5} & \tiny{0}\\ 
        \hline
        \multicolumn{2}{|c|}{\texttt{#1}} 
            & \multicolumn{2}{|c|}{\texttt{#2}} \\ 
        \hline
    \end{tabular}
}}

\newcommand{\FloatESShort}[3]{\text{\upshape
    \begin{tabular}{clrlr}
        \tiny{31}  &\tiny{30} & \tiny{24} & \tiny{23} & \tiny{0}\\ 
        \hline
        \multicolumn{1}{|c|}{\texttt{#1}} 
            & \multicolumn{2}{|c|}{\texttt{#2}} 
                & \multicolumn{2}{|c|}{\texttt{#3}} 
                    \\ 
        \hline
    \end{tabular}
}}

\newcommand{\FloatPCShort}[3]{\text{\upshape
    \begin{tabular}{clrlr}
        \tiny{31}  &\tiny{30} & \tiny{23} & \tiny{22} & \tiny{0}\\ 
        \hline
        \multicolumn{1}{|c|}{\texttt{#1}} 
            & \multicolumn{2}{|c|}{\texttt{#2}} 
                & \multicolumn{2}{|c|}{\texttt{#3}} 
                    \\ 
        \hline
    \end{tabular}
}}

\newcommand{\FloatMyOrderX}[4]{\text{\upshape
    \begin{tabular}{clrclr}
        \tiny{9}  &\tiny{8} & \tiny{4} & \tiny{3} & \tiny{2} & \tiny{0}\\ 
        \hline
        \multicolumn{1}{|c|}{\texttt{#1}} 
            & \multicolumn{2}{|c|}{\texttt{#2}} 
                & \multicolumn{1}{|c|}{\texttt{#3}} 
                    & \multicolumn{2}{|c|}{\texttt{#4}} \\ 
        \hline
    \end{tabular}
}}

\newcommand{\FloatMyCharX}[3]{\text{\upshape
    \begin{tabular}{clrlr}
        \tiny{9}  &\tiny{8} & \tiny{4} & \tiny{3} & \tiny{0}\\ 
        \hline
        \multicolumn{1}{|c|}{\texttt{#1}} 
            & \multicolumn{2}{|c|}{\texttt{#2}} 
                & \multicolumn{2}{|c|}{\texttt{#3}} \\ 
        \hline
    \end{tabular}
}}

\newcommand{\FloatMyDcOrderX}[3]{\text{\upshape
    \begin{tabular}{lrclr}
        \tiny{9} & \tiny{4} & \tiny{3} & \tiny{2} & \tiny{0}\\ 
        \hline
        \multicolumn{2}{|c|}{\texttt{#1}} 
            & \multicolumn{1}{|c|}{\texttt{#2}} 
                & \multicolumn{2}{|c|}{\texttt{#3}} \\ 
        \hline
    \end{tabular}
}}

\newcommand{\FloatMyDcCharX}[2]{\text{\upshape
    \begin{tabular}{lrlr}
        \tiny{9} & \tiny{4} & \tiny{3} & \tiny{0}\\ 
        \hline
        \multicolumn{2}{|c|}{\texttt{#1}} 
            & \multicolumn{2}{|c|}{\texttt{#2}} \\ 
        \hline
    \end{tabular}
}}


%--- СПЕЦИФИЧНЫЕ ДЛЯ УМНОЖЕНИЯ КОМАНДЫ ---------------------------------------------------------------------------------------------


\newcommand{\Number}[1]{
    \texttt{#1}
}

\newcommand{\NumberHi}[2]{
    \underline{\underline{\texttt{#1}}}\texttt{#2}
}

\newcommand{\NumberMid}[3]{
    \texttt{#1}\underline{\underline{\texttt{#2}}}\texttt{#3}
}

\newcommand{\NumberLo}[2]{
    \texttt{#1}\underline{\underline{\texttt{#2}}}
}

\newcommand{\Stack}[2]{
    \begin{tabular}[t]{@{}r@{}}
        {#1}\\ \hline
        {#2}\\ 
    \end{tabular}
}

\newcommand{\StackThree}[3]{
    \begin{tabular}[t]{@{}r@{}}
        {#1}\\ \hline
        {#2}\\ \hline
        {#3}\\
    \end{tabular}
}

\newcommand{\StackFour}[4]{
    \begin{tabular}[t]{@{}r@{}}
        {#1}\\ \hline
        {#2}\\ \hline
        {#3}\\ \hline
        {#4}\\
    \end{tabular}
}

\newcommand{\Operation}[4]{
    \begin{tabular}[t]{@{}r@{}}
        \texttt{#4}
        \begin{tabular}{@{}r@{}}
            \Number{#1}\\
            \Number{#2}\\ \hline
        \end{tabular} \\ 
        \Number{#3}\\
    \end{tabular}
}

\newcommand{\Addition}[3]{\Operation{#1}{#2}{#3}{+}}

\newcommand{\Subtraction}[3]{\Operation{#1}{#2}{#3}{-}}

\newcommand{\Register}[2]{\Number{#1:#2}}

\newcommand{\Mantiss}{m}
\newcommand{\Order}{p}
\newcommand{\Char}{c}

\newcommand{\MantissOf}[1]{\Mantiss_{#1}}
\newcommand{\OrderOf}[1]{\Order_{#1}}
\newcommand{\CharOf}[1]{\Char_{#1}}

\newcommand{\FloatExpression}[2]{\MantissOf{#1}\cdot {#2}^{\OrderOf{#1}}}

\newcommand{\DivAnswer}[2]{(\texttt{$#1$ rem $#2$})}

\newenvironment{Solve}[1]%
    {\begin{proof}[Решение]#1}
    {\end{proof}}
    
    
%определённые мной команды логической разметки
\newcommand{\Signs}[2]{\fbox{{\small{\underbar{#1}}}{#2}}}
\newcommand{\Sign}[1]{\fbox{#1}}
\newcommand{\DC}[1]{\text{ДК}(#1)}
\newcommand{\OC}[1]{\text{ОК}(#1)}
\newcommand{\PC}[1]{\text{ПК}(#1)}

\newcommand{\MyProc}{\text{$\mathbf{R}_8$}}

\newboolean{IsNeedAnswer}
\setboolean{IsNeedAnswer}{false} %true/false

\newcommand{\Machine}[1]{\texttt{#1}}
\newcommand{\Opcode}[1]{\texttt{\bf{#1}}}
\newcommand{\Operand}[1]{\texttt{#1}}
\newcommand{\CmdOneAddr}[2]{\text{\Opcode{#1} \Operand{#2}}}
\newcommand{\CmdTwoAddr}[3]{\text{\Opcode{#1} \Operand{#2}, \Operand{#3}}}
\newcommand{\CmdThreeAddr}[4]{\text{\Opcode{#1} {\Operand{#2}}, \Operand{#3}, \Operand{#4}}}

\newcommand{\ProofAnswer}[1]{
    \ifthenelse{\boolean{IsNeedAnswer}}{
        \begin{proof}[Ответ]
            #1
        \end{proof}
    }{}
}

\newcommand{\LabeledAnswer}[1]{
    \ifthenelse{\boolean{IsNeedAnswer}}{
        \emph{Отв.}: #1 \qed
    }{}
}

\newcommand{\PlainAnswer}[1]{
    \ifthenelse{\boolean{IsNeedAnswer}}{#1}{}
}

\newcommand{\UnsignedAny}[2]{\text{\upshape
    \begin{tabular}{lr}
        \tiny{#1} & \tiny{0}\\ 
        \hline
        \multicolumn{2}{|c|}{\texttt{#2}} \\ 
        \hline
    \end{tabular}
}}

\newcommand{\UnsignedByte}[1]{\UnsignedAny{7}{#1}}

\newcommand{\UnsignedTwoBytes}[1]{\UnsignedAny{15}{#1}}

\newcommand{\FixedAny}[4]{\text{\upshape
    \begin{tabular}{clr}
        \tiny{#1}  &\tiny{#2} & \tiny{0}\\ 
        \hline
        \multicolumn{1}{|c|}{\texttt{#3}} & \multicolumn{2}{|c|}{\texttt{#4}} \\ 
        \hline
    \end{tabular}
}}

\newcommand{\FixedByte}[2]{\FixedAny{7}{6}{#1}{#2}}

\newcommand{\FixedTwoBytes}[2]{\FixedAny{15}{14}{#1}{#2}}

\newcommand{\SignedAny}[4]{\text{\upshape
    \begin{tabular}{clr}
        \tiny{#1}  &\tiny{#2} & \tiny{0}\\ 
        \hline
        \multicolumn{1}{|c|}{\Machine{#3}} & \multicolumn{2}{|c|}{\Machine{#4}} \\ 
        \hline
    \end{tabular}
}}

\newcommand{\SignedNibble}[2]{\SignedAny{3}{2}{#1}{#2}}

\newcommand{\SignedByte}[2]{\SignedAny{7}{6}{#1}{#2}}

\newcommand{\SignedTwoBytes}[2]{\SignedAny{15}{14}{#1}{#2}}

\newcommand{\FloatMyHex}[4]{\text{\upshape
    \begin{tabular}{clrclr}
        \tiny{15}  &\tiny{14} & \tiny{6} & \tiny{5} & \tiny{4} & \tiny{0}\\ 
        \hline
        \multicolumn{1}{|c|}{\texttt{#1}} 
            & \multicolumn{2}{|c|}{\texttt{#2}} 
                & \multicolumn{1}{|c|}{\texttt{#3}} 
                    & \multicolumn{2}{|c|}{\texttt{#4}} \\ 
        \hline
    \end{tabular}
}}

\newcommand{\FloatMyCharHex}[3]{\text{\upshape
    \begin{tabular}{clrlr}
        \tiny{15}  &\tiny{14} & \tiny{6} & \tiny{5} & \tiny{0}\\ 
        \hline
        \multicolumn{1}{|c|}{\texttt{#1}} 
            & \multicolumn{2}{|c|}{\texttt{#2}} 
                & \multicolumn{2}{|c|}{\texttt{#3}} \\ 
        \hline
    \end{tabular}
}}

\newcommand{\FloatMyDcMantCharHex}[2]{\text{\upshape
    \begin{tabular}{lrlr}
        \tiny{15} & \tiny{6} & \tiny{5} & \tiny{0}\\ 
        \hline
        \multicolumn{2}{|c|}{\texttt{#1}} 
            & \multicolumn{2}{|c|}{\texttt{#2}} \\ 
        \hline
    \end{tabular}
}}

\newcommand{\FloatESShort}[3]{\text{\upshape
    \begin{tabular}{clrlr}
        \tiny{31}  &\tiny{30} & \tiny{24} & \tiny{23} & \tiny{0}\\ 
        \hline
        \multicolumn{1}{|c|}{\texttt{#1}} 
            & \multicolumn{2}{|c|}{\texttt{#2}} 
                & \multicolumn{2}{|c|}{\texttt{#3}} 
                    \\ 
        \hline
    \end{tabular}
}}

\newcommand{\FloatPCShort}[3]{\text{\upshape
    \begin{tabular}{clrlr}
        \tiny{31}  &\tiny{30} & \tiny{23} & \tiny{22} & \tiny{0}\\ 
        \hline
        \multicolumn{1}{|c|}{\texttt{#1}} 
            & \multicolumn{2}{|c|}{\texttt{#2}} 
                & \multicolumn{2}{|c|}{\texttt{#3}} 
                    \\ 
        \hline
    \end{tabular}
}}

\newcommand{\FloatMyOrderX}[4]{\text{\upshape
    \begin{tabular}{clrclr}
        \tiny{9}  &\tiny{8} & \tiny{4} & \tiny{3} & \tiny{2} & \tiny{0}\\ 
        \hline
        \multicolumn{1}{|c|}{\texttt{#1}} 
            & \multicolumn{2}{|c|}{\texttt{#2}} 
                & \multicolumn{1}{|c|}{\texttt{#3}} 
                    & \multicolumn{2}{|c|}{\texttt{#4}} \\ 
        \hline
    \end{tabular}
}}

\newcommand{\FloatMyCharX}[3]{\text{\upshape
    \begin{tabular}{clrlr}
        \tiny{9}  &\tiny{8} & \tiny{4} & \tiny{3} & \tiny{0}\\ 
        \hline
        \multicolumn{1}{|c|}{\texttt{#1}} 
            & \multicolumn{2}{|c|}{\texttt{#2}} 
                & \multicolumn{2}{|c|}{\texttt{#3}} \\ 
        \hline
    \end{tabular}
}}

\newcommand{\FloatMyDcOrderX}[3]{\text{\upshape
    \begin{tabular}{lrclr}
        \tiny{9} & \tiny{4} & \tiny{3} & \tiny{2} & \tiny{0}\\ 
        \hline
        \multicolumn{2}{|c|}{\texttt{#1}} 
            & \multicolumn{1}{|c|}{\texttt{#2}} 
                & \multicolumn{2}{|c|}{\texttt{#3}} \\ 
        \hline
    \end{tabular}
}}

\newcommand{\FloatMyDcCharX}[2]{\text{\upshape
    \begin{tabular}{lrlr}
        \tiny{9} & \tiny{4} & \tiny{3} & \tiny{0}\\ 
        \hline
        \multicolumn{2}{|c|}{\texttt{#1}} 
            & \multicolumn{2}{|c|}{\texttt{#2}} \\ 
        \hline
    \end{tabular}
}}


%--- СПЕЦИФИЧНЫЕ ДЛЯ УМНОЖЕНИЯ КОМАНДЫ ---------------------------------------------------------------------------------------------


\newcommand{\Number}[1]{
    \texttt{#1}
}

\newcommand{\NumberHi}[2]{
    \underline{\underline{\texttt{#1}}}\texttt{#2}
}

\newcommand{\NumberMid}[3]{
    \texttt{#1}\underline{\underline{\texttt{#2}}}\texttt{#3}
}

\newcommand{\NumberLo}[2]{
    \texttt{#1}\underline{\underline{\texttt{#2}}}
}

\newcommand{\Stack}[2]{
    \begin{tabular}[t]{@{}r@{}}
        {#1}\\ \hline
        {#2}\\ 
    \end{tabular}
}

\newcommand{\StackThree}[3]{
    \begin{tabular}[t]{@{}r@{}}
        {#1}\\ \hline
        {#2}\\ \hline
        {#3}\\
    \end{tabular}
}

\newcommand{\StackFour}[4]{
    \begin{tabular}[t]{@{}r@{}}
        {#1}\\ \hline
        {#2}\\ \hline
        {#3}\\ \hline
        {#4}\\
    \end{tabular}
}

\newcommand{\Operation}[4]{
    \begin{tabular}[t]{@{}r@{}}
        \texttt{#4}
        \begin{tabular}{@{}r@{}}
            \Number{#1}\\
            \Number{#2}\\ \hline
        \end{tabular} \\ 
        \Number{#3}\\
    \end{tabular}
}

\newcommand{\Addition}[3]{\Operation{#1}{#2}{#3}{+}}

\newcommand{\Subtraction}[3]{\Operation{#1}{#2}{#3}{-}}

\newcommand{\Register}[2]{\Number{#1:#2}}

\newcommand{\Mantiss}{m}
\newcommand{\Order}{p}
\newcommand{\Char}{c}

\newcommand{\MantissOf}[1]{\Mantiss_{#1}}
\newcommand{\OrderOf}[1]{\Order_{#1}}
\newcommand{\CharOf}[1]{\Char_{#1}}

\newcommand{\FloatExpression}[2]{\MantissOf{#1}\cdot {#2}^{\OrderOf{#1}}}

\newcommand{\DivAnswer}[2]{(\texttt{$#1$ rem $#2$})}

\newenvironment{Solve}[1]%
    {\begin{proof}[Решение]#1}
    {\end{proof}}
    
    

\title[Умножение в ДК (АК)]{Умножение в дополнительном коде\\(с автоматической коррекцией)}

\newcounter{TaskSimpleCtr}
\setcounter{TaskSimpleCtr}{1}
\newcommand{\TaskSimpleNumber}{ \arabic{TaskSimpleCtr}) \addtocounter{TaskSimpleCtr}{1} }

%вставка изображений из metapost (post script)
\DeclareGraphicsRule{*}{mps}{*}{}

\begin{document}

\mode<article>{\maketitle\tableofcontents}
\frame<presentation>{\titlepage}
\begin{frame}<presentation>
    \frametitle{Содержание}
    \tableofcontents
\end{frame}


\section{Обоснование корректности}


\subsection{Структурная точка зрения на число}


\begin{frame}
    \frametitle{Точка зрения на число}

    Положительное двоичное число будем рассматривать как 
    \begin{block}{}
        последовательность непрерывных групп единиц, разделённых непрерывными группами нулей. 
    \end{block}
    
    Минимальная длина группы при этом будет равна одному символу.

    \begin{block}{}
        Рассмотрим группу:
        \[
            \Number{...0001111100...}
        \]
    \end{block}
    
    Соответствующее ей число можно получить как разность:
    
    \begin{block}{}
        \[
            \Subtraction{...0010000000...}
                        {...0000000100...}
                        {...0001111100...}
        \]
    \end{block}
\end{frame}

\begin{frame}
    \frametitle{Вклады разрядов}

    \[
        \Subtraction{...0010000000...}
                    {...0000000100...}
                    {...0001111100...}
    \]
        
    \begin{block}{}
        Ненулевой вклад вносят только \emph{переходы из группы в группу}: 
        \[\text{\Number{01} и \Number{10}}\]
    \end{block}
    
    Обозначим пару разрядов, образующих переход $(a_i,a_{i-1})$. Тогда:
    
    \begin{block}{}
        \begin{itemize}
            \item переход $(a_i,a_{i-1})=\Number{01}$ вносит положительный вклад $2^i$, 
            \item переход $(a_i,a_{i-1})=\Number{10}$ вносит отрицательный вклад $({-2^i})$.
        \end{itemize}
    \end{block}
\end{frame}


\subsection{Не нужна коррекция}

\begin{frame}
    \frametitle{Отрицательное число}
    
    Например, представление положительного числа $118$:
    \[
        \Number{...0001110110}
    \] 
    
    Две группы вносят следующие вклады:
    \begin{itemize}
        \item первая группа вносит: $-2^1$ и $2^3$;
        \item вторая: $-2^4$ и $2^7$.
        \item \emph{итого:} $(2^7-2^4+2^3-2^1)=(128-16+8-2)=118.$
    \end{itemize}
\end{frame}

\begin{frame}
    \frametitle{Отрицательное число? Дополнительный код}
    
    Инверсия разрядов числа $118$:
    \[
        \Number{...0001110110}
    \] 
    
    не приводит к результату с противоположным знаком:
    \[
        \Number{...1110001001}
    \] 
    
    \begin{itemize}
        \item Ожидалось: $(-2^7+2^4-2^3+2^1)=(-128+16-8+2)=-118.$
        \item Но?: $(-2^7+2^4-2^3+2^1-2^0)=(-128+16-8+2-1)=-119$.
    \end{itemize}
    
    Корректируем на единицу:
    \[
        \Number{...1110001010}
    \] 
    
    Получаем: $(-2^7+2^4-2^3+2^2-2^1)=(-128+16-8+4-2)=-118$.
\end{frame}


\section{Автоматичесая коррекция}


\subsection{Технические ограничения}


\subsection{Примеры}

\begin{frame}
    \frametitle{Операнды для примеров}

    В качестве примера будем перемножать числа $9$ и $11$ с различными комбинациями знаков. 
    
    Выбрав масштаб $M=2^5$, получим следующие представления:
    
    \begin{align*}
        \DC{9}   & = \Number{,01001}\\
        \DC{-9}  & = \Number{,10111}\\
        \DC{11}  & = \Number{,01011}\\
        \DC{-11} & = \Number{,10101}
    \end{align*}
\end{frame}

\begin{frame}
    \frametitle{I-способ: $-9\cdot 11$. $\DC{-99}=\Number{,11100 11101}$}

    \begin{tabular}{c|r|l}
        \hline\hline
        мн-ль $\rightarrow$ & 
                                \multicolumn{1}{|c|}{СЧП $\rightarrow$}       
                                                        & прим. \\ 
        \hline\hline
        \NumberLo{,1011}{1|0} & \Addition{,00000 00000}
                                         {,11010 1....}
                                         {,11010 10000} & -мн-е/2; сдвиг; \\ \hline
        \NumberLo{,.101}{1|1} &   \Number{,11101 01000} & сдвиг; \\ \hline
        \NumberLo{,..10}{1|1} &   \Number{,11110 10100} & сдвиг; \\ \hline
        \NumberLo{,...1}{0|1} & \Addition{,11111 01010}
                                         {,.0101 1....}
                                         {,00100 11010} & +мн-е/2; сдвиг; \\ \hline
        \NumberLo{,....}{1|0} & \Addition{,00010 01101}
                                         {,11010 1....}
                                         {,11100 11101} & -мн-е/2; Рез-т!
    \end{tabular}
\end{frame}

\begin{frame}
    \frametitle{II-способ: $-9\cdot -11$. $\DC{99}=\Number{,00011 00011}$}

    \begin{tabular}{c|r|r|l}
                                                                   \hline\hline
        мн-ль $\rightarrow$ 
                            & \multicolumn{1}{|c|}{мн-е $\leftarrow$}       
                                                    & \multicolumn{1}{|c|}{СЧП}       
                                                                                  & прим. \\ \hline\hline
        \NumberLo{,1011}{1|0} & \Number{,11111 10101} & \Addition {,00000 00000} 
                                                                  {,00000 01011}
                                                                  {,00000 01011} & -мн-е; сдвиг;\\ \hline
        \NumberLo{,.101}{1|1} & \Number{,11111 0101.} &                          & сдвиг;\\ \hline
        \NumberLo{,..10}{1|1} & \Number{,11110 101..} &                          & сдвиг;\\ \hline
        \NumberLo{,...1}{0|1} & \Number{,11101 01...} & \Addition {,00000 01011} 
                                                                  {,11101 01...}
                                                                  {,11101 10011} & +мн-е; сдвиг;\\ \hline
        \NumberLo{,....}{1|0} & \Number{,11010 1....} & \Addition {,11101 10011} 
                                                                  {,00101 1....}
                                                                  {,00011 00011} & -мн-е; Рез-т!\\ 
    \end{tabular}
\end{frame}

\begin{frame}
    \frametitle{III-способ: $11\cdot -9$. $\DC{-99}=\Number{,11100 11101}$}

    \begin{tabular}{c|r|l}
                                                                   \hline\hline
        мн-ль $\leftarrow$ 
                               & \multicolumn{1}{|c|}{СЧП $\leftarrow$}       
                                                           & прим.      \\ \hline\hline
        \NumberMid{,}{01}{011} & \Addition{,00000 00000}
                                          {,11111 10111}
                                          {,11111 10111} & +мн-е; сдвиг; \\ \hline
        \NumberMid{,}{10}{11.} & \Addition{,11111 0111.}
                                          {,..... 01001}
                                          {,11111 10111} & -мн-е; сдвиг; \\ \hline
        \NumberMid{,}{01}{1..} & \Addition{,11111 0111.}
                                          {,11111 10111}
                                          {,11111 00101} & +мн-е; сдвиг; \\ \hline
        \NumberMid{,}{11}{...} &   \Number{,11110 0101.} &        сдвиг; \\ \hline
        \NumberMid{,}{1.}{...} & \Addition{,11100 101..}
                                          {,..... 01001}
                                          {,11100 11101} & -мн-е; Рез-т!
    \end{tabular}
\end{frame}

\begin{frame}
    \frametitle{IV-способ:: $11\cdot 9$. $\DC{99}=\Number{,00011 00011}$}

    \begin{tabular}{c|r|r|l}
                                                                   \hline\hline
        мн-ль $\leftarrow$ 
                               & \multicolumn{1}{|c|}{мн-е $\rightarrow$}       
                                                       & \multicolumn{1}{|c|}{СЧП}       
                                                                                 & прим. \\ \hline\hline
        \NumberMid{,}{01}{011} & \Number{,.0100 1....} & \Addition{,00000 00000} 
                                                                  {,.0100 1....}
                                                                  {,00100 10000} & +мн-е; сдвиг;\\ \hline
        \NumberMid{,}{10}{11.} & \Number{,..010 01...} & \Addition{,00100 10000} 
                                                                  {,11101 11...}
                                                                  {,00010 01000} & -мн-е; сдвиг;\\ \hline
        \NumberMid{,}{01}{1..} & \Number{,...01 001..} & \Addition{,00010 01000} 
                                                                  {,...01 001..}
                                                                  {,00011 01100} & +мн-е; сдвиг\\ \hline
        \NumberMid{,}{11}{...} & \Number{,....0 1001.} &                         &        сдвиг;\\ \hline
        \NumberMid{,}{1.}{...} & \Number{,..... 01001} & \Addition{,00011 01100} 
                                                                  {,11111 10111}
                                                                  {,00011 00011} & -мн-е; Рез-т!\\ \hline
    \end{tabular}
\end{frame}


\subsection{Заключение}


\appendix


\section{Задания на практику}


\subsection{Проходное}

\begin{frame}
    \frametitle{\TaskSimpleNumber}
    Какая разрядность результата должна получиться, если дополнительные коды операндов занимают $n$ бит?
\end{frame}

\begin{frame}
    \frametitle{\TaskSimpleNumber}

    Перемножить числа:

    \begin{enumerate}
        \item $26$ и $-13$ I-м способом; 
        \item $-26$ и $13$ II-м способом; 
        \item $-26$ и $-13$ III-м способом; 
        \item $-13$ и $-26$ IV-м способом.
    \end{enumerate}
    
    Обосновать выбор масштаба.
\end{frame}

\begin{frame}
    \frametitle{\TaskSimpleNumber}
    Прорешать одним из методов <<краевые>> случаи в $n$-разрядной сетке:
    \begin{itemize}
        \item $-2^n\cdot -2^n$;
        \item $-2^n\cdot x$, где $x>0$;
        \item $(2^n-1)\cdot(2^n-1)$.
    \end{itemize}
\end{frame}

\begin{frame}
    \frametitle{\TaskSimpleNumber}
    
\end{frame}

\section{Самообучение}

\begin{frame}
    \frametitle{Советы самоучке}
    
    Рекомендуется почитать разделы посвященные работе с битами в \cite{bib:warren:algTriks}.
\end{frame}

\begin{frame}[allowframebreaks]{Библиография}
    \bibliographystyle{gost780u}
    \bibliography{./../../../bibliobase}
\end{frame}

\end{document}