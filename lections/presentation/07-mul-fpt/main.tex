%include part: see main.beamer.tex and main.article.tex
%include common packages and settings
\usepackage{etex} %эта магическая херь избавляет от переполнения регистров TeX а!!!

\mode<article>{\usepackage{fullpage}}
\mode<presentation>{
    \usetheme{Madrid} %%Boadilla,Madrid,AnnArbor,CambridgeUS,Malmoe,Singapore,Berlin
    \useoutertheme{shadow}
} 

\usepackage[utf8]{inputenc}
\usepackage[russian]{babel}
\usepackage{indentfirst}
\usepackage{graphicx}

\usepackage{amsmath}
\usepackage{amsfonts}
\usepackage{amsthm}
\usepackage{algorithm}
\usepackage{algorithmic}

\usepackage[all]{xy}

\date{Лекция по дисциплине <<информатика>>\\(\today)}
\author[М.~М.~Шихов]{Михаил Шихов \\ \texttt{\underline{m.m.shihov@gmail.com}}}

%для рисования графов пакетом xy-pic
\entrymodifiers={++[o][F-]}

%для псевдокода алгоритмов (algorithm,algorithmic)
\renewcommand{\algorithmicrequire}{\textbf{Вход:}}
\renewcommand{\algorithmicensure}{\textbf{Выход:}}
\renewcommand{\algorithmiccomment}[1]{// #1}
\floatname{algorithm}{Псевдокод}

%%определённые мной команды логической разметки
\newcommand{\Signs}[2]{\fbox{{\small{\underbar{#1}}}{#2}}}
\newcommand{\Sign}[1]{\fbox{#1}}
\newcommand{\DC}[1]{\text{ДК}(#1)}
\newcommand{\OC}[1]{\text{ОК}(#1)}
\newcommand{\PC}[1]{\text{ПК}(#1)}

\newcommand{\MyProc}{\text{$\mathbf{R}_8$}}

\newboolean{IsNeedAnswer}
\setboolean{IsNeedAnswer}{false} %true/false

\newcommand{\Machine}[1]{\texttt{#1}}
\newcommand{\Opcode}[1]{\texttt{\bf{#1}}}
\newcommand{\Operand}[1]{\texttt{#1}}
\newcommand{\CmdOneAddr}[2]{\text{\Opcode{#1} \Operand{#2}}}
\newcommand{\CmdTwoAddr}[3]{\text{\Opcode{#1} \Operand{#2}, \Operand{#3}}}
\newcommand{\CmdThreeAddr}[4]{\text{\Opcode{#1} {\Operand{#2}}, \Operand{#3}, \Operand{#4}}}

\newcommand{\ProofAnswer}[1]{
    \ifthenelse{\boolean{IsNeedAnswer}}{
        \begin{proof}[Ответ]
            #1
        \end{proof}
    }{}
}

\newcommand{\LabeledAnswer}[1]{
    \ifthenelse{\boolean{IsNeedAnswer}}{
        \emph{Отв.}: #1 \qed
    }{}
}

\newcommand{\PlainAnswer}[1]{
    \ifthenelse{\boolean{IsNeedAnswer}}{#1}{}
}

\newcommand{\UnsignedAny}[2]{\text{\upshape
    \begin{tabular}{lr}
        \tiny{#1} & \tiny{0}\\ 
        \hline
        \multicolumn{2}{|c|}{\texttt{#2}} \\ 
        \hline
    \end{tabular}
}}

\newcommand{\UnsignedByte}[1]{\UnsignedAny{7}{#1}}

\newcommand{\UnsignedTwoBytes}[1]{\UnsignedAny{15}{#1}}

\newcommand{\FixedAny}[4]{\text{\upshape
    \begin{tabular}{clr}
        \tiny{#1}  &\tiny{#2} & \tiny{0}\\ 
        \hline
        \multicolumn{1}{|c|}{\texttt{#3}} & \multicolumn{2}{|c|}{\texttt{#4}} \\ 
        \hline
    \end{tabular}
}}

\newcommand{\FixedByte}[2]{\FixedAny{7}{6}{#1}{#2}}

\newcommand{\FixedTwoBytes}[2]{\FixedAny{15}{14}{#1}{#2}}

\newcommand{\SignedAny}[4]{\text{\upshape
    \begin{tabular}{clr}
        \tiny{#1}  &\tiny{#2} & \tiny{0}\\ 
        \hline
        \multicolumn{1}{|c|}{\Machine{#3}} & \multicolumn{2}{|c|}{\Machine{#4}} \\ 
        \hline
    \end{tabular}
}}

\newcommand{\SignedNibble}[2]{\SignedAny{3}{2}{#1}{#2}}

\newcommand{\SignedByte}[2]{\SignedAny{7}{6}{#1}{#2}}

\newcommand{\SignedTwoBytes}[2]{\SignedAny{15}{14}{#1}{#2}}

\newcommand{\FloatMyHex}[4]{\text{\upshape
    \begin{tabular}{clrclr}
        \tiny{15}  &\tiny{14} & \tiny{6} & \tiny{5} & \tiny{4} & \tiny{0}\\ 
        \hline
        \multicolumn{1}{|c|}{\texttt{#1}} 
            & \multicolumn{2}{|c|}{\texttt{#2}} 
                & \multicolumn{1}{|c|}{\texttt{#3}} 
                    & \multicolumn{2}{|c|}{\texttt{#4}} \\ 
        \hline
    \end{tabular}
}}

\newcommand{\FloatMyCharHex}[3]{\text{\upshape
    \begin{tabular}{clrlr}
        \tiny{15}  &\tiny{14} & \tiny{6} & \tiny{5} & \tiny{0}\\ 
        \hline
        \multicolumn{1}{|c|}{\texttt{#1}} 
            & \multicolumn{2}{|c|}{\texttt{#2}} 
                & \multicolumn{2}{|c|}{\texttt{#3}} \\ 
        \hline
    \end{tabular}
}}

\newcommand{\FloatMyDcMantCharHex}[2]{\text{\upshape
    \begin{tabular}{lrlr}
        \tiny{15} & \tiny{6} & \tiny{5} & \tiny{0}\\ 
        \hline
        \multicolumn{2}{|c|}{\texttt{#1}} 
            & \multicolumn{2}{|c|}{\texttt{#2}} \\ 
        \hline
    \end{tabular}
}}

\newcommand{\FloatESShort}[3]{\text{\upshape
    \begin{tabular}{clrlr}
        \tiny{31}  &\tiny{30} & \tiny{24} & \tiny{23} & \tiny{0}\\ 
        \hline
        \multicolumn{1}{|c|}{\texttt{#1}} 
            & \multicolumn{2}{|c|}{\texttt{#2}} 
                & \multicolumn{2}{|c|}{\texttt{#3}} 
                    \\ 
        \hline
    \end{tabular}
}}

\newcommand{\FloatPCShort}[3]{\text{\upshape
    \begin{tabular}{clrlr}
        \tiny{31}  &\tiny{30} & \tiny{23} & \tiny{22} & \tiny{0}\\ 
        \hline
        \multicolumn{1}{|c|}{\texttt{#1}} 
            & \multicolumn{2}{|c|}{\texttt{#2}} 
                & \multicolumn{2}{|c|}{\texttt{#3}} 
                    \\ 
        \hline
    \end{tabular}
}}

\newcommand{\FloatMyOrderX}[4]{\text{\upshape
    \begin{tabular}{clrclr}
        \tiny{9}  &\tiny{8} & \tiny{4} & \tiny{3} & \tiny{2} & \tiny{0}\\ 
        \hline
        \multicolumn{1}{|c|}{\texttt{#1}} 
            & \multicolumn{2}{|c|}{\texttt{#2}} 
                & \multicolumn{1}{|c|}{\texttt{#3}} 
                    & \multicolumn{2}{|c|}{\texttt{#4}} \\ 
        \hline
    \end{tabular}
}}

\newcommand{\FloatMyCharX}[3]{\text{\upshape
    \begin{tabular}{clrlr}
        \tiny{9}  &\tiny{8} & \tiny{4} & \tiny{3} & \tiny{0}\\ 
        \hline
        \multicolumn{1}{|c|}{\texttt{#1}} 
            & \multicolumn{2}{|c|}{\texttt{#2}} 
                & \multicolumn{2}{|c|}{\texttt{#3}} \\ 
        \hline
    \end{tabular}
}}

\newcommand{\FloatMyDcOrderX}[3]{\text{\upshape
    \begin{tabular}{lrclr}
        \tiny{9} & \tiny{4} & \tiny{3} & \tiny{2} & \tiny{0}\\ 
        \hline
        \multicolumn{2}{|c|}{\texttt{#1}} 
            & \multicolumn{1}{|c|}{\texttt{#2}} 
                & \multicolumn{2}{|c|}{\texttt{#3}} \\ 
        \hline
    \end{tabular}
}}

\newcommand{\FloatMyDcCharX}[2]{\text{\upshape
    \begin{tabular}{lrlr}
        \tiny{9} & \tiny{4} & \tiny{3} & \tiny{0}\\ 
        \hline
        \multicolumn{2}{|c|}{\texttt{#1}} 
            & \multicolumn{2}{|c|}{\texttt{#2}} \\ 
        \hline
    \end{tabular}
}}


%--- СПЕЦИФИЧНЫЕ ДЛЯ УМНОЖЕНИЯ КОМАНДЫ ---------------------------------------------------------------------------------------------


\newcommand{\Number}[1]{
    \texttt{#1}
}

\newcommand{\NumberHi}[2]{
    \underline{\underline{\texttt{#1}}}\texttt{#2}
}

\newcommand{\NumberMid}[3]{
    \texttt{#1}\underline{\underline{\texttt{#2}}}\texttt{#3}
}

\newcommand{\NumberLo}[2]{
    \texttt{#1}\underline{\underline{\texttt{#2}}}
}

\newcommand{\Stack}[2]{
    \begin{tabular}[t]{@{}r@{}}
        {#1}\\ \hline
        {#2}\\ 
    \end{tabular}
}

\newcommand{\StackThree}[3]{
    \begin{tabular}[t]{@{}r@{}}
        {#1}\\ \hline
        {#2}\\ \hline
        {#3}\\
    \end{tabular}
}

\newcommand{\StackFour}[4]{
    \begin{tabular}[t]{@{}r@{}}
        {#1}\\ \hline
        {#2}\\ \hline
        {#3}\\ \hline
        {#4}\\
    \end{tabular}
}

\newcommand{\Operation}[4]{
    \begin{tabular}[t]{@{}r@{}}
        \texttt{#4}
        \begin{tabular}{@{}r@{}}
            \Number{#1}\\
            \Number{#2}\\ \hline
        \end{tabular} \\ 
        \Number{#3}\\
    \end{tabular}
}

\newcommand{\Addition}[3]{\Operation{#1}{#2}{#3}{+}}

\newcommand{\Subtraction}[3]{\Operation{#1}{#2}{#3}{-}}

\newcommand{\Register}[2]{\Number{#1:#2}}

\newcommand{\Mantiss}{m}
\newcommand{\Order}{p}
\newcommand{\Char}{c}

\newcommand{\MantissOf}[1]{\Mantiss_{#1}}
\newcommand{\OrderOf}[1]{\Order_{#1}}
\newcommand{\CharOf}[1]{\Char_{#1}}

\newcommand{\FloatExpression}[2]{\MantissOf{#1}\cdot {#2}^{\OrderOf{#1}}}

\newcommand{\DivAnswer}[2]{(\texttt{$#1$ rem $#2$})}

\newenvironment{Solve}[1]%
    {\begin{proof}[Решение]#1}
    {\end{proof}}
    
    
%определённые мной команды логической разметки
\newcommand{\Signs}[2]{\fbox{{\small{\underbar{#1}}}{#2}}}
\newcommand{\Sign}[1]{\fbox{#1}}
\newcommand{\DC}[1]{\text{ДК}(#1)}
\newcommand{\OC}[1]{\text{ОК}(#1)}
\newcommand{\PC}[1]{\text{ПК}(#1)}

\newcommand{\MyProc}{\text{$\mathbf{R}_8$}}

\newboolean{IsNeedAnswer}
\setboolean{IsNeedAnswer}{false} %true/false

\newcommand{\Machine}[1]{\texttt{#1}}
\newcommand{\Opcode}[1]{\texttt{\bf{#1}}}
\newcommand{\Operand}[1]{\texttt{#1}}
\newcommand{\CmdOneAddr}[2]{\text{\Opcode{#1} \Operand{#2}}}
\newcommand{\CmdTwoAddr}[3]{\text{\Opcode{#1} \Operand{#2}, \Operand{#3}}}
\newcommand{\CmdThreeAddr}[4]{\text{\Opcode{#1} {\Operand{#2}}, \Operand{#3}, \Operand{#4}}}

\newcommand{\ProofAnswer}[1]{
    \ifthenelse{\boolean{IsNeedAnswer}}{
        \begin{proof}[Ответ]
            #1
        \end{proof}
    }{}
}

\newcommand{\LabeledAnswer}[1]{
    \ifthenelse{\boolean{IsNeedAnswer}}{
        \emph{Отв.}: #1 \qed
    }{}
}

\newcommand{\PlainAnswer}[1]{
    \ifthenelse{\boolean{IsNeedAnswer}}{#1}{}
}

\newcommand{\UnsignedAny}[2]{\text{\upshape
    \begin{tabular}{lr}
        \tiny{#1} & \tiny{0}\\ 
        \hline
        \multicolumn{2}{|c|}{\texttt{#2}} \\ 
        \hline
    \end{tabular}
}}

\newcommand{\UnsignedByte}[1]{\UnsignedAny{7}{#1}}

\newcommand{\UnsignedTwoBytes}[1]{\UnsignedAny{15}{#1}}

\newcommand{\FixedAny}[4]{\text{\upshape
    \begin{tabular}{clr}
        \tiny{#1}  &\tiny{#2} & \tiny{0}\\ 
        \hline
        \multicolumn{1}{|c|}{\texttt{#3}} & \multicolumn{2}{|c|}{\texttt{#4}} \\ 
        \hline
    \end{tabular}
}}

\newcommand{\FixedByte}[2]{\FixedAny{7}{6}{#1}{#2}}

\newcommand{\FixedTwoBytes}[2]{\FixedAny{15}{14}{#1}{#2}}

\newcommand{\SignedAny}[4]{\text{\upshape
    \begin{tabular}{clr}
        \tiny{#1}  &\tiny{#2} & \tiny{0}\\ 
        \hline
        \multicolumn{1}{|c|}{\Machine{#3}} & \multicolumn{2}{|c|}{\Machine{#4}} \\ 
        \hline
    \end{tabular}
}}

\newcommand{\SignedNibble}[2]{\SignedAny{3}{2}{#1}{#2}}

\newcommand{\SignedByte}[2]{\SignedAny{7}{6}{#1}{#2}}

\newcommand{\SignedTwoBytes}[2]{\SignedAny{15}{14}{#1}{#2}}

\newcommand{\FloatMyHex}[4]{\text{\upshape
    \begin{tabular}{clrclr}
        \tiny{15}  &\tiny{14} & \tiny{6} & \tiny{5} & \tiny{4} & \tiny{0}\\ 
        \hline
        \multicolumn{1}{|c|}{\texttt{#1}} 
            & \multicolumn{2}{|c|}{\texttt{#2}} 
                & \multicolumn{1}{|c|}{\texttt{#3}} 
                    & \multicolumn{2}{|c|}{\texttt{#4}} \\ 
        \hline
    \end{tabular}
}}

\newcommand{\FloatMyCharHex}[3]{\text{\upshape
    \begin{tabular}{clrlr}
        \tiny{15}  &\tiny{14} & \tiny{6} & \tiny{5} & \tiny{0}\\ 
        \hline
        \multicolumn{1}{|c|}{\texttt{#1}} 
            & \multicolumn{2}{|c|}{\texttt{#2}} 
                & \multicolumn{2}{|c|}{\texttt{#3}} \\ 
        \hline
    \end{tabular}
}}

\newcommand{\FloatMyDcMantCharHex}[2]{\text{\upshape
    \begin{tabular}{lrlr}
        \tiny{15} & \tiny{6} & \tiny{5} & \tiny{0}\\ 
        \hline
        \multicolumn{2}{|c|}{\texttt{#1}} 
            & \multicolumn{2}{|c|}{\texttt{#2}} \\ 
        \hline
    \end{tabular}
}}

\newcommand{\FloatESShort}[3]{\text{\upshape
    \begin{tabular}{clrlr}
        \tiny{31}  &\tiny{30} & \tiny{24} & \tiny{23} & \tiny{0}\\ 
        \hline
        \multicolumn{1}{|c|}{\texttt{#1}} 
            & \multicolumn{2}{|c|}{\texttt{#2}} 
                & \multicolumn{2}{|c|}{\texttt{#3}} 
                    \\ 
        \hline
    \end{tabular}
}}

\newcommand{\FloatPCShort}[3]{\text{\upshape
    \begin{tabular}{clrlr}
        \tiny{31}  &\tiny{30} & \tiny{23} & \tiny{22} & \tiny{0}\\ 
        \hline
        \multicolumn{1}{|c|}{\texttt{#1}} 
            & \multicolumn{2}{|c|}{\texttt{#2}} 
                & \multicolumn{2}{|c|}{\texttt{#3}} 
                    \\ 
        \hline
    \end{tabular}
}}

\newcommand{\FloatMyOrderX}[4]{\text{\upshape
    \begin{tabular}{clrclr}
        \tiny{9}  &\tiny{8} & \tiny{4} & \tiny{3} & \tiny{2} & \tiny{0}\\ 
        \hline
        \multicolumn{1}{|c|}{\texttt{#1}} 
            & \multicolumn{2}{|c|}{\texttt{#2}} 
                & \multicolumn{1}{|c|}{\texttt{#3}} 
                    & \multicolumn{2}{|c|}{\texttt{#4}} \\ 
        \hline
    \end{tabular}
}}

\newcommand{\FloatMyCharX}[3]{\text{\upshape
    \begin{tabular}{clrlr}
        \tiny{9}  &\tiny{8} & \tiny{4} & \tiny{3} & \tiny{0}\\ 
        \hline
        \multicolumn{1}{|c|}{\texttt{#1}} 
            & \multicolumn{2}{|c|}{\texttt{#2}} 
                & \multicolumn{2}{|c|}{\texttt{#3}} \\ 
        \hline
    \end{tabular}
}}

\newcommand{\FloatMyDcOrderX}[3]{\text{\upshape
    \begin{tabular}{lrclr}
        \tiny{9} & \tiny{4} & \tiny{3} & \tiny{2} & \tiny{0}\\ 
        \hline
        \multicolumn{2}{|c|}{\texttt{#1}} 
            & \multicolumn{1}{|c|}{\texttt{#2}} 
                & \multicolumn{2}{|c|}{\texttt{#3}} \\ 
        \hline
    \end{tabular}
}}

\newcommand{\FloatMyDcCharX}[2]{\text{\upshape
    \begin{tabular}{lrlr}
        \tiny{9} & \tiny{4} & \tiny{3} & \tiny{0}\\ 
        \hline
        \multicolumn{2}{|c|}{\texttt{#1}} 
            & \multicolumn{2}{|c|}{\texttt{#2}} \\ 
        \hline
    \end{tabular}
}}


%--- СПЕЦИФИЧНЫЕ ДЛЯ УМНОЖЕНИЯ КОМАНДЫ ---------------------------------------------------------------------------------------------


\newcommand{\Number}[1]{
    \texttt{#1}
}

\newcommand{\NumberHi}[2]{
    \underline{\underline{\texttt{#1}}}\texttt{#2}
}

\newcommand{\NumberMid}[3]{
    \texttt{#1}\underline{\underline{\texttt{#2}}}\texttt{#3}
}

\newcommand{\NumberLo}[2]{
    \texttt{#1}\underline{\underline{\texttt{#2}}}
}

\newcommand{\Stack}[2]{
    \begin{tabular}[t]{@{}r@{}}
        {#1}\\ \hline
        {#2}\\ 
    \end{tabular}
}

\newcommand{\StackThree}[3]{
    \begin{tabular}[t]{@{}r@{}}
        {#1}\\ \hline
        {#2}\\ \hline
        {#3}\\
    \end{tabular}
}

\newcommand{\StackFour}[4]{
    \begin{tabular}[t]{@{}r@{}}
        {#1}\\ \hline
        {#2}\\ \hline
        {#3}\\ \hline
        {#4}\\
    \end{tabular}
}

\newcommand{\Operation}[4]{
    \begin{tabular}[t]{@{}r@{}}
        \texttt{#4}
        \begin{tabular}{@{}r@{}}
            \Number{#1}\\
            \Number{#2}\\ \hline
        \end{tabular} \\ 
        \Number{#3}\\
    \end{tabular}
}

\newcommand{\Addition}[3]{\Operation{#1}{#2}{#3}{+}}

\newcommand{\Subtraction}[3]{\Operation{#1}{#2}{#3}{-}}

\newcommand{\Register}[2]{\Number{#1:#2}}

\newcommand{\Mantiss}{m}
\newcommand{\Order}{p}
\newcommand{\Char}{c}

\newcommand{\MantissOf}[1]{\Mantiss_{#1}}
\newcommand{\OrderOf}[1]{\Order_{#1}}
\newcommand{\CharOf}[1]{\Char_{#1}}

\newcommand{\FloatExpression}[2]{\MantissOf{#1}\cdot {#2}^{\OrderOf{#1}}}

\newcommand{\DivAnswer}[2]{(\texttt{$#1$ rem $#2$})}

\newenvironment{Solve}[1]%
    {\begin{proof}[Решение]#1}
    {\end{proof}}
    
    

\title[Умножение в ПЗ]{Умножение в формате с плавающей точкой}

\newcounter{TaskSimpleCtr}
\setcounter{TaskSimpleCtr}{1}
\newcommand{\TaskSimpleNumber}{ \arabic{TaskSimpleCtr}) \addtocounter{TaskSimpleCtr}{1} }

%вставка изображений из metapost (post script)
\DeclareGraphicsRule{*}{mps}{*}{}

\begin{document}

\mode<article>{\maketitle\tableofcontents}
\frame<presentation>{\titlepage}
\begin{frame}<presentation>
    \frametitle{Содержание}
    \tableofcontents
\end{frame}



\begin{frame}
    \frametitle{Формат с плавающей точкой}
    
    \[
        X = \FloatExpression{X}{2},
    \]
    где $\MantissOf{X}$ --- нормализованная мантисса числа $X$, $\OrderOf{X}$ --- порядок числа $X$, подобранный так, чтобы $\MantissOf{X}$ была нормализованной.
    \begin{block}{Правила нормализации $X\neq 0$}
        Мантисса $\MantissOf{X}$ получается из двоичного представления $X$ переносом точки в такую позицию, чтобы целая часть была равна нулю, а в старшем разряде дробной части была единица:
        \[(0.\underbrace{1xxx\cdots xxx}_{\text{\tiny{разряды мантиссы}}})_2\]
        порядок $\OrderOf{X}$ определяет на сколько разрядов нужно передвинуть запятую в мантиссе, чтобы получить исходное число.
    \end{block}
\end{frame}


\section{Порядок}


\begin{frame}
    \frametitle{Формат для примеров}

    \[
        \FloatMyHex{X}{XXXXXXXXX}{X}{XXXXX}
    \]
    \begin{itemize}
        \item Разряды нормализованной мантиссы в прямом коде хранятся в разрядах $[15:6]$.
        \item Порядок в прямом коде хранится в разрядах $[5:0]$.
    \end{itemize}
\end{frame}


\subsection{Правила умножения в формате}


\begin{frame}
    \frametitle{Правила умножения}

    \[R=X\cdot Y=\FloatExpression{X}{2}\cdot\FloatExpression{Y}{2}=(\MantissOf{X}\cdot\MantissOf{Y})\cdot 2^{(\OrderOf{X}+\OrderOf{Y})}.\]
        
    \begin{enumerate}
        \item Порядок результата определяется сложением порядков операндов: $\OrderOf{R}\gets\OrderOf{X}+\OrderOf{Y}$.  
        \item Мантисса результата определяется перемножением мантисс операндов по правилам умножения чисел с фиксированной запятой: $\MantissOf{R}\gets\MantissOf{X}\cdot\MantissOf{Y}$.
        \item Выполняется нормализация результата, если он получился не нормализованым.
        \item Фиксируется результат, или ситуации ПРС/ПМР.
    \end{enumerate}
\end{frame}


\subsection{Примеры умножения}


\begin{frame}
    \frametitle{$37\cdot 86=3182$}

    \begin{center}
        \FloatMyHex{0}{100101000}{0}{00110} $\times$
        \FloatMyHex{0}{101011000}{0}{00111}
    \end{center}
    
    \resizebox{!}{.75\height}{
        \begin{tabular}{r|r|l}
            \multicolumn{1}{c|}{$\Mantiss$}
                                        & \multicolumn{1}{|c|}{$\Order$}
                                                                & \multicolumn{1}{|c}{прим.} \\ \hline\hline
            \Number{0,100101000}        & \Number{0,00110}      & $X=37$, ПК\\ 
            \Number{0,101011000}        & \Number{0,00111}      & $Y=86$, ПК\\ \hline
                                        & \Addition{00,00110}
                                                   {00,00111}
                                                   {00,01101}   & сложение порядков в МДК \\ \hline
            \Multiplication{0,100101000}     
                           {0,101011000}     
                           {0,011000110 111000000}     
                                        &                       & $\MantissOf{R} = \MantissOf{X}\cdot\MantissOf{Y}$, ненормализованная \\ \hline
            \Number{0,011000110 111000000}   
                                        & \Number{00,01101}     & Нормализация! $\MantissOf{R}\gets \MantissOf{R}\ll 1$; 
                                                                                 $\OrderOf{R}\gets\OrderOf{R}-1$ \\ \hline
            \Number{0,110001101}        & \Number{0,01100}      & Рез-т!
        \end{tabular}
    }
    \[
        \FloatMyHex{0}{110001101}{0}{01100}
    \]
    $37\cdot 86\approx 3176$. $\Delta = 6, \delta\approx 0,00189$
\end{frame}

\begin{frame}
    \frametitle{ПРС}

    \begin{center}
        \FloatMyHex{1}{110000000}{0}{11000} $\times$
        \FloatMyHex{0}{110000000}{0}{10000}
    \end{center}
    
    \resizebox{!}{.75\height}{
        \begin{tabular}{r|r|l}
            \multicolumn{1}{c|}{$\Mantiss$}
                                        & \multicolumn{1}{|c|}{$\Order$}
                                                                & \multicolumn{1}{|c}{прим.} \\ \hline\hline
            \Number{1,110000000}        & \Number{0,11000}      & $X$, ПК\\ 
            \Number{0,110000000}        & \Number{0,10000}      & $Y$, ПК\\ \hline
                                        & \Addition{00,11000}
                                                   {00,10000}
                                                   {01,01000}   & Сложение в МДК. ПРС порядков \\ \hline
            \Multiplication{1,110000000}     
                           {0,110000000}     
                           {1,100100000 000000000}     
                                        &                       & $\MantissOf{R} = \MantissOf{X}\cdot\MantissOf{Y}$, нормализованная \\ \hline
            \Number{1,100100000 000000000}   
                                        & \Number{01,01000}     & ПРС! Слишком большой порядок.
        \end{tabular}
    }

    Ошибка вычислений --- ПРС формата с плавающей запятой.
\end{frame}
    

\begin{frame}
    \frametitle{Устранимое ПРС}

    \begin{center}
        \FloatMyHex{1}{100000000}{0}{11000} $\times$
        \FloatMyHex{1}{100000000}{0}{01000}
    \end{center}
    
    \resizebox{!}{.75\height}{
        \begin{tabular}{r|r|l}
            \multicolumn{1}{c|}{$\Mantiss$}
                                        & \multicolumn{1}{|c|}{$\Order$}
                                                                & \multicolumn{1}{|c}{прим.} \\ \hline\hline
            \Number{1,100000000}        & \Number{0,11000}      & $X$, ПК\\ 
            \Number{1,100000000}        & \Number{0,10000}      & $Y$, ПК\\ \hline
                                        & \Addition{00,11000}
                                                   {00,01000}
                                                   {01,00000}   & Сложение в МДК. ПРС порядков? \\ \hline
            \Multiplication{1,100000000}     
                           {1,100000000}     
                           {0,010000000 000000000}     
                                        &                       & $\MantissOf{R} = \MantissOf{X}\cdot\MantissOf{Y}$, ненормализованная \\ \hline
            \Number{0,010000000 000000000}   
                                        & \Number{01,00000}     & Нормализация! $\MantissOf{R}\gets \MantissOf{R}\ll 1$; 
                                                                                 $\OrderOf{R}\gets\OrderOf{R}-1$ \\ \hline
            \Number{0,100000000 000000000}   
                                        & \Number{00,11111}     & ПРС нет! \\ \hline
            \Number{0,100000000}        & \Number{0,11111}      & Рез-т!
        \end{tabular}
    }
    \[
        \FloatMyHex{0}{100000000}{0}{11111}
    \]
\end{frame}

\begin{frame}
    \frametitle{ПМР}

    \begin{center}
        \FloatMyHex{1}{100000000}{1}{11000} $\times$
        \FloatMyHex{0}{100000000}{1}{01100}
    \end{center}
    
    \resizebox{!}{.75\height}{
        \begin{tabular}{r|r|l}
            \multicolumn{1}{c|}{$\Mantiss$}
                                        & \multicolumn{1}{|c|}{$\Order$}
                                                                & \multicolumn{1}{|c}{прим.} \\ \hline\hline
            \Number{1,100000000}        & \Number{1,11000}      & $X$, ПК\\ 
            \Number{0,100000000}        & \Number{1,01100}      & $Y$, ПК\\ \hline
                                        & \Addition{11,01000}
                                                   {11,10100}
                                                   {10,11100}   & Сложение в МДК. ПРС порядков <<в минус>>. \\ 
        \end{tabular}
    }
    
    ПМР не устряняется!
    \[
        \FloatMyHex{0}{000000000}{0}{00000}
    \]
\end{frame}

    
\section{Характеристика}


\begin{frame}
    \frametitle{Характеристика}

    \[
        X = \FloatExpression{X}{2}.
    \]
    
    Диапазон представления порядка $\OrderOf{X}$ в $n$-разрядной сетке будет\footnote{Если использовать дополнительный код}:
    \[
        \OrderOf{X}\in[-2^{n-1}, +(2^{n-1}-1)]
    \]
    
    Характеристика получается из порядка прибавлением фиксированной поправки $\Delta$, такой, что левая граница представления обращается в ноль. Таким образом, 
    \begin{block}{}
        характеристика $\CharOf{X}$ --- всегда положительное число.
        \begin{equation}
            \CharOf{X} = \OrderOf{X} + \Delta,
            \label{add:fpt:char}
        \end{equation}
        где\footnote{Опять же только в случае использования дополнительного кода} $\Delta=+2^{n-1}$, а $\CharOf{X}\in[0,2^n-1]$.
    \end{block}
\end{frame}

\begin{frame}
    \frametitle{Свойства $n$-разрядной характеристики}

    \begin{itemize}
        \item Характеристика --- положительное число.
        \item Разность характеристик равна разности порядков.
        \item Если в процессе нормализации (или денормализации) порядок увеличивается (или уменьшается), то то же самое происходит и с характеристикой.
        \item Если для работы с характеристиками использовать ДК или МДК, о ПРС при нормализации легко судить по знаковому разряду: он не должен быть 1.
        \item Если использвется поправка $\Delta=2^{n-1}$, то характеристика получается из дополнительного кода порядка инверсией знакового разряда.
    \end{itemize}
\end{frame}

\begin{frame}
    \frametitle{Формат для примеров}

    \[
        \FloatMyCharHex{X}{XXXXXXXXX}{XXXXXX}
    \]
    \begin{itemize}
        \item Разряды нормализованной мантиссы в прямом коде хранятся в разрядах $[15:6]$.
        \item Характеристика хранится в разрядах $[5:0]$.
        \item $\Delta=2^5=32=(100000)_2$
    \end{itemize}
\end{frame}


\subsection{Правила умножения}


\begin{frame}
    \frametitle{Правила умножения}

    \[
        R=X\cdot Y=
            \MantissOf{X}\cdot 2^{\CharOf{X}-\Delta}\cdot
            \MantissOf{Y}\cdot 2^{\CharOf{Y}-\Delta}=
            (\MantissOf{X}\cdot\MantissOf{Y})\cdot 2^{\CharOf{X} + \CharOf{Y}-2\Delta}
    \]
        
    \begin{enumerate}
        \item Хаарактеристика результата определяется по формуле: $\CharOf{R}\gets(\CharOf{X}+\CharOf{Y}-\Delta)$.  
        \item Мантисса результата определяется перемножением мантисс операндов по правилам умножения чисел с фиксированной запятой: $\MantissOf{R}\gets\MantissOf{X}\cdot\MantissOf{Y}$.
        \item Выполняется нормализация результата, если он получился не нормализованым.
        \item Фиксируется результат, или ситуации ПРС/ПМР.
    \end{enumerate}
\end{frame}


\subsection{Примеры умножения}


\begin{frame}
    \frametitle{$2.5\cdot 6=15$}

    \begin{center}
        \FloatMyCharHex{0}{101000000}{100010} $\times$
        \FloatMyCharHex{0}{110000000}{100011}
    \end{center}
    
    
    \resizebox{!}{.7\height}{
        \begin{tabular}{r|r|l}
            \multicolumn{1}{c|}{$\Mantiss$}
                                        & \multicolumn{1}{|c|}{$\Char$}
                                                               & \multicolumn{1}{|c}{прим.} \\ \hline\hline
            \Number{0,101000000}        & \Number{100010}      & $X=2.5$, ПК\\ 
            \Number{0,110000000}        & \Number{100011}      & $Y=6$, ПК\\ \hline
                                        & \Addition{00,100010}
                                                   {00,100011}
                                                   {01,000101}  & $(\CharOf{X}+\CharOf{Y})$ в МДК \\ \hline
                                        & \Addition{01,000101} 
                                                   {11,100000} 
                                                   {00,100101}  & $(\CharOf{X}+\CharOf{Y}-\Delta)$ в МДК \\ \hline
            \Multiplication{0,101000000}                       
                           {0,110000000}                       
                           {0,011110000 000000000}             
                                        &                      & $\MantissOf{R} = \MantissOf{X}\cdot\MantissOf{Y}$, ненормализованная \\ \hline
            \Number{0,011110000 000000000}                     
                                        & \Number{00,100101}   & Нормализация! $\MantissOf{R}\gets \MantissOf{R}\ll 1$; 
                                                                               $\CharOf{R}\gets \CharOf{R}-1$; \\ \hline
            \Number{0,111100000}        & \Number{100100}      & Рез-т!
        \end{tabular}
    }
    \[
        \FloatMyCharHex{0}{111100000}{100100}
    \]
\end{frame}

\begin{frame}
    \frametitle{ПРС}

    \begin{center}
        \FloatMyCharHex{0}{100000000}{111000} $\times$
        \FloatMyCharHex{0}{100000000}{101100}
    \end{center}
    
    
    \resizebox{!}{.7\height}{
        \begin{tabular}{r|r|l}
            \multicolumn{1}{c|}{$\Mantiss$}
                                        & \multicolumn{1}{|c|}{$\Char$}
                                                               & \multicolumn{1}{|c}{прим.} \\ \hline\hline
            \Number{0,100000000}        & \Number{111000}      & $X=0.5\cdot 2^{56-32}$, ПК\\ 
            \Number{0,100000000}        & \Number{101100}      & $Y=0.5\cdot 2^{44-32}$, ПК\\ \hline
                                        & \Addition{00,111000}
                                                   {00,101100}
                                                   {01,100100}  & $(\CharOf{X}+\CharOf{Y})$ в МДК \\ \hline
                                        & \Addition{01,100100} 
                                                   {11,100000} 
                                                   {01,000100}  & $(\CharOf{X}+\CharOf{Y}-\Delta)$ в МДК, ПРС \\ \hline
            \Multiplication{0,100000000}                       
                           {0,100000000}                       
                           {0,010000000 000000000}             
                                        &                      & $\MantissOf{R} = \MantissOf{X}\cdot\MantissOf{Y}$, ненормализованная \\ \hline
            \Number{0,010000000 000000000}                     
                                        & \Number{01,000100}   & Нормализация! $\MantissOf{R}\gets \MantissOf{R}\ll 1$; $\CharOf{R}\gets \CharOf{R}-1$; \\ \hline
            \Number{0,100000000 000000000}                     
                                        & \Number{01,000011}   & ПРС! Слишком большой порядок. \\ \hline
        \end{tabular}
    }
    
    Ошибка вычислений --- ПРС формата с плавающей запятой.
\end{frame}

\begin{frame}
    \frametitle{Устранимое ПРС}

    \begin{center}
        \FloatMyCharHex{0}{100000000}{111000} $\times$
        \FloatMyCharHex{0}{100000000}{101000}
    \end{center}
    
    
    \resizebox{!}{.67\height}{
        \begin{tabular}{r|r|l}
            \multicolumn{1}{c|}{$\Mantiss$}
                                        & \multicolumn{1}{|c|}{$\Char$}
                                                               & \multicolumn{1}{|c}{прим.} \\ \hline\hline
            \Number{0,100000000}        & \Number{111000}      & $X=0.5\cdot 2^{56-32}$, ПК\\ 
            \Number{0,100000000}        & \Number{101000}      & $Y=0.5\cdot 2^{40-32}$, ПК\\ \hline
                                        & \Addition{00,111000}
                                                   {00,101000}
                                                   {01,100000}  & $(\CharOf{X}+\CharOf{Y})$ в МДК \\ \hline
                                        & \Addition{01,100000} 
                                                   {11,100000} 
                                                   {01,000000}  & $(\CharOf{X}+\CharOf{Y}-\Delta)$ в МДК, ПРС \\ \hline
            \Multiplication{0,100000000}                       
                           {0,100000000}                       
                           {0,010000000 000000000}             
                                        &                      & $\MantissOf{R} = \MantissOf{X}\cdot\MantissOf{Y}$, ненормализованная \\ \hline
            \Number{0,010000000 000000000}                     
                                        & \Number{01,000000}   & Нормализация! $\MantissOf{R}\gets \MantissOf{R}\ll 1$; $\CharOf{R}\gets \CharOf{R}-1$; \\ \hline
            \Number{0,100000000 000000000}                     
                                        & \Number{00,111111}   & ПРС устраняется.  \\ \hline
            \Number{0,100000000}                     
                                        & \Number{111111}      & $R=0.5\cdot 2^{63-32}$  \\ \hline
        \end{tabular}
    }
    \[
        \FloatMyCharHex{0}{100000000}{111111}
    \]
\end{frame}


\begin{frame}
    \frametitle{ПМР}

    \begin{center}
        \FloatMyCharHex{0}{100000000}{000001} $\times$
        \FloatMyCharHex{0}{100000000}{011110}
    \end{center}
    
    
    \resizebox{!}{.86\height}{
        \begin{tabular}{r|r|l}
            \multicolumn{1}{c|}{$\Mantiss$}
                                        & \multicolumn{1}{|c|}{$\Char$}
                                                               & \multicolumn{1}{|c}{прим.} \\ \hline\hline
            \Number{0,100000000}        & \Number{111000}      & $X=0.5\cdot 2^{1-32}$, ПК\\ 
            \Number{0,100000000}        & \Number{101000}      & $Y=0.5\cdot 2^{30-32}$, ПК\\ \hline
                                        & \Addition{00,000001}
                                                   {00,011110}
                                                   {00,011111}  & $(\CharOf{X}+\CharOf{Y})$ в МДК \\ \hline
                                        & \Addition{00,011111} 
                                                   {11,100000} 
                                                   {11,111111}  & $(\CharOf{X}+\CharOf{Y}-\Delta)$ в МДК, ПРС <<в минус>>
        \end{tabular}
    }
    
    ПМР не устраняется!
    \[
        \FloatMyCharHex{0}{000000000}{000000}
    \]
\end{frame}

\appendix


\section{Задания на практику}


\subsection{Проходное}

\begin{frame}
    \frametitle{\TaskSimpleNumber}
    
    Придумать правила определения ПРС/ПМР при работе 
    \begin{itemize}
        \item с порядками в прямом коде;
        \item с характеристиками.
    \end{itemize}
\end{frame}

\begin{frame}
    \frametitle{\TaskSimpleNumber}

    Разработать собственный 10-разрядный формат\footnote{Преподавателю: обязательно проследить, чтобы были использованы и порядки и характеристики} и перемножить в нем числа:

    \begin{enumerate}
        \item $9$ и $-11$;
        \item $10$ и $7$;
        \item $0.625$ и $0.75$;
        \item $-0.625$ и $0.375$.
    \end{enumerate}
\end{frame}

\subsection{Мегамозг}

\begin{frame}
    \frametitle{\TaskSimpleNumber}
    
    Придумать пример из последовательности трех чисел, произведение которых зависимсит от порядка перемножения. 
\end{frame}

\section{Самообучение}

\begin{frame}
    \frametitle{Советы самоучке}
    
    Классика жанра: \cite{bib:lisikov:automateBase}.
\end{frame}

\begin{frame}[allowframebreaks]{Библиография}
    \bibliographystyle{gost780u}
    \bibliography{./../../../bibliobase}
\end{frame}

\end{document}