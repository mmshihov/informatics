%include part: see main.beamer.tex and main.article.tex
%include common packages and settings
\usepackage{etex} %эта магическая херь избавляет от переполнения регистров TeX а!!!

\mode<article>{\usepackage{fullpage}}
\mode<presentation>{
    \usetheme{Madrid} %%Boadilla,Madrid,AnnArbor,CambridgeUS,Malmoe,Singapore,Berlin
    \useoutertheme{shadow}
} 

\usepackage[utf8]{inputenc}
\usepackage[russian]{babel}
\usepackage{indentfirst}
\usepackage{graphicx}

\usepackage{amsmath}
\usepackage{amsfonts}
\usepackage{amsthm}
\usepackage{algorithm}
\usepackage{algorithmic}

\usepackage[all]{xy}

\date{Лекция по дисциплине <<информатика>>\\(\today)}
\author[М.~М.~Шихов]{Михаил Шихов \\ \texttt{\underline{m.m.shihov@gmail.com}}}

%для рисования графов пакетом xy-pic
\entrymodifiers={++[o][F-]}

%для псевдокода алгоритмов (algorithm,algorithmic)
\renewcommand{\algorithmicrequire}{\textbf{Вход:}}
\renewcommand{\algorithmicensure}{\textbf{Выход:}}
\renewcommand{\algorithmiccomment}[1]{// #1}
\floatname{algorithm}{Псевдокод}

%%определённые мной команды логической разметки
\newcommand{\DC}[1]{\text{ДК}(#1)}
\newcommand{\MDC}[1]{\text{MДК}(#1)}
\newcommand{\OC}[1]{\text{ОК}(#1)}
\newcommand{\MOC}[1]{\text{МОК}(#1)}
\newcommand{\PC}[1]{\text{ПК}(#1)}

\newcommand{\Machine}[1]{\texttt{#1}}

\newcommand{\UnsignedAny}[2]{\text{\upshape
    \begin{tabular}{lr}
        \tiny{#1} & \tiny{0}\\ 
        \hline
        \multicolumn{2}{|c|}{\Machine{#2}} \\ 
        \hline
    \end{tabular}
}}

\newcommand{\UnsignedByte}[1]{\UnsignedAny{7}{#1}}

\newcommand{\UnsignedTwoBytes}[1]{\UnsignedAny{15}{#1}}

\newcommand{\SignedAny}[4]{\text{\upshape
    \begin{tabular}{clr}
        \tiny{#1}  &\tiny{#2} & \tiny{0}\\ 
        \hline
        \multicolumn{1}{|c|}{\Machine{#3}} & \multicolumn{2}{|c|}{\Machine{#4}} \\ 
        \hline
    \end{tabular}
}}

\newcommand{\SignedNibble}[2]{\SignedAny{3}{2}{#1}{#2}}

\newcommand{\SignedByte}[2]{\SignedAny{7}{6}{#1}{#2}}

\newcommand{\SignedTwoBytes}[2]{\SignedAny{15}{14}{#1}{#2}}

\newcommand{\FloatMyHex}[4]{\text{\upshape
    \begin{tabular}{clrclr}
        \tiny{15}  &\tiny{14} & \tiny{6} & \tiny{5} & \tiny{4} & \tiny{0}\\ 
        \hline
        \multicolumn{1}{|c|}{\texttt{#1}} 
            & \multicolumn{2}{|c|}{\texttt{#2}} 
                & \multicolumn{1}{|c|}{\texttt{#3}} 
                    & \multicolumn{2}{|c|}{\texttt{#4}} \\ 
        \hline
    \end{tabular}
}}

\newcommand{\FloatMyCharHex}[3]{\text{\upshape
    \begin{tabular}{clrlr}
        \tiny{15}  &\tiny{14} & \tiny{6} & \tiny{5} & \tiny{0}\\ 
        \hline
        \multicolumn{1}{|c|}{\texttt{#1}} 
            & \multicolumn{2}{|c|}{\texttt{#2}} 
                & \multicolumn{2}{|c|}{\texttt{#3}} \\ 
        \hline
    \end{tabular}
}}

\newcommand{\FloatMySimpleHex}[2]{\text{\upshape
    \begin{tabular}{lrlr}
        \tiny{15}  & \tiny{6} & \tiny{5} & \tiny{0} \\ 
        \hline
        \multicolumn{2}{|c|}{\texttt{#1}} 
            & \multicolumn{2}{|c|}{\texttt{#2}} \\ 
        \hline
    \end{tabular}
}}

\newcommand{\FloatMyOrderX}[4]{\text{\upshape
    \begin{tabular}{clrclr}
        \tiny{9}  &\tiny{8} & \tiny{4} & \tiny{3} & \tiny{2} & \tiny{0}\\ 
        \hline
        \multicolumn{1}{|c|}{\texttt{#1}} 
            & \multicolumn{2}{|c|}{\texttt{#2}} 
                & \multicolumn{1}{|c|}{\texttt{#3}} 
                    & \multicolumn{2}{|c|}{\texttt{#4}} \\ 
        \hline
    \end{tabular}
}}

\newcommand{\FloatMyDcOrderX}[3]{\text{\upshape
    \begin{tabular}{lrclr}
        \tiny{9} & \tiny{4} & \tiny{3} & \tiny{2} & \tiny{0}\\ 
        \hline
        \multicolumn{2}{|c|}{\texttt{#1}} 
            & \multicolumn{1}{|c|}{\texttt{#2}} 
                & \multicolumn{2}{|c|}{\texttt{#3}} \\ 
        \hline
    \end{tabular}
}}

\newcommand{\FloatMyDcCharX}[2]{\text{\upshape
    \begin{tabular}{lrlr}
        \tiny{9} & \tiny{4} & \tiny{3} & \tiny{0}\\ 
        \hline
        \multicolumn{2}{|c|}{\texttt{#1}} 
            & \multicolumn{2}{|c|}{\texttt{#2}} \\ 
        \hline
    \end{tabular}
}}

\newcommand{\FloatMyCharX}[3]{\text{\upshape
    \begin{tabular}{clrlr}
        \tiny{9}  &\tiny{8} & \tiny{4} & \tiny{3} & \tiny{0}\\ 
        \hline
        \multicolumn{1}{|c|}{\texttt{#1}} 
            & \multicolumn{2}{|c|}{\texttt{#2}} 
                & \multicolumn{2}{|c|}{\texttt{#3}} \\ 
        \hline
    \end{tabular}
}}

\newcommand{\FloatESShort}[3]{\text{\upshape
    \begin{tabular}{clrlr}
        \tiny{31}  &\tiny{30} & \tiny{24} & \tiny{23} & \tiny{0}\\ 
        \hline
        \multicolumn{1}{|c|}{\texttt{#1}} 
            & \multicolumn{2}{|c|}{\texttt{#2}} 
                & \multicolumn{2}{|c|}{\texttt{#3}} 
                    \\ 
        \hline
    \end{tabular}
}}

\newcommand{\FloatPCShort}[3]{\text{\upshape
    \begin{tabular}{clrlr}
        \tiny{31}  &\tiny{30} & \tiny{23} & \tiny{22} & \tiny{0}\\ 
        \hline
        \multicolumn{1}{|c|}{\texttt{#1}} 
            & \multicolumn{2}{|c|}{\texttt{#2}} 
                & \multicolumn{2}{|c|}{\texttt{#3}} 
                    \\ 
        \hline
    \end{tabular}
}}


%--- СПЕЦИФИЧНЫЕ ДЛЯ УМНОЖЕНИЯ КОМАНДЫ ---------------------------------------------------------------------------------------------


\newcommand{\Number}[1]{
    \texttt{#1}
}

\newcommand{\NumberHi}[2]{
    \underline{\underline{\texttt{#1}}}\texttt{#2}
}

\newcommand{\NumberMid}[3]{
    \texttt{#1}\underline{\underline{\texttt{#2}}}\texttt{#3}
}

\newcommand{\NumberLo}[2]{
    \texttt{#1}\underline{\underline{\texttt{#2}}}
}

\newcommand{\Stack}[2]{
    \begin{tabular}[t]{@{}r@{}}
        {#1}\\ \hline
        {#2}\\ 
    \end{tabular}
}

\newcommand{\Operation}[4]{
    \begin{tabular}[t]{@{}r@{}}
        \texttt{#4}
        \begin{tabular}{@{}r@{}}
            \Number{#1}\\
            \Number{#2}\\ \hline
        \end{tabular} \\ 
        \Number{#3}\\
    \end{tabular}
}

\newcommand{\Addition}[3]{\Operation{#1}{#2}{#3}{+}}

\newcommand{\Subtraction}[3]{\Operation{#1}{#2}{#3}{-}}

\newcommand{\Multiplication}[3]{\Operation{#1}{#2}{#3}{$\times$}}

\newcommand{\Register}[2]{\Number{#1:#2}}

\newcommand{\Mantiss}{m}
\newcommand{\Order}{p}
\newcommand{\Char}{c}

\newcommand{\MantissOf}[1]{\Mantiss_{#1}}
\newcommand{\OrderOf}[1]{\Order_{#1}}
\newcommand{\CharOf}[1]{\Char_{#1}}

\newcommand{\FloatExpression}[2]{\MantissOf{#1}\cdot {#2}^{\OrderOf{#1}}}

\newenvironment{Solve}[1]%
    {\begin{proof}[Решение]#1}
    {\end{proof}}
    
    
%определённые мной команды логической разметки
\newcommand{\DC}[1]{\text{ДК}(#1)}
\newcommand{\MDC}[1]{\text{MДК}(#1)}
\newcommand{\OC}[1]{\text{ОК}(#1)}
\newcommand{\MOC}[1]{\text{МОК}(#1)}
\newcommand{\PC}[1]{\text{ПК}(#1)}

\newcommand{\Machine}[1]{\texttt{#1}}

\newcommand{\UnsignedAny}[2]{\text{\upshape
    \begin{tabular}{lr}
        \tiny{#1} & \tiny{0}\\ 
        \hline
        \multicolumn{2}{|c|}{\Machine{#2}} \\ 
        \hline
    \end{tabular}
}}

\newcommand{\UnsignedByte}[1]{\UnsignedAny{7}{#1}}

\newcommand{\UnsignedTwoBytes}[1]{\UnsignedAny{15}{#1}}

\newcommand{\SignedAny}[4]{\text{\upshape
    \begin{tabular}{clr}
        \tiny{#1}  &\tiny{#2} & \tiny{0}\\ 
        \hline
        \multicolumn{1}{|c|}{\Machine{#3}} & \multicolumn{2}{|c|}{\Machine{#4}} \\ 
        \hline
    \end{tabular}
}}

\newcommand{\SignedNibble}[2]{\SignedAny{3}{2}{#1}{#2}}

\newcommand{\SignedByte}[2]{\SignedAny{7}{6}{#1}{#2}}

\newcommand{\SignedTwoBytes}[2]{\SignedAny{15}{14}{#1}{#2}}

\newcommand{\FloatMyHex}[4]{\text{\upshape
    \begin{tabular}{clrclr}
        \tiny{15}  &\tiny{14} & \tiny{6} & \tiny{5} & \tiny{4} & \tiny{0}\\ 
        \hline
        \multicolumn{1}{|c|}{\texttt{#1}} 
            & \multicolumn{2}{|c|}{\texttt{#2}} 
                & \multicolumn{1}{|c|}{\texttt{#3}} 
                    & \multicolumn{2}{|c|}{\texttt{#4}} \\ 
        \hline
    \end{tabular}
}}

\newcommand{\FloatMyCharHex}[3]{\text{\upshape
    \begin{tabular}{clrlr}
        \tiny{15}  &\tiny{14} & \tiny{6} & \tiny{5} & \tiny{0}\\ 
        \hline
        \multicolumn{1}{|c|}{\texttt{#1}} 
            & \multicolumn{2}{|c|}{\texttt{#2}} 
                & \multicolumn{2}{|c|}{\texttt{#3}} \\ 
        \hline
    \end{tabular}
}}

\newcommand{\FloatMySimpleHex}[2]{\text{\upshape
    \begin{tabular}{lrlr}
        \tiny{15}  & \tiny{6} & \tiny{5} & \tiny{0} \\ 
        \hline
        \multicolumn{2}{|c|}{\texttt{#1}} 
            & \multicolumn{2}{|c|}{\texttt{#2}} \\ 
        \hline
    \end{tabular}
}}

\newcommand{\FloatMyOrderX}[4]{\text{\upshape
    \begin{tabular}{clrclr}
        \tiny{9}  &\tiny{8} & \tiny{4} & \tiny{3} & \tiny{2} & \tiny{0}\\ 
        \hline
        \multicolumn{1}{|c|}{\texttt{#1}} 
            & \multicolumn{2}{|c|}{\texttt{#2}} 
                & \multicolumn{1}{|c|}{\texttt{#3}} 
                    & \multicolumn{2}{|c|}{\texttt{#4}} \\ 
        \hline
    \end{tabular}
}}

\newcommand{\FloatMyDcOrderX}[3]{\text{\upshape
    \begin{tabular}{lrclr}
        \tiny{9} & \tiny{4} & \tiny{3} & \tiny{2} & \tiny{0}\\ 
        \hline
        \multicolumn{2}{|c|}{\texttt{#1}} 
            & \multicolumn{1}{|c|}{\texttt{#2}} 
                & \multicolumn{2}{|c|}{\texttt{#3}} \\ 
        \hline
    \end{tabular}
}}

\newcommand{\FloatMyDcCharX}[2]{\text{\upshape
    \begin{tabular}{lrlr}
        \tiny{9} & \tiny{4} & \tiny{3} & \tiny{0}\\ 
        \hline
        \multicolumn{2}{|c|}{\texttt{#1}} 
            & \multicolumn{2}{|c|}{\texttt{#2}} \\ 
        \hline
    \end{tabular}
}}

\newcommand{\FloatMyCharX}[3]{\text{\upshape
    \begin{tabular}{clrlr}
        \tiny{9}  &\tiny{8} & \tiny{4} & \tiny{3} & \tiny{0}\\ 
        \hline
        \multicolumn{1}{|c|}{\texttt{#1}} 
            & \multicolumn{2}{|c|}{\texttt{#2}} 
                & \multicolumn{2}{|c|}{\texttt{#3}} \\ 
        \hline
    \end{tabular}
}}

\newcommand{\FloatESShort}[3]{\text{\upshape
    \begin{tabular}{clrlr}
        \tiny{31}  &\tiny{30} & \tiny{24} & \tiny{23} & \tiny{0}\\ 
        \hline
        \multicolumn{1}{|c|}{\texttt{#1}} 
            & \multicolumn{2}{|c|}{\texttt{#2}} 
                & \multicolumn{2}{|c|}{\texttt{#3}} 
                    \\ 
        \hline
    \end{tabular}
}}

\newcommand{\FloatPCShort}[3]{\text{\upshape
    \begin{tabular}{clrlr}
        \tiny{31}  &\tiny{30} & \tiny{23} & \tiny{22} & \tiny{0}\\ 
        \hline
        \multicolumn{1}{|c|}{\texttt{#1}} 
            & \multicolumn{2}{|c|}{\texttt{#2}} 
                & \multicolumn{2}{|c|}{\texttt{#3}} 
                    \\ 
        \hline
    \end{tabular}
}}


%--- СПЕЦИФИЧНЫЕ ДЛЯ УМНОЖЕНИЯ КОМАНДЫ ---------------------------------------------------------------------------------------------


\newcommand{\Number}[1]{
    \texttt{#1}
}

\newcommand{\NumberHi}[2]{
    \underline{\underline{\texttt{#1}}}\texttt{#2}
}

\newcommand{\NumberMid}[3]{
    \texttt{#1}\underline{\underline{\texttt{#2}}}\texttt{#3}
}

\newcommand{\NumberLo}[2]{
    \texttt{#1}\underline{\underline{\texttt{#2}}}
}

\newcommand{\Stack}[2]{
    \begin{tabular}[t]{@{}r@{}}
        {#1}\\ \hline
        {#2}\\ 
    \end{tabular}
}

\newcommand{\Operation}[4]{
    \begin{tabular}[t]{@{}r@{}}
        \texttt{#4}
        \begin{tabular}{@{}r@{}}
            \Number{#1}\\
            \Number{#2}\\ \hline
        \end{tabular} \\ 
        \Number{#3}\\
    \end{tabular}
}

\newcommand{\Addition}[3]{\Operation{#1}{#2}{#3}{+}}

\newcommand{\Subtraction}[3]{\Operation{#1}{#2}{#3}{-}}

\newcommand{\Multiplication}[3]{\Operation{#1}{#2}{#3}{$\times$}}

\newcommand{\Register}[2]{\Number{#1:#2}}

\newcommand{\Mantiss}{m}
\newcommand{\Order}{p}
\newcommand{\Char}{c}

\newcommand{\MantissOf}[1]{\Mantiss_{#1}}
\newcommand{\OrderOf}[1]{\Order_{#1}}
\newcommand{\CharOf}[1]{\Char_{#1}}

\newcommand{\FloatExpression}[2]{\MantissOf{#1}\cdot {#2}^{\OrderOf{#1}}}

\newenvironment{Solve}[1]%
    {\begin{proof}[Решение]#1}
    {\end{proof}}
    
    

\title[2-10 коды]{Двоично-десятичные коды}

\newcounter{TaskSimpleCtr}
\setcounter{TaskSimpleCtr}{1}
\newcommand{\TaskSimpleNumber}{ \arabic{TaskSimpleCtr}) \addtocounter{TaskSimpleCtr}{1} }

%вставка изображений из metapost (post script)
\DeclareGraphicsRule{*}{mps}{*}{}

\newcommand{\NaturalLabel}{\text{``\texttt{8421}''}}
\newcommand{\Natural}[1]{\NaturalLabel(#1)}

\newcommand{\PlusThreeLabel}{\text{``\texttt{8421+3}''}}
\newcommand{\PlusThree}[1]{\PlusThreeLabel(#1)}

\newcommand{\AikenLabel}{\text{``\texttt{2421}''}}
\newcommand{\Aiken}[1]{\AikenLabel(#1)}

\newcommand{\PentaLabel}{\text{``\texttt{3a+2}''}}
\newcommand{\Penta}[1]{\PentaLabel(#1)}


\begin{document}

\mode<article>{\maketitle\tableofcontents}
\frame<presentation>{\titlepage}
\begin{frame}<presentation>
    \frametitle{Содержание}
    \tableofcontents
\end{frame}


\begin{frame}
    \frametitle{Введение}

    \begin{block}{}
        Двоично-десятичные коды могут быть использваны для выполнения высокоточных вычислений в десятичной системе счисления с помощью двоичной вычислительной техники.
    \end{block}
\end{frame}

\begin{frame}
    \frametitle{Обратный код в десятичной системе счисления}

    \[
        \OC{X} = 
        \begin{cases}
            \overline{|X|}, &\text{ если $X<0$,}\\
            |X|,            &\text{ если $X\ge 0$,}
        \end{cases}
    \]
    где $\overline{|X|}$ --- порязрядное дополнение цифр десятичного числа $X$ до $9$, то есть разряд $x_i$ числа находится как $(9-x_i)$.    

    \[
        X = 
        \begin{cases}
            -(\overline{\OC{X}}), &\text{ если $msb(\OC{X})=9$,}\\
             \OC{X},              &\text{ если $msb(\OC{X})=0$,}
        \end{cases}
    \]
    где $msb(x)$ --- функция, возвращающая старший значащий бит последовательности $x$.    
    
    \begin{block}{}
        При сложении, как в двоичной системе счисления, единицу переноса из старшего разрядя следует прибавить к младшему разряду.
    \end{block}
\end{frame}

\begin{frame}
    \frametitle{Примеры сложения в обратном коде в 10СС}
    \framesubtitle{4-х разрядная сетка}
    
    \begin{columns}
        \column{.30\textwidth}
            \begin{block}{$731-485=246$}
                 \center
                 \Addition{0731}
                          {9514}
                         {10245}
                         
                 Поправка:                         
                 
                 \Addition{0245}
                          {0001}
                          {0246}
            \end{block}
        \column{.30\textwidth}
            \begin{block}{$204-690=-486$}
                 \center
                 \Addition{0204}
                          {9309}
                          {9513}
            \end{block}
        \column{.30\textwidth}
            \begin{block}{$-247-585=-832$}
                 \center
                 \Addition{9752}
                          {9414}
                         {19166}
                         
                 Поправка:                         
                 
                 \Addition{9166}
                          {0001}
                          {9167}
            \end{block}
    \end{columns}

    Признаками ПРС являются значения в знаковом разряде, отличные от 0 или 9: 1 --- положительное переполнение, 8 --- отрицательное.
\end{frame}


\section{Четырехбитные коды}


\begin{frame}
    \frametitle{Инверсия разрядов тетрады}

    Арифметически, инверсия разрядов двоичной тетрады соответствует дополнению до 15:
    \[
        \overline{(\Number{xxxx})}_2 \Leftrightarrow (\Number{1111})_2-(\Number{xxxx})_2.
    \]
    Например $(\Number{1001})_2=9$:
    \[
        \overline{(\Number{1001})}_2=(\Number{0110})_2=6=(15-9).
    \]
\end{frame}


\subsection{Код \NaturalLabel}


\begin{frame}
    \frametitle{Код \NaturalLabel}
    \framesubtitle{Код с естественными весами: $\Natural{a}=a$}
    
    \begin{center}
    \begin{tabular}{|c|c|c|}
        \hline\hline
        $a$ в 10СС  & $\Natural{a}$         & $\Natural{9-a}$\\
        \hline\hline
        $0$         & $\Number{0000}$       & $\Number{1001}$ \\
        $1$         & $\Number{0001}$       & $\Number{1000}$ \\
        $2$         & $\Number{0010}$       & $\Number{0111}$ \\
        $3$         & $\Number{0011}$       & $\Number{0110}$ \\
        $4$         & $\Number{0100}$       & $\Number{0101}$ \\
        $5$         & $\Number{0101}$       & $\Number{0100}$ \\
        $6$         & $\Number{0110}$       & $\Number{0011}$ \\
        $7$         & $\Number{0111}$       & $\Number{0010}$ \\
        $8$         & $\Number{1000}$       & $\Number{0001}$ \\
        $9$         & $\Number{1001}$       & $\Number{0000}$ \\
        \hline
    \end{tabular}
    \end{center}
    $\Natural{9-a}=9-a=(15-a)-6=\overline{\Natural{a}}-6=\overline{\Natural{a}}+\Number{1010}.$
\end{frame}

\begin{frame}
    \frametitle{Код \NaturalLabel}
    \framesubtitle{Сложение $S=A+B$, где $A=(a_{n-1}\cdots a_0)$ и $B=(b_{n-1}\cdots b_0)$}

    \[
        s_k=a_k+b_k+c_k,
    \]
    где $c_k$ --- перенос в $k$-й разряд, а $s_k,a_k,b_k$ --- десятичные цифры.
    
    \begin{enumerate}
        \item $(a_k+b_k+c_k)<10$; $c_{k+1}=0$. Код $(a_k+b_k+c_k)$ корректен.
        
        \item $10\leq (a_k+b_k+c_k)\leq 15$; $c_{k+1}=0$. Неверно! Перенос в 10СС должен быть, но в 16СС его не случилось. Правильная 10СС цифра $(a_k+b_k+c_k-10)$, и перенос: $(a_k+b_k+c_k-10)+16=(a_k+b_k+c_k+6)$. Поправка: $+6=\Number{0110}$.
        
        \item $(a_k+b_k+c_k)\geq 16$; $c_{k+1}=1$. Неверно! Перенос корректен, а полученная цифра $(a_k+b_k+c_k-16)$ неправильна. Правильный код $(a_k+b_k+c_k-10)$. Поправка: $+6=\Number{0110}$. При такой поправке переноса не будет.
    \end{enumerate}
\end{frame}

\begin{frame}
    \frametitle{Код \NaturalLabel}
    \framesubtitle{Упрощение условий поправок}

    \[
        s_k=a_k+b_k+c_k,
    \]
    
    Из предыдущего слайда ясно, что:
    \begin{enumerate}
        \item\label{bcd:fixup} Если $0\leq(a_k+b_k+c_k)\leq 9$, то поправок не надо.
        \item Если $10\leq(a_k+b_k+c_k)\leq 19$, то нужна поправка $(+6)=\Number{0110}$. Переносы из тетрады в тетраду при этом возникают автоматически.
    \end{enumerate}
    
    Поэтому поправку $(+6)=\Number{0110}$ можно прибавить к обратному коду одного из операндов заранее. И тогда, если переноса из тетрады не будет (только в случае условия из п. \ref{bcd:fixup}) поправку нужно вычесть из тетрады.
\end{frame}


\begin{frame}
    \frametitle{Алгоритм сложения в коде \NaturalLabel}

    \begin{enumerate}
        \item Перевести слагаемые в обратный $\NaturalLabel$-код. Каждая тетрада модуля отрицательного слагаемого инвертируются и к результату прибавляется код $(-6)=\Number{1010}$. Переносы между тетрадами не распространяются.
        \item К каждой тетраде одного из слагаемых прибавляется поправка $(+6)=\Number{0110}$. Переносов между тетрадами при этом не возникает\footnote{Если одно из слагаемых отрицательно, то поправки перевода в ОК (-6) и поправка данного шага (+6) друг друга компенсируют!}.
        \item Выполняется сложение по правилам двоичной арифметики. Переносы распространяются.
        \item Корректируются тетрады, из которых не было переносов. К каждой такой тетраде прибавляется $-6$, т.е. тетрада $\Number{1010}$. Переносы не распространяются.
        \item Результат получен в обратном \NaturalLabel-коде.
    \end{enumerate}
\end{frame}

\begin{frame}[allowframebreaks]
    \frametitle{Код \NaturalLabel}
    \framesubtitle{Пример сложения $-57$ и $894$. }

    \begin{enumerate}
        \item Перевод в ОК (добавлены два знаковых двоичных разряда МОК):
        \begin{align*}
            -57\Rightarrow-\Number{0000 0101 0111}\Rightarrow\\
            \Rightarrow\Addition{1111 1010 1000}
                                {1010 1010 1010}
                                {1001 0100 0010}\\
            \OC{-57}=\Number{11 1001 0100 0010}\\
            \OC{894}=\Number{00 1000 1001 0100}
        \end{align*}
        
        \item К каждой тетраде $\OC{-57}$ прибавлена тетрда $\Number{0110}$.
        \[
            \Addition{11 1001 0100 0010}
                     {.. 0110 0110 0110}
                     {11 1111 1010 1000}
        \]

        \item Выполняется сложение полученного числа с $\OC{894}$. 
        \[
            \Addition{11 1111 1010 1000}
                     {00 1000 1001 0100}
                   {1*00*1000*0011 1100}
        \]
        Коррекция переносом из знакового разряда:
        \[
            \Addition{00*1000*0011 1100}
                     {.. .... .... ...1}
                     {00*1000*0011 1101}
        \]

        \item Корректируются тетрады из которых не было переносов. Переносы между тетрадами не распространяются.
        \[
            \Addition{00*1000*0011 1101}
                     {.. .... .... 1010}
                     {00 1000 0011 0111}
        \]
    \end{enumerate}
    
    ПРС не возникло, $\OC{S}=\Number{00 1000 0011 0111}$, $S=837$
\end{frame}


\subsection{Код \PlusThreeLabel}


\begin{frame}
    \frametitle{Код \PlusThreeLabel}
    \framesubtitle{Код с избытком 3: $\PlusThree{a}=(a+3)$}
    
    \begin{center}
    \begin{tabular}{|c|c|c|}
        \hline\hline
        $a$ в 10СС  & $\PlusThree{a}$  & $\PlusThree{9-a}$\\
        \hline\hline
        $0$         & $\Number{0011}$  & $\Number{1100}$ \\
        $1$         & $\Number{0100}$  & $\Number{1011}$ \\
        $2$         & $\Number{0101}$  & $\Number{1010}$ \\
        $3$         & $\Number{0110}$  & $\Number{1001}$ \\
        $4$         & $\Number{0111}$  & $\Number{1000}$ \\
        $5$         & $\Number{1000}$  & $\Number{0111}$ \\
        $6$         & $\Number{1001}$  & $\Number{0110}$ \\
        $7$         & $\Number{1010}$  & $\Number{0101}$ \\
        $8$         & $\Number{1011}$  & $\Number{0100}$ \\
        $9$         & $\Number{1100}$  & $\Number{0011}$ \\
        \hline
    \end{tabular}
    \end{center}
    
    $\overline{\PlusThree{a}}=15-(a+3)=12-a=(9-a)+3=\PlusThree{9-a}.$
\end{frame}

\begin{frame}
    \frametitle{Код \PlusThreeLabel}
    \framesubtitle{Сложение $S=A+B$, где $A=(a_{n-1}\cdots a_0)$ и $B=(b_{n-1}\cdots b_0)$}

    \[
        s_k=a_k+b_k+c_k.
    \]
    
    \begin{enumerate}
        \item $s_k<10$; При сложении кодов: \[(a_k+3)+(b_k+3)+c_k=(a_k+b_k+c_k)+6=(s_k+6).\] 

        Так как $(s_k+6)\leq 15$, то переноса не возникает. Правильная тетрада должна быть $(s_k+3)$, следовательно, нужна поправка $-3=\Number{1101}$. Перенос игнорируется.
        
        \item $s_k\geq 10$; При сложении кодов возникнет перенос:
        \[(a_k+3)+(b_k+3)+c_k-16=(a_k+b_k+c_k)-10=s_k-10.\] 

        Для получения правильного: $(s_k-10)+3$, нужна поправка $+3=\Number{0011}$. Перенос игнорируется.
    \end{enumerate}
\end{frame}

\begin{frame}
    \frametitle{Код \PlusThreeLabel}
    \framesubtitle{Алгоритм сложения}
    
    \begin{enumerate}
        \item Перевести слагаемые в обратном $\PlusThreeLabel$-коде. Каждая тетрада модуля отрицательного числа инвертируется.
        \item Выполняется сложение полученных операднов по правилам двоичной арифметики.
        \item К тетрадам, из которых не было переноса, прибавляется $\Number{1101}$, а к остальным прибавляется $\Number{0011}$. Переносы игноритуются.
        \item Результат получен в обратном \PlusThreeLabel-коде.
    \end{enumerate}
\end{frame}

\begin{frame}[allowframebreaks]
    \frametitle{Код \PlusThreeLabel}
    \framesubtitle{Пример сложения $-894$ и $57$. }

    \begin{enumerate}
        \item Перевод в ОК (добавлены два знаковых двоичных разряда МОК):
        \begin{align*}
            -894\Rightarrow-\Number{1011 1100 0111}\\
            \OC{-894}=\Number{11 0100 0011 1000}\\
            \OC{57}  =\Number{00 0011 1000 1010}\\
        \end{align*}
        
        \item Выполняется сложение обратных кодов. 
        \[
            \Addition{11 0100 0011 1000}
                     {00 0011 1000 1010}
                     {11 0111 1100*0010}
        \]

        \item Выполняется коррекция. Переносы между тетрадами не распространяются.
        \[
            \Addition{11 0111 1100*0010}
                     {.. 1101 1101 0011}
                     {11 0100 1001 0101}
        \]
    \end{enumerate}
    
    ПРС не возникло, \[\OC{S}=\Number{11 0100 1001 0101}, \overline{\OC{S}}=\Number{00 1011 0110 1010},\] $S=-837$.
\end{frame}


\subsection{Код \AikenLabel}


\begin{frame}
    \frametitle{Код \AikenLabel}
    \framesubtitle{Код Айкена: $\Aiken{a}\equiv t_{3}t_{2}t_{1}t_{0}$, $a=2t_{3}+4t_{2}+2t_{1}+1t_{0}$}
    
    \begin{center}
    \begin{tabular}{|c|c|c|}
        \hline\hline
        $a$ в 10СС  & $\Aiken{a}$                                   & $\Aiken{9-a}$\\
        \hline\hline
        $0$         & $\Number{0000}$                               & $\Number{1111}$ \\
        $1$         & $\Number{0001}$                               & $\Number{1110}$ \\
        %                  
        $2$         & $\Number{0010}\leftrightarrow\Number{1000}$   & $\Number{1101}\leftrightarrow\Number{0111}$ \\
        $3$         & $\Number{0011}\leftrightarrow\Number{1001}$   & $\Number{1100}\leftrightarrow\Number{0110}$ \\
        $4$         & $\Number{0100}\leftrightarrow\Number{1010}$   & $\Number{1011}\leftrightarrow\Number{0101}$ \\
        $5$         & $\Number{0101}\leftrightarrow\Number{1011}$   & $\Number{1010}\leftrightarrow\Number{0100}$ \\
        $6$         & $\Number{0110}\leftrightarrow\Number{1100}$   & $\Number{1001}\leftrightarrow\Number{0011}$ \\
        $7$         & $\Number{0111}\leftrightarrow\Number{1101}$   & $\Number{1000}\leftrightarrow\Number{0010}$ \\
        %                  
        $8$         & $\Number{1110}$                               & $\Number{0001}$ \\
        $9$         & $\Number{1111}$                               & $\Number{0000}$ \\
        \hline
    \end{tabular}
    \end{center}
    
    \begin{align*}
        &\overline{T}=\overline{t_{3}t_{2}t_{1}t_{0}}=(2-2t_3) + (4-4t_3) + (2-2t_2) + (1-1t_0)=9-T,\\
        &\overline{\Aiken{a}} = \Aiken{9-a}.
    \end{align*}
\end{frame}

\begin{frame}
    \frametitle{Код \AikenLabel}
    \framesubtitle{Код Айкена: $\Aiken{a}\equiv t_{3}t_{2}t_{1}t_{0}$, $a=2t_{3}+4t_{2}+2t_{1}+1t_{0}$}
    
    Обобщая таблицу с предыдущего слайда:
    \begin{enumerate}
        \item если $0\leq a \leq 1$, то $\Aiken{a}=a$;
        \item если $2\leq a \leq 7$, то $\Aiken{a}=a$ или $\Aiken{a}=(a+6)$;
        \item если $8\leq a \leq 9$, то $\Aiken{a}=(a+6)$.
    \end{enumerate}
    
    Преследуя цель <<пусть будет>>: <<однозначность>>, <<перенос>> и <<самодополняемость>>, формулируются новые правила:
    \begin{enumerate}
        \item\label{bcd:aiken:ruleI} если $0\leq a \leq 4$, то $\Aiken{a}=a$:
        \[\overline{\Aiken{a}}=(15-a)=\underbrace{(9-a)+6}_{\text{см. п.\ref{bcd:aiken:ruleII}}}=\Aiken{9-a};\]
        
        \item\label{bcd:aiken:ruleII} если $5\leq a \leq 9$, то $\Aiken{a}=(a+6)$:
        \[\overline{\Aiken{a}}=15-(a+6)=\underbrace{(9-a)}_{\text{см. п.\ref{bcd:aiken:ruleI}}}=\Aiken{9-a}.\]
    \end{enumerate}
    
\end{frame}

\begin{frame}
    \frametitle{Код \AikenLabel}
    
    \begin{center}
    \begin{tabular}{|c|c|c|}
        \hline\hline
        $a$ в 10СС  & $\Aiken{a}$       & $\Aiken{9-a}$\\
        \hline\hline                   
        $0$         & $\Number{0000}$   & $\Number{1111}$ \\
        $1$         & $\Number{0001}$   & $\Number{1110}$ \\
        $2$         & $\Number{0010}$   & $\Number{1101}$ \\
        $3$         & $\Number{0011}$   & $\Number{1100}$ \\
        $4$         & $\Number{0100}$   & $\Number{1011}$ \\
        $5$         & $\Number{1011}$   & $\Number{0100}$ \\
        $6$         & $\Number{1100}$   & $\Number{0011}$ \\
        $7$         & $\Number{1101}$   & $\Number{0010}$ \\
        $8$         & $\Number{1110}$   & $\Number{0001}$ \\
        $9$         & $\Number{1111}$   & $\Number{0000}$ \\
        \hline
    \end{tabular}
    \end{center}
    
    \[
        \overline{\Aiken{a}} = \Aiken{9-a}.
    \]
\end{frame}

\begin{frame}
    \frametitle{Код \AikenLabel}

    \[
        s_k=a_k+b_k+c_k.
    \]
    
    \begin{enumerate}
        \item Если $0\leq a_{k},b_{k}\leq 4$, то сложение кодов $(a_{k}+b_{k}+c_{k})$:
        \begin{enumerate}
            \item если $0\leq (a_{k}+b_{k}+c_{k}) \leq 4$, то поправок не нужно;
            \item если $5\leq(a_{k}+b_{k}+c_{k}) \leq 9$, то неверно! Должно быть: $(a_{k}+b_{k}+c_{k})+6$. Поправка: $+6=\Number{0110}$.
        \end{enumerate}

        \item\label{bcd:aiken:diff} Если $0\leq a_{k}\leq 4$ и $5\leq b_{k}\leq 9$, то код $(a_{k}+b_{k}+c_{k}+6)$:
        \begin{enumerate}
            \item если $5\leq (a_{k}+b_{k}+c_{k}) \leq 9$, то код верен;
            \item если $10\leq(a_{k}+b_{k}+c_{k}) \leq 14$, то формируется перенос и $(a_{k}+b_{k}+c_{k}+6)-16$. Код $(a_{k}+b_{k}+c_{k}-10)$ верен.
        \end{enumerate}
        
        \item Если $5\leq a_{k}\leq 9$ и $0\leq b_{k}\leq 4$ код верен (аналогично п.\ref{bcd:aiken:diff}).

        \item Если $5\leq a_{k},b_{k}\leq 9$, то код $(a_{k}+b_{k}+c_{k}-10)+6$:
        \begin{enumerate}
            \item если $0\leq (a_{k}+b_{k}+c_{k}-10) \leq 4$, то неверно! Должно быть: $(a_{k}+b_{k}+c_{k} - 10)$. Поправка: $-6=\Number{1010}$.
            \item если $5\leq(a_{k}+b_{k}+c_{k}-10) \leq 9$, то код верен!
        \end{enumerate}
    \end{enumerate}
\end{frame}

\begin{frame}
    \frametitle{Код \AikenLabel}
    \framesubtitle{Когда нужны поправки?}

    \[
        s_k=a_k+b_k+c_k.
    \]
    Запрещенные комбинации кода:
    \[
        \AikenLabel\notin\{\Number{0101},\Number{0110},\Number{0111},\Number{1000},\Number{1001},\Number{1010}\}.
    \]
    Правила коррекции:
    \begin{enumerate}
        \item Если $0\leq a_{k},b_{k}\leq 4$ и $5\leq(a_{k}+b_{k}+c_{k})\leq 9$, то поправка: $+6=\Number{0110}$. 
        При этом $msb(\Aiken{a_k})=msb(\Aiken{b_k})=0$ и в результате получается одна из запрещенных комбинаций.
        
        \item Если $5\leq a_{k},b_{k}\leq 9$ и $0\leq(a_{k}+b_{k}+c_{k}-10)\leq 4$, то поправка: $-6=\Number{1010}$.
        При этом $msb(\Aiken{a_k})=msb(\Aiken{b_k})=1$ и в результате получается одна из запрещенных комбинаций.
    \end{enumerate}
\end{frame}

\begin{frame}
    \frametitle{Код \AikenLabel}
    \framesubtitle{Алгоритм сложения}
    
    \begin{enumerate}
        \item Перевести слагаемые в обратный $\AikenLabel$-код. Каждая тетрада модуля отрицательного числа инвертируется.
        \item Выполняется сложение полученных операднов по правилам двоичной арифметики.
        \item В соответствии с изложенными выше правилами, выполняется коррекция результата. Переносы при коррекциях не распространяются.
        \item Результат получен в обратном $\AikenLabel$-коде.
    \end{enumerate}
\end{frame}

\begin{frame}[allowframebreaks]
    \frametitle{Код $\AikenLabel$}
    \framesubtitle{Пример сложения $-365$ и $783$.}

    \begin{enumerate}
        \item Перевод в ОК (добавлены два знаковых двоичных разряда МОК):
        \begin{align*}
            -365\Rightarrow-\Number{0011 1100 1011}\\
            \OC{-365} =\Number{11 1100 0011 0100}\\
            \OC{783}  =\Number{00 1101 1110 0011}\\
        \end{align*}
        
        \item Выполняется сложение обратных кодов. 
        \[
            \Addition{11 1100 0011 0100}
                     {00 1101 1110 0011}
                   {1*00 1010 0001 0111}
        \]
        Коррекция переносом:
        \[
            \Addition{00 1010 0001 0111}
                     {.. .... .... ...1}
                     {00 1010 0001 1000}
        \]
        
        \item Выполняется коррекция. Переносы между тетрадами не распространяются.
        \begin{align*}
            \OC{-365} =\Number{11 1100 0011 0100}\\
            \OC{783}  =\Number{00 1101 1110 0011}\\
                     \Addition{00 1010 0001 1000}
                              {.. 1010 .... 0110}
                              {00 0100 0001 1110}
        \end{align*}
    \end{enumerate}
    
    ПРС не возникло, \[\OC{S}=\Number{00 0100 0001 1110},\] $S=418$.
\end{frame}


\section{Пятибитные коды}


\subsection{Код $\PentaLabel$}


\begin{frame}
    \frametitle{Код \PentaLabel}
    \framesubtitle{Пентадный код: $\Penta{a}=(3\cdot a + 2)$}
    
    \begin{center}
    \begin{tabular}{|c|c|c|}
        \hline\hline
        $a$ в 10СС  & $\Penta{a}$       & $\Penta{9-a}$\\
        \hline\hline
        $0$         & $\Number{00010}$  & $\Number{11101}$ \\
        $1$         & $\Number{00101}$  & $\Number{11010}$ \\
        $2$         & $\Number{01000}$  & $\Number{10111}$ \\
        $3$         & $\Number{01011}$  & $\Number{10100}$ \\
        $4$         & $\Number{01110}$  & $\Number{10001}$ \\
        $5$         & $\Number{10001}$  & $\Number{01110}$ \\
        $6$         & $\Number{10100}$  & $\Number{01011}$ \\
        $7$         & $\Number{10111}$  & $\Number{01000}$ \\
        $8$         & $\Number{11010}$  & $\Number{00101}$ \\
        $9$         & $\Number{11101}$  & $\Number{00010}$ \\
        \hline
    \end{tabular}
    \end{center}
    
    \[
        \overline{\Penta{a}}=31-(3a+2)=3(9-a)+2=\Penta{9-a}
    \]
\end{frame}

\begin{frame}
    \frametitle{Код \PentaLabel}
    \framesubtitle{Сложение $S=A+B$, где $A=(a_{n-1}\cdots a_0)$ и $B=(b_{n-1}\cdots b_0)$}

    \[
        s_k=a_k+b_k+c_k.
    \]
    
    \begin{enumerate}
        \item Если $0\leq a_{k}+b_{k}+c_{k}\leq 9$:
        \begin{enumerate}
            \item если $c_{k}=0$, то код $3(a_{k}+b_{k}) + 4$. Верный код: $3(a_{k}+b_{k})+2$. Поправка: $-2=\Number{11110}$:
            \item если $c_{k}=1$, то код $(3a_k+2)+(3b_{k}+2)+1=3(a_{k}+b_{k}+1)+2$. Код верен!
        \end{enumerate}

        \item Если $a_{k}+b_{k}+c_{k}\geq 10$:
        \begin{enumerate}
            \item если $c_{k}=0$, то код $(3a_{k}+2)+(3b_{k}+2)-32=3(a_{k}+ b_{k}-10)+2$. Код верен!
            \item если $c_{k}=1$, то код $(3a_{k}+2)+(3b_{k}+2)+1-32$, т.е. $3((a_{k}+b_{k}+1)-10)$. Верный код:
            $3((a_{k}+b_{k}+1)-10)+2$. Поправка: $+2=\Number{00010}$.
        \end{enumerate}
    \end{enumerate}
\end{frame}

\begin{frame}
    \frametitle{Код \PentaLabel}
    \framesubtitle{Алгоритм сложения}
    
    \begin{enumerate}
        \item Перевести слагаемые в обратный $\PentaLabel$-код. Каждая тетрада модуля отрицательного числа инвертируется.
        \item Выполняется сложение полученных операднов по правилам двоичной арифметики.
        \item Выполняется коррекция. Прибавляется код $11110_{2}$ к пентадам, в которые и из которых не формировались единицы переноса. Прибавляется код $00010_{2}$ к пентадам, в которые и из которых формировались единицы переноса. В процессе коррекции переносы из пентады в пентаду не распространяются.
        \item Результат получен в обратном $\PentaLabel$-коде.
    \end{enumerate}
\end{frame}

\begin{frame}[allowframebreaks]
    \frametitle{Код \PentaLabel}
    \framesubtitle{Пример сложения $-6425$ и $4985$}

    \begin{enumerate}
        \item Перевод в ОК (добавлены два знаковых двоичных разряда МОК):
        \begin{align*}
            -6425\Rightarrow-\Number{10100 01110 01000 10001}\\
            \OC{-6425}   =\Number{11 01011 10001 10111 01110}\\
            \OC{4985}    =\Number{00 01110 11101 11010 10001}
        \end{align*}
        
        \item Выполняется сложение обратных кодов. 
        \[
            \Addition{11 01011 10001 10111 01110}
                     {00 01110 11101 11010 10001}
                     {11 11010*01111*10001 11111}
        \]
        
        \item Выполняется коррекция. Переносы между тетрадами не распространяются.
        \[
            \Addition{11 11010*01111*10001 11111}
                     {.. ..... 00010 ..... 11110}
                     {11 11010 10001 10001 11101}
        \]
    \end{enumerate}
    
    ПРС не возникло, 
    \begin{align*}
        \OC{S}=\Number{11 11010 10001 10001 11101},\\
        \overline{\OC{S}}=\Number{00 00101 01110 01110 00010},\\
        S=-1440.
    \end{align*} 
\end{frame}


\appendix


\section{Задания на практику}


\subsection{Проходное}

\begin{frame}
    \frametitle{\TaskSimpleNumber}
    
    Всеми рассмотренными способами решить <<универсальные>> примеры:
    \begin{enumerate}
        \item \(
            \Addition{+27837}
                     {-14657}
                     {+13180}
        \)

        \item \(
            \Addition{-53043}
                     {-31776}
                     {-84819}
        \)
    \end{enumerate}
\end{frame}


\section{Самообучение}


\begin{frame}
    \frametitle{Советы самоучке}

    Двоично-десятичные коды обсуждаются, например, в \cite{bib:zubchuk:schemo, bib:lisikov:automateBase}.
\end{frame}

\begin{frame}[allowframebreaks]{Библиография}
    \bibliographystyle{gost780u}
    \bibliography{./../../../bibliobase}
\end{frame}

\end{document}




