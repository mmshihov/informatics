%include part: see main.beamer.tex and main.article.tex
%include common packages and settings
\usepackage{etex} %эта магическая херь избавляет от переполнения регистров TeX а!!!

\mode<article>{\usepackage{fullpage}}
\mode<presentation>{
    \usetheme{Madrid} %%Boadilla,Madrid,AnnArbor,CambridgeUS,Malmoe,Singapore,Berlin
    \useoutertheme{shadow}
} 

\usepackage[utf8]{inputenc}
\usepackage[russian]{babel}
\usepackage{indentfirst}
\usepackage{graphicx}

\usepackage{amsmath}
\usepackage{amsfonts}
\usepackage{amsthm}
\usepackage{algorithm}
\usepackage{algorithmic}

\usepackage[all]{xy}

\date{Лекция по дисциплине <<информатика>>\\(\today)}
\author[М.~М.~Шихов]{Михаил Шихов \\ \texttt{\underline{m.m.shihov@gmail.com}}}

%для рисования графов пакетом xy-pic
\entrymodifiers={++[o][F-]}

%для псевдокода алгоритмов (algorithm,algorithmic)
\renewcommand{\algorithmicrequire}{\textbf{Вход:}}
\renewcommand{\algorithmicensure}{\textbf{Выход:}}
\renewcommand{\algorithmiccomment}[1]{// #1}
\floatname{algorithm}{Псевдокод}

%%определённые мной команды логической разметки
\newcommand{\Signs}[2]{\fbox{{\small{\underbar{#1}}}{#2}}}
\newcommand{\Sign}[1]{\fbox{#1}}
\newcommand{\DC}[1]{\text{ДК}(#1)}
\newcommand{\OC}[1]{\text{ОК}(#1)}
\newcommand{\PC}[1]{\text{ПК}(#1)}

\newcommand{\MyProc}{\text{$\mathbf{R}_8$}}

\newboolean{IsNeedAnswer}
\setboolean{IsNeedAnswer}{false} %true/false

\newcommand{\Machine}[1]{\texttt{#1}}
\newcommand{\Opcode}[1]{\texttt{\bf{#1}}}
\newcommand{\Operand}[1]{\texttt{#1}}
\newcommand{\CmdOneAddr}[2]{\text{\Opcode{#1} \Operand{#2}}}
\newcommand{\CmdTwoAddr}[3]{\text{\Opcode{#1} \Operand{#2}, \Operand{#3}}}
\newcommand{\CmdThreeAddr}[4]{\text{\Opcode{#1} {\Operand{#2}}, \Operand{#3}, \Operand{#4}}}

\newcommand{\ProofAnswer}[1]{
    \ifthenelse{\boolean{IsNeedAnswer}}{
        \begin{proof}[Ответ]
            #1
        \end{proof}
    }{}
}

\newcommand{\LabeledAnswer}[1]{
    \ifthenelse{\boolean{IsNeedAnswer}}{
        \emph{Отв.}: #1 \qed
    }{}
}

\newcommand{\PlainAnswer}[1]{
    \ifthenelse{\boolean{IsNeedAnswer}}{#1}{}
}

\newcommand{\UnsignedAny}[2]{\text{\upshape
    \begin{tabular}{lr}
        \tiny{#1} & \tiny{0}\\ 
        \hline
        \multicolumn{2}{|c|}{\texttt{#2}} \\ 
        \hline
    \end{tabular}
}}

\newcommand{\UnsignedByte}[1]{\UnsignedAny{7}{#1}}

\newcommand{\UnsignedTwoBytes}[1]{\UnsignedAny{15}{#1}}

\newcommand{\FixedAny}[4]{\text{\upshape
    \begin{tabular}{clr}
        \tiny{#1}  &\tiny{#2} & \tiny{0}\\ 
        \hline
        \multicolumn{1}{|c|}{\texttt{#3}} & \multicolumn{2}{|c|}{\texttt{#4}} \\ 
        \hline
    \end{tabular}
}}

\newcommand{\FixedByte}[2]{\FixedAny{7}{6}{#1}{#2}}

\newcommand{\FixedTwoBytes}[2]{\FixedAny{15}{14}{#1}{#2}}

\newcommand{\SignedAny}[4]{\text{\upshape
    \begin{tabular}{clr}
        \tiny{#1}  &\tiny{#2} & \tiny{0}\\ 
        \hline
        \multicolumn{1}{|c|}{\Machine{#3}} & \multicolumn{2}{|c|}{\Machine{#4}} \\ 
        \hline
    \end{tabular}
}}

\newcommand{\SignedNibble}[2]{\SignedAny{3}{2}{#1}{#2}}

\newcommand{\SignedByte}[2]{\SignedAny{7}{6}{#1}{#2}}

\newcommand{\SignedTwoBytes}[2]{\SignedAny{15}{14}{#1}{#2}}

\newcommand{\FloatMyHex}[4]{\text{\upshape
    \begin{tabular}{clrclr}
        \tiny{15}  &\tiny{14} & \tiny{6} & \tiny{5} & \tiny{4} & \tiny{0}\\ 
        \hline
        \multicolumn{1}{|c|}{\texttt{#1}} 
            & \multicolumn{2}{|c|}{\texttt{#2}} 
                & \multicolumn{1}{|c|}{\texttt{#3}} 
                    & \multicolumn{2}{|c|}{\texttt{#4}} \\ 
        \hline
    \end{tabular}
}}

\newcommand{\FloatMyCharHex}[3]{\text{\upshape
    \begin{tabular}{clrlr}
        \tiny{15}  &\tiny{14} & \tiny{6} & \tiny{5} & \tiny{0}\\ 
        \hline
        \multicolumn{1}{|c|}{\texttt{#1}} 
            & \multicolumn{2}{|c|}{\texttt{#2}} 
                & \multicolumn{2}{|c|}{\texttt{#3}} \\ 
        \hline
    \end{tabular}
}}

\newcommand{\FloatMyDcMantCharHex}[2]{\text{\upshape
    \begin{tabular}{lrlr}
        \tiny{15} & \tiny{6} & \tiny{5} & \tiny{0}\\ 
        \hline
        \multicolumn{2}{|c|}{\texttt{#1}} 
            & \multicolumn{2}{|c|}{\texttt{#2}} \\ 
        \hline
    \end{tabular}
}}

\newcommand{\FloatESShort}[3]{\text{\upshape
    \begin{tabular}{clrlr}
        \tiny{31}  &\tiny{30} & \tiny{24} & \tiny{23} & \tiny{0}\\ 
        \hline
        \multicolumn{1}{|c|}{\texttt{#1}} 
            & \multicolumn{2}{|c|}{\texttt{#2}} 
                & \multicolumn{2}{|c|}{\texttt{#3}} 
                    \\ 
        \hline
    \end{tabular}
}}

\newcommand{\FloatPCShort}[3]{\text{\upshape
    \begin{tabular}{clrlr}
        \tiny{31}  &\tiny{30} & \tiny{23} & \tiny{22} & \tiny{0}\\ 
        \hline
        \multicolumn{1}{|c|}{\texttt{#1}} 
            & \multicolumn{2}{|c|}{\texttt{#2}} 
                & \multicolumn{2}{|c|}{\texttt{#3}} 
                    \\ 
        \hline
    \end{tabular}
}}

\newcommand{\FloatMyOrderX}[4]{\text{\upshape
    \begin{tabular}{clrclr}
        \tiny{9}  &\tiny{8} & \tiny{4} & \tiny{3} & \tiny{2} & \tiny{0}\\ 
        \hline
        \multicolumn{1}{|c|}{\texttt{#1}} 
            & \multicolumn{2}{|c|}{\texttt{#2}} 
                & \multicolumn{1}{|c|}{\texttt{#3}} 
                    & \multicolumn{2}{|c|}{\texttt{#4}} \\ 
        \hline
    \end{tabular}
}}

\newcommand{\FloatMyCharX}[3]{\text{\upshape
    \begin{tabular}{clrlr}
        \tiny{9}  &\tiny{8} & \tiny{4} & \tiny{3} & \tiny{0}\\ 
        \hline
        \multicolumn{1}{|c|}{\texttt{#1}} 
            & \multicolumn{2}{|c|}{\texttt{#2}} 
                & \multicolumn{2}{|c|}{\texttt{#3}} \\ 
        \hline
    \end{tabular}
}}

\newcommand{\FloatMyDcOrderX}[3]{\text{\upshape
    \begin{tabular}{lrclr}
        \tiny{9} & \tiny{4} & \tiny{3} & \tiny{2} & \tiny{0}\\ 
        \hline
        \multicolumn{2}{|c|}{\texttt{#1}} 
            & \multicolumn{1}{|c|}{\texttt{#2}} 
                & \multicolumn{2}{|c|}{\texttt{#3}} \\ 
        \hline
    \end{tabular}
}}

\newcommand{\FloatMyDcCharX}[2]{\text{\upshape
    \begin{tabular}{lrlr}
        \tiny{9} & \tiny{4} & \tiny{3} & \tiny{0}\\ 
        \hline
        \multicolumn{2}{|c|}{\texttt{#1}} 
            & \multicolumn{2}{|c|}{\texttt{#2}} \\ 
        \hline
    \end{tabular}
}}


%--- СПЕЦИФИЧНЫЕ ДЛЯ УМНОЖЕНИЯ КОМАНДЫ ---------------------------------------------------------------------------------------------


\newcommand{\Number}[1]{
    \texttt{#1}
}

\newcommand{\NumberHi}[2]{
    \underline{\underline{\texttt{#1}}}\texttt{#2}
}

\newcommand{\NumberMid}[3]{
    \texttt{#1}\underline{\underline{\texttt{#2}}}\texttt{#3}
}

\newcommand{\NumberLo}[2]{
    \texttt{#1}\underline{\underline{\texttt{#2}}}
}

\newcommand{\Stack}[2]{
    \begin{tabular}[t]{@{}r@{}}
        {#1}\\ \hline
        {#2}\\ 
    \end{tabular}
}

\newcommand{\StackThree}[3]{
    \begin{tabular}[t]{@{}r@{}}
        {#1}\\ \hline
        {#2}\\ \hline
        {#3}\\
    \end{tabular}
}

\newcommand{\StackFour}[4]{
    \begin{tabular}[t]{@{}r@{}}
        {#1}\\ \hline
        {#2}\\ \hline
        {#3}\\ \hline
        {#4}\\
    \end{tabular}
}

\newcommand{\Operation}[4]{
    \begin{tabular}[t]{@{}r@{}}
        \texttt{#4}
        \begin{tabular}{@{}r@{}}
            \Number{#1}\\
            \Number{#2}\\ \hline
        \end{tabular} \\ 
        \Number{#3}\\
    \end{tabular}
}

\newcommand{\Addition}[3]{\Operation{#1}{#2}{#3}{+}}

\newcommand{\Subtraction}[3]{\Operation{#1}{#2}{#3}{-}}

\newcommand{\Register}[2]{\Number{#1:#2}}

\newcommand{\Mantiss}{m}
\newcommand{\Order}{p}
\newcommand{\Char}{c}

\newcommand{\MantissOf}[1]{\Mantiss_{#1}}
\newcommand{\OrderOf}[1]{\Order_{#1}}
\newcommand{\CharOf}[1]{\Char_{#1}}

\newcommand{\FloatExpression}[2]{\MantissOf{#1}\cdot {#2}^{\OrderOf{#1}}}

\newcommand{\DivAnswer}[2]{(\texttt{$#1$ rem $#2$})}

\newenvironment{Solve}[1]%
    {\begin{proof}[Решение]#1}
    {\end{proof}}
    
    
%определённые мной команды логической разметки
\newcommand{\Signs}[2]{\fbox{{\small{\underbar{#1}}}{#2}}}
\newcommand{\Sign}[1]{\fbox{#1}}
\newcommand{\DC}[1]{\text{ДК}(#1)}
\newcommand{\OC}[1]{\text{ОК}(#1)}
\newcommand{\PC}[1]{\text{ПК}(#1)}

\newcommand{\MyProc}{\text{$\mathbf{R}_8$}}

\newboolean{IsNeedAnswer}
\setboolean{IsNeedAnswer}{false} %true/false

\newcommand{\Machine}[1]{\texttt{#1}}
\newcommand{\Opcode}[1]{\texttt{\bf{#1}}}
\newcommand{\Operand}[1]{\texttt{#1}}
\newcommand{\CmdOneAddr}[2]{\text{\Opcode{#1} \Operand{#2}}}
\newcommand{\CmdTwoAddr}[3]{\text{\Opcode{#1} \Operand{#2}, \Operand{#3}}}
\newcommand{\CmdThreeAddr}[4]{\text{\Opcode{#1} {\Operand{#2}}, \Operand{#3}, \Operand{#4}}}

\newcommand{\ProofAnswer}[1]{
    \ifthenelse{\boolean{IsNeedAnswer}}{
        \begin{proof}[Ответ]
            #1
        \end{proof}
    }{}
}

\newcommand{\LabeledAnswer}[1]{
    \ifthenelse{\boolean{IsNeedAnswer}}{
        \emph{Отв.}: #1 \qed
    }{}
}

\newcommand{\PlainAnswer}[1]{
    \ifthenelse{\boolean{IsNeedAnswer}}{#1}{}
}

\newcommand{\UnsignedAny}[2]{\text{\upshape
    \begin{tabular}{lr}
        \tiny{#1} & \tiny{0}\\ 
        \hline
        \multicolumn{2}{|c|}{\texttt{#2}} \\ 
        \hline
    \end{tabular}
}}

\newcommand{\UnsignedByte}[1]{\UnsignedAny{7}{#1}}

\newcommand{\UnsignedTwoBytes}[1]{\UnsignedAny{15}{#1}}

\newcommand{\FixedAny}[4]{\text{\upshape
    \begin{tabular}{clr}
        \tiny{#1}  &\tiny{#2} & \tiny{0}\\ 
        \hline
        \multicolumn{1}{|c|}{\texttt{#3}} & \multicolumn{2}{|c|}{\texttt{#4}} \\ 
        \hline
    \end{tabular}
}}

\newcommand{\FixedByte}[2]{\FixedAny{7}{6}{#1}{#2}}

\newcommand{\FixedTwoBytes}[2]{\FixedAny{15}{14}{#1}{#2}}

\newcommand{\SignedAny}[4]{\text{\upshape
    \begin{tabular}{clr}
        \tiny{#1}  &\tiny{#2} & \tiny{0}\\ 
        \hline
        \multicolumn{1}{|c|}{\Machine{#3}} & \multicolumn{2}{|c|}{\Machine{#4}} \\ 
        \hline
    \end{tabular}
}}

\newcommand{\SignedNibble}[2]{\SignedAny{3}{2}{#1}{#2}}

\newcommand{\SignedByte}[2]{\SignedAny{7}{6}{#1}{#2}}

\newcommand{\SignedTwoBytes}[2]{\SignedAny{15}{14}{#1}{#2}}

\newcommand{\FloatMyHex}[4]{\text{\upshape
    \begin{tabular}{clrclr}
        \tiny{15}  &\tiny{14} & \tiny{6} & \tiny{5} & \tiny{4} & \tiny{0}\\ 
        \hline
        \multicolumn{1}{|c|}{\texttt{#1}} 
            & \multicolumn{2}{|c|}{\texttt{#2}} 
                & \multicolumn{1}{|c|}{\texttt{#3}} 
                    & \multicolumn{2}{|c|}{\texttt{#4}} \\ 
        \hline
    \end{tabular}
}}

\newcommand{\FloatMyCharHex}[3]{\text{\upshape
    \begin{tabular}{clrlr}
        \tiny{15}  &\tiny{14} & \tiny{6} & \tiny{5} & \tiny{0}\\ 
        \hline
        \multicolumn{1}{|c|}{\texttt{#1}} 
            & \multicolumn{2}{|c|}{\texttt{#2}} 
                & \multicolumn{2}{|c|}{\texttt{#3}} \\ 
        \hline
    \end{tabular}
}}

\newcommand{\FloatMyDcMantCharHex}[2]{\text{\upshape
    \begin{tabular}{lrlr}
        \tiny{15} & \tiny{6} & \tiny{5} & \tiny{0}\\ 
        \hline
        \multicolumn{2}{|c|}{\texttt{#1}} 
            & \multicolumn{2}{|c|}{\texttt{#2}} \\ 
        \hline
    \end{tabular}
}}

\newcommand{\FloatESShort}[3]{\text{\upshape
    \begin{tabular}{clrlr}
        \tiny{31}  &\tiny{30} & \tiny{24} & \tiny{23} & \tiny{0}\\ 
        \hline
        \multicolumn{1}{|c|}{\texttt{#1}} 
            & \multicolumn{2}{|c|}{\texttt{#2}} 
                & \multicolumn{2}{|c|}{\texttt{#3}} 
                    \\ 
        \hline
    \end{tabular}
}}

\newcommand{\FloatPCShort}[3]{\text{\upshape
    \begin{tabular}{clrlr}
        \tiny{31}  &\tiny{30} & \tiny{23} & \tiny{22} & \tiny{0}\\ 
        \hline
        \multicolumn{1}{|c|}{\texttt{#1}} 
            & \multicolumn{2}{|c|}{\texttt{#2}} 
                & \multicolumn{2}{|c|}{\texttt{#3}} 
                    \\ 
        \hline
    \end{tabular}
}}

\newcommand{\FloatMyOrderX}[4]{\text{\upshape
    \begin{tabular}{clrclr}
        \tiny{9}  &\tiny{8} & \tiny{4} & \tiny{3} & \tiny{2} & \tiny{0}\\ 
        \hline
        \multicolumn{1}{|c|}{\texttt{#1}} 
            & \multicolumn{2}{|c|}{\texttt{#2}} 
                & \multicolumn{1}{|c|}{\texttt{#3}} 
                    & \multicolumn{2}{|c|}{\texttt{#4}} \\ 
        \hline
    \end{tabular}
}}

\newcommand{\FloatMyCharX}[3]{\text{\upshape
    \begin{tabular}{clrlr}
        \tiny{9}  &\tiny{8} & \tiny{4} & \tiny{3} & \tiny{0}\\ 
        \hline
        \multicolumn{1}{|c|}{\texttt{#1}} 
            & \multicolumn{2}{|c|}{\texttt{#2}} 
                & \multicolumn{2}{|c|}{\texttt{#3}} \\ 
        \hline
    \end{tabular}
}}

\newcommand{\FloatMyDcOrderX}[3]{\text{\upshape
    \begin{tabular}{lrclr}
        \tiny{9} & \tiny{4} & \tiny{3} & \tiny{2} & \tiny{0}\\ 
        \hline
        \multicolumn{2}{|c|}{\texttt{#1}} 
            & \multicolumn{1}{|c|}{\texttt{#2}} 
                & \multicolumn{2}{|c|}{\texttt{#3}} \\ 
        \hline
    \end{tabular}
}}

\newcommand{\FloatMyDcCharX}[2]{\text{\upshape
    \begin{tabular}{lrlr}
        \tiny{9} & \tiny{4} & \tiny{3} & \tiny{0}\\ 
        \hline
        \multicolumn{2}{|c|}{\texttt{#1}} 
            & \multicolumn{2}{|c|}{\texttt{#2}} \\ 
        \hline
    \end{tabular}
}}


%--- СПЕЦИФИЧНЫЕ ДЛЯ УМНОЖЕНИЯ КОМАНДЫ ---------------------------------------------------------------------------------------------


\newcommand{\Number}[1]{
    \texttt{#1}
}

\newcommand{\NumberHi}[2]{
    \underline{\underline{\texttt{#1}}}\texttt{#2}
}

\newcommand{\NumberMid}[3]{
    \texttt{#1}\underline{\underline{\texttt{#2}}}\texttt{#3}
}

\newcommand{\NumberLo}[2]{
    \texttt{#1}\underline{\underline{\texttt{#2}}}
}

\newcommand{\Stack}[2]{
    \begin{tabular}[t]{@{}r@{}}
        {#1}\\ \hline
        {#2}\\ 
    \end{tabular}
}

\newcommand{\StackThree}[3]{
    \begin{tabular}[t]{@{}r@{}}
        {#1}\\ \hline
        {#2}\\ \hline
        {#3}\\
    \end{tabular}
}

\newcommand{\StackFour}[4]{
    \begin{tabular}[t]{@{}r@{}}
        {#1}\\ \hline
        {#2}\\ \hline
        {#3}\\ \hline
        {#4}\\
    \end{tabular}
}

\newcommand{\Operation}[4]{
    \begin{tabular}[t]{@{}r@{}}
        \texttt{#4}
        \begin{tabular}{@{}r@{}}
            \Number{#1}\\
            \Number{#2}\\ \hline
        \end{tabular} \\ 
        \Number{#3}\\
    \end{tabular}
}

\newcommand{\Addition}[3]{\Operation{#1}{#2}{#3}{+}}

\newcommand{\Subtraction}[3]{\Operation{#1}{#2}{#3}{-}}

\newcommand{\Register}[2]{\Number{#1:#2}}

\newcommand{\Mantiss}{m}
\newcommand{\Order}{p}
\newcommand{\Char}{c}

\newcommand{\MantissOf}[1]{\Mantiss_{#1}}
\newcommand{\OrderOf}[1]{\Order_{#1}}
\newcommand{\CharOf}[1]{\Char_{#1}}

\newcommand{\FloatExpression}[2]{\MantissOf{#1}\cdot {#2}^{\OrderOf{#1}}}

\newcommand{\DivAnswer}[2]{(\texttt{$#1$ rem $#2$})}

\newenvironment{Solve}[1]%
    {\begin{proof}[Решение]#1}
    {\end{proof}}
    
    

\title[Целочисленное деление]{Целочисленное деление}

\newcounter{TaskSimpleCtr}
\setcounter{TaskSimpleCtr}{1}
\newcommand{\TaskSimpleNumber}{ \arabic{TaskSimpleCtr}) \addtocounter{TaskSimpleCtr}{1} }

%вставка изображений из metapost (post script)
\DeclareGraphicsRule{*}{mps}{*}{}

\begin{document}

\mode<article>{\maketitle\tableofcontents}
\frame<presentation>{\titlepage}
\begin{frame}<presentation>
    \frametitle{Содержание}
    \tableofcontents
\end{frame}


\begin{frame}
    \frametitle{Целочисленное деление}
    
    В результате деления числа $A$ (делимое) на число $d$ (делитель) получается частное $q$ и остаток $\Delta$, такие, что выполняется равенство:
    \[
        A = q \cdot d + \Delta,
    \]
    где $A, q, d, \Delta$ --- целые, а $|\Delta|<|d|$.

    Результат деления будем записывать как $A\div d=q \Machine{ rem } \Delta$, например:
    \[
        7\div 3=\Machine{2 rem 1}.
    \]
    
    \begin{block}{}
        Результат целочисленного деления, как обратной умножению операции, получается серией вычитаний и сдвигов.
    \end{block}
\end{frame}

\begin{frame}
    \frametitle{Целочисленное деление (10CC), $52938\div 43$}
    \framesubtitle{Англо-американская система}
    
    \begin{center}
    \resizebox{!}{.8\height} {
    \begin{tabular}{cccccc|c|l}
                      &\uncover<2->{0}&\uncover<3->{1}&\uncover<4->{2}&\uncover<5->{3}&\uncover<6->{1}&        & частное\\ \hline\hline
                      &5              &2              &9              &3              &8              &$\div43$& делимое $\div$ делитель\\ \hline\hline
                      &\uncover<2->{5}&               &               &               &               &        & \\
       \uncover<2->{-}&\uncover<2->{0}&               &               &               &               &        & \\
       \uncover<2->{=}&\uncover<2->{5}&               &               &               &               &        & \uncover<2->{$\Delta_1$}\\ \hline
                      &\uncover<3->{5}&\uncover<3->{2}&               &               &               &        & \\
       \uncover<3->{-}&\uncover<3->{4}&\uncover<3->{3}&               &               &               &        & \\
       \uncover<3->{=}&               &\uncover<3->{9}&               &               &               &        & \uncover<3->{$\Delta_2$}\\ \hline
                      &               &\uncover<4->{9}&\uncover<4->{9}&               &               &        & \\
                      &\uncover<4->{-}&\uncover<4->{8}&\uncover<4->{6}&               &               &        & \\
                      &\uncover<4->{=}&\uncover<4->{1}&\uncover<4->{3}&               &               &        & \uncover<4->{$\Delta_3$}\\ \hline
                      &               &\uncover<5->{1}&\uncover<5->{3}&\uncover<5->{3}&               &        & \\
                      &\uncover<5->{-}&\uncover<5->{1}&\uncover<5->{2}&\uncover<5->{9}&               &        & \\
                      &\uncover<5->{=}&               &               &\uncover<5->{4}&               &        & \uncover<5->{$\Delta_4$}\\ \hline
                      &               &               &               &\uncover<6->{4}&\uncover<6->{8}&        & \\
                      &               &               &\uncover<6->{-}&\uncover<6->{4}&\uncover<6->{3}&        & \\
                      &               &               &\uncover<6->{=}&               &\uncover<6->{5}&        & \uncover<6->{$\Delta_5$}\\ \hline
                      &               &               &               &               &\uncover<7->{5}&        & \uncover<7->{$52938\div 43=\Machine{1231 rem 5}$.}
    \end{tabular}
    }
    \end{center}
\end{frame}


\begin{frame}
    \frametitle{Целочисленное деление (2CC), $10\div 3$}
    \framesubtitle{Англо-американская система}
    
    \begin{center}
    \resizebox{!}{.9\height} {
    \begin{tabular}{ccccc|c|l}
                       &\uncover<2->{0}&\uncover<3->{0}&\uncover<4->{1}&\uncover<5->{1}&          & частное \\ \hline\hline
                       &1              &0              &1              &0              &$\div$11  & делимое $\div$ делитель\\ \hline\hline
                       &\uncover<2->{1}&               &               &               &          & \\
        \uncover<2->{-}&\uncover<2->{0}&               &               &               &          & \\ 
        \uncover<2->{=}&\uncover<2->{1}&               &               &               &          &\uncover<2->{$\Delta_1$} \\ \hline
                       &\uncover<3->{1}&\uncover<3->{0}&               &               &          & \\
        \uncover<3->{-}&               &\uncover<3->{0}&               &               &          & \\ 
        \uncover<3->{=}&\uncover<3->{1}&\uncover<3->{0}&               &               &          &\uncover<3->{$\Delta_2$} \\ \hline
                       &\uncover<4->{1}&\uncover<4->{0}&\uncover<4->{1}&               &          & \\
        \uncover<4->{-}&               &\uncover<4->{1}&\uncover<4->{1}&               &          & \\ 
        \uncover<4->{=}&               &\uncover<4->{1}&\uncover<4->{0}&               &          &\uncover<4->{$\Delta_3$} \\ \hline
                       &               &\uncover<5->{1}&\uncover<5->{0}&\uncover<5->{0}&          & \\
                       &\uncover<5->{-}&               &\uncover<5->{1}&\uncover<5->{1}&          & \\ 
                       &\uncover<5->{=}&               &               &\uncover<5->{1}&          &\uncover<5->{$\Delta_4$} \\ \hline
                       &               &               &               &\uncover<6->{1}&          &\uncover<6->{$(1010)_2\div (11)_2=\Machine{$(11)_2$ rem 1}$}\\ 
    \end{tabular}
    }
    \end{center}
\end{frame}

\begin{frame}
    \frametitle{Беззнаковое целочисленное деление}
    \framesubtitle{Первый способ}

    \only<1> {
        Начальное состояние:
        \begin{center}
            \includegraphics[width=0.85\textwidth]{fig/ibegin}
        \end{center}
    }
    \only<2> {
        Конечное состояние:
        \begin{center}
            \includegraphics[width=0.85\textwidth]{fig/iend}
        \end{center}
    }
\end{frame}

\begin{frame}
    \frametitle{Целочисленное деление}
    \framesubtitle{Второй способ}

    \only<1> {
        Начальное состояние:
        \begin{center}
            \includegraphics[width=0.85\textwidth]{fig/iibegin}
        \end{center}
    }
    \only<2> {
        Конечное состояние:
        \begin{center}
            \includegraphics[width=0.85\textwidth]{fig/iiend}
        \end{center}
    }
\end{frame}

\begin{frame}
    \frametitle{На ноль делить нельзя!}

    Все приведенные ниже алгоритмы работают при условии, что не получают на входе нулевой делитель.
\end{frame}


\section{Беззнаковое целочисленное деление}


\begin{frame}
    \frametitle{Беззнаковое целочисленное деление $A\div d=q\Machine{ rem } \Delta$}
    
    \[
        A = d \cdot q + \Delta,
    \]
    
    В результате деления положительных чисел делимого $A$ на делитель $d$ получаемые в результате частное $q$ и остаток $\Delta$ --- также положительны. 
\end{frame}


\subsection{Деление с восстановлением остатков}


\begin{frame}
    \frametitle{Алгоритм деления с восстановлением остатков}
    \framesubtitle{$n$-разрядные беззнаковые целые}

    \begin{enumerate}
        \item $i\gets 1$; В младшую часть регистра остатков заносится делимое, в старшую часть регистра делителя --- делитель. Далее состояние регистра остатков --- остаток ($\Delta$), регистра делителя --- делитель ($d$), регистра частного ($q$) --- частное.
        \item\label{div:vo:start} Выполнить сдвиги: частное $q$ влево, остаток $\Delta$ влево (в первом способе), делитель $d$ вправо (во втором способе).
        \item Получить новый остаток $\Delta\gets(\Delta - d)$;
        \item Если $\Delta < 0$, то в младший разряд частного занести 0, иначе 1.
        \item Если $\Delta < 0$, то выполнить восстановление старого значения остатка: $\Delta\gets(\Delta + d)$.
        \item $i\gets (i + 1)$. Если $i\le n$, то к шагу \ref{div:vo:start}.
        \item В регистре частного получено значение частного, в регистре остатков --- $n$-разрядный остаток (в первом способе в старших разрядах, во втором --- в младших).
    \end{enumerate}
\end{frame}

\begin{frame}
    \frametitle{Алгоритм деления с восстановлением остатков}
    \framesubtitle{Примечания}

    \begin{itemize}
        \item В регистр остатков и регистр делителя добавлены знаковые разряды.
    \end{itemize}
\end{frame}

\begin{frame}
    \frametitle{Деление с восстановлением остатков I-й способ}
    \framesubtitle{$46\div 5$}
    
    \resizebox{.85\textwidth}{!}{
        \begin{tabular}{c|r|r|l}
            \hline\hline
            Частное $q, \leftarrow$ 
                & \multicolumn{1}{|c|}{дел-е, $\Delta$ $\leftarrow$}
                    & \multicolumn{1}{|c|}{дел-ль, $d$}
                        & прим. \\ 
            \hline\hline
            \Number{......}
                & \Number{.,...... 101110}
                    & \Number{.,...101}
                        & операнды;\\ \hline\hline
            \Number{......}
                & \Number{.,.....1 01110.}
                    & 
                        & сдвиг;\\ \hline
            \Number{.....0}
                & \Addition{.,.....1 01110.}
                           {1,111011 ......}
                           {1,111100 01110.}
                    & 
                        & $\Delta_1<0$;\\ \hline
            \Number{.....0}
                & \Addition{1,111100 01110.}
                           {.,...101 ......}
                           {.,.....1 01110.}
                    & 
                        & Восст. $\Delta_1$;
        \end{tabular}
    }
\end{frame}

\begin{frame}
    \frametitle{Деление с восстановлением остатков I-й способ (2)}
    \framesubtitle{$46\div 5$}
    
    \resizebox{.85\textwidth}{!}{
        \begin{tabular}{c|r|r|l}
            Частное $q, \leftarrow$ 
                & \multicolumn{1}{|c|}{дел-е, $\Delta$ $\leftarrow$}
                    & \multicolumn{1}{|c|}{дел-ль, $d$}
                        & прим. \\ 
            \hline\hline
            \Number{.....0}
                & \Number{.,.....1 01110.}
                    & \Number{.,...101}
                        & $\uparrow$ \\ \hline\hline
            \Number{....0.}
                & \Number{.,....10 1110..}
                    & 
                        & сдвиг;\\ \hline
            \Number{....00}
                & \Addition{.,....10 1110..}
                           {1,111011 ......}
                           {1,111101 1110..}
                    & 
                        & $\Delta_2<0$;\\ \hline
            \Number{....00}
                & \Addition{1,111101 1110..}
                           {.,...101 ......}
                           {.,....10 1110..}
                    & 
                        & Восст. $\Delta_2$;\\ \hline
            \Number{...00.}
                & \Number{.,...101 110...}
                    & 
                        & сдвиг;\\ \hline
            \Number{...001}
                & \Addition{.,...101 110...}
                           {1,111011 ......}
                           {.,...... 110...}
                    & 
                        & $\Delta_3\ge 0$;\\ \hline
            \Number{..001.}
                & \Number{.,.....1 10....}
                    & 
                        & сдвиг;
        \end{tabular}
    }
\end{frame}

\begin{frame}
    \frametitle{Деление с восстановлением остатков I-й способ (3)}
    \framesubtitle{$46\div 5$}
    
    \resizebox{.85\textwidth}{!}{
        \begin{tabular}{c|r|r|l}
            Частное $q, \leftarrow$ 
                & \multicolumn{1}{|c|}{дел-е, $\Delta$ $\leftarrow$}
                    & \multicolumn{1}{|c|}{дел-ль, $d$}
                        & прим. \\ 
            \hline\hline
            \Number{..001.}
                & \Number{.,.....1 10....}
                    & \Number{.,...101}
                        & $\uparrow$ \\ \hline\hline
            \Number{..0010}
                & \Addition{.,.....1 10....}
                           {1,111011 ......}
                           {1,111100 10....}
                    & 
                        & $\Delta_4<0$;\\ \hline
            \Number{..0010}
                & \Addition{1,111100 10....}
                           {.,...101 ......}
                           {.,.....1 10....}
                    & 
                        & Восст. $\Delta_4$;\\ \hline
            \Number{.0010.}
                & \Number{.,....11 0.....}
                    & 
                        & сдвиг;\\ \hline
            \Number{.00100}
                & \Addition{.,....11 0.....}
                           {1,111011 ......}
                           {1,111111 0.....}
                    & 
                        & $\Delta_5<0$;\\ \hline
            \Number{.00100}
                & \Addition{1,111111 0.....}
                           {.,...101 ......}
                           {.,....11 0.....}
                    & 
                        & Восст. $\Delta_5$;
        \end{tabular}
    }
\end{frame}

\begin{frame}
    \frametitle{Деление с восстановлением остатков I-й способ (4)}
    \framesubtitle{$46\div 5 = \Machine{9 rem 1}$}
    
    \resizebox{.85\textwidth}{!}{
        \begin{tabular}{c|r|r|l}
            Частное $q, \leftarrow$ 
                & \multicolumn{1}{|c|}{дел-е, $\Delta$ $\leftarrow$}
                    & \multicolumn{1}{|c|}{дел-ль, $d$}
                        & прим. \\ 
            \hline\hline
            \Number{.00100}
                & \Number{.,....11 0.....}
                    & \Number{.,...101}
                        & $\uparrow$ \\ \hline\hline
            \Number{00100.}
                & \Number{.,...110 ......}
                    & 
                        & сдвиг;\\ \hline
            \Number{001001}
                & \Addition{.,...110 ......}
                           {1,111011 ......}
                           {.,000001 ......}
                    & 
                        & $\Delta_6>0$;\\ \hline
        \end{tabular}
    }
    
    \begin{align*}
        q       &= (001001)_2 = 9\\
        \Delta  &= (000001)_2 = 1
    \end{align*}
\end{frame}


\begin{frame}
    \frametitle{Деление с восстановлением остатков II-й способ}
    \framesubtitle{$53\div 5$}
    
    \resizebox{.95\textwidth}{!}{
        \begin{tabular}{c|r|r|l}
            \hline\hline
            Частное $q, \leftarrow$ 
                & \multicolumn{1}{|c|}{дел-е, $\Delta$}
                    & \multicolumn{1}{|c|}{дел-ль, $d$ $\rightarrow$}
                        & прим. \\ 
            \hline\hline
            \Number{......}
                & \Number{.,...... 110101}
                    & \Number{.,...101 ......}
                        & операнды;\\ \hline\hline
            \Number{......}
                & \Number{.,...... 110101}
                    & \Number{.,....10 1.....}
                        & сдвиг;\\ \hline
            \Number{.....0}
                & \Addition{.,...... 110101}
                           {1,111101 1.....}
                           {1,111110 010101}
                    & 
                        & $\Delta_1<0$;\\ \hline
            \Number{.....0}
                & \Addition{1,111110 010101}
                           {.,....10 1.....}
                           {.,...... 110101}
                    & 
                        & Восст. $\Delta_1$;
        \end{tabular}
    }
\end{frame}

\begin{frame}
    \frametitle{Деление с восстановлением остатков II-й способ (2)}
    \framesubtitle{$53\div 5$}
    
    \resizebox{.85\textwidth}{!}{
        \begin{tabular}{c|r|r|l}
            Частное $q, \leftarrow$ 
                & \multicolumn{1}{|c|}{дел-е, $\Delta$}
                    & \multicolumn{1}{|c|}{дел-ль, $d$ $\rightarrow$}
                        & прим. \\ 
            \hline\hline
            \Number{.....0}
                & \Number{.,...... 110101}
                    & \Number{.,....10 1.....}
                        & $\uparrow$ \\ \hline\hline
            \Number{....0.}
                & \Number{.,...... 110101}
                    & \Number{.,.....1 01....}
                        & сдвиг;\\ \hline
            \Number{....00}
                & \Addition{.,...... 110101}
                           {1,111110 11....}
                           {1,111111 100101}
                    & 
                        & $\Delta_2<0$;\\ \hline
            \Number{....00}
                & \Addition{1,111111 100101}
                           {.,.....1 01....}
                           {.,...... 110101}
                    & 
                        & Восст. $\Delta_2$;\\ \hline
            \Number{...00.}
                & \Number{.,...... 110101}
                    & \Number{.,...... 101...}
                        & сдвиг;\\ \hline
            \Number{...001}
                & \Addition{.,...... 110101}
                           {1,111111 011...}
                           {.,...... ..1101}
                    & 
                        & $\Delta_3\ge 0$;\\ \hline
            \Number{..001.}
                & \Number{.,...... ..1101}
                    & \Number{.,...... .101..}
                        & сдвиг;
        \end{tabular}
    }
\end{frame}

\begin{frame}
    \frametitle{Деление с восстановлением остатков II-й способ (3)}
    \framesubtitle{$53\div 5$}
    
    \resizebox{.85\textwidth}{!}{
        \begin{tabular}{c|r|r|l}
            Частное $q, \leftarrow$ 
                & \multicolumn{1}{|c|}{дел-е, $\Delta$}
                    & \multicolumn{1}{|c|}{дел-ль, $d$ $\rightarrow$}
                        & прим. \\ 
            \hline\hline
            \Number{..001.}
                & \Number{.,...... ..1101}
                    & \Number{.,...... .101..}
                        & $\uparrow$ \\ \hline\hline
            \Number{..0010}
                & \Addition{.,...... ..1101}
                           {1,111111 1011..}
                           {1,111111 111001}
                    & 
                        & $\Delta_4<0$;\\ \hline
            \Number{..0010}
                & \Addition{1,111111 111001}
                           {.,...... .101..}
                           {.,...... ..1101}
                    & 
                        & Восст. $\Delta_4$;\\ \hline
            \Number{.0010.}
                & \Number{.,...... ..1101}
                    & \Number{.,...... ..101.}
                        & сдвиг;\\ \hline
            \Number{.00101}
                & \Addition{.,...... ..1101}
                           {1,111111 11011.}
                           {.,...... ....11}
                    & 
                        & $\Delta_5\ge 0$;\\ \hline
            \Number{00101.}
                & \Number{.,...... ....11}
                    & \Number{.,...... ...101}
                        & сдвиг;
        \end{tabular}
    }
\end{frame}

\begin{frame}
    \frametitle{Деление с восстановлением остатков II-й способ (4)}
    \framesubtitle{$53\div 5 = \Machine{10 rem 3}$}
    
    \resizebox{.85\textwidth}{!}{
        \begin{tabular}{c|r|r|l}
            Частное $q, \leftarrow$ 
                & \multicolumn{1}{|c|}{дел-е, $\Delta$}
                    & \multicolumn{1}{|c|}{дел-ль, $d$ $\rightarrow$}
                        & прим. \\ 
            \hline\hline
            \Number{00101.}
                & \Number{.,...... ....11}
                    & \Number{.,...... ...101}
                        & $\uparrow$ \\ \hline\hline
            \Number{001010}
                & \Addition{.,...... ....11}
                           {1,111111 111011}
                           {1,111111 111110}
                    & 
                        & $\Delta_6<0$;\\ \hline
            \Number{001010}
                & \Addition{1,111111 111110}
                           {.,...... ...101}
                           {.,...... 000011}
                    & 
                        & Восст. $\Delta_6$;\\ \hline
        \end{tabular}
    }
    
    \begin{align*}
        q       &= (001010)_2 = 10\\
        \Delta  &= (000011)_2 = 3
    \end{align*}
\end{frame}


\subsection{Деление без восстановления остатков}


\begin{frame}
    \frametitle{Деление без восстановления остатков}

    Если новый остаток $\Delta$ получается отрицательным, то к нему прибавляется делитель, чтобы восстановить старое (положительное) значение остатка. Чтобы не тратить на это время --- проследим, что происходит к моменту получения следующего остатка $\Delta'$.
    
    \begin{itemize}
        \item В первом способе: 
        \[
            \Delta' = 
                \begin{cases}
                    2\cdot\Delta + d, & \text{ если $\Delta<0$: $2\cdot(\underbrace{\Delta + d}_\text{В.О.}) - d = 2\cdot\Delta + d$,}\\
                    2\cdot\Delta - d, & \text{ если $\Delta\ge 0$.}
                \end{cases}
        \]
        \item Во втором способе:
        \[
            \Delta' = 
                \begin{cases}
                    \Delta + d/2, & \text{ если $\Delta<0$: $(\underbrace{\Delta + d}_\text{В.О.}) - d/2 = \Delta + d/2$,}\\
                    \Delta - d/2, & \text{ если $\Delta\ge 0$.}
                \end{cases}
        \]
    \end{itemize}
\end{frame}

\begin{frame}
    \frametitle{Алгоритм деления без восстановления остатков}
    \framesubtitle{$n$-разрядные беззнаковые целые}

    \begin{enumerate}
        \item $i\gets 1$; В младшую часть регистра остатков заносится делимое, в старшую часть регистра делителя --- делитель. Далее состояние регистра остатков --- остаток ($\Delta$), регистра делителя --- делитель ($d$), регистра частного ($q$) --- частное.
        \item\label{div:wvo:start} Выполнить сдвиги: частное $q$ влево, остаток $\Delta$ влево (в первом способе), делитель $d$ вправо (во втором способе).
        \item Если $\Delta < 0$, то $\Delta\gets(\Delta + d)$, иначе $\Delta\gets(\Delta - d)$;
        \item Если $\Delta < 0$, то в младший разряд частного занести 0, иначе 1.
        \item $i\gets (i + 1)$. Если $\le n$, то к шагу \ref{div:wvo:start}.
        \item Восстановим остаток. Если $\Delta < 0$, то $\Delta\gets(\Delta + d)$.
        \item В регистре частного получено значение частного, в регистре остатков --- $n$-разрядный остаток (в первом способе в старших разрядах, во втором --- в младших).
    \end{enumerate}
\end{frame}

\begin{frame}
    \frametitle{Алгоритм деления без восстановления остатков}
    \framesubtitle{Примечания}

    \begin{itemize}
        \item В первом способе в регистре остатка добавлено \emph{два} разряда под знак: по младшему знаковому разряду судят о знаке полученного остатка, а по старшему судят о знаке остатка до его сдвига вправо.
    \end{itemize}
\end{frame}

\begin{frame}
    \frametitle{Деление без восстановления остатков I-й способ}
    \framesubtitle{$46\div 5$}
    
    \resizebox{.85\textwidth}{!}{
        \begin{tabular}{c|r|r|l}
            \hline\hline
            Частное $q, \leftarrow$ 
                & \multicolumn{1}{|c|}{дел-е, $\Delta$ $\leftarrow$}
                    & \multicolumn{1}{|c|}{дел-ль, $d$}
                        & прим. \\ 
            \hline\hline
            \Number{......}
                & \Number{00,...... 101110}
                    & \Number{..,...101}
                        & операнды;\\ \hline\hline
            \Number{......}
                & \Number{00,.....1 01110.}
                    & 
                        & сдвиг;\\ \hline
            \Number{.....0}
                & \Addition{00,.....1 01110.}
                           {11,111011 ......}
                           {11,111100 01110.}
                    & 
                        & $-d,\Delta_1<0$;\\ \hline
            \Number{....0.}
                & \Number{11,111000 1110..}
                    & 
                        & сдвиг;\\ \hline
            \Number{....00}
                & \Addition{11,111000 1110..}
                           {..,...101 ......}
                           {11,111101 1110..}
                    & 
                        & $+d,\Delta_2<0$;\\ \hline
            \Number{...00.}
                & \Number{11,111011 110...}
                    & 
                        & сдвиг;\\ \hline
            \Number{...001}
                & \Addition{11,111011 110...}
                           {..,...101 ......}
                           {00,000000 110...}
                    & 
                        & $+d,\Delta_3\ge 0$;\\ \hline
            \Number{..001.}
                & \Number{00,000001 10....}
                    & 
                        & сдвиг;\\ 
        \end{tabular}
    }
\end{frame}

\begin{frame}
    \frametitle{Деление без восстановления остатков I-й способ (2)}
    \framesubtitle{$46\div 5 = \Machine{9 rem 1}$}
    
    \resizebox{.85\textwidth}{!}{
        \begin{tabular}{c|r|r|l}
            Частное $q, \leftarrow$ 
                & \multicolumn{1}{|c|}{дел-е, $\Delta$ $\leftarrow$}
                    & \multicolumn{1}{|c|}{дел-ль, $d$}
                        & прим. \\ 
            \hline\hline
            \Number{..001.}
                & \Number{00,000001 10....}
                    & 
                        & $\uparrow$\\ \hline\hline
            \Number{..0010}
                & \Addition{00,000001 10....}
                           {11,111011 ......}
                           {11,111100 10....}
                    & 
                        & $-d,\Delta_4<0$;\\ \hline
            \Number{.0010.}
                & \Number{11,111001 0.....}
                    & 
                        & сдвиг;\\ \hline
            \Number{.00100}
                & \Addition{11,111001 0.....}
                           {..,...101 ......}
                           {11,111110 0.....}
                    & 
                        & $+d,\Delta_5<0$;\\ \hline
            \Number{00100.}
                & \Number{11,111100 ......}
                    & 
                        & сдвиг;\\ \hline
            \Number{001001}
                & \Addition{11,111100 ......}
                           {..,...101 ......}
                           {00,000001 ......}
                    & 
                        & $+d,\Delta_6\ge 0$;\\ \hline
        \end{tabular}
    }
    $q=(001001)_2=9; \Delta=(000001)_2=1$.
\end{frame}

\begin{frame}
    \frametitle{Деление без восстановления остатков II-й способ}
    \framesubtitle{$53\div 5$}
    
    \resizebox{.85\textwidth}{!}{
        \begin{tabular}{c|r|r|l}
            \hline\hline
            Частное $q, \leftarrow$ 
                & \multicolumn{1}{|c|}{дел-е, $\Delta$}
                    & \multicolumn{1}{|c|}{дел-ль, $d$ $\rightarrow$}
                        & прим. \\ 
            \hline\hline
            \Number{......}
                & \Number{.,...... 110101}
                    & \Number{.,...101 ......}
                        & операнды;\\ \hline\hline
            \Number{......}
                & 
                    & \Number{.,....10 1.....}
                        & сдвиг;\\ \hline
            \Number{.....0}
                & \Addition{.,...... 110101}
                           {1,111101 1.....}
                           {1,111110 010101}
                    & 
                        & $-d,\Delta_1<0$;\\ \hline
            \Number{....0.}
                & 
                    & \Number{.,.....1 01....}
                        & сдвиг;\\ \hline
            \Number{....00}
                & \Addition{1,111110 010101}
                           {.,.....1 01....}
                           {1,111111 100101}
                    & 
                        & $+d,\Delta_2<0$;\\ \hline
            \Number{...00.}
                & 
                    & \Number{.,...... 101...}
                        & сдвиг;\\ \hline
            \Number{...001}
                & \Addition{1,111111 100101}
                           {.,...... 101...}
                           {.,...... ..1101}
                    & 
                        & $+d,\Delta_3\ge 0$;\\ \hline
            \Number{..001.}
                & 
                    & \Number{.,...... .101..}
                        & сдвиг;\\ 
        \end{tabular}
    }
\end{frame}

\begin{frame}
    \frametitle{Деление без восстановления остатков II-й способ (2)}
    \framesubtitle{$53\div 5$}
    
    \resizebox{.85\textwidth}{!}{
        \begin{tabular}{c|r|r|l}
            Частное $q, \leftarrow$ 
                & \multicolumn{1}{|c|}{дел-е, $\Delta$}
                    & \multicolumn{1}{|c|}{дел-ль, $d$ $\rightarrow$}
                        & прим. \\ 
            \hline\hline
            \Number{..001.}
                & \Number{.,...... ..1101}
                    & \Number{.,...... .101..}
                        & $\uparrow$\\ \hline\hline
            \Number{..0010}
                & \Addition{.,...... ..1101}
                           {1,111111 1011..}
                           {1,111111 111001}
                    & 
                        & $-d,\Delta_4<0$;\\ \hline
            \Number{.0010.}
                & 
                    & \Number{.,...... ..101.}
                        & сдвиг;\\ \hline
            \Number{.00101}
                & \Addition{1,111111 111001}
                           {.,...... ..101.}
                           {.,...... ....11}
                    & 
                        & $+d,\Delta_5\ge 0$;\\ \hline
            \Number{00101.}
                & 
                    & \Number{.,...... ...101}
                        & сдвиг;\\ \hline
            \Number{001010}
                & \Addition{.,...... ....11}
                           {1,111111 111011}
                           {1,111111 111110}
                    & 
                        & $-d,\Delta_6<0$;\\ 
        \end{tabular}
    }
\end{frame}

\begin{frame}
    \frametitle{Деление без восстановления остатков II-й способ (3)}
    \framesubtitle{$53\div 5 = \Machine{10 rem 3}$}
    
    \resizebox{.85\textwidth}{!}{
        \begin{tabular}{c|r|r|l}
            Частное $q, \leftarrow$ 
                & \multicolumn{1}{|c|}{дел-е, $\Delta$}
                    & \multicolumn{1}{|c|}{дел-ль, $d$ $\rightarrow$}
                        & прим. \\ 
            \hline\hline
            \Number{001010}
                & \Number{1,111111 111110}
                    & \Number{.,...... ...101}
                        & $\uparrow$\\ \hline\hline
            \Number{001010}
                & \Addition{1,111111 111110}
                           {.,...... ...101}
                           {.,...... 000011}
                    & 
                        & Восст. остатка\\ \hline
        \end{tabular}
    }
    
    \begin{align*}
        q &= (001010)_2 = 10\\
        \Delta &= (000011)_2 = 3
    \end{align*}
\end{frame}


\section{Деление чисел со знаком}

\begin{frame}
    \frametitle{Деление чисел со знаком}
    \framesubtitle{Неоднозначность результатов, см. подробнее в \cite{bib:warren:algTriks}}

    \[
    \begin{tabular}{lc||l|l|l}
        \hline\hline
        Пример        & &Отсечение           &\multicolumn{1}{|p{.15\textwidth}|}{Модуль $\Delta\ge 0$}
                                                                 & \multicolumn{1}{p{.22\textwidth}}{Округление $q$ к меньшему значению} \\
        \hline\hline
        $7\div 3$     &=&\Machine{ 2 rem  1} &\Machine{ 2 rem 1} & \Machine{ 2 rem  1} \\
        $(-7)\div 3$  &=&\Machine{-2 rem -1} &\Machine{-3 rem 2} & \Machine{-3 rem  2} \\
        $7\div(-3)$   &=&\Machine{-2 rem  1} &\Machine{-2 rem 1} & \Machine{-3 rem -2} \\
        $(-7)\div(-3)$&=&\Machine{ 2 rem -1} &\Machine{ 3 rem 2} & \Machine{ 2 rem -1} \\ \hline
    \end{tabular}
    \]
    
    Остановимся на варианте <<Отсечение>>.
\end{frame}


\subsection{Деление без восстановления остатков}

\begin{frame}
    \frametitle{Определение разряда частного $q_0$}

    Пусть $S(x)$ --- функция возвращающая знак $x$. Исходное правило:
    \begin{enumerate}
        \item Если знаки делимого $A$ и текущего остатка $\Delta$ совпадают, то разряд частного (модуля) $q_0\gets 1$, иначе $q_0\gets 0$.
        \item Если знаки делимого $A$ и делителя $d$ различны, то $q_0\gets (\lnot q_0)$. (Т.е. инверсия модуля частного --- \emph{обратный код}!)
    \end{enumerate}
    
    
    Так как $(x=y)\Leftrightarrow \lnot(x\oplus y) \Leftrightarrow (1\oplus x\oplus y)$,
    то исходное правило можно выразить одной формулой и упростить:
    \begin{align*}
        q_0\gets \lnot(S(A)=S(\Delta))\oplus(S(A)\oplus S(d)) \Leftrightarrow \\
        \Leftrightarrow(1\oplus S(A)\oplus S(\Delta))\oplus (S(A)\oplus S(d)) \Leftrightarrow (1\oplus S(d)\oplus S(\Delta))\Leftrightarrow \\
        \Leftrightarrow\lnot(S(d)\oplus S(\Delta)).
    \end{align*}
    \begin{block}{$\lnot(S(d)\oplus S(\Delta))$}
        Если знаки $d$ и $\Delta$ совпадают, то $q_0\gets 1$, иначе $q_0\gets 0$.
    \end{block}
\end{frame}

\begin{frame}
    \frametitle{Алгоритм деления в ДК без восстановления остатков}
    \framesubtitle{$n$-разрядные знаковые целые в ДК}

    \begin{enumerate}
        \item $i\gets 1$; Инициализируются регистры $q$, $\Delta$ и $d$.
        \item\label{div:dcvo:start} Выполняются сдвиги: $q$ --- влево, $\Delta$ --- влево (I сп.), $d$ --- вправо (II~сп., с учётом знака).
        \item Если знаки $\Delta$ и $d$ совпадают, то $\Delta\gets(\Delta - d)$, иначе $\Delta\gets(\Delta + d)$.
        \item $q_0\gets\lnot(S(d)\oplus S(\Delta))$. Т.е. если знаки $d$ и $\Delta$ совпадают, то $q_0\gets 1$, иначе $q_0\gets 0$. Обр. код!
        \item $i\gets (i + 1)$. Если $i<=n$, то к шагу \ref{div:dcvo:start}.
        \item Выполняется процедура коррекции остатка и частного (см. следующий слайд).
    \end{enumerate}
\end{frame}

\begin{frame}
    \frametitle{Процедура коррекции остатка и частного (отсечение)}
    \framesubtitle{Вход: $A$ --- делимое, $d$ --- делитель, $q$ --- частное, $\Delta$ --- остаток. Выход: $q$, $\Delta$}
    
    \resizebox{\textwidth}{!}{
        \begin{tabular}{c||c|c|}
                & $d > 0$
                    & $d < 0$\\
            \hline\hline
            \rotatebox{90}{$A\ge 0$}
                & 
                \parbox[c]{.3\textwidth}{
                    \begin{align*}
                        q&\gets q,\\
                        \Delta&\gets 
                        \begin{cases}
                            (\Delta + d), & \Delta < 0, \\
                            \Delta,      &\text{ иначе,}
                        \end{cases}
                    \end{align*}
                }
                    & 
                    \parbox[c]{.3\textwidth}{
                        \begin{align*}
                            q&\gets (q+1),\\
                            \Delta&\gets 
                            \begin{cases}
                                (\Delta - d), & \Delta < 0, \\
                                \Delta,       &\text{ иначе,}
                            \end{cases}
                        \end{align*}
                    }
                    \\
            \hline
            \rotatebox{90}{$A < 0$}
                & 
                \parbox[c]{.3\textwidth}{
                    \begin{align*}
                        q&\gets 
                        \begin{cases}
                            q,      & (\Delta = 0) \lor (\Delta = -d) \\
                            (q+1),  &\text{ иначе,}
                        \end{cases}
                        \\
                        \Delta&\gets 
                        \begin{cases}
                            0,          & (\Delta = 0) \lor (\Delta = -d), \\
                            (\Delta-d), & \Delta > 0, \\
                            \Delta,     & \text{ иначе,}
                        \end{cases}
                    \end{align*}
                }
                    & 
                    \parbox[c]{.3\textwidth}{
                        \begin{align*}
                            q&\gets 
                            \begin{cases}
                                q+1, & (\Delta = 0) \lor (\Delta = d), \\
                                q,   &\text{ иначе,}
                            \end{cases}
                            \\
                            \Delta&\gets 
                            \begin{cases}
                                0,        & (\Delta = 0) \lor (\Delta = d), \\
                                \Delta+d, & \Delta > 0, \\
                                \Delta,   & \text{ иначе.}
                            \end{cases}
                        \end{align*}
                    } \\ \hline
        \end{tabular}
    }
\end{frame}

\begin{frame}
    \frametitle{Деление без ВО в ДК I-й способ}
    \framesubtitle{$27\div(-5)$}
    
    \resizebox{.85\textwidth}{!}{
        \begin{tabular}{c|r|r|l}
            \hline\hline
            Частное $q, \leftarrow$ 
                & \multicolumn{1}{|c|}{дел-е, $\Delta$ $\leftarrow$}
                    & \multicolumn{1}{|c|}{дел-ль, $d$}
                        & прим. \\ 
            \hline\hline
            \Number{......}
                & \Number{0,...... 011011}
                    & \Number{1,111011}
                        & операнды;\\ \hline\hline
            \Number{......}
                & \Number{0,.....0 11011.}
                    & 
                        & сдвиг;\\ \hline
            \Number{.....1}
                & \Addition{0,.....0 11011.}
                           {1,111011 ......}
                           {1,111011 11011.}
                    & 
                        & $+d, \Delta_1<0$;\\ \hline
            \Number{....1.}
                & \Number{1,110111 1011..}
                    & 
                        & сдвиг;\\ \hline
            \Number{....11}
                & \Addition{1,110111 1011..}
                           {.,...101 ......}
                           {1,111100 1011..}
                    & 
                        & $-d,\Delta_2<0$;\\ \hline
            \Number{...11.}
                & \Number{1,111001 011...}
                    & 
                        & сдвиг;\\ \hline
            \Number{...111}
                & \Addition{1,111001 011...}
                           {.,...101 ......}
                           {1,111110 011...}
                    & 
                        & $-d,\Delta_3<0$;\\ \hline
            \Number{..111.}
                & \Number{1,111100 11....}
                    & 
                        & сдвиг;\\ 
        \end{tabular}
    }
\end{frame}

\begin{frame}
    \frametitle{Деление без ВО в ДК I-й способ (2)}
    \framesubtitle{$27\div(-5)$}
    
    \resizebox{.85\textwidth}{!}{
        \begin{tabular}{c|r|r|l}
            Частное $q, \leftarrow$ 
                & \multicolumn{1}{|c|}{дел-е, $\Delta$ $\leftarrow$}
                    & \multicolumn{1}{|c|}{дел-ль, $d$}
                        & прим. \\ 
            \hline\hline
            \Number{..111.}
                & \Number{1,111100 11....}
                    & 
                        & $\uparrow$\\ \hline\hline
            \Number{..1110}
                & \Addition{1,111100 11....}
                           {0,000101 ......}
                           {0,000001 11....}
                    & 
                        & $-d,\Delta_4\ge 0$;\\ \hline
            \Number{.1110.}
                & \Number{0,000011 1.....}
                    & 
                        & сдвиг;\\ \hline
            \Number{.11101}
                & \Addition{0,000011 1.....}
                           {1,111011 ......}
                           {1,111110 1.....}
                    & 
                        & $+d,\Delta_5<0$;\\ \hline
            \Number{11101.}
                & \Number{1,111101 ......}
                    & 
                        & сдвиг;\\ \hline
            \Number{111010}
                & \Addition{1,111101 ......}
                           {.,...101 ......}
                           {0,000010 ......}
                    & 
                        & $-d,\Delta_6\ge 0$;\\ 
        \end{tabular}
    }
\end{frame}

\begin{frame}
    \frametitle{Деление без ВО в ДК I-й способ (3)}
    \framesubtitle{$27\div(-5)=\Machine{-5 rem 2}$}
    
    \resizebox{.85\textwidth}{!}{
        \begin{tabular}{r|r|r|l}
            Частное $q, \leftarrow$ 
                & \multicolumn{1}{|c|}{дел-е, $\Delta$ $\leftarrow$}
                    & \multicolumn{1}{|c|}{дел-ль, $d$}
                        & прим. \\ 
            \hline\hline
            \Number{111010}
                & \Number{0,000010 ......}
                    & 
                        & $\uparrow$\\ \hline\hline
            \Addition{111010}{000001}{111011}
                & \Number{0,000010 ......}
                    & 
                        & $q\gets(q+1)$;\\ \hline
        \end{tabular}
    }
    
    \begin{align*}
        \DC{q}      &= (111011)_2 \Rightarrow -5\\
        \DC{\Delta} &= (000010)_2 \Rightarrow 2
    \end{align*}
\end{frame}

\begin{frame}
    \frametitle{Деление без восстановления остатков II-й способ}
    \framesubtitle{$(-25)\div 6$}
    
    \resizebox{.85\textwidth}{!}{
        \begin{tabular}{r|r|r|l}
            \hline\hline
            Частное $q, \leftarrow$ 
                & \multicolumn{1}{|c|}{дел-е, $\Delta$}
                    & \multicolumn{1}{|c|}{дел-ль, $d$ $\rightarrow$}
                        & прим. \\ 
            \hline\hline
            \Number{......}
                & \Number{1,111111 100111}
                    & \Number{.,...110 ......}
                        & операнды;\\ \hline\hline
            \Number{......}
                & 
                    & \Number{.,....11 0.....}
                        & сдвиг;\\ \hline
            \Number{.....1}
                & \Addition{1,111111 100111}
                           {.,....11 0.....}
                           {0,....10 100111}
                    & 
                        & $+d,\Delta_1\ge 0$;\\ \hline
            \Number{....1.}
                & 
                    & \Number{.,.....1 10....}
                        & сдвиг;\\ \hline
            \Number{....11}
                & \Addition{0,....10 100111}
                           {1,111110 10....}
                           {0,.....1 000111}
                    & 
                        & $-d,\Delta_2\ge 0$;\\ \hline
            \Number{...11.}
                & 
                    & \Number{.,...... 110...}
                        & сдвиг;\\ \hline
            \Number{...111}
                & \Addition{0,.....1 000111}
                           {1,111111 010...}
                           {0,...... .10111}
                    & 
                        & $-d,\Delta_3\ge 0$;\\ \hline
            \Number{..111.}
                & 
                    & \Number{.,...... .110..}
                        & сдвиг;\\ 
        \end{tabular}
    }
\end{frame}

\begin{frame}
    \frametitle{Деление без восстановления остатков II-й способ (2)}
    \framesubtitle{$(-25)\div 6$}
    
    \resizebox{.85\textwidth}{!}{
        \begin{tabular}{c|r|r|l}
            Частное $q, \leftarrow$ 
                & \multicolumn{1}{|c|}{дел-е, $\Delta$}
                    & \multicolumn{1}{|c|}{дел-ль, $d$ $\rightarrow$}
                        & прим. \\ 
            \hline\hline
            \Number{..111.}
                & \Number{0,...... .10111}
                    & \Number{.,...... .110..}
                        & $\uparrow$\\ \hline\hline
            \Number{..1110}
                & \Addition{0,...... .10111}
                           {1,111111 1010..}
                           {1,111111 111111}
                    & 
                        & $-d,\Delta_4\ge 0$;\\ \hline
            \Number{.1110.}
                & 
                    & \Number{.,...... ..110.}
                        & сдвиг;\\ \hline
            \Number{.11101}
                & \Addition{1,111111 111111}
                           {.,...... ..110.}
                           {0,...... ..1011}
                    & 
                        & $+d,\Delta_5\ge 0$;\\ \hline
            \Number{11101.}
                & 
                    & \Number{.,...... ...110}
                        & сдвиг;\\ \hline
            \Number{111011}
                & \Addition{0,...... ..1011}
                           {1,111111 111010}
                           {0,...... ...101}
                    & 
                        & $-d,\Delta_6\ge 0$;\\ 
        \end{tabular}
    }
\end{frame}

\begin{frame}
    \frametitle{Деление без восстановления остатков II-й способ (3)}
    \framesubtitle{$(-25)\div 6 = \Machine{-4 rem -1}$}
    
    \resizebox{.85\textwidth}{!}{
        \begin{tabular}{c|r|r|l}
            Частное $q, \leftarrow$ 
                & \multicolumn{1}{|c|}{дел-е, $\Delta$}
                    & \multicolumn{1}{|c|}{дел-ль, $d$ $\rightarrow$}
                        & прим. \\ 
            \hline\hline
            \Number{111011}
                & \Number{0,...... ...101}
                    & \Number{.,...... ...110}
                        & $\uparrow$\\ \hline\hline
            \Number{111011}
                & \Addition{.,...... ...101}
                           {1,111111 111010}
                           {1,111111 111111}
                    & 
                        & Восст. остатка, $-d$\\ \hline
            \Addition{111011}
                     {000001}
                     {111100}
                & \NumberLo{1,111111 }{111111}
                    & 
                        & $q\gets(q+1)$\\ \hline
        \end{tabular}
    }
    
    \begin{align*}
        \DC{q}      &= (111100)_2 \Rightarrow -4\\
        \DC{\Delta} &= (111111)_2 \Rightarrow -1
    \end{align*}
\end{frame}


\appendix


\section{Задания на практику}


\subsection{Проходное}

\begin{frame}
    \frametitle{\TaskSimpleNumber}
    
    Выполнить беззнаковое деление чисел:
    \begin{enumerate}
        \item $27\div 9$, первым способом без восстановления остатков;

        \item $39\div 10$, вторым способом без восстановления остатков;
    \end{enumerate}
\end{frame}

\begin{frame}
    \frametitle{\TaskSimpleNumber}
    
    Выполнить целочисленное деление в дополнительном коде чисел со знаком:
    \begin{enumerate}
        \item $122\div  22$, первым и вторым способами;
        
        \item $122\div  19$, первым способом;
        
        \item $(-119) \div 11$, первым способом;
        
        \item $(-118)\div(-11)$, вторым способом;

        \item $123\div (-12)$, вторым способом.
    \end{enumerate}
\end{frame}

\subsection{Мегамозг}

\begin{frame}
    \frametitle{\TaskSimpleNumber}
    
    Выполнить целочисленное деление в дополнительном коде в 8-и разрядной сетке чисел со знаком (любым способом):
    \begin{enumerate}
        \item $(-128)\div(-127)$;
        \item $(-128)\div 127$;
        \item $127\div(-128)$;
        \item $0\div(127)$.
    \end{enumerate}
\end{frame}

\section{Самообучение}

\begin{frame}
    \frametitle{Советы самоучке}

    Подробно об особенностях целочисленного деления см. в \cite{bib:warren:algTriks}. <<Длинные>> алгоритмы умножения и деления, см. в четвертой главе <<Арифметика>> \cite{bib:knuth:artOfProgramming2}
\end{frame}

\begin{frame}[allowframebreaks]{Библиография}
    \bibliographystyle{gost780u}
    \bibliography{./../../../bibliobase}
\end{frame}

\end{document}




