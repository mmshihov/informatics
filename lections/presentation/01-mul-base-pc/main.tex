%include part: see main.beamer.tex and main.article.tex
%include common packages and settings
\usepackage{etex} %эта магическая херь избавляет от переполнения регистров TeX а!!!

\mode<article>{\usepackage{fullpage}}
\mode<presentation>{
    \usetheme{Madrid} %%Boadilla,Madrid,AnnArbor,CambridgeUS,Malmoe,Singapore,Berlin
    \useoutertheme{shadow}
} 

\usepackage[utf8]{inputenc}
\usepackage[russian]{babel}
\usepackage{indentfirst}
\usepackage{graphicx}

\usepackage{amsmath}
\usepackage{amsfonts}
\usepackage{amsthm}
\usepackage{algorithm}
\usepackage{algorithmic}

\usepackage[all]{xy}

\date{Лекция по дисциплине <<информатика>>\\(\today)}
\author[М.~М.~Шихов]{Михаил Шихов \\ \texttt{\underline{m.m.shihov@gmail.com}}}

%для рисования графов пакетом xy-pic
\entrymodifiers={++[o][F-]}

%для псевдокода алгоритмов (algorithm,algorithmic)
\renewcommand{\algorithmicrequire}{\textbf{Вход:}}
\renewcommand{\algorithmicensure}{\textbf{Выход:}}
\renewcommand{\algorithmiccomment}[1]{// #1}
\floatname{algorithm}{Псевдокод}

%%определённые мной команды логической разметки
\newcommand{\Signs}[2]{\fbox{{\small{\underbar{#1}}}{#2}}}
\newcommand{\Sign}[1]{\fbox{#1}}
\newcommand{\DC}[1]{\text{ДК}(#1)}
\newcommand{\OC}[1]{\text{ОК}(#1)}
\newcommand{\PC}[1]{\text{ПК}(#1)}

\newcommand{\MyProc}{\text{$\mathbf{R}_8$}}

\newboolean{IsNeedAnswer}
\setboolean{IsNeedAnswer}{false} %true/false

\newcommand{\Machine}[1]{\texttt{#1}}
\newcommand{\Opcode}[1]{\texttt{\bf{#1}}}
\newcommand{\Operand}[1]{\texttt{#1}}
\newcommand{\CmdOneAddr}[2]{\text{\Opcode{#1} \Operand{#2}}}
\newcommand{\CmdTwoAddr}[3]{\text{\Opcode{#1} \Operand{#2}, \Operand{#3}}}
\newcommand{\CmdThreeAddr}[4]{\text{\Opcode{#1} {\Operand{#2}}, \Operand{#3}, \Operand{#4}}}

\newcommand{\ProofAnswer}[1]{
    \ifthenelse{\boolean{IsNeedAnswer}}{
        \begin{proof}[Ответ]
            #1
        \end{proof}
    }{}
}

\newcommand{\LabeledAnswer}[1]{
    \ifthenelse{\boolean{IsNeedAnswer}}{
        \emph{Отв.}: #1 \qed
    }{}
}

\newcommand{\PlainAnswer}[1]{
    \ifthenelse{\boolean{IsNeedAnswer}}{#1}{}
}

\newcommand{\UnsignedAny}[2]{\text{\upshape
    \begin{tabular}{lr}
        \tiny{#1} & \tiny{0}\\ 
        \hline
        \multicolumn{2}{|c|}{\texttt{#2}} \\ 
        \hline
    \end{tabular}
}}

\newcommand{\UnsignedByte}[1]{\UnsignedAny{7}{#1}}

\newcommand{\UnsignedTwoBytes}[1]{\UnsignedAny{15}{#1}}

\newcommand{\FixedAny}[4]{\text{\upshape
    \begin{tabular}{clr}
        \tiny{#1}  &\tiny{#2} & \tiny{0}\\ 
        \hline
        \multicolumn{1}{|c|}{\texttt{#3}} & \multicolumn{2}{|c|}{\texttt{#4}} \\ 
        \hline
    \end{tabular}
}}

\newcommand{\FixedByte}[2]{\FixedAny{7}{6}{#1}{#2}}

\newcommand{\FixedTwoBytes}[2]{\FixedAny{15}{14}{#1}{#2}}

\newcommand{\SignedAny}[4]{\text{\upshape
    \begin{tabular}{clr}
        \tiny{#1}  &\tiny{#2} & \tiny{0}\\ 
        \hline
        \multicolumn{1}{|c|}{\Machine{#3}} & \multicolumn{2}{|c|}{\Machine{#4}} \\ 
        \hline
    \end{tabular}
}}

\newcommand{\SignedNibble}[2]{\SignedAny{3}{2}{#1}{#2}}

\newcommand{\SignedByte}[2]{\SignedAny{7}{6}{#1}{#2}}

\newcommand{\SignedTwoBytes}[2]{\SignedAny{15}{14}{#1}{#2}}

\newcommand{\FloatMyHex}[4]{\text{\upshape
    \begin{tabular}{clrclr}
        \tiny{15}  &\tiny{14} & \tiny{6} & \tiny{5} & \tiny{4} & \tiny{0}\\ 
        \hline
        \multicolumn{1}{|c|}{\texttt{#1}} 
            & \multicolumn{2}{|c|}{\texttt{#2}} 
                & \multicolumn{1}{|c|}{\texttt{#3}} 
                    & \multicolumn{2}{|c|}{\texttt{#4}} \\ 
        \hline
    \end{tabular}
}}

\newcommand{\FloatMyCharHex}[3]{\text{\upshape
    \begin{tabular}{clrlr}
        \tiny{15}  &\tiny{14} & \tiny{6} & \tiny{5} & \tiny{0}\\ 
        \hline
        \multicolumn{1}{|c|}{\texttt{#1}} 
            & \multicolumn{2}{|c|}{\texttt{#2}} 
                & \multicolumn{2}{|c|}{\texttt{#3}} \\ 
        \hline
    \end{tabular}
}}

\newcommand{\FloatMyDcMantCharHex}[2]{\text{\upshape
    \begin{tabular}{lrlr}
        \tiny{15} & \tiny{6} & \tiny{5} & \tiny{0}\\ 
        \hline
        \multicolumn{2}{|c|}{\texttt{#1}} 
            & \multicolumn{2}{|c|}{\texttt{#2}} \\ 
        \hline
    \end{tabular}
}}

\newcommand{\FloatESShort}[3]{\text{\upshape
    \begin{tabular}{clrlr}
        \tiny{31}  &\tiny{30} & \tiny{24} & \tiny{23} & \tiny{0}\\ 
        \hline
        \multicolumn{1}{|c|}{\texttt{#1}} 
            & \multicolumn{2}{|c|}{\texttt{#2}} 
                & \multicolumn{2}{|c|}{\texttt{#3}} 
                    \\ 
        \hline
    \end{tabular}
}}

\newcommand{\FloatPCShort}[3]{\text{\upshape
    \begin{tabular}{clrlr}
        \tiny{31}  &\tiny{30} & \tiny{23} & \tiny{22} & \tiny{0}\\ 
        \hline
        \multicolumn{1}{|c|}{\texttt{#1}} 
            & \multicolumn{2}{|c|}{\texttt{#2}} 
                & \multicolumn{2}{|c|}{\texttt{#3}} 
                    \\ 
        \hline
    \end{tabular}
}}

\newcommand{\FloatMyOrderX}[4]{\text{\upshape
    \begin{tabular}{clrclr}
        \tiny{9}  &\tiny{8} & \tiny{4} & \tiny{3} & \tiny{2} & \tiny{0}\\ 
        \hline
        \multicolumn{1}{|c|}{\texttt{#1}} 
            & \multicolumn{2}{|c|}{\texttt{#2}} 
                & \multicolumn{1}{|c|}{\texttt{#3}} 
                    & \multicolumn{2}{|c|}{\texttt{#4}} \\ 
        \hline
    \end{tabular}
}}

\newcommand{\FloatMyCharX}[3]{\text{\upshape
    \begin{tabular}{clrlr}
        \tiny{9}  &\tiny{8} & \tiny{4} & \tiny{3} & \tiny{0}\\ 
        \hline
        \multicolumn{1}{|c|}{\texttt{#1}} 
            & \multicolumn{2}{|c|}{\texttt{#2}} 
                & \multicolumn{2}{|c|}{\texttt{#3}} \\ 
        \hline
    \end{tabular}
}}

\newcommand{\FloatMyDcOrderX}[3]{\text{\upshape
    \begin{tabular}{lrclr}
        \tiny{9} & \tiny{4} & \tiny{3} & \tiny{2} & \tiny{0}\\ 
        \hline
        \multicolumn{2}{|c|}{\texttt{#1}} 
            & \multicolumn{1}{|c|}{\texttt{#2}} 
                & \multicolumn{2}{|c|}{\texttt{#3}} \\ 
        \hline
    \end{tabular}
}}

\newcommand{\FloatMyDcCharX}[2]{\text{\upshape
    \begin{tabular}{lrlr}
        \tiny{9} & \tiny{4} & \tiny{3} & \tiny{0}\\ 
        \hline
        \multicolumn{2}{|c|}{\texttt{#1}} 
            & \multicolumn{2}{|c|}{\texttt{#2}} \\ 
        \hline
    \end{tabular}
}}


%--- СПЕЦИФИЧНЫЕ ДЛЯ УМНОЖЕНИЯ КОМАНДЫ ---------------------------------------------------------------------------------------------


\newcommand{\Number}[1]{
    \texttt{#1}
}

\newcommand{\NumberHi}[2]{
    \underline{\underline{\texttt{#1}}}\texttt{#2}
}

\newcommand{\NumberMid}[3]{
    \texttt{#1}\underline{\underline{\texttt{#2}}}\texttt{#3}
}

\newcommand{\NumberLo}[2]{
    \texttt{#1}\underline{\underline{\texttt{#2}}}
}

\newcommand{\Stack}[2]{
    \begin{tabular}[t]{@{}r@{}}
        {#1}\\ \hline
        {#2}\\ 
    \end{tabular}
}

\newcommand{\StackThree}[3]{
    \begin{tabular}[t]{@{}r@{}}
        {#1}\\ \hline
        {#2}\\ \hline
        {#3}\\
    \end{tabular}
}

\newcommand{\StackFour}[4]{
    \begin{tabular}[t]{@{}r@{}}
        {#1}\\ \hline
        {#2}\\ \hline
        {#3}\\ \hline
        {#4}\\
    \end{tabular}
}

\newcommand{\Operation}[4]{
    \begin{tabular}[t]{@{}r@{}}
        \texttt{#4}
        \begin{tabular}{@{}r@{}}
            \Number{#1}\\
            \Number{#2}\\ \hline
        \end{tabular} \\ 
        \Number{#3}\\
    \end{tabular}
}

\newcommand{\Addition}[3]{\Operation{#1}{#2}{#3}{+}}

\newcommand{\Subtraction}[3]{\Operation{#1}{#2}{#3}{-}}

\newcommand{\Register}[2]{\Number{#1:#2}}

\newcommand{\Mantiss}{m}
\newcommand{\Order}{p}
\newcommand{\Char}{c}

\newcommand{\MantissOf}[1]{\Mantiss_{#1}}
\newcommand{\OrderOf}[1]{\Order_{#1}}
\newcommand{\CharOf}[1]{\Char_{#1}}

\newcommand{\FloatExpression}[2]{\MantissOf{#1}\cdot {#2}^{\OrderOf{#1}}}

\newcommand{\DivAnswer}[2]{(\texttt{$#1$ rem $#2$})}

\newenvironment{Solve}[1]%
    {\begin{proof}[Решение]#1}
    {\end{proof}}
    
    
%определённые мной команды логической разметки
\newcommand{\Signs}[2]{\fbox{{\small{\underbar{#1}}}{#2}}}
\newcommand{\Sign}[1]{\fbox{#1}}
\newcommand{\DC}[1]{\text{ДК}(#1)}
\newcommand{\OC}[1]{\text{ОК}(#1)}
\newcommand{\PC}[1]{\text{ПК}(#1)}

\newcommand{\MyProc}{\text{$\mathbf{R}_8$}}

\newboolean{IsNeedAnswer}
\setboolean{IsNeedAnswer}{false} %true/false

\newcommand{\Machine}[1]{\texttt{#1}}
\newcommand{\Opcode}[1]{\texttt{\bf{#1}}}
\newcommand{\Operand}[1]{\texttt{#1}}
\newcommand{\CmdOneAddr}[2]{\text{\Opcode{#1} \Operand{#2}}}
\newcommand{\CmdTwoAddr}[3]{\text{\Opcode{#1} \Operand{#2}, \Operand{#3}}}
\newcommand{\CmdThreeAddr}[4]{\text{\Opcode{#1} {\Operand{#2}}, \Operand{#3}, \Operand{#4}}}

\newcommand{\ProofAnswer}[1]{
    \ifthenelse{\boolean{IsNeedAnswer}}{
        \begin{proof}[Ответ]
            #1
        \end{proof}
    }{}
}

\newcommand{\LabeledAnswer}[1]{
    \ifthenelse{\boolean{IsNeedAnswer}}{
        \emph{Отв.}: #1 \qed
    }{}
}

\newcommand{\PlainAnswer}[1]{
    \ifthenelse{\boolean{IsNeedAnswer}}{#1}{}
}

\newcommand{\UnsignedAny}[2]{\text{\upshape
    \begin{tabular}{lr}
        \tiny{#1} & \tiny{0}\\ 
        \hline
        \multicolumn{2}{|c|}{\texttt{#2}} \\ 
        \hline
    \end{tabular}
}}

\newcommand{\UnsignedByte}[1]{\UnsignedAny{7}{#1}}

\newcommand{\UnsignedTwoBytes}[1]{\UnsignedAny{15}{#1}}

\newcommand{\FixedAny}[4]{\text{\upshape
    \begin{tabular}{clr}
        \tiny{#1}  &\tiny{#2} & \tiny{0}\\ 
        \hline
        \multicolumn{1}{|c|}{\texttt{#3}} & \multicolumn{2}{|c|}{\texttt{#4}} \\ 
        \hline
    \end{tabular}
}}

\newcommand{\FixedByte}[2]{\FixedAny{7}{6}{#1}{#2}}

\newcommand{\FixedTwoBytes}[2]{\FixedAny{15}{14}{#1}{#2}}

\newcommand{\SignedAny}[4]{\text{\upshape
    \begin{tabular}{clr}
        \tiny{#1}  &\tiny{#2} & \tiny{0}\\ 
        \hline
        \multicolumn{1}{|c|}{\Machine{#3}} & \multicolumn{2}{|c|}{\Machine{#4}} \\ 
        \hline
    \end{tabular}
}}

\newcommand{\SignedNibble}[2]{\SignedAny{3}{2}{#1}{#2}}

\newcommand{\SignedByte}[2]{\SignedAny{7}{6}{#1}{#2}}

\newcommand{\SignedTwoBytes}[2]{\SignedAny{15}{14}{#1}{#2}}

\newcommand{\FloatMyHex}[4]{\text{\upshape
    \begin{tabular}{clrclr}
        \tiny{15}  &\tiny{14} & \tiny{6} & \tiny{5} & \tiny{4} & \tiny{0}\\ 
        \hline
        \multicolumn{1}{|c|}{\texttt{#1}} 
            & \multicolumn{2}{|c|}{\texttt{#2}} 
                & \multicolumn{1}{|c|}{\texttt{#3}} 
                    & \multicolumn{2}{|c|}{\texttt{#4}} \\ 
        \hline
    \end{tabular}
}}

\newcommand{\FloatMyCharHex}[3]{\text{\upshape
    \begin{tabular}{clrlr}
        \tiny{15}  &\tiny{14} & \tiny{6} & \tiny{5} & \tiny{0}\\ 
        \hline
        \multicolumn{1}{|c|}{\texttt{#1}} 
            & \multicolumn{2}{|c|}{\texttt{#2}} 
                & \multicolumn{2}{|c|}{\texttt{#3}} \\ 
        \hline
    \end{tabular}
}}

\newcommand{\FloatMyDcMantCharHex}[2]{\text{\upshape
    \begin{tabular}{lrlr}
        \tiny{15} & \tiny{6} & \tiny{5} & \tiny{0}\\ 
        \hline
        \multicolumn{2}{|c|}{\texttt{#1}} 
            & \multicolumn{2}{|c|}{\texttt{#2}} \\ 
        \hline
    \end{tabular}
}}

\newcommand{\FloatESShort}[3]{\text{\upshape
    \begin{tabular}{clrlr}
        \tiny{31}  &\tiny{30} & \tiny{24} & \tiny{23} & \tiny{0}\\ 
        \hline
        \multicolumn{1}{|c|}{\texttt{#1}} 
            & \multicolumn{2}{|c|}{\texttt{#2}} 
                & \multicolumn{2}{|c|}{\texttt{#3}} 
                    \\ 
        \hline
    \end{tabular}
}}

\newcommand{\FloatPCShort}[3]{\text{\upshape
    \begin{tabular}{clrlr}
        \tiny{31}  &\tiny{30} & \tiny{23} & \tiny{22} & \tiny{0}\\ 
        \hline
        \multicolumn{1}{|c|}{\texttt{#1}} 
            & \multicolumn{2}{|c|}{\texttt{#2}} 
                & \multicolumn{2}{|c|}{\texttt{#3}} 
                    \\ 
        \hline
    \end{tabular}
}}

\newcommand{\FloatMyOrderX}[4]{\text{\upshape
    \begin{tabular}{clrclr}
        \tiny{9}  &\tiny{8} & \tiny{4} & \tiny{3} & \tiny{2} & \tiny{0}\\ 
        \hline
        \multicolumn{1}{|c|}{\texttt{#1}} 
            & \multicolumn{2}{|c|}{\texttt{#2}} 
                & \multicolumn{1}{|c|}{\texttt{#3}} 
                    & \multicolumn{2}{|c|}{\texttt{#4}} \\ 
        \hline
    \end{tabular}
}}

\newcommand{\FloatMyCharX}[3]{\text{\upshape
    \begin{tabular}{clrlr}
        \tiny{9}  &\tiny{8} & \tiny{4} & \tiny{3} & \tiny{0}\\ 
        \hline
        \multicolumn{1}{|c|}{\texttt{#1}} 
            & \multicolumn{2}{|c|}{\texttt{#2}} 
                & \multicolumn{2}{|c|}{\texttt{#3}} \\ 
        \hline
    \end{tabular}
}}

\newcommand{\FloatMyDcOrderX}[3]{\text{\upshape
    \begin{tabular}{lrclr}
        \tiny{9} & \tiny{4} & \tiny{3} & \tiny{2} & \tiny{0}\\ 
        \hline
        \multicolumn{2}{|c|}{\texttt{#1}} 
            & \multicolumn{1}{|c|}{\texttt{#2}} 
                & \multicolumn{2}{|c|}{\texttt{#3}} \\ 
        \hline
    \end{tabular}
}}

\newcommand{\FloatMyDcCharX}[2]{\text{\upshape
    \begin{tabular}{lrlr}
        \tiny{9} & \tiny{4} & \tiny{3} & \tiny{0}\\ 
        \hline
        \multicolumn{2}{|c|}{\texttt{#1}} 
            & \multicolumn{2}{|c|}{\texttt{#2}} \\ 
        \hline
    \end{tabular}
}}


%--- СПЕЦИФИЧНЫЕ ДЛЯ УМНОЖЕНИЯ КОМАНДЫ ---------------------------------------------------------------------------------------------


\newcommand{\Number}[1]{
    \texttt{#1}
}

\newcommand{\NumberHi}[2]{
    \underline{\underline{\texttt{#1}}}\texttt{#2}
}

\newcommand{\NumberMid}[3]{
    \texttt{#1}\underline{\underline{\texttt{#2}}}\texttt{#3}
}

\newcommand{\NumberLo}[2]{
    \texttt{#1}\underline{\underline{\texttt{#2}}}
}

\newcommand{\Stack}[2]{
    \begin{tabular}[t]{@{}r@{}}
        {#1}\\ \hline
        {#2}\\ 
    \end{tabular}
}

\newcommand{\StackThree}[3]{
    \begin{tabular}[t]{@{}r@{}}
        {#1}\\ \hline
        {#2}\\ \hline
        {#3}\\
    \end{tabular}
}

\newcommand{\StackFour}[4]{
    \begin{tabular}[t]{@{}r@{}}
        {#1}\\ \hline
        {#2}\\ \hline
        {#3}\\ \hline
        {#4}\\
    \end{tabular}
}

\newcommand{\Operation}[4]{
    \begin{tabular}[t]{@{}r@{}}
        \texttt{#4}
        \begin{tabular}{@{}r@{}}
            \Number{#1}\\
            \Number{#2}\\ \hline
        \end{tabular} \\ 
        \Number{#3}\\
    \end{tabular}
}

\newcommand{\Addition}[3]{\Operation{#1}{#2}{#3}{+}}

\newcommand{\Subtraction}[3]{\Operation{#1}{#2}{#3}{-}}

\newcommand{\Register}[2]{\Number{#1:#2}}

\newcommand{\Mantiss}{m}
\newcommand{\Order}{p}
\newcommand{\Char}{c}

\newcommand{\MantissOf}[1]{\Mantiss_{#1}}
\newcommand{\OrderOf}[1]{\Order_{#1}}
\newcommand{\CharOf}[1]{\Char_{#1}}

\newcommand{\FloatExpression}[2]{\MantissOf{#1}\cdot {#2}^{\OrderOf{#1}}}

\newcommand{\DivAnswer}[2]{(\texttt{$#1$ rem $#2$})}

\newenvironment{Solve}[1]%
    {\begin{proof}[Решение]#1}
    {\end{proof}}
    
    

\title[Умножение в ПК]{Умножение с фиксированной точкой\\(прямой код)}

\newcounter{TaskSimpleCtr}
\setcounter{TaskSimpleCtr}{1}
\newcommand{\TaskSimpleNumber}{ \arabic{TaskSimpleCtr}) \addtocounter{TaskSimpleCtr}{1} }

%вставка изображений из metapost (post script)
\DeclareGraphicsRule{*}{mps}{*}{}

\begin{document}

\mode<article>{\maketitle\tableofcontents}
\frame<presentation>{\titlepage}
\begin{frame}<presentation>
    \frametitle{Содержание}
    \tableofcontents
\end{frame}

\begin{frame}
    \frametitle{Дробно-нормализованные числа}
    \framesubtitle{Для обоснований выкладок мы используем дробное масштабирование}
    
    \[x=y\cdot M\]

    Для обоснования выкладок будут использоваться $y$.

    \begin{block}{}
        $y$ --- дробное число, его целая часть $y$ равна нолю.
    \end{block}
    
    Рарзядная сетка хранит разряды дробной части $y$.
    \[
        \raisebox{1.\height}{0.}
        \raisebox{1.\height}{
            \UnsignedAny{n-1}{yyyyy$\cdots$yyyyy}
        }
    \]
    
    Чтобы зафиксировать запятую между $k$ и $(k-1)$ разрядами $n$-разрядной сетки, выбирается масштаб
    \[
        M=2^{(n-k)}.
    \]
\end{frame}

\section{Процесс умножения в 2СС}


\subsection{Умножение <<столбиком>>}
\begin{frame}
    \frametitle{Умножение <<столбиком>>}
    
    Пусть для положительных чисел $A$ и $B$ имеются дробно-масштабированные представления в двоичной системе счисления. Пусть
    \[
        A\equiv(0,a_{1}\cdots a_{n})_2.
    \]

    Тогда результат произведения $A\times B$ в двоичной системе счисления будет определяться по формуле:
    \[
        A\times B = B\cdot\sum_{i=1}^{n} a_{i}\cdot 2^{-i} = \sum_{i=1}^{n} (B\cdot 2^{-i})\cdot a_i = \sum_{i=1}^{n} (B \gg i)\cdot a_i.
    \]
\end{frame}

\subsection{Пример умножения}

\begin{frame}
    \frametitle{Пример умножения}
    \framesubtitle{Операнды}
    
    Требуется найти произведение $A\times B$, где $A=23$ и $B=25$. Дробные представления (с масштабирующим множителем $M=2^5$) будут:
    \begin{align*}
        A\equiv\Number{0,10111},\\
        B\equiv\Number{0,11001}.
    \end{align*}
    
    \begin{block}{}
        Результатоом произведения будет дробное число, но с масштабом 
        \[
            M^2 = 2^{10}
        \]
    \end{block}
\end{frame}

\begin{frame}
    \frametitle{Пример умножения <<столбиком>> чисел без знака}
    
    \begin{align*}
        A\equiv\Number{,10111},\\
        B\equiv\Number{,11001}.
    \end{align*}
    
    \[    
        \begin{tabular}{c|c|c}
                                                                     \hline\hline
            $i$ & Разряды $a_i$    & $(B\cdot 2^{-i}) \cdot a_i$  \\ \hline\hline
            -1  & \Number{1....} & \Number{~.1100 1....} \\
            -2  & \Number{.0...} & \Number{~..... .....} \\
            -3  & \Number{..1..} & \Number{~...11 001..} \\
            -4  & \Number{...1.} & \Number{~....1 1001.} \\
            -5  & \Number{....1} & \Number{~..... 11001} \\ \hline
            \multicolumn{2}{r}{Результат:} 
                                   & \Number{,10001 11111} \\
        \end{tabular}
    \]
    
    \[(\Number{,10001 11111})_2\cdot 2^{10}=(\Number{10001 11111,})_2=575=23\cdot 25.\]
\end{frame}


\section{Основные способы умножения}


\subsection{I способ}

\begin{frame}
    \begin{center}
        I-й способ
    \end{center}
\end{frame}

\begin{frame}
    \frametitle{I-й способ}
    
    \begin{align*}
        A\equiv\Number{,10111},\\
        B\equiv\Number{,11001}.
    \end{align*}
    
    \[    
        \begin{tabular}{c|c|c}
                                                                     \hline\hline
            $i$ & Разряды $a_i$    & $(B\cdot 2^{-i}) \cdot a_i$  \\ \hline\hline
            \uncover<5->{-1}  & \uncover<5->{\Number{1....}} & \only<5>{\Number{.1100 1....~~~~}}%
                                                               \\
            \uncover<4->{-2}  & \uncover<4->{\Number{.0...}} & \only<5>{\Number{..... .....~~~~}}%
                                                               \only<4>{\Number{~..... .....~~~}}% 
                                                               \\
            \uncover<3->{-3}  & \uncover<3->{\Number{..1..}} & \only<5>{\Number{...11 001..~~~~}}%
                                                               \only<4>{\Number{~..110 01...~~~}}% 
                                                               \only<3>{\Number{~~.1100 1....~~}}% 
                                                               \\
            \uncover<2->{-4}  & \uncover<2->{\Number{...1.}} & \only<5>{\Number{....1 1001.~~~~}}%
                                                               \only<4>{\Number{~...11 001..~~~}}% 
                                                               \only<3>{\Number{~~..110 01...~~}}% 
                                                               \only<2>{\Number{~~~.1100 1....~}}% 
                                                               \\
            \uncover<1->{-5}  & \uncover<1->{\Number{....1}} & \only<5>{\Number{..... 11001~~~~}}% 
                                                               \only<4>{\Number{~....1 1001.~~~}}% 
                                                               \only<3>{\Number{~~...11 001..~~}}% 
                                                               \only<2>{\Number{~~~..110 01...~}}% 
                                                               \only<1>{\Number{~~~~.1100 1....}}% 
                                                               \\ \hline
            \multicolumn{2}{r}{\uncover<5>{Результат:}} 
                                   &    \only<5>{\Number{10001 11111~~~~}}%
                                        \only<4>{\Number{~01010 11110~~~}}%
                                        \only<3>{\Number{~~10101 11100~~}}%
                                        \only<2>{\Number{~~~10010 11000~}}%
                                        \only<1>{\Number{~~~~01100 10000}}%
                                        \\
        \end{tabular}
    \]
\end{frame}

\begin{frame}
    \frametitle{I способ: аналитически}

    \begin{gather*}
        a_{n}B\cdot2^{-n}+a_{(n-1)}B\cdot2^{-(n-1)}+ \cdots 
        +a_{2}B\cdot2^{-2}+a_{1}B\cdot2^{-1}
        \\
        \Leftrightarrow
        \\
        \left(\left(\cdots\left(\left(a_{n}\frac{B}{2}\right)\cdot 2^{-1} 
            + a_{(n-1)}\frac{B}{2}\right)\cdot 2^{-1} 
                + \cdots 
                    + a_{2}\frac{B}{2}\right)\cdot 2^{-1}
                        + a_{1}\frac{B}{2}\right)
    \end{gather*}

    В реккурентной форме:
    \[
        S_i=
        \begin{cases}
            \displaystyle a_{n}\frac{B}{2},                         &\text{если $i=n$},\\
            \displaystyle S_{(i+1)}\cdot 2^{-1} + a_i\frac{B}{2},   &\text{если $i<n$}.\\
        \end{cases}
    \]
    
    $S_1$ --- результат ($S_n\to S_{n-1}\to\cdots\to S_2\to S_1$)
\end{frame}

\begin{frame}
    \frametitle{I способ: реализация}
    
    \begin{center}
        \includegraphics[width=.8\textwidth]{fig/mult-methods.1}
    \end{center}
\end{frame}

\subsection{II способ}

\begin{frame}
    \begin{center}
        II-й способ
    \end{center}
\end{frame}

\begin{frame}
    \frametitle{II-й способ}
    
    \begin{align*}
        A\equiv\Number{,10111},\\
        B\equiv\Number{,11001}.
    \end{align*}
    
    \[    
        \begin{tabular}{c|c|c}
                                                                     \hline\hline
            $i$ & Разряды $a_i$    & $(B\cdot 2^{-i}) \cdot a_i$  \\ \hline\hline
            \uncover<5->{-1}  & \uncover<5->{\Number{1....}} & \uncover<5->{\Number{.1100 1....}} \\
            \uncover<4->{-2}  & \uncover<4->{\Number{.0...}} & \uncover<4->{\Number{..... .....}} \\
            \uncover<3->{-3}  & \uncover<3->{\Number{..1..}} & \uncover<3->{\Number{...11 001..}} \\
            \uncover<2->{-4}  & \uncover<2->{\Number{...1.}} & \uncover<2->{\Number{....1 1001.}} \\
            \uncover<1->{-5}  & \uncover<1->{\Number{....1}} & \uncover<1->{\Number{..... 11001}} \\ \hline
            \multicolumn{2}{r}{\uncover<5>{Результат:}} 
                                   &    \only<5>{\Number{10001 11111}}%
                                        \only<4>{\Number{00101 01111}}%
                                        \only<3>{\Number{00101 01111}}%
                                        \only<2>{\Number{00010 01011}}%
                                        \only<1>{\Number{00000 11001}}%
                                        \\
        \end{tabular}
    \]
\end{frame}

\begin{frame}
    \frametitle{II способ: аналитически}

    \begin{gather*}
        a_{n}\cdot\frac{B}{2^{n}}
            +a_{(n-1)}\cdot\frac{B}{2^{(n-1)}}
                +a_{(n-2)}\cdot\frac{B}{2^{(n-2)}}
                    + \cdots 
                        +a_{1}\cdot\frac{B}{2^{-1}}
        \\
        \Leftrightarrow
        \\
        \left(\cdots\left(\left(\left(a_{n}\cdot\frac{B}{2^{n}}\right)
            +a_{(n-1)}\cdot\frac{B}{2^{(n-1)}}\right)
                +a_{(n-2)}\cdot\frac{B}{2^{(n-2)}}\right)
                    + \cdots 
                        +a_{1}\cdot\frac{B}{2^{-1}}\right)
    \end{gather*}
    
    В реккурентной форме:
    \[
        S_i=
        \begin{cases}
            \displaystyle a_{n}\cdot\frac{B}{2^{n}},            &\text{если $i=n$},\\
            \displaystyle S_{(i+1)} + a_i\cdot\frac{B}{2^{i}}, &\text{если $i<n$}.\\
        \end{cases}
    \]
    $S_1$ --- результат ($S_n\to S_{n-1}\to\cdots\to S_2\to S_1$)
\end{frame}

\begin{frame}
    \frametitle{II-способ: реализация}
    
    \begin{center}
        \includegraphics[width=.8\textwidth]{fig/mult-methods.2}
    \end{center}
\end{frame}


\subsection{III способ}

\begin{frame}
    \begin{center}
        III-й способ
    \end{center}
\end{frame}

\begin{frame}
    \frametitle{III-й способ}
    
    \begin{align*}
        A\equiv\Number{,10111},\\
        B\equiv\Number{,11001}.
    \end{align*}
    
    \[    
        \begin{tabular}{c|c|c}
                                                                     \hline\hline
            $i$ & Разряды $a_i$    & $(B\cdot 2^{-i}) \cdot a_i$  \\ \hline\hline
            \uncover<1->{-1}  & \uncover<1->{\Number{1....}} & \only<1>{\Number{..... 11001~~~~}}% 
                                                               \only<2>{\Number{~....1 1001.~~~}}%
                                                               \only<3>{\Number{~~...11 001..~~}}%
                                                               \only<4>{\Number{~~~..110 01...~}}%
                                                               \only<5>{\Number{~~~~.1100 1....}}%
                                                               \\
            \uncover<2->{-2}  & \uncover<2->{\Number{.0...}} & \only<2>{\Number{~..... .....~~~}}%
                                                               \only<3>{\Number{~~..... .....~~}}%
                                                               \only<4>{\Number{~~~..... .....~}}%
                                                               \only<5>{\Number{~~~~..... .....}}%
                                                               \\
            \uncover<3->{-3}  & \uncover<3->{\Number{..1..}} & \only<3>{\Number{~~.....11 001~~}}% 
                                                               \only<4>{\Number{~~~....11 001.~}}%
                                                               \only<5>{\Number{~~~~...11 001..}}%
                                                               \\
            \uncover<4->{-4}  & \uncover<4->{\Number{...1.}} & \only<4>{\Number{~~~.....1 1001~}}%
                                                               \only<5>{\Number{~~~~....1 1001.}}%
                                                               \\
            \uncover<5->{-5}  & \uncover<5->{\Number{....1}} & \only<5>{\Number{~~~~..... 11001}}%
                                                               \\ \hline
            \multicolumn{2}{r}{\uncover<5>{Результат:}} 
                                   &    \only<1>{\Number{00000 11001~~~~}}%
                                        \only<2>{\Number{~00001 10010~~~}}%
                                        \only<3>{\Number{~~00011 11101~~}}%
                                        \only<4>{\Number{~~~01000 10011~}}%
                                        \only<5>{\Number{~~~~10001 11111}}%
                                        \\
        \end{tabular}
    \]
\end{frame}

\begin{frame}
    \frametitle{III способ: аналитически}

    \begin{gather*}
        a_{1}B\cdot2^{-1}+a_{2}B\cdot2^{-2}+ \cdots 
        +a_{(n-1)}B\cdot2^{-(n-1)}+a_{n}B\cdot2^{-n}
        \\
        \Leftrightarrow
        \\
        \left(\left(\cdots\left(\left(a_{1}\frac{B}{2^n}\right)\cdot 2
            +a_{2}\frac{B}{2^n}\right)\cdot 2
                + \cdots             
                    +a_{(n-1)}\frac{B}{2^n}\right)\cdot 2
                        +a_{n}\frac{B}{2^n}\right)
    \end{gather*}
    
    В реккурентной форме:
    \[
        S_i=
        \begin{cases}
            \displaystyle a_{1}\frac{B}{2^n},                     &\text{если $i=1$},\\
            \displaystyle S_{(i-1)}\cdot 2 + a_{i}\frac{B}{2^n},  &\text{если $i>1$}.\\
        \end{cases}
    \]
    $S_n$ --- результат ($S_1\to S_2\to\cdots\to S_{n-1}\to S_n$)
\end{frame}

\begin{frame}
    \frametitle{III-способ: реализация}
    
    \begin{center}
        \includegraphics[width=.8\textwidth]{fig/mult-methods.3}
    \end{center}
\end{frame}


\subsection{IV способ}

\begin{frame}
    \begin{center}
        IV способ
    \end{center}
\end{frame}

\begin{frame}
    \frametitle{IV способ}
    
    \begin{align*}
        A\equiv\Number{,10111},\\
        B\equiv\Number{,11001}.
    \end{align*}
    
    \[    
        \begin{tabular}{c|c|c}
                                                                     \hline\hline
            $i$ & Разряды $a_i$    & $(B\cdot 2^{-i}) \cdot a_i$  \\ \hline\hline
            \uncover<1->{-1}  & \uncover<1->{\Number{1....}} & \uncover<1->{\Number{.1100 1....}} \\
            \uncover<2->{-2}  & \uncover<2->{\Number{.0...}} & \uncover<2->{\Number{..... .....}} \\
            \uncover<3->{-3}  & \uncover<3->{\Number{..1..}} & \uncover<3->{\Number{...11 001..}} \\
            \uncover<4->{-4}  & \uncover<4->{\Number{...1.}} & \uncover<4->{\Number{....1 1001.}} \\
            \uncover<5->{-5}  & \uncover<5->{\Number{....1}} & \uncover<5->{\Number{..... 11001}} \\ \hline
            \multicolumn{2}{r}{\uncover<5>{Результат:}} 
                                   &    \only<1>{\Number{01100 10000}}%
                                        \only<2>{\Number{01100 10000}}%
                                        \only<3>{\Number{01111 10100}}%
                                        \only<4>{\Number{10001 00110}}%
                                        \only<5>{\Number{10001 11111}}%
                                        \\
        \end{tabular}
    \]
\end{frame}

\begin{frame}
    \frametitle{IV способ: аналитически}

    \begin{gather*}
        a_{1}B\cdot2^{-1}+a_{2}B\cdot2^{-2}+a_{3}B\cdot2^{-3}+ \cdots 
        +a_{(n)}B\cdot2^{-n}
        \\
        \Leftrightarrow
        \\
        (\cdots(((a_{1}B\cdot2^{-1})+a_{2}B\cdot2^{-2})+a_{3}B\cdot2^{-3})+ \cdots 
        +a_{n}B\cdot2^{-n})
    \end{gather*}

    В реккурентной форме:
    \[
        S_i=
        \begin{cases}
            a_1B\cdot 2^{-1},&\text{если $i=1$},\\
            S_{(i-1)} + a_iB\cdot 2^{-i},&\text{если $i>1$}.\\
        \end{cases}
    \]
    $S_n$ --- результат ($S_1\to S_2\to\cdots\to S_{n-1}\to S_n$)
\end{frame}

\begin{frame}
    \frametitle{IV-способ: реализация}
    
    \begin{center}
        \includegraphics[width=.8\textwidth]{fig/mult-methods.4}\\
    \end{center}
\end{frame}

\begin{frame}
    Резюме:
    \begin{center}
        \begin{tabular}{c||c||c}
                & Сдвиг СЧП
                    & Сдвиг множимого\\
            \hline\hline
            \rotatebox{90}{\texttt{shr}(Мн-ль)}
                & \includegraphics[width=.43\textwidth]{fig/mult-methods.1}
                    & \includegraphics[width=.43\textwidth]{fig/mult-methods.2}\\
            \hline\hline
            \rotatebox{90}{\texttt{shl}(Мн-ль)}
                & \includegraphics[width=.43\textwidth]{fig/mult-methods.3}
                    & \includegraphics[width=.43\textwidth]{fig/mult-methods.4}\\
        \end{tabular}
    \end{center}
\end{frame}


\section{Примеры умножения в прямом коде}

\begin{frame}
    \frametitle{Умножение в прямом коде}

    При умножении в прямом коде знак результата и результат умножения модулей формируются независимо. 
    
    \begin{enumerate}
        \item Знак результата
        \begin{block}{}
            получается сложением <<по модулю два>> (XOR) знаков множителя и множимого.
        \end{block}
        
        \item Модуль (мантисса) результата получается беззнаковым перемножением мантисс операндов.
        
        \item Корректируется, если нужно, запрещенная комбинация <<минус ноль>>.
    \end{enumerate}
\end{frame}

\begin{frame}
    \frametitle{Операнды}

    Перемножить числа в прямом коде
    
    Множитель: \[-25=(-11001)_2.\]
    
    Множимое: \[-23=(-10111)_2.\]
    
    Используем дробное масштабирование с множителем $2^5$.
    
    Знак получаем отдельно: $1\oplus 1=0$. Результат положителен.
    
    Далее показано только получение мантиссы результата разными способами. В таблицах отражаеются состояния основных регистров по тактам.
\end{frame}


\subsection{I способ}

\begin{frame}
    \frametitle{I-способ}

    \begin{tabular}{c|r|l}
                                                                   \hline\hline
        мн-ль $\rightarrow$ & 
                                \multicolumn{1}{|c|}{СЧП $\rightarrow$}       
                                                        & прим. \\ \hline\hline
        \NumberLo{,1100}{1} & \Addition{,00000 00000}
                                       {,.1011 1....}
                                       {,01011 10000} & +мн-е/2; сдвиг\\ \hline
        \NumberLo{,.110}{0} &   \Number{,00101 11000} & сдвиг\\ \hline
        \NumberLo{,..11}{0} &   \Number{,00010 11100} & сдвиг\\ \hline
        \NumberLo{,...1}{1} & \Addition{,00001 01110}
                                       {,.1011 1....}
                                       {,01100 11110} & +мн-е/2; сдвиг\\ \hline
        \NumberLo{,....}{1} & \Addition{,00110 01111}
                                       {,.1011 1....}
                                       {,10001 11111} & +мн-е/2; Рез-т!\\
    \end{tabular}
\end{frame}


\subsection{II способ}

\begin{frame}
    \frametitle{II-способ}

    \begin{tabular}{c|r|r|l}
                                                                   \hline\hline
        множитель $\rightarrow$ 
                            & \multicolumn{1}{|c|}{мн-е $\leftarrow$}       
                                                     & \multicolumn{1}{|c|}{СЧП}       
                                                                                  & прим. \\ \hline\hline
        \NumberLo{,1100}{1} & \Number{,..... 10111} & \Addition {,00000 00000} 
                                                                {,..... 10111}
                                                                {,00000 10111} & +мн-е; сдвиг\\ \hline
        \NumberLo{,.110}{0} & \Number{,....1 0111.} &                           & сдвиг\\ \hline
        \NumberLo{,..11}{0} & \Number{,...10 111..} &                           & сдвиг\\ \hline
        \NumberLo{,...1}{1} & \Number{,..101 11...} & \Addition {,00000 10111} 
                                                                {,..101 11...}
                                                                {,00110 01111} & +мн-е; сдвиг\\ \hline
        \NumberLo{,....}{1} & \Number{,.1011 1....} & \Addition {,00110 01111} 
                                                                {,.1011 1....}
                                                                {,10001 11111} & +мн-е; Рез-т!\\
    \end{tabular}
\end{frame}


\subsection{III способ}

\begin{frame}
    \frametitle{III-способ}

    \begin{tabular}{c|r|l}
                                                                   \hline\hline
        мн-ль $\leftarrow$ 
                               & \multicolumn{1}{|c|}{СЧП $\leftarrow$}       
                                                           & прим.\\ \hline\hline
        \NumberMid{,}{1}{1001} & \Addition{,00000 00000}
                                          {,..... 10111}
                                          {,00000 10111} & +мн-е; сдвиг\\ \hline
        \NumberMid{,}{1}{001.} & \Addition{,00001 0111.}
                                          {,..... 10111}
                                          {,00010 00101} & +мн-е; сдвиг\\ \hline
        \NumberMid{,}{0}{01..} &   \Number{,00100 0101.} & сдвиг\\ \hline
        \NumberMid{,}{0}{1...} &   \Number{,01000 101..} & сдвиг\\ \hline
        \NumberMid{,}{1}{....} & \Addition{,10001 01...}
                                          {,..... 10111}
                                          {,10001 11111} & +мн-е; Рез-т!\\ 
    \end{tabular}
\end{frame}


\subsection{IV способ}

\begin{frame}
    \frametitle{IV-способ}

    \begin{tabular}{c|r|r|l}
                                                                   \hline\hline
        мн-ль $\leftarrow$ 
                               & \multicolumn{1}{|c|}{мн-е $\rightarrow$}       
                                                       & \multicolumn{1}{|c|}{СЧП}       
                                                                                 & прим. \\ \hline\hline
        \NumberMid{,}{1}{1001} & \Number{,.1011 1....} & \Addition{,00000 00000} 
                                                                  {,.1011 1....}
                                                                  {,01011 10000} & +мн-е; сдвиг;\\ \hline
        \NumberMid{,}{1}{001.} & \Number{,..101 11...} & \Addition{,01011 10000} 
                                                                  {,..101 11...}
                                                                  {,10001 01000} & +мн-е; сдвиг\\ \hline
        \NumberMid{,}{0}{01..} & \Number{,...10 111..} &                         & сдвиг\\ \hline
        \NumberMid{,}{0}{1...} & \Number{,....1 0111.} &                         & сдвиг\\ \hline
        \NumberMid{,}{1}{....} & \Number{,..... 10111} & \Addition{,10001 01000} 
                                                                  {,..... 10111}
                                                                  {,10001 11111} & +мн-е; Рез-т!\\
    \end{tabular}
\end{frame}

\appendix


\section{Задания на практику}


\subsection{Проходное}

\begin{frame}
    \frametitle{\TaskSimpleNumber}
    Какова минимальная разрядность результата перемножения $n$-разрядных прямых кодов?
\end{frame}

\begin{frame}
    \frametitle{\TaskSimpleNumber}
    Перемножить числа $-91$ и $114$. Самостоятельно выбрать масштаб.
\end{frame}

\begin{frame}
    \frametitle{\TaskSimpleNumber}
    Выявить ситуации получения неправильных прямых кодов.
\end{frame}

\begin{frame}
    \frametitle{\TaskSimpleNumber}
    Обосновать, работает ли схема умножения модулей первым способом:
    \begin{center}
        \includegraphics{fig/mult-methods.5}
    \end{center}
    
    \begin{itemize}
        \item Как её модифицировать, если она работает неправильно?
        \item Где получается результат?
    \end{itemize}
\end{frame}

\section{Самообучение}

\begin{frame}
    \frametitle{Советы самоучке}
    
    Классика жанра: \cite{bib:lisikov:automateBase,bib:saveliev:automateTheory}.

    Рекомендуется почитать разделы, посвященные работе с суммами и рекуррентными соотношениями в книге \cite{bib:knuth:concretemath}.
\end{frame}

\begin{frame}[allowframebreaks]{Библиография}
    \bibliographystyle{gost780u}
    \bibliography{./../../../bibliobase}
\end{frame}

\end{document}