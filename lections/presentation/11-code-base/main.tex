%include part: see main.beamer.tex and main.article.tex
%include common packages and settings
\usepackage{etex} %эта магическая херь избавляет от переполнения регистров TeX а!!!

\mode<article>{\usepackage{fullpage}}
\mode<presentation>{
    \usetheme{Madrid} %%Boadilla,Madrid,AnnArbor,CambridgeUS,Malmoe,Singapore,Berlin
    \useoutertheme{shadow}
} 

\usepackage[utf8]{inputenc}
\usepackage[russian]{babel}
\usepackage{indentfirst}
\usepackage{graphicx}

\usepackage{amsmath}
\usepackage{amsfonts}
\usepackage{amsthm}
\usepackage{algorithm}
\usepackage{algorithmic}

\usepackage[all]{xy}

\date{Лекция по дисциплине <<информатика>>\\(\today)}
\author[М.~М.~Шихов]{Михаил Шихов \\ \texttt{\underline{m.m.shihov@gmail.com}}}

%для рисования графов пакетом xy-pic
\entrymodifiers={++[o][F-]}

%для псевдокода алгоритмов (algorithm,algorithmic)
\renewcommand{\algorithmicrequire}{\textbf{Вход:}}
\renewcommand{\algorithmicensure}{\textbf{Выход:}}
\renewcommand{\algorithmiccomment}[1]{// #1}
\floatname{algorithm}{Псевдокод}

%%определённые мной команды логической разметки
\newcommand{\Signs}[2]{\fbox{{\small{\underbar{#1}}}{#2}}}
\newcommand{\Sign}[1]{\fbox{#1}}
\newcommand{\DC}[1]{\text{ДК}(#1)}
\newcommand{\OC}[1]{\text{ОК}(#1)}
\newcommand{\PC}[1]{\text{ПК}(#1)}

\newcommand{\MyProc}{\text{$\mathbf{R}_8$}}

\newboolean{IsNeedAnswer}
\setboolean{IsNeedAnswer}{false} %true/false

\newcommand{\Machine}[1]{\texttt{#1}}
\newcommand{\Opcode}[1]{\texttt{\bf{#1}}}
\newcommand{\Operand}[1]{\texttt{#1}}
\newcommand{\CmdOneAddr}[2]{\text{\Opcode{#1} \Operand{#2}}}
\newcommand{\CmdTwoAddr}[3]{\text{\Opcode{#1} \Operand{#2}, \Operand{#3}}}
\newcommand{\CmdThreeAddr}[4]{\text{\Opcode{#1} {\Operand{#2}}, \Operand{#3}, \Operand{#4}}}

\newcommand{\ProofAnswer}[1]{
    \ifthenelse{\boolean{IsNeedAnswer}}{
        \begin{proof}[Ответ]
            #1
        \end{proof}
    }{}
}

\newcommand{\LabeledAnswer}[1]{
    \ifthenelse{\boolean{IsNeedAnswer}}{
        \emph{Отв.}: #1 \qed
    }{}
}

\newcommand{\PlainAnswer}[1]{
    \ifthenelse{\boolean{IsNeedAnswer}}{#1}{}
}

\newcommand{\UnsignedAny}[2]{\text{\upshape
    \begin{tabular}{lr}
        \tiny{#1} & \tiny{0}\\ 
        \hline
        \multicolumn{2}{|c|}{\texttt{#2}} \\ 
        \hline
    \end{tabular}
}}

\newcommand{\UnsignedByte}[1]{\UnsignedAny{7}{#1}}

\newcommand{\UnsignedTwoBytes}[1]{\UnsignedAny{15}{#1}}

\newcommand{\FixedAny}[4]{\text{\upshape
    \begin{tabular}{clr}
        \tiny{#1}  &\tiny{#2} & \tiny{0}\\ 
        \hline
        \multicolumn{1}{|c|}{\texttt{#3}} & \multicolumn{2}{|c|}{\texttt{#4}} \\ 
        \hline
    \end{tabular}
}}

\newcommand{\FixedByte}[2]{\FixedAny{7}{6}{#1}{#2}}

\newcommand{\FixedTwoBytes}[2]{\FixedAny{15}{14}{#1}{#2}}

\newcommand{\SignedAny}[4]{\text{\upshape
    \begin{tabular}{clr}
        \tiny{#1}  &\tiny{#2} & \tiny{0}\\ 
        \hline
        \multicolumn{1}{|c|}{\Machine{#3}} & \multicolumn{2}{|c|}{\Machine{#4}} \\ 
        \hline
    \end{tabular}
}}

\newcommand{\SignedNibble}[2]{\SignedAny{3}{2}{#1}{#2}}

\newcommand{\SignedByte}[2]{\SignedAny{7}{6}{#1}{#2}}

\newcommand{\SignedTwoBytes}[2]{\SignedAny{15}{14}{#1}{#2}}

\newcommand{\FloatMyHex}[4]{\text{\upshape
    \begin{tabular}{clrclr}
        \tiny{15}  &\tiny{14} & \tiny{6} & \tiny{5} & \tiny{4} & \tiny{0}\\ 
        \hline
        \multicolumn{1}{|c|}{\texttt{#1}} 
            & \multicolumn{2}{|c|}{\texttt{#2}} 
                & \multicolumn{1}{|c|}{\texttt{#3}} 
                    & \multicolumn{2}{|c|}{\texttt{#4}} \\ 
        \hline
    \end{tabular}
}}

\newcommand{\FloatMyCharHex}[3]{\text{\upshape
    \begin{tabular}{clrlr}
        \tiny{15}  &\tiny{14} & \tiny{6} & \tiny{5} & \tiny{0}\\ 
        \hline
        \multicolumn{1}{|c|}{\texttt{#1}} 
            & \multicolumn{2}{|c|}{\texttt{#2}} 
                & \multicolumn{2}{|c|}{\texttt{#3}} \\ 
        \hline
    \end{tabular}
}}

\newcommand{\FloatMyDcMantCharHex}[2]{\text{\upshape
    \begin{tabular}{lrlr}
        \tiny{15} & \tiny{6} & \tiny{5} & \tiny{0}\\ 
        \hline
        \multicolumn{2}{|c|}{\texttt{#1}} 
            & \multicolumn{2}{|c|}{\texttt{#2}} \\ 
        \hline
    \end{tabular}
}}

\newcommand{\FloatESShort}[3]{\text{\upshape
    \begin{tabular}{clrlr}
        \tiny{31}  &\tiny{30} & \tiny{24} & \tiny{23} & \tiny{0}\\ 
        \hline
        \multicolumn{1}{|c|}{\texttt{#1}} 
            & \multicolumn{2}{|c|}{\texttt{#2}} 
                & \multicolumn{2}{|c|}{\texttt{#3}} 
                    \\ 
        \hline
    \end{tabular}
}}

\newcommand{\FloatPCShort}[3]{\text{\upshape
    \begin{tabular}{clrlr}
        \tiny{31}  &\tiny{30} & \tiny{23} & \tiny{22} & \tiny{0}\\ 
        \hline
        \multicolumn{1}{|c|}{\texttt{#1}} 
            & \multicolumn{2}{|c|}{\texttt{#2}} 
                & \multicolumn{2}{|c|}{\texttt{#3}} 
                    \\ 
        \hline
    \end{tabular}
}}

\newcommand{\FloatMyOrderX}[4]{\text{\upshape
    \begin{tabular}{clrclr}
        \tiny{9}  &\tiny{8} & \tiny{4} & \tiny{3} & \tiny{2} & \tiny{0}\\ 
        \hline
        \multicolumn{1}{|c|}{\texttt{#1}} 
            & \multicolumn{2}{|c|}{\texttt{#2}} 
                & \multicolumn{1}{|c|}{\texttt{#3}} 
                    & \multicolumn{2}{|c|}{\texttt{#4}} \\ 
        \hline
    \end{tabular}
}}

\newcommand{\FloatMyCharX}[3]{\text{\upshape
    \begin{tabular}{clrlr}
        \tiny{9}  &\tiny{8} & \tiny{4} & \tiny{3} & \tiny{0}\\ 
        \hline
        \multicolumn{1}{|c|}{\texttt{#1}} 
            & \multicolumn{2}{|c|}{\texttt{#2}} 
                & \multicolumn{2}{|c|}{\texttt{#3}} \\ 
        \hline
    \end{tabular}
}}

\newcommand{\FloatMyDcOrderX}[3]{\text{\upshape
    \begin{tabular}{lrclr}
        \tiny{9} & \tiny{4} & \tiny{3} & \tiny{2} & \tiny{0}\\ 
        \hline
        \multicolumn{2}{|c|}{\texttt{#1}} 
            & \multicolumn{1}{|c|}{\texttt{#2}} 
                & \multicolumn{2}{|c|}{\texttt{#3}} \\ 
        \hline
    \end{tabular}
}}

\newcommand{\FloatMyDcCharX}[2]{\text{\upshape
    \begin{tabular}{lrlr}
        \tiny{9} & \tiny{4} & \tiny{3} & \tiny{0}\\ 
        \hline
        \multicolumn{2}{|c|}{\texttt{#1}} 
            & \multicolumn{2}{|c|}{\texttt{#2}} \\ 
        \hline
    \end{tabular}
}}


%--- СПЕЦИФИЧНЫЕ ДЛЯ УМНОЖЕНИЯ КОМАНДЫ ---------------------------------------------------------------------------------------------


\newcommand{\Number}[1]{
    \texttt{#1}
}

\newcommand{\NumberHi}[2]{
    \underline{\underline{\texttt{#1}}}\texttt{#2}
}

\newcommand{\NumberMid}[3]{
    \texttt{#1}\underline{\underline{\texttt{#2}}}\texttt{#3}
}

\newcommand{\NumberLo}[2]{
    \texttt{#1}\underline{\underline{\texttt{#2}}}
}

\newcommand{\Stack}[2]{
    \begin{tabular}[t]{@{}r@{}}
        {#1}\\ \hline
        {#2}\\ 
    \end{tabular}
}

\newcommand{\StackThree}[3]{
    \begin{tabular}[t]{@{}r@{}}
        {#1}\\ \hline
        {#2}\\ \hline
        {#3}\\
    \end{tabular}
}

\newcommand{\StackFour}[4]{
    \begin{tabular}[t]{@{}r@{}}
        {#1}\\ \hline
        {#2}\\ \hline
        {#3}\\ \hline
        {#4}\\
    \end{tabular}
}

\newcommand{\Operation}[4]{
    \begin{tabular}[t]{@{}r@{}}
        \texttt{#4}
        \begin{tabular}{@{}r@{}}
            \Number{#1}\\
            \Number{#2}\\ \hline
        \end{tabular} \\ 
        \Number{#3}\\
    \end{tabular}
}

\newcommand{\Addition}[3]{\Operation{#1}{#2}{#3}{+}}

\newcommand{\Subtraction}[3]{\Operation{#1}{#2}{#3}{-}}

\newcommand{\Register}[2]{\Number{#1:#2}}

\newcommand{\Mantiss}{m}
\newcommand{\Order}{p}
\newcommand{\Char}{c}

\newcommand{\MantissOf}[1]{\Mantiss_{#1}}
\newcommand{\OrderOf}[1]{\Order_{#1}}
\newcommand{\CharOf}[1]{\Char_{#1}}

\newcommand{\FloatExpression}[2]{\MantissOf{#1}\cdot {#2}^{\OrderOf{#1}}}

\newcommand{\DivAnswer}[2]{(\texttt{$#1$ rem $#2$})}

\newenvironment{Solve}[1]%
    {\begin{proof}[Решение]#1}
    {\end{proof}}
    
    


\title{Основы кодирования}


\begin{document}

%титул и содержание статьи
\mode<article>{\maketitle\tableofcontents}

%титул и содержание презентации
\frame<presentation>{\titlepage}
\begin{frame}<presentation>
    \frametitle{Содержание}
    \tableofcontents
\end{frame}


\begin{frame}
    \frametitle{Определение информации}
    
    \begin{definition}[Юридическое]
        \alert{Информация} --- это сведения (сообщения, данные) независимо от формы их представления\footnote{№149-ФЗ от 27 июля 2006 г <<Об информации, информационных технологиях и о защите информации>>}.
    \end{definition}
    
    \begin{definition}
        \alert{Информация} --- это упорядоченная последовательность (цепочка) \alert{кодовых символов}, принадлежащих конечному алфавиту. При этом каждый символ последовательности несёт определённую смысловую нагрузку.
    \end{definition}
\end{frame}

\begin{frame}
    \begin{center}
        \includegraphics[width=.95\textwidth]{fig/iproperties}
    \end{center}
\end{frame}

\begin{frame}
    \frametitle{Уровни доступа к информации}
    
    \begin{center}
        \begin{tabular}[c]{l|c|c|c|}
            \hline
            Носителя 
                & \includegraphics[width=.12\textwidth]{fig/book}
                    & \includegraphics[width=.12\textwidth]{fig/flash}
                        & \includegraphics[width=.12\textwidth]{fig/frog}
                            \\ \hline
            Средств взаимодействия
                & \includegraphics[width=.12\textwidth]{fig/book-tool}
                    & \includegraphics[width=.12\textwidth]{fig/flash-tool}
                        & \includegraphics[width=.12\textwidth]{fig/frog-tool}
                            \\ \hline
            Представления (код, формат)
                & \includegraphics[width=.12\textwidth]{fig/book-code}
                    & \includegraphics[width=.12\textwidth]{fig/flash-code}
                        & \includegraphics[width=.12\textwidth]{fig/frog-code}
                            \\ \hline
            Семантики (понимания)
                & \includegraphics[width=.10\textwidth]{fig/eurika}
                    & \includegraphics[width=.10\textwidth]{fig/eurika}
                        & \includegraphics[width=.10\textwidth]{fig/eurika}
                            \\ \hline
        \end{tabular}
    \end{center}
    
\end{frame}

\begin{frame}
    \frametitle{Трудности оценки количества информации}

    \begin{itemize}
        \item Просто оценить количество явно заданной информации: достаточно посчитать количество кодовых символов последовательности. 
    
        \item Сложно оценить необходимое количество информации для адекватного <<отражения>> исходного объекта.
    \end{itemize}
\end{frame}

Отражаемый объект рассматривается как источник сообщений, которые для потребителя этих сообщений несут определенную смысловую нагрузку. Поэтому оценка адекватности информации зависит в первую очередь от способности потребителя принимать (понимать, воспринимать) сообщения\footnote{Звонок в службу поддержки оператора сотовой связи. Клиент: <<Девушка, до меня не доходят sms сообщения!!!>>.  Девушка: <<Прочтите еще раз\ldots>>. Бородатый анекдот.}. 

Обсуждая синтаксический подход, мы приводили пример двух сообщений, несущих разное (с точки зрения синтаксического подхода) количество информации: <<Жучка укусила Иванова>> и <<Иванов укусил Жучку>> и просили сравнить ощущения. Теперь сравните свои ощущения от следующих двух сообщений: <<бутявка укусила калушу>> и <<калуша укусила бутявку>>\footnote{Цикл лингвистических сказок <<Пуськи бятые>> Петрушевской Людмилы Стефановны}. Не доходят сообщения? Ноль адекватной информации в обоих случаях? Вероятно, это происходит потому, что в вашем личном тезаурусе нет определения смысла для <<бутявки>> и <<калуши>>, а также неизвестно в каких отношениях они находятся.

Слову <<тезаурус>> есть много определений. От древнегреческого: <<словарь с примерами>>, до современного лаконичного: <<совокупность выражающих смысл единиц языка с описанием отношений между ними>>. Оценки на основе тезауруса получили наибольшее распространение. При этом в большинстве случаев предполагается, что получатель получает именно информацию (цепочку символов), а не наблюдает события (то есть отражение уже свершилось). Следовательно, под сообщением понимается информация, а не событие.

Тезаурус пользователя мы обозначим $S_p$. Если $S_p$ пуст ($S_p=\emptyset$), то пользователь не понимает (не воспринимает) информацию и для него она --- информационный шум. Ежели $S_p$ всеобъемлющ ($S_p\rightarrow\infty$), то пользователь знает все, и ему нет необходимости понимать (воспринимать) информацию, так как она уже не обогатит его тезаурус. 

В большинстве случаев тезаурус пользователя находится между этими крайностями ($\emptyset<S_p<\infty$) и в этом случае поступающую информацию можно разделить на три части. Допустим, что потребитель получает информационную цепочку определенной длины ($I$). Причем в этой цепочке можно выделить фрагменты, несущие смысловые единицы и отношения которые: уже имеются в тезаурусе пользователя ($I_s$); отсутствуют в тезаурусе, но поняты (восприняты) и обогатят тезаурус ($I_c$); не поняты и представляют собой бесполезный информационный шум ($I_n$). Если длина фрагмента $I_n=0$, то поступающая информация идеально согласована с тезаурусом пользователя. При этом важной основной оценкой является коэффициент содержательности
\[C=\frac{I_c}{I}.\]
Как видно, этот коэффициент представляет собой отношение количества новой, осмысленной информации к общему её количеству.

\begin{frame}
    \frametitle{Семантический подход к количественной оценке}
    \framesubtitle{Зависимость $I_c$ в сообщении от тезауруса $S_p$ получателя}

    \[I=I_s + I_c + I_n.\]
    где $I_s$ --- известна, $I_c$ --- неизвестна и понятна, $I_n$ --- шум. $C=\frac{I_c}{I}$ называется коэффициентом содержательности информации.
    \begin{center}
        \includegraphics[height=.45\textheight]{fig/semantic} 
    \end{center}
\end{frame}

Семантический подход учитывает особенности конкретного потребителя и дает количественную оценку адекватности информации с точки зрения её «осмысленности» для него. Одно и то же сообщение может оказаться полезным для компетентного, бессмысленным для некомпетентного и ненужным для всезнающего получателя. Внимание уделяется такой категории, как знание, а информация в данном случае --- это всего лишь способ (транстпорт) доставки знаний до познающего потребителя.


\begin{frame}
    \frametitle{Прагматический подход к количественной оценке}
    \framesubtitle{Оценка Александра Александровича Харкевича}
    
    \begin{equation}
        I=\log_{m}\frac{p_2}{p_1}=\log_{m}p_2 - \log_{m}p_1,
    \end{equation}
    где $m$ --- основание логарифма, определяющее единицы измерения ($m>1$), $p_1$ --- вероятность достижения потребителем \alert{цели} до получения информации, $p_2$ --- вероятность достижения потребителем цели после получения информации. Ценность информации в случае $p_1>p_2>0$ положительна, в случае $0<p_1<p_2$ отрицательна, а в случае $p_1=p_2$ равна нулю.
\end{frame}

В заключение следует отметить, что результаты оценок синтаксического подхода являются наиболее объективными, так как учитывают только отражаемый объект и избавлены от влияний со стороны конкретного потребителя, что характерно для остальных подходов.

\begin{frame}
    \frametitle{Синтаксический подход}
    \framesubtitle{Задача о картах (постановка)}
    
    \begin{example}
        Имеется колода из восьми карт. По две карты (туз и двойка) каждой масти. Некто вытягивает наугад карту и готов честно отвечать только да или нет на любые задаваемые вопросы. Требуется минимальным количеством вопросов угадать вытащенную карту.
    \end{example} 
\end{frame}

\begin{frame}
    \frametitle{Задача о картах (решение)}
    
    \begin{center}
        \includegraphics[width=.45\textwidth]{fig/cards} 
    \end{center} 
\end{frame}

\begin{frame}
    \frametitle{Количественная оценка Ральфа Хартли}
    \framesubtitle{$m^H\geq N$}
    
    \[
        H=\log_{m}N,
    \]
    где $m$ --- количество \alert{кодовых символов}; $N$ --- количество состояний \alert{отражаемого объекта}.

    \begin{example}
        В случае примера с картами: количество состояний $N=8$, количество символов $m=2$. Количество информации по Хартли: $H=\log_{2}8=3$ бита.
    \end{example}
\end{frame}

\begin{frame}
    \frametitle{Количественная оценка по Клоду Шеннону}
    \framesubtitle{Постулаты}
    
    Отражаемый объект --- источник \alert{событий}.
    \begin{enumerate}
        \item Количество информации есть непрерывная функция от вероятности события.

        \item Количество информации $I_i$ одиночного $i$-го события $s_i\in S$, $1\leq i\leq N$ происходящего с вероятностью $p_i$, имеет положительное значение. 
        \[I_i\geq 0; I_i=I(p_i); \sum_{i=1}^{N}p_i = 1.\]

        \item Количество информации $I_{ij}$ двух независимых событий $s_i,s_j\in S$ с вероятностью $p_{ij}=p_i\cdot p_j$, равно сумме количеств информаций событий в отдельности:
        \(I_{ij}=I_i+I_j; I(p_i\cdot p_j)=I(p_i) + I(p_j).\)
    \end{enumerate}
\end{frame}

\begin{example}
    Количество воспринимаемой информации влияет на эмоциональное состояние человека и постулаты Шеннона легко проверить на себе.
    
    Согласно Шеннону, количество информации зависит от вероятности наступления события. Сравните свои эмоции от двух событий: <<Жучка укусила Иванова>> и <<Иванов укусил Жучку>>. Первое событие, хоть собака и друг человека, будничное, а вот второе вызывает улыбку --- сенсация! Вероятность первого события весьма велика, вероятность второго близка к нулю. Первое событие несет мало информации, а второе несет большое её количество, отсюда и эмоции.

    В то же время, если мы на следующий день услышим, что после Иванова Жучку укусил еще и Петров, то внутренне мы будем готовы к тому, что на третий день Жучку укусит и Сидоров --- только ленивый не кусает Жучку! Хотя по правилам теории вероятности, с точки зрения обывателя, который не в курсе отношений Жучки с людьми, вероятность того, что Жучку укусит Иванов также мала, как и вероятность быть покусанной Петровым, и равна $p$. Вероятность, того, что Жучка будет покусана обоими, равна произведению вероятностей этих событий: $p^2$ --- это практически невероятно, так как $p^2$ много меньше $p$ и следует ожидать большой сенсации. Но! Никто (разве что Жучка) не падает в обморок от совместных действий Иванова и Петрова, так как количество информации, которое несет данное событие, есть лишь сумма количеств информаций для каждого факта оскорбления действием Жучки в отдельности.
    \qed
\end{example}

\begin{frame}
    \frametitle{Количественная оценка по Клоду Шеннону}
    \framesubtitle{Зависимость информации от вероятности}
    
    \[
        I(p)=-\log_{m}(p),
    \]
    где $I(p)$ --- информация события, происходящего с вероятностью $p$; $m$ --- количество \alert{кодовых символов}.
    
    \begin{example}[$m$ определяет единицы измерения информации]
        \begin{itemize}
            \item $m=2$. \alert{бит}.
            \item $m=e$. \alert{нат}.
            \item $m=3$. \alert{трит}.
            \item $m=10$. \alert{дит}.
        \end{itemize}    
    \end{example}
\end{frame}

\begin{frame}
    \frametitle{Зависимость количества информации от вероятности}
    
    \begin{center}
        \includegraphics[width=.8\textwidth]{fig/ip}
    \end{center} 
\end{frame}

\begin{frame}
    \frametitle{Задача о биллиардных шарах (постановка)}
    \begin{example}
        Имеется восемь биллиардных шаров с номерами 1-8 соответственно. Все шары одинаковой массы, кроме одного, который тяжелее остальных. Имеются весы Фемиды (чашечные). Какое количество взвешиваний потребуется, чтобы определить номер тяжелого шара?
    \end{example} 
\end{frame}

\begin{frame}
    \frametitle{Задача о биллиардных шарах (решение)}
    \framesubtitle{Решение. $H=\log_{3}9=I(p)=-\log_{3}\frac{1}{9}=2$ трита}
    
    \begin{center}
        \includegraphics[width=.7\textwidth]{fig/trit}  
    \end{center} 
\end{frame}

\begin{frame}
    \frametitle{Энтропия}
    \framesubtitle{Мера информативности источника событий (сколько выдаёт $I$ в среднем за раз)}
    
    \[
        E=-\sum_{i=1}^{N}p_{i}\cdot \log_{m}p_i.
        \label{eq:e}
    \]
    
    \begin{figure}[!ht]
        \centering
        \includegraphics[width=.6\textwidth]{fig/ecoin} 
        \caption{Энтропия для источника с двумя состояниями}\label{pict:ecoin}
    \end{figure} 
\end{frame}

\begin{frame}
    \frametitle{Война префиксов закончилась 19 марта 2005 года}
    \framesubtitle{Принят стандарт IEEE 1541. 1000 байт --- 1 kB (килобайт), 1024 байт --- 1KiB (кибибайт)}
    
    \begin{center}
        \begin{tabular}[c]{|l|l||l|l|}
            \hline\hline
            Множитель          & СИ/SI                  & Множитель        & IEEE 1541 \\
            \hline\hline
            $10^3  = 1000^1$   & \emph{kilo} (k) кило   & $2^{10} =1024^1$ & \emph{kibi} (Ki) киби\\ \hline
            $10^6  = 1000^2$   & \emph{mega} (M) мега   & $2^{20} =1024^2$ & \emph{mebi} (Mi) меби \\ \hline
            $10^9  = 1000^3$   & \emph{giga} (G) гига   & $2^{30} =1024^3$ & \emph{gibi} (Gi) гиби\\ \hline
            $10^{12} = 1000^4$ & \emph{tera} (T) тера   & $2^{40} =1024^4$ & \emph{tebi} (Ti) теби\\ \hline
            $10^{15} = 1000^5$ & \emph{peta} (P) пета   & $2^{50} =1024^5$ & \emph{pebi} (Pi) пеби\\ \hline
            $10^{18} = 1000^6$ & \emph{exa} (E) экса    & $2^{60} =1024^6$ & \emph{exbi} (Ei) эксби\\ \hline
            $10^{21} = 1000^7$ & \emph{zetta} (Z) зетта & $2^{70} =1024^7$ & \emph{zebi} (Zi) зеби\\ \hline
            $10^{24} = 1000^8$ & \emph{yotta} (Y) йотта & $2^{80} =1024^8$ & \emph{yobi} (Yi) йоби\\ \hline
        \end{tabular}
    \end{center}
\end{frame}

\begin{frame}
    \frametitle{Кодирование}
    
    \begin{definition}
        \alert{Кодирование} --- процесс перехода от \alert{источника событий} к \alert{источнику информации}. Т.е.  сопоставление \alert{событиям} цепочек из \alert{кодовых символов}.
    \end{definition}
    
    Некоторые назначения кодирования:
    \begin{enumerate}
        \item принципиальная возможность описания мира с помощью символов конечного алфавита;
        \item устранение избыточности, сжатие информации, экономия памяти и снижение нагрузки на каналы передачи информации;
        \item обеспечение устойчивости к помехам;
        \item защита важных свойств информации (конфиденциальность, целостность, принадлежность и т.д.).
    \end{enumerate}
\end{frame}

\begin{frame}
    \frametitle{Формальное определение кодирования}
    
    \begin{definition}
        Дано:
        \begin{itemize}
            \item Алфавит \alert{событий}: $S=\{s_1,\ldots,s_N\}$;
            \item Алфавит кодовых \alert{символов}: $T=\{t_1,\ldots,t_m\}$;
        \end{itemize}
        
        Требуется задать отображение $\delta:S\to T^{+}$ (таблицу кодов, \alert{схему кодирования}):
        \[
            \delta=\langle s_1\mapsto \omega_1,\ldots,s_N\mapsto \omega_N\rangle,
        \]
        где $\omega_i=t_{i_1}\cdots t_{i_{k_i}}$, причем слово $\varsigma_j=s_{j_1}\cdots s_{j_t}$ будет кодироваться символами кодового алфавита как $\varsigma_j=\omega_{j_1}\cdots \omega_{j_t}$.
        Множество кодовых слов $\omega_i$, соответствующих $s_i$ называется множеством \alert{элементарных} кодов:
        \[\Omega=\{\omega_1,\ldots,\omega_N\}.\]
    \end{definition} 
\end{frame}

\begin{frame}
    \frametitle{Примеры кодирования}

    \begin{example}[Неоднозначное декодирование. $\delta$ --- не биекция]
        $S=\{A,B,C,D,E,F,G,H\}$; $T=\{0,1\}$; $\delta=\langle A\to 0,B\to 1,C\to 10,D\to 11,E\to 100,F\to 101,G\to 110,H\to 111 \rangle$. 
        \begin{itemize}
            \item Кодирование однозначно: $ABAB\mapsto 0101$. 
            \item Декодирование нет: $0101$ разделяется на слова $ABAB$, $AF$ и $ACB$.
        \end{itemize}
    \end{example} 

    \begin{example}[Однозначное декодирование.  $\delta$ --- биекция]
        $S=\{A,B,C,D,E,F,G,H\}$; $T=\{0,1\}$; $\delta=\langle A\mapsto 000, B\mapsto 001, C\mapsto 010, D\mapsto 011, E\mapsto 100, F\mapsto 101, G\mapsto 110, H\mapsto 111 \rangle$.
        
        \begin{itemize}
            \item Кодирование: $ABBA\mapsto 000001000001$. 
            \item Декодирование: $000 001 000 001\mapsto ABBA$.
        \end{itemize}
    \end{example} 
\end{frame}

\begin{frame}
    \frametitle{Схемы кодирования}

    \begin{definition}
        Схема кодирования $\delta$ является \alert{разделимой}, если любое слово $\varsigma$, составленное из элементарных кодов $\omega_i$ единственным образом разлагается на элементарные коды.
    \end{definition} 
    Разделимая схема допускает декодирование. Важным частным случаем \alert{разделимых} схем являются \alert{префиксные} схемы.
    \begin{definition}
        Схема называется \alert{префиксной}, если ни один элементарный код $\omega_i$ из множества $\Omega$ не является префиксом\footnote{Префиксом слова $\omega$ называется слово $\omega_1$, если $\omega=\omega_1\omega_2$} другого кода из того же множества.
    \end{definition} 
\end{frame}

\begin{frame}
    \frametitle{<<Равномерное>> кодирование}
    
    Наиболее простым вариантом кодирования является \alert{равномерное} кодирование, когда все элементарные коды равной длины.
    
    Для кодирования $N$ событий требуется использовать цепочки длины
    \[
        I(n)=\lceil\log_m(n)\rceil,
    \]
    где $m$ --- количество кодовых символов; $\lceil X \rceil$ --- наименьшее целое, большее или равное $X$.
    
    Эта же оценка на основе постулатов Шеннона:
    \[
        I(n)=\left\lceil -\log_m\left(\frac{1}{n}\right) \right\rceil=\lceil \log_m(n) \rceil.
    \]
\end{frame}

\begin{frame}
    \frametitle{Равномерное кодирование}

    \begin{example}
        В соревновании учавствуют $33$ спортсмена. Для регистрации пересечения финишной черты каждому спортсмену выдается радио-брелок. В момент пересечения финишной черты спортсменом, брелок передает двоичный код для идентификации спортсмена. Все брелки передают код одинаковой длины. Какое минимально необходиоме количество бит в общем случае должен передать брелок?
    \end{example}
    \begin{proof}<2>[Решение] 
        $\lceil \log_2(33)\rceil = 6$.
    \end{proof}
\end{frame}

В ряде случаев в процессе кодирования имеются знания о вероятности возникновения тех или иных событий. Если это так, то можно использовать методики оптимального кодирования для экономии памяти (или снижения нагрузки на каналы передачи данных).

Другим важным понятием является сигнал. Сигнал – это изменение (во времени или пространстве) физической величины, несущее информацию, т.е. способ, позволяющий фиксировать символ в материально-энергетическом носителе. Различают аналоговые (непрерывные) и цифровые (дискретные) сигналы. Соответственно различают аналоговую и цифровую технику. В чем принципиальная разница между аналоговым и цифровым сигналом?

\begin{frame}
    \frametitle{Сигнал}
    
    \begin{definition}
        \alert{Сигнал} --- это изменение (во времени или пространстве) физической величины, несущее информацию, т.е. способ, позволяющий фиксировать \alert{символ} в материально-энергетическом носителе
    \end{definition}
    
    Выделяют два вида сигналов:
    \begin{enumerate}
        \item аналоговые;
        \item цифровые.
    \end{enumerate}
\end{frame}


Представьте, что нужно передать, например, звук. Звук это сигнал – изменение физической величины «давления воздуха» в определенной точке (пусть в ухе) во времени. Это непрерывное изменение: для любого момента на оси времени можно определить значение давления. Увы, этот сигнал для передачи на большие расстояния не годится, требуется информацию передаваемую звуком представить другим сигналом, более удобным для передачи на большие расстояния.

Итак, путь первый, аналоговый. Заместить исходный сигнал более подходящим аналогом. Например, раскрутить со скоростью 33 оборота в секунду виниловый диск и алмазной иглой, которая благодаря мембране, повторяющей колебания воздуха, прочертит на виниле спиральную дорожку – пространственный аналог звука. Дальше такой сигнал можно передавать самолетом или поездом. Конечно, способов создать аналог масса, например, можно модулировать радиоволну звуковыми колебаниями. См. Рис. \ref{pict:analog}  Конечно, корень аналог не подразумевает, что аналоговый сигнал обязательно является непрерывным, но исторически так сложилось.

\begin{frame}
    \frametitle{Аналоговый сигнал}
    
    \begin{center}
        \includegraphics[width=\textwidth]{fig/analog} 
    \end{center}
\end{frame}


Цифровой сигнал по определению прерывен, дискретен. Он содержит значения изменяющейся физической величины только в определенных точках времени или пространства. Значение физической величины определяется числом. Продолжая пример со звуком можно выполнить замеры давления воздуха в определенные моменты времени. Чем меньше промежуток между этими моментами, тем лучше. Получив набор чисел, их можно закодировать с помощью конечного набора символов-цифр. На Рис. \ref{pict:digital} числа закодированы в двоичной системе счисления (используются только два символа-цифры <<1>> и <<0>>).


\begin{frame}
    \frametitle{Цифровой сигнал}
    \begin{center}
        \includegraphics[height=.75\textheight]{fig/digital} 
    \end{center}
\end{frame}


Естественно, цифровая форма представления по определению менее качественна, но обладает одним замечательным достоинством: число можно закодировать на носителе с $m$ устойчивыми состояниями (2-мя в современной технике), которые практически невозможно спутать. Поэтому цифровой сигнал намного более устойчив к помехам. Тысячная копия с копии лазерного диска (цифровой сигнал) будет идентичной оригиналу, а уже на сотой производной копии грампластинки вы не услышите ничего, кроме шума. Еще одно преимущество цифрового кодирования информации --- возможность использования всей мощи математики, для представления и обработки информации. Вычислительный узел, изначально предназначенный лишь для проведения сложных математических расчетов, в наше время играет ведущую роль в любых манипуляциях с цифровым представлением информации (например, очистка от помех, сжатие и т.д.)

Цифровая техника в настоящее время сильно потеснила аналоговую, но ценители качества, например меломаны, тяготеют к изделиям, в названии которых еще остается вхождения заветного корня <<аналог>>.

И хоть в процессе перехода от одного сигнала к другому иногда требуется кодирование (сигнал рассматривается как отражаемый объект) определения цифровое и аналоговое к кодированию отношения не имеют.


\section{Оптимальное кодирование}


\subsection{Энтропия}

\begin{frame}
    \frametitle{Информативность источника \alert{событий}}
    
    Источнику событий после кодирования соответствует источник информации, выдающий коды событий. Оценку информативности \alert{источника событий} дает величина, называемая \alert{энтропией} источника событий:
    \begin{equation}
        \label{eq:code:entrophyS}
        E=-\sum_{i=1}^N {p_i\cdot\log_m p_i},
    \end{equation}
    где $p_i$ --- вероятность $i$-го события $s_i\in S$ на выходе источника событий, $m$ --- количество кодовых символов, $N$ --- количество событий $N=|S|$.
\end{frame}

\begin{frame}
    \frametitle{Информативность источника \alert{информации}}
    
    Так как вероятности появления кодов событий останутся прежними, то энтропия \alert{источника информации} $E'$ будет равна
    \begin{equation}
        \label{eq:code:entrophyI}
        E'=\sum_{i=1}^n {p_i\cdot I_i},
    \end{equation}
    где $I_i$ --- длина кода $\omega_i$ для $i$-го события.
    
    \begin{center}
        \includegraphics[width=.5\textwidth]{fig/reflection}
    \end{center}
\end{frame}

\begin{frame}
    \frametitle{Задача оптимального кодирования}
    
    \begin{block}{Аксиома}
        Энтропия источника информации всегда больше энтропии отражаемого источника событий. 
    \end{block}
    Задача оптимального кодирования: максимально приблизить энтропию источника информации к энтропии источника событий.
\end{frame}

\begin{frame}
    \frametitle{Оценка оптимальности кодирования}
    
    Пусть имеется источник событий $s_i$, о вероятности появления которых на его выходе известно следующее:
    
    \begin{center}
        \begin{tabular}{|l|c|c|c|c|}
            \hline
            Событие $s_i$                   &А      &Б      &В      &Г   \\ \hline
            Вероятность $p_i$ события $s_i$ &0.5    &0.3    &0.1    &0.1 \\ \hline
        \end{tabular}
    \end{center}
    
    Энтропия источника событий, формула \eqref{eq:code:entrophyS}, составляет:
    \[
        \begin{split}
            E=-(0.5\cdot\log_2 0.5+0.3\cdot\log_2 0.3+0.1\cdot log_2 0.1+0.1\cdot log_2 0.1)\approx \\
            \approx(0.5+0.521089678+0.332192809+0.332192809)\approx \\
            \approx 1.685475297\text{ бит}.
        \end{split}
    \]
\end{frame}

\begin{frame}
    \frametitle{Оценка оптимальности кодирования}
    
    Для равномерного кодирования битами может быть получен такой вариант:

    \begin{center}
        \begin{tabular}{|l|c|c|c|c|}
            \hline
            Событие $s_i$                   &А      &Б      &В      &Г   \\ \hline
            Вероятность $p_i$ события $s_i$ &0.5    &0.3    &0.1    &0.1 \\ \hline
            Код события $\omega_i$          &00     &01     &10     &11  \\ \hline
        \end{tabular}
    \end{center}
    
    Энтропия данного источника информации составит, по формуле \eqref{eq:code:entrophyI}:
    \[
        E'=(0.5\cdot 2+0.3\cdot 2+0.1\cdot 2+0.1\cdot 2)=2 \text{ бита}.
    \]
\end{frame}
    
Видно, что энтропия источника информации значительно больше. Можно ли её уменьшить, приблизить к энтропии источника событий? Очевидно, что если кодировать символы с большей вероятностью появления кодом с меньшей длиной, то результаты должны получиться лучше. Попробуем следующую схему:
    
\begin{frame}
    \frametitle{Оценка оптимальности кодирования}
    
    \begin{center}
        \begin{tabular}{|l|c|c|c|c|}
            \hline
            Событие $s_i$                   &А      &Б      &В      &Г   \\ \hline
            Вероятность $p_i$ события $s_i$ &0.5    &0.3    &0.1    &0.1 \\ \hline
            Код события $\omega_i$          &0      &10     &110    &111 \\ \hline
        \end{tabular}
    \end{center}
    
    Так же как и предыдущая, эта схема префиксная и разделимая, но неравномерная. Энтропия источника информации теперь составляет
    \[
        E'=(0.5\cdot 1+0.3\cdot 2+0.1\cdot 3+0.1\cdot 3)=1.7\text{ бита}.
    \]
\end{frame}

\begin{frame}
    \frametitle{Оценка оптимальности кодирования}
    
    Представленные источники \alert{эквивалентны}. Если запустить источник информации на выдачу, например, $100$ кодов событий, то первый вариант кодирования выдаст цепочку длины $\approx 200$, а второй --- $\approx 170$ бит.
\end{frame}

Далее рассматриваются два алгоритма оптимального кодирования источника событий: алгоритм Хаффмана и алгоритм Фано. 


\subsection{Алгоритм Хаффмана для $m=2$}

\begin{frame}
    \frametitle{Алгоритм Хаффмана для $m=2$}
    
    \begin{enumerate}
        \item\label{enum:code:haffSort} 
        События сортируются по убыванию вероятности.
        
        \item\label{enum:code:haffGlue} 
        Два события с минимальными вероятностями объединяются в одно составное событие c суммарной вероятностью исходных. При этом одно из исходных событий помечается кодовым символом $0$, а второе --- символом $1$. Исходные события исключаются из множества событий, вместо них остается одно составное.
        
        \item Шаги \ref{enum:code:haffSort} и \ref{enum:code:haffGlue} последовательно повторяются до тех пор, пока все события не склеятся в единственное составное событие, вероятность которого равна $1$. После этого кодовое слово $\omega_i$ для исходного события $s_i$ есть цепочка из кодовых символов, которыми помечены все составные события от  корня до $s_i$.
    \end{enumerate}
\end{frame}

\begin{frame}
    \frametitle{Оптимальное кодирование по Хаффману}
    
    \begin{center}
        \begin{tabular}{|l|c|c|c|c|}
            \hline
            Событие $s_i$                   &А      &Б      &В      &Г   \\ \hline
            Вероятность $p_i$ события $s_i$ &0.5    &0.3    &0.1    &0.1 \\ \hline
            Код события $\omega_i$          &\uncover<6>{0}      
                                                    &\uncover<6>{10}     
                                                            &\uncover<6>{110}    
                                                                    &\uncover<6>{111} 
                                                                         \\ \hline
        \end{tabular}
    \end{center}
    \begin{center}
        \only<1>{ \includegraphics[width=0.6\textwidth]{fig/huffmanX1} }
        \only<2>{ \includegraphics[width=0.6\textwidth]{fig/huffmanX2} }
        \only<3>{ \includegraphics[width=0.6\textwidth]{fig/huffmanX3} }
        \only<4>{ \includegraphics[width=0.6\textwidth]{fig/huffmanX4} }
        \only<5->{ \includegraphics[width=0.6\textwidth]{fig/huffmanXE} }
    \end{center}
\end{frame}


\begin{frame}
    \frametitle{Оптимальное кодирование по Хаффману (задача)}
    
    \begin{center}
        \begin{tabular}{|l|c|c|c|c|c|}
            \hline
            Событие $s_i$                   &А      &Б      &В      &Г      &Д      \\ \hline
            Вероятность $p_i$ события $s_i$ &0.5    &0.125  &0.125  &0.125  &0.125  \\ \hline
            Код события $\omega_i$          &\uncover<2->{0} 
                                                    &\uncover<2->{100}    
                                                            &\uncover<2->{101}    
                                                                    &\uncover<2->{110}    
                                                                            &\uncover<2->{111}    
                                                                                    \\ \hline
        \end{tabular}
    \end{center}
    \uncover<2->{
        \begin{center}
            \includegraphics[width=0.6\textwidth]{fig/haffman2Ex}
        \end{center}
    }    
    \uncover<2->{$10001101110111\to\uncover<3->{\text{БАГДАД}}$}
\end{frame}


\subsection{Алгоритм Фано для $m=2$}

\begin{frame}
    \frametitle{Алгоритм Фано для $m=2$}
    
    \begin{enumerate}
        \item Исходный массив событий, сортируется в порядке убывания вероятностей. 
    
        \item \label{en:code:fanoSplit} Массив разбивается на две части, так, чтобы разница сумм вероятностей событий каждой части была минимальна. Первый кодовый символ элементарного кода $\omega_i$ находится так: для всех событий левой части разбитого массива кодовый символ будет $0$, а для всех событий правой части --- $1$. 
    
        \item Второй и последующие кодовые символы определяется так: каждая часть разбитого исходного массива, в которой более одного события, становится исходным массивом, и её разбиение выполняется так же, как исходного массива (шаг \ref{en:code:fanoSplit}).
    \end{enumerate}
\end{frame}

\begin{frame} 
    \frametitle{Оптимальное кодирование по алгоритму Фано}

    \begin{center}
        \begin{tabular}{|l|c|c|c|c|}
            \hline
            Событие $s_i$                   &А      &Б      &В      &Г   \\ \hline
            Вероятность $p_i$ события $s_i$ &0.5    &0.3    &0.1    &0.1 \\ \hline
            Код события $\omega_i$          &\uncover<5>{0}      
                                                    &\uncover<5>{10}     
                                                            &\uncover<5>{110}    
                                                                    &\uncover<5>{111} 
                                                                         \\ \hline
        \end{tabular}
    \end{center}
    
    \only<1>{
        \begin{center}
            \begin{tabular}{|l|l||}
                \hline\hline
                $s_i$   &$p_i$  \\ \hline\hline
                А       &0.5    \\
                Б       &0.3    \\
                В       &0.1    \\
                Г       &0.1    \\ \hline
            \end{tabular}
        \end{center}
    }
    \only<2>{
        \begin{center}
            \begin{tabular}{|l|l||c}
                \hline\hline
                $s_i$   &$p_i$  &\multicolumn{1}{c}{$\omega_i$} \\ \hline\hline
                А       &0.5    &0                              \\ \cline{3-3}
                Б       &0.3    &1                              \\ 
                В       &0.1    &1                              \\ 
                Г       &0.1    &1                              \\ \hline
            \end{tabular}
        \end{center}
    }
    \only<3>{
        \begin{center}
            \begin{tabular}{|l|l||c|c}
                \hline\hline
                $s_i$   &$p_i$  &\multicolumn{2}{c}{$\omega_i$} \\ \hline\hline
                А       &0.5    &0&\multicolumn{1}{c}{}         \\ \cline{3-4}
                Б       &0.3    &1&0                            \\ \cline{4-4}
                В       &0.1    &1&1                            \\ 
                Г       &0.1    &1&1                            \\ \hline
            \end{tabular}
        \end{center}
    }
    \only<4->{
        \begin{center}
            \begin{tabular}{|l|l||c|c|c|}
                \hline\hline
                $s_i$   &$p_i$  &\multicolumn{3}{c|}{$\omega_i$} \\ \hline\hline
                А       &0.5    &0&\multicolumn{2}{c|}{}         \\ \cline{3-4}
                Б       &0.3    &1&0&\multicolumn{1}{c|}{}       \\ \cline{4-5}
                В       &0.1    &1&1&0                           \\ \cline{5-5}
                Г       &0.1    &1&1&1                           \\ \hline
            \end{tabular}
        \end{center}
    }    
\end{frame}

\begin{frame} 
    \frametitle{Оптимальное кодирование по алгоритму Фано (задача)}

    \begin{center}
        \begin{tabular}{|l|c|c|c|c|c|c|c|c|}
            \hline
            $s_i$       &А      &Б      &В      &Г      &Д      &Е      &Ж      \\ \hline
            $p_i$       &0.135  &0.24   &0.25   &0.125  &0.0635 &0.124  &0.0625 \\ \hline
            $\omega_i$  &\uncover<3->{100}  
                                &\uncover<3->{01}   
                                        &\uncover<3->{00}   
                                                &\uncover<3->{101}  
                                                        &\uncover<3->{1110} 
                                                                &\uncover<3->{110}  
                                                                        &\uncover<3->{1111} 
                                                                                \\ \hline
        \end{tabular}
    \end{center}
    
    \uncover<2->{
        \begin{center}
            \begin{tabular}{|l|l||c|c|c|c|}
                \hline\hline
                $s_i$   &$p_i$  &\multicolumn{4}{c|}{$\omega_i$}\\ \hline\hline
                В       &0.25   &0&0&\multicolumn{2}{c|}{}      \\ \cline{4-4}
                Б       &0.24   &0&1&\multicolumn{2}{c|}{}      \\ \cline{3-5}
                А       &0.135  &1&0&0&                         \\ \cline{5-5}
                Г       &0.125  &1&0&1&                         \\ \cline{4-5}
                Е       &0.124  &1&1&0&                         \\ \cline{5-6}
                Д       &0.0635 &1&1&1&0                        \\ \cline{6-6}
                Ж       &0.0625 &1&1&1&1                        \\ \hline            
            \end{tabular}
        \end{center}
    }
\end{frame}


\section{Кодирование с целью сжатия информации}

Следует выделить различия между между \emph{оптимальным кодированием} и \emph{кодированием с целью сжатия}? Оптимальное кодирование ставит себе в задачу сопоставить источнику событий минимальное количество адекватной ему информации. 


\subsection{Сжатие}

\begin{frame}
    \frametitle{Сжатие}
    
    Кодирование с целью сжатия (или просто \alert{сжатие}) ставит себе в задачу уменьшить количество информации, не теряя (или оставаясь в допустимых рамках) при этом свойство адекватности отражаемому объекту. В случае сжатия, события $s_i$ уже представляют собой слова в алфавите $T$ (коды). При сжатии информация \alert{перекодируется} в том же алфавите $T$.
    
    Классы алгоритмов сжатия:
    \begin{itemize}
        \item сжатие с потерями (адекватности);
        \item сжатие без потерь.
    \end{itemize}
\end{frame}

Выделяют два больших класса алгоритмов сжатия информации: сжатие с потерями и без потерь. Теряется, данном случае, конечно, адекватность отражения. При сжатии без потерь из сжатой информации можно восстановить исходную информацию в точности такую же, как до сжатия. При сжатии с потерями восстановленная информация будет отличаться от исходной. Ярким примером сжатия с потерями является сжатие изображений: используя определенные особенности восприятия цвета человеком, такие алгоритмы отбрасывают <<лишнюю>> информацию. Потеря адекватности отражаемому объекту в этом случае значительная, но для человека-потребителя эти потери адекватности незаметны.

Часто алгоритмы сжатия весьма специфичны и учитывают особенности отражаемого источника событий. Одним из важнейших таких источников в жизни человека является речь. Письмо --- способ кодирования речи с помощью символов конечного алфавита --- азбуки. На основе, например, русского алфавита можно построить бесконечное количество слов, но в реальной жизни словарный запас редко превышает сотню тысяч слов. На практике для универсального представления текста байтами кодируются буквы, цифры, знаки препинания, пробелы и т.д., но если мы знаем, что кодируется именно осмысленный текст (содержащий осмысленные слова), то можно сильно сэкономить, кодируя в качестве сообщений $s_i$ не буквы, а слова. Такие методы сжатия называются \emph{словарными}. 

Впрочем, словарные методы могут использоваться не только для кодирования текста, но для произвольных информационных цепочек. Причем словарь может строиться динамически и совершенно не учитывать смысловой нагрузки слов в словаре.

Далее будет рассмотрен алгоритм Лемпела-Зива, относящийся к группе словарных. В основе алгоритма Лемпела-Зива лежит идея \emph{адаптивного} сжатия: за один проход по цепочке одновременно строится и словарь и код, причем словарь не хранится, так как при декодировании он динамически восстанавливается.


\subsection{Алгоритм Лемпела-Зива}

\begin{frame}
    \frametitle{Алгоритм Лемпела-Зива (LZ)}
    \framesubtitle{Кодирование}
    
	\begin{enumerate}
		\item В словарь нулевым элементом помещается пустая цепочка $\varepsilon$. Пустое слово $\varepsilon$ не содержит букв, и для любого слова $\omega$ справедливо $\omega=\varepsilon\omega=\omega\varepsilon$.
        
        \item\label{enum:code:lzWord} От исходной цепочки $t$ отделяется слово $\omega a$, где $\omega$ --- максимально длинное слово из словаря, $a$ --- расширяющая буква. Т.е. $t=\omega at'$.
        
        \item\label{enum:code:lzDict} В конец словаря добавляется новое слово $\omega a$. К коду $c$ добавляется пара $\langle i_{\omega},a\rangle$, где $i_{\omega}$ --- индекс слова $\omega$ в словаре. От исходного текста отделяется слово $\omega a$: $t=t'$.
        
        \item Пункты \ref{enum:code:lzWord}-\ref{enum:code:lzDict} последовательно повторяются до тех пор, пока в тексте $t$ остается хоть одна буква.
	\end{enumerate}
    В результате получается код $c=\langle i_1,a_1\rangle\cdots\langle i_n,a_n\rangle$.
\end{frame}

Нужно отметить, что данный алгоритм хорошо сжимает тексты большого объема, в которых так или иначе будут присутствовать одинаковые и достаточно длинные слова. В следующем примере такие вхождения были созданы искусственно.

\begin{frame}
    \frametitle{Алгоритм Лемпела-Зива}
    \framesubtitle{Пример сжатия текста: <<АБАКАНКАНКАНКИАНКИН>>}
    
    \begin{center}
        \resizebox{.8\textwidth}{!}{
            \begin{tabular}[c]{|l|l|l|l|}
            \hline\hline
            $i$ & $t$                                            & $\omega a$                   & $c=\langle i_\omega,a\rangle$ \\ 
            \hline\hline
              &                                                  & $0\rightarrow\varepsilon $   & \\ \hline
            1 &	$\varepsilon\text{\textbf{А}БАКАНКАНКАНКИАНКИН}$ & $1\rightarrow\text{A}    $   & $\langle\text{0,А}\rangle$ \\ \hline
            2 &	$\varepsilon\text{\textbf{Б}АКАНКАНКАНКИАНКИН} $ & $2\rightarrow\text{Б}    $   & $\langle\text{0,Б}\rangle$ \\ \hline
            3 &	$           \text{\textbf{АК}АНКАНКАНКИАНКИН}  $ & $3\rightarrow\text{АК}   $   & $\langle\text{1,К}\rangle$ \\ \hline
            4 &	$           \text{\textbf{АН}КАНКАНКИАНКИН}    $ & $4\rightarrow\text{АН}   $   & $\langle\text{1,Н}\rangle$ \\ \hline
            5 &	$\varepsilon\text{\textbf{К}АНКАНКИАНКИН}      $ & $5\rightarrow\text{К}    $   & $\langle\text{0,К}\rangle$ \\ \hline
            6 &	$           \text{\textbf{АНК}АНКИАНКИН}       $ & $6\rightarrow\text{АНК}  $   & $\langle\text{4,К}\rangle$ \\ \hline
            7 &	$           \text{\textbf{АНКИ}АНКИН}          $ & $7\rightarrow\text{АНКИ} $   & $\langle\text{6,И}\rangle$ \\ \hline
            8 &	$           \text{\textbf{АНКИН}}              $ & $8\rightarrow\text{АНКИН}$   & $\langle\text{7,Н}\rangle$ \\ \hline
            \end{tabular}
        }        
    \end{center}
    \[
        \overbrace{\varepsilon}^0
        \overbrace{\text{А}}^1_{\langle 0,\text{А}\rangle}
        \overbrace{\text{Б}}^2_{\langle 0,\text{Б}\rangle}
        \overbrace{\text{АК}}^3_{\langle 1,\text{К}\rangle}
        \overbrace{\text{АН}}^4_{\langle 1,\text{Н}\rangle}
        \overbrace{\text{К}}^5_{\langle 0,\text{К}\rangle}
        \overbrace{\text{АНК}}^6_{\langle 4,\text{К}\rangle}
        \overbrace{\text{АНКИ}}^7_{\langle 6,\text{И}\rangle}
        \overbrace{\text{АНКИН}}^8_{\langle 7,\text{Н}\rangle}
    \]
    
    Дать оценку длин кода и текста
\end{frame}

Полученный код (сжатый текст) занимает 8*2=16 байт информации (один байт на индекс, второй на букву), а код исходного текста занимает 19 байт (один байт на букву).

\begin{frame}
    \frametitle{Алгоритм Лемпела-Зива}
    \framesubtitle{Задание}
    
    Сжать текст:
    \[
        \text{<<тартарарамитамтамывтартарарах>>}
    \]
    \uncover<2>{
        \[
            \overbrace{\varepsilon}^0
            \overbrace{\text{т}}^1_{\langle 0,\text{т}\rangle}
            \overbrace{\text{а}}^2_{\langle 0,\text{а}\rangle}
            \overbrace{\text{р}}^3_{\langle 0,\text{р}\rangle}
            \overbrace{\text{та}}^4_{\langle 1,\text{а}\rangle}
            \overbrace{\text{ра}}^5_{\langle 3,\text{а}\rangle}
            \overbrace{\text{рам}}^6_{\langle 5,\text{м}\rangle}
            \overbrace{\text{и}}^7_{\langle 0,\text{и}\rangle}
            \overbrace{\text{там}}^8_{\langle 4,\text{м}\rangle}
            \overbrace{\text{тамы}}^9_{\langle 8,\text{ы}\rangle}
            \overbrace{\text{в}}^{10}_{\langle 0,\text{в}\rangle}
            \overbrace{\text{тар}}^{11}_{\langle 4,\text{р}\rangle}
            \overbrace{\text{тара}}^{12}_{\langle 11,\text{а}\rangle}
            \overbrace{\text{рах}}^{13}_{\langle 5,\text{х}\rangle}
        \]
        Код:
        \[
            \begin{split}
                \langle 0,\text{т}\rangle,
                \langle 0,\text{а}\rangle,
                \langle 0,\text{р}\rangle,
                \langle 1,\text{а}\rangle,
                \langle 3,\text{а}\rangle,
                \langle 5,\text{м}\rangle,
                \langle 0,\text{и}\rangle,\\
                \langle 4,\text{м}\rangle,
                \langle 8,\text{ы}\rangle,
                \langle 0,\text{в}\rangle,
                \langle 4,\text{р}\rangle,
                \langle 11,\text{а}\rangle,
                \langle 5,\text{х}\rangle
            \end{split}
        \]
    }
\end{frame}

\begin{frame}
    \frametitle{Алгоритм Лемпела-Зива (LZ)}
    \framesubtitle{Декодирование}

    \begin{enumerate}
        \item В словарь нулевым элементом помещается пустая цепочка $\varepsilon$. Текст $t$ не содержит букв: $t=\varepsilon$.
        
        \item\label{enum:code:lzDecode} От исходного кода $c$ отделяется пара $\langle i,a\rangle$, в словарь добавляется слово $\omega_i a$, где $\omega_i$ --- $i$-е слово из словаря. Восстанавливается текст $t=t\omega_i a$.
        
        \item Пункт \ref{enum:code:lzDecode} последовательно повторяется до тех пор, пока в коде $c$ остается хоть одна пара.
    \end{enumerate}
\end{frame}

\begin{frame}
    \frametitle{Алгоритм Лемпела-Зива}
    \framesubtitle{Пример декодирования <<0А,0Б,1К,1Н,0К,4К,6И,7Н>>}
    
    \begin{center}
        \resizebox{.8\textwidth}{!}{
            \begin{tabular}[c]{|l|l|l|l|}
                \hline\hline
                $i$ & $c=\langle i_\omega,a\rangle$ & $\omega a$              & $t$ \\ 
                \hline\hline
                  &                            & $0\rightarrow\varepsilon $   &                                                 \\ \hline
                1 & $\langle\text{0,А}\rangle$ & $1\rightarrow\text{A}    $   &	$\text{}      \varepsilon\text{\textbf{А}}    $ \\ \hline
                2 & $\langle\text{0,Б}\rangle$ & $2\rightarrow\text{Б}    $   &	$\text{А}     \varepsilon\text{\textbf{Б}}    $ \\ \hline
                3 & $\langle\text{1,К}\rangle$ & $3\rightarrow\text{АК}   $   &	$\text{АБ}               \text{\textbf{АК}}   $ \\ \hline
                4 & $\langle\text{1,Н}\rangle$ & $4\rightarrow\text{АН}   $   &	$\text{АБАК}             \text{\textbf{АН}}   $ \\ \hline
                5 & $\langle\text{0,К}\rangle$ & $5\rightarrow\text{К}    $   &	$\text{АБАКАН}\varepsilon\text{\textbf{К}}    $ \\ \hline
                6 & $\langle\text{4,К}\rangle$ & $6\rightarrow\text{АНК}  $   &	$\text{АБАКАНК}          \text{\textbf{АНК}}  $ \\ \hline
                7 & $\langle\text{6,И}\rangle$ & $7\rightarrow\text{АНКИ} $   &	$\text{АБАКАНКАНК}       \text{\textbf{АНКИ}} $ \\ \hline
                8 & $\langle\text{7,Н}\rangle$ & $8\rightarrow\text{АНКИН}$   &	$\text{АБАКАНКАНКАНКИ}   \text{\textbf{АНКИН}}$ \\ \hline
            \end{tabular}
        }
    \end{center}
    \[
        \overbrace{\varepsilon}^0
        \overbrace{\langle 0,\text{А}\rangle}^1_{\text{А}}
        \overbrace{\langle 0,\text{Б}\rangle}^2_{\text{Б}}
        \overbrace{\langle 1,\text{К}\rangle}^3_{\text{АК}}
        \overbrace{\langle 1,\text{Н}\rangle}^4_{\text{АН}}
        \overbrace{\langle 0,\text{К}\rangle}^5_{\text{К}}
        \overbrace{\langle 4,\text{К}\rangle}^6_{\text{АНК}}
        \overbrace{\langle 6,\text{И}\rangle}^7_{\text{АНКИ}}
        \overbrace{\langle 7,\text{Н}\rangle}^8_{\text{АНКИН}}
    \]    
\end{frame}

\begin{frame}
    \frametitle{Алгоритм Лемпела-Зива}
    \framesubtitle{Задание}
    
    Восстановить текст из кода:
    \[
        \langle 0,\text{в}\rangle,
        \langle 0,\text{о}\rangle,
        \langle 0,\text{т}\rangle,
        \langle 1,\text{а}\rangle,
        \langle 0,\text{м}\rangle,
        \langle 4,\text{р}\rangle,
        \langle 6,\text{ы}\rangle,
        \langle 2,\text{т}\rangle,
        \langle 6,\text{ы}\rangle    
    \]
    \uncover<2>{
        \[
            \overbrace{\varepsilon}^0
            \overbrace{\langle 0,\text{в}\rangle}^1_{\text{в}}
            \overbrace{\langle 0,\text{о}\rangle}^2_{\text{о}}
            \overbrace{\langle 0,\text{т}\rangle}^3_{\text{т}}
            \overbrace{\langle 1,\text{а}\rangle}^4_{\text{ва}}
            \overbrace{\langle 0,\text{м}\rangle}^5_{\text{м}}
            \overbrace{\langle 4,\text{р}\rangle}^6_{\text{вар}}
            \overbrace{\langle 6,\text{ы}\rangle}^7_{\text{вары}}
            \overbrace{\langle 2,\text{т}\rangle}^8_{\text{от}}
            \overbrace{\langle 6,\text{ы}\rangle}^9_{\text{вары}}
        \]
        
        Текст:
        \[
            \text{<<вотвамварварыотвары>>}
        \]
    }
\end{frame}


\section{Кодирование с целью защиты свойств информации}


\begin{frame}
    \frametitle{Свойства информации с точки зрения её защиты}
    
    \begin{itemize}
        \item целостность: \uncover<2->{контрольные суммы; корректирующие коды;}
        \item конфиденциальность: \uncover<2->{шифрование; скрытая передача;}
        \item принадлежность: \uncover<2->{цифровая подпись;}
        \item доступность: \uncover<2->{надежность информационных систем.}
    \end{itemize}
\end{frame}

Информация имеет несколько свойств, важных с точки зрения их защиты: \emph{целостность}, \emph{конфиденциальность}, \emph{принадлежность} и \emph{доступность}. Далее рассматриваются лишь методы кодирования с целью защиты целостности информации. \emph{Целостность} --- это неизменность информации относительно некоторого фиксированного значения. Защита этого свойства дает пользователю уверенность в том, что информация, полученная им по каналу связи, доставлена в том виде, в котором была отправлена. \emph{Конфиденциальность} --- недоступность третьим лицам (я,мы---1е, ты,вы--2е, он,она,оно,они---3е.) \emph{Принадлежность} --- свойство информации, позволяющее однозначно установить факты передачи или приема конкретным лицом. \emph{Доступность} --- свойство, позволяющее первым двум лицам (я, мы, ты, вы)


\subsection{Защита целостности}


\begin{frame}
    \frametitle{Классификация ошибок}
    
    Ошибки, возникающие в цифровом (двоичном) канале могут быть следующими:
    \begin{itemize}
        \item замещения кодового символа;
        \item вставка кодового символа;
        \item выпадение кодового символа.
    \end{itemize}
    
    Далее рассматриваются только ошибки замещения. Существуют две стратегии защиты от ошибок замещения:
    \begin{itemize}
        \item с обнаружением и запросом на повторную передачу (ARQ --- Automatic Repeat Request);
        \item с обнаружением и непосредственным исправлением на стороне получателя (FEC --- Forward Error Correction).
    \end{itemize}
\end{frame}

\begin{frame}
    \frametitle{ARQ}
    
    Примером стратегии ARQ может считаться контроль по четности (нечетности). 
    \[
        \begin{split}
            p_{\text{чётн}}=d_{n-1}\oplus\ldots\oplus d_1\oplus d_0,\\
            p_{\text{нечётн}}=d_{n-1}\oplus\ldots\oplus d_1\oplus d_0\oplus 1.
        \end{split}
    \]
\end{frame}

Если рассчитанный по той же формуле бит не совпадает с переданным --- в канале произошла ошибка, требуется повторная передача. Ошибки четной кратности данным кодом не распознаются.

Стратегия FEC позволяет не только выявлять ошибки, но и исправлять их на месте. Рассмотрим несколько примеров кодирования для исправления одиночной ошибки.

\begin{frame}
    \frametitle{FEC}
    
    Можно кодировать каждый бит исходной последовательности по схеме
    \[\delta=\{0\mapsto 000, 1\mapsto 111\},\]
    а декодировать по схеме
    \[
        \begin{split}
            \delta'=\{
                000\mapsto 0,001\mapsto 0,010\mapsto 0,100\mapsto 0,\\
                111\mapsto 1,110\mapsto 1,101\mapsto 1,011\mapsto 1
            \},
        \end{split}
    \]
    \begin{example}
        Пусть передается слово $101$. Кодируется: $111000111$. Поступает в канал. Возникает одиночная ошибка: $11\fbox{$0$}000111$. Декодируется: $101$. При этом декодер обнаруживает и исправляет одиночную ошибку. \qed
    \end{example}
\end{frame}

\begin{frame}
    \frametitle{FEC}
    \framesubtitle{Пример кодирования <<утроением>>}
    
    \begin{center}
        \includegraphics[width=.8\textwidth]{fig/fecTriplet} 
    \end{center}
\end{frame}

\begin{frame}
    \frametitle{Схема канала передачи данных}
    
    \begin{center}
        \includegraphics[width=.9\textwidth]{fig/channel} 
    \end{center}
\end{frame}

\begin{frame}
    \frametitle{FEC --- Ошибка обнаружена и верно исправлена}
    
    \begin{center}
        \includegraphics[width=.8\textwidth]{fig/fecOk} 
    \end{center}
\end{frame}


\begin{frame}
    \frametitle{FEC --- ошибка обнаружена, но исправлена неверно}
    
    \begin{center}
        \includegraphics[width=.8\textwidth]{fig/fecIncorrect} 
    \end{center}
\end{frame}


\begin{frame}
    \frametitle{FEC --- необнаружимая ошибка}
    
    \begin{center}
        \includegraphics[width=.8\textwidth]{fig/fecFail} 
    \end{center}
\end{frame}

\begin{frame}
    \frametitle{FEC: Код Хемминга}
    
    \begin{itemize}    
        \item Код Хемминга формирует номер ошибочного разряда. 
        \item Признаком отсутствия ошибок является нулевой номер. 
        \item Поэтому вводится <<фиктивный>> нулевой разряд. 
        \item Если исходное слово имеет длину $n$ бит, тогда к нему нужно добавить $m$ дополнительных бит, исходя из неравенства
        
        \begin{equation}
            \label{eq:code:hammingM}
            2^m\geq n + m + 1,
        \end{equation}
        где левая часть неравенства --- количество $m$-разрядных двоичных чисел, правая --- общая длина кода с учетом <<фиктивного>> разряда. Выбирается минимальное $m$ из возможных.
    \end{itemize}    
\end{frame}

\begin{frame}
    \frametitle{Кодирование и декодирование по Хеммингу}
    
    Алгоритм кодирования
    \begin{enumerate}
        \item В двоичном числе длиной $m+n$ бит (без фиктивного разряда) контрольные $m$ бит размещаются в разрядах с номерами, равными степени двойки ($2^i,0\leq i<m$). А $n$ бит исходного слова размещаются в оставшихся разрядах. Контрольный биты при этом инициализируются нулевыми значениями.
        
        \item\label{en:code:hammingCount} Каждый контрольный бит $c_{2^i}$ в разряде $2^i$ пересчитывается как сумма по XOR бит кода, находящихся в разрядах с номерами, двоичное представление которых содержит единицу в $i$ разряде (включая и сам контрольный разряд).
    \end{enumerate}
    
    При декодировании контрольные разряды пересчитываются в соответствии с пунктом \ref{en:code:hammingCount} алгоритма кодирования. В результате в контрольных разрядах будет получено двоичное представление номера разряда ошибочного бита.
\end{frame}

\begin{frame}
    \frametitle{Построить код Хемминга для слова $u=0011$}
    
    $n=4$. Исходя из формулы \eqref{eq:code:hammingM} выбирается $m=3$.
    
    \begin{center}
        \includegraphics[width=0.6\textwidth]{fig/hammingEncode}
    \end{center}
\end{frame}

\begin{frame}
    \frametitle{Код Хемминга}
    \framesubtitle{Обнаружение и исправление одиночных ошибок}
    
    \begin{center}
        \includegraphics[width=0.6\textwidth]{fig/hammingDecode}
    \end{center}
\end{frame}

\begin{frame}
    \frametitle{Код Хемминга}
    \framesubtitle{Задание}
    
    Получена последовательность бит $r$. Перед передачей из исходного 4-х битного слова был получен код Хемминга. Выяснить, были ли ошибки в процессе передачи. Если были, то выполнить коррекцию, предполагая, что ошибка одиночная. Выделить исходное 4-х битное слово.
    \begin{center}
        \only<1>{ \includegraphics[width=0.6\textwidth]{fig/hammingTask} }
        \only<2>{ \includegraphics[width=0.6\textwidth]{fig/hammingAnswer} }
    \end{center}
\end{frame}    


\subsection{Защита конфиденциальности}


\begin{frame}
    \frametitle{Базовая схема передачи информации}
    
    \begin{center}
        \includegraphics[width=.75\textwidth]{fig/basechannel} 
    \end{center}

    Можно выделить следующие виды \alert{каналов связи}:
    \begin{itemize}
        \item \alert{Секретный} гарантирует конфиденциальность, целостность и принадлежность M;
        \item \alert{Аутентичный} гарантирует только целостность и принадлежность M;
        \item \alert{Открытый} не гарантирует ничего в отношении M.
    \end{itemize}
\end{frame}

\begin{frame}
    \frametitle{Схемы шифрования}
    \framesubtitle{Симметричная схема}
    
    \begin{center}
        \includegraphics[width=.95\textwidth]{fig/symmcipher} 
    \end{center}
\end{frame}


\begin{frame}
    \frametitle{Схемы шифрования}
    \framesubtitle{Ассиметричная схема}
    
    \begin{center}
        \includegraphics[width=.95\textwidth]{fig/asymmcipher} 
    \end{center}
\end{frame}


\subsection{Защита принадлежности}


\begin{frame}
    \frametitle{Цифровая подпись}
    
    \begin{center}
        \includegraphics[width=.95\textwidth]{fig/signature} 
    \end{center}
\end{frame}

\appendix


\begin{frame}
    \frametitle{В заключение}
    
    Изложение  математических основ кодирования можно найти, например, в \cite{bib:novic:discrmathprogrammer,bib:yablonsky:discreteintro}. По основам теории информации можно рекомендовать книгу \cite{bib:panin:informationTheory}. Основы кодирования подробно изложены в \cite{bib:verner:codingBase}. Заинтересовавшимся алгоритмами сжатия можно рекомендовать книгу \cite{bib:salmon:compressing}.
    
    Как введение по вопросам безопасности информации можно рекомендовать \cite{bib:tannen:os}. Обзор задач и протоколов информационной безопасности прекрасно описан в \cite{bib:shneir:applCrypto}. Математические основы шифрования для сильно интересующихся можно найти в \cite{bib:shneir:applCrypto, bib:smart:crypto, bib:mao:modernCrypto}.
\end{frame}

\begin{frame}[allowframebreaks]{Библиография}
    \bibliographystyle{gost780u}
    \bibliography{./../../../bibliobase}
\end{frame}

\end{document}