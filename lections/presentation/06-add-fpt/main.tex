%include part: see main.beamer.tex and main.article.tex
%include common packages and settings
\usepackage{etex} %эта магическая херь избавляет от переполнения регистров TeX а!!!

\mode<article>{\usepackage{fullpage}}
\mode<presentation>{
    \usetheme{Madrid} %%Boadilla,Madrid,AnnArbor,CambridgeUS,Malmoe,Singapore,Berlin
    \useoutertheme{shadow}
} 

\usepackage[utf8]{inputenc}
\usepackage[russian]{babel}
\usepackage{indentfirst}
\usepackage{graphicx}

\usepackage{amsmath}
\usepackage{amsfonts}
\usepackage{amsthm}
\usepackage{algorithm}
\usepackage{algorithmic}

\usepackage[all]{xy}

\date{Лекция по дисциплине <<информатика>>\\(\today)}
\author[М.~М.~Шихов]{Михаил Шихов \\ \texttt{\underline{m.m.shihov@gmail.com}}}

%для рисования графов пакетом xy-pic
\entrymodifiers={++[o][F-]}

%для псевдокода алгоритмов (algorithm,algorithmic)
\renewcommand{\algorithmicrequire}{\textbf{Вход:}}
\renewcommand{\algorithmicensure}{\textbf{Выход:}}
\renewcommand{\algorithmiccomment}[1]{// #1}
\floatname{algorithm}{Псевдокод}

%%определённые мной команды логической разметки
\newcommand{\DC}[1]{\text{ДК}(#1)}
\newcommand{\MDC}[1]{\text{MДК}(#1)}
\newcommand{\OC}[1]{\text{ОК}(#1)}
\newcommand{\MOC}[1]{\text{МОК}(#1)}
\newcommand{\PC}[1]{\text{ПК}(#1)}

\newcommand{\Machine}[1]{\texttt{#1}}

\newcommand{\UnsignedAny}[2]{\text{\upshape
    \begin{tabular}{lr}
        \tiny{#1} & \tiny{0}\\ 
        \hline
        \multicolumn{2}{|c|}{\Machine{#2}} \\ 
        \hline
    \end{tabular}
}}

\newcommand{\UnsignedByte}[1]{\UnsignedAny{7}{#1}}

\newcommand{\UnsignedTwoBytes}[1]{\UnsignedAny{15}{#1}}

\newcommand{\SignedAny}[4]{\text{\upshape
    \begin{tabular}{clr}
        \tiny{#1}  &\tiny{#2} & \tiny{0}\\ 
        \hline
        \multicolumn{1}{|c|}{\Machine{#3}} & \multicolumn{2}{|c|}{\Machine{#4}} \\ 
        \hline
    \end{tabular}
}}

\newcommand{\SignedNibble}[2]{\SignedAny{3}{2}{#1}{#2}}

\newcommand{\SignedByte}[2]{\SignedAny{7}{6}{#1}{#2}}

\newcommand{\SignedTwoBytes}[2]{\SignedAny{15}{14}{#1}{#2}}

\newcommand{\FloatMyHex}[4]{\text{\upshape
    \begin{tabular}{clrclr}
        \tiny{15}  &\tiny{14} & \tiny{6} & \tiny{5} & \tiny{4} & \tiny{0}\\ 
        \hline
        \multicolumn{1}{|c|}{\texttt{#1}} 
            & \multicolumn{2}{|c|}{\texttt{#2}} 
                & \multicolumn{1}{|c|}{\texttt{#3}} 
                    & \multicolumn{2}{|c|}{\texttt{#4}} \\ 
        \hline
    \end{tabular}
}}

\newcommand{\FloatMyCharHex}[3]{\text{\upshape
    \begin{tabular}{clrlr}
        \tiny{15}  &\tiny{14} & \tiny{6} & \tiny{5} & \tiny{0}\\ 
        \hline
        \multicolumn{1}{|c|}{\texttt{#1}} 
            & \multicolumn{2}{|c|}{\texttt{#2}} 
                & \multicolumn{2}{|c|}{\texttt{#3}} \\ 
        \hline
    \end{tabular}
}}

\newcommand{\FloatMySimpleHex}[2]{\text{\upshape
    \begin{tabular}{lrlr}
        \tiny{15}  & \tiny{6} & \tiny{5} & \tiny{0} \\ 
        \hline
        \multicolumn{2}{|c|}{\texttt{#1}} 
            & \multicolumn{2}{|c|}{\texttt{#2}} \\ 
        \hline
    \end{tabular}
}}

\newcommand{\FloatMyOrderX}[4]{\text{\upshape
    \begin{tabular}{clrclr}
        \tiny{9}  &\tiny{8} & \tiny{4} & \tiny{3} & \tiny{2} & \tiny{0}\\ 
        \hline
        \multicolumn{1}{|c|}{\texttt{#1}} 
            & \multicolumn{2}{|c|}{\texttt{#2}} 
                & \multicolumn{1}{|c|}{\texttt{#3}} 
                    & \multicolumn{2}{|c|}{\texttt{#4}} \\ 
        \hline
    \end{tabular}
}}

\newcommand{\FloatMyDcOrderX}[3]{\text{\upshape
    \begin{tabular}{lrclr}
        \tiny{9} & \tiny{4} & \tiny{3} & \tiny{2} & \tiny{0}\\ 
        \hline
        \multicolumn{2}{|c|}{\texttt{#1}} 
            & \multicolumn{1}{|c|}{\texttt{#2}} 
                & \multicolumn{2}{|c|}{\texttt{#3}} \\ 
        \hline
    \end{tabular}
}}

\newcommand{\FloatMyDcCharX}[2]{\text{\upshape
    \begin{tabular}{lrlr}
        \tiny{9} & \tiny{4} & \tiny{3} & \tiny{0}\\ 
        \hline
        \multicolumn{2}{|c|}{\texttt{#1}} 
            & \multicolumn{2}{|c|}{\texttt{#2}} \\ 
        \hline
    \end{tabular}
}}

\newcommand{\FloatMyCharX}[3]{\text{\upshape
    \begin{tabular}{clrlr}
        \tiny{9}  &\tiny{8} & \tiny{4} & \tiny{3} & \tiny{0}\\ 
        \hline
        \multicolumn{1}{|c|}{\texttt{#1}} 
            & \multicolumn{2}{|c|}{\texttt{#2}} 
                & \multicolumn{2}{|c|}{\texttt{#3}} \\ 
        \hline
    \end{tabular}
}}

\newcommand{\FloatESShort}[3]{\text{\upshape
    \begin{tabular}{clrlr}
        \tiny{31}  &\tiny{30} & \tiny{24} & \tiny{23} & \tiny{0}\\ 
        \hline
        \multicolumn{1}{|c|}{\texttt{#1}} 
            & \multicolumn{2}{|c|}{\texttt{#2}} 
                & \multicolumn{2}{|c|}{\texttt{#3}} 
                    \\ 
        \hline
    \end{tabular}
}}

\newcommand{\FloatPCShort}[3]{\text{\upshape
    \begin{tabular}{clrlr}
        \tiny{31}  &\tiny{30} & \tiny{23} & \tiny{22} & \tiny{0}\\ 
        \hline
        \multicolumn{1}{|c|}{\texttt{#1}} 
            & \multicolumn{2}{|c|}{\texttt{#2}} 
                & \multicolumn{2}{|c|}{\texttt{#3}} 
                    \\ 
        \hline
    \end{tabular}
}}


%--- СПЕЦИФИЧНЫЕ ДЛЯ УМНОЖЕНИЯ КОМАНДЫ ---------------------------------------------------------------------------------------------


\newcommand{\Number}[1]{
    \texttt{#1}
}

\newcommand{\NumberHi}[2]{
    \underline{\underline{\texttt{#1}}}\texttt{#2}
}

\newcommand{\NumberMid}[3]{
    \texttt{#1}\underline{\underline{\texttt{#2}}}\texttt{#3}
}

\newcommand{\NumberLo}[2]{
    \texttt{#1}\underline{\underline{\texttt{#2}}}
}

\newcommand{\Stack}[2]{
    \begin{tabular}[t]{@{}r@{}}
        {#1}\\ \hline
        {#2}\\ 
    \end{tabular}
}

\newcommand{\Operation}[4]{
    \begin{tabular}[t]{@{}r@{}}
        \texttt{#4}
        \begin{tabular}{@{}r@{}}
            \Number{#1}\\
            \Number{#2}\\ \hline
        \end{tabular} \\ 
        \Number{#3}\\
    \end{tabular}
}

\newcommand{\Addition}[3]{\Operation{#1}{#2}{#3}{+}}

\newcommand{\Subtraction}[3]{\Operation{#1}{#2}{#3}{-}}

\newcommand{\Multiplication}[3]{\Operation{#1}{#2}{#3}{$\times$}}

\newcommand{\Register}[2]{\Number{#1:#2}}

\newcommand{\Mantiss}{m}
\newcommand{\Order}{p}
\newcommand{\Char}{c}

\newcommand{\MantissOf}[1]{\Mantiss_{#1}}
\newcommand{\OrderOf}[1]{\Order_{#1}}
\newcommand{\CharOf}[1]{\Char_{#1}}

\newcommand{\FloatExpression}[2]{\MantissOf{#1}\cdot {#2}^{\OrderOf{#1}}}

\newenvironment{Solve}[1]%
    {\begin{proof}[Решение]#1}
    {\end{proof}}
    
    
%определённые мной команды логической разметки
\newcommand{\DC}[1]{\text{ДК}(#1)}
\newcommand{\MDC}[1]{\text{MДК}(#1)}
\newcommand{\OC}[1]{\text{ОК}(#1)}
\newcommand{\MOC}[1]{\text{МОК}(#1)}
\newcommand{\PC}[1]{\text{ПК}(#1)}

\newcommand{\Machine}[1]{\texttt{#1}}

\newcommand{\UnsignedAny}[2]{\text{\upshape
    \begin{tabular}{lr}
        \tiny{#1} & \tiny{0}\\ 
        \hline
        \multicolumn{2}{|c|}{\Machine{#2}} \\ 
        \hline
    \end{tabular}
}}

\newcommand{\UnsignedByte}[1]{\UnsignedAny{7}{#1}}

\newcommand{\UnsignedTwoBytes}[1]{\UnsignedAny{15}{#1}}

\newcommand{\SignedAny}[4]{\text{\upshape
    \begin{tabular}{clr}
        \tiny{#1}  &\tiny{#2} & \tiny{0}\\ 
        \hline
        \multicolumn{1}{|c|}{\Machine{#3}} & \multicolumn{2}{|c|}{\Machine{#4}} \\ 
        \hline
    \end{tabular}
}}

\newcommand{\SignedNibble}[2]{\SignedAny{3}{2}{#1}{#2}}

\newcommand{\SignedByte}[2]{\SignedAny{7}{6}{#1}{#2}}

\newcommand{\SignedTwoBytes}[2]{\SignedAny{15}{14}{#1}{#2}}

\newcommand{\FloatMyHex}[4]{\text{\upshape
    \begin{tabular}{clrclr}
        \tiny{15}  &\tiny{14} & \tiny{6} & \tiny{5} & \tiny{4} & \tiny{0}\\ 
        \hline
        \multicolumn{1}{|c|}{\texttt{#1}} 
            & \multicolumn{2}{|c|}{\texttt{#2}} 
                & \multicolumn{1}{|c|}{\texttt{#3}} 
                    & \multicolumn{2}{|c|}{\texttt{#4}} \\ 
        \hline
    \end{tabular}
}}

\newcommand{\FloatMyCharHex}[3]{\text{\upshape
    \begin{tabular}{clrlr}
        \tiny{15}  &\tiny{14} & \tiny{6} & \tiny{5} & \tiny{0}\\ 
        \hline
        \multicolumn{1}{|c|}{\texttt{#1}} 
            & \multicolumn{2}{|c|}{\texttt{#2}} 
                & \multicolumn{2}{|c|}{\texttt{#3}} \\ 
        \hline
    \end{tabular}
}}

\newcommand{\FloatMySimpleHex}[2]{\text{\upshape
    \begin{tabular}{lrlr}
        \tiny{15}  & \tiny{6} & \tiny{5} & \tiny{0} \\ 
        \hline
        \multicolumn{2}{|c|}{\texttt{#1}} 
            & \multicolumn{2}{|c|}{\texttt{#2}} \\ 
        \hline
    \end{tabular}
}}

\newcommand{\FloatMyOrderX}[4]{\text{\upshape
    \begin{tabular}{clrclr}
        \tiny{9}  &\tiny{8} & \tiny{4} & \tiny{3} & \tiny{2} & \tiny{0}\\ 
        \hline
        \multicolumn{1}{|c|}{\texttt{#1}} 
            & \multicolumn{2}{|c|}{\texttt{#2}} 
                & \multicolumn{1}{|c|}{\texttt{#3}} 
                    & \multicolumn{2}{|c|}{\texttt{#4}} \\ 
        \hline
    \end{tabular}
}}

\newcommand{\FloatMyDcOrderX}[3]{\text{\upshape
    \begin{tabular}{lrclr}
        \tiny{9} & \tiny{4} & \tiny{3} & \tiny{2} & \tiny{0}\\ 
        \hline
        \multicolumn{2}{|c|}{\texttt{#1}} 
            & \multicolumn{1}{|c|}{\texttt{#2}} 
                & \multicolumn{2}{|c|}{\texttt{#3}} \\ 
        \hline
    \end{tabular}
}}

\newcommand{\FloatMyDcCharX}[2]{\text{\upshape
    \begin{tabular}{lrlr}
        \tiny{9} & \tiny{4} & \tiny{3} & \tiny{0}\\ 
        \hline
        \multicolumn{2}{|c|}{\texttt{#1}} 
            & \multicolumn{2}{|c|}{\texttt{#2}} \\ 
        \hline
    \end{tabular}
}}

\newcommand{\FloatMyCharX}[3]{\text{\upshape
    \begin{tabular}{clrlr}
        \tiny{9}  &\tiny{8} & \tiny{4} & \tiny{3} & \tiny{0}\\ 
        \hline
        \multicolumn{1}{|c|}{\texttt{#1}} 
            & \multicolumn{2}{|c|}{\texttt{#2}} 
                & \multicolumn{2}{|c|}{\texttt{#3}} \\ 
        \hline
    \end{tabular}
}}

\newcommand{\FloatESShort}[3]{\text{\upshape
    \begin{tabular}{clrlr}
        \tiny{31}  &\tiny{30} & \tiny{24} & \tiny{23} & \tiny{0}\\ 
        \hline
        \multicolumn{1}{|c|}{\texttt{#1}} 
            & \multicolumn{2}{|c|}{\texttt{#2}} 
                & \multicolumn{2}{|c|}{\texttt{#3}} 
                    \\ 
        \hline
    \end{tabular}
}}

\newcommand{\FloatPCShort}[3]{\text{\upshape
    \begin{tabular}{clrlr}
        \tiny{31}  &\tiny{30} & \tiny{23} & \tiny{22} & \tiny{0}\\ 
        \hline
        \multicolumn{1}{|c|}{\texttt{#1}} 
            & \multicolumn{2}{|c|}{\texttt{#2}} 
                & \multicolumn{2}{|c|}{\texttt{#3}} 
                    \\ 
        \hline
    \end{tabular}
}}


%--- СПЕЦИФИЧНЫЕ ДЛЯ УМНОЖЕНИЯ КОМАНДЫ ---------------------------------------------------------------------------------------------


\newcommand{\Number}[1]{
    \texttt{#1}
}

\newcommand{\NumberHi}[2]{
    \underline{\underline{\texttt{#1}}}\texttt{#2}
}

\newcommand{\NumberMid}[3]{
    \texttt{#1}\underline{\underline{\texttt{#2}}}\texttt{#3}
}

\newcommand{\NumberLo}[2]{
    \texttt{#1}\underline{\underline{\texttt{#2}}}
}

\newcommand{\Stack}[2]{
    \begin{tabular}[t]{@{}r@{}}
        {#1}\\ \hline
        {#2}\\ 
    \end{tabular}
}

\newcommand{\Operation}[4]{
    \begin{tabular}[t]{@{}r@{}}
        \texttt{#4}
        \begin{tabular}{@{}r@{}}
            \Number{#1}\\
            \Number{#2}\\ \hline
        \end{tabular} \\ 
        \Number{#3}\\
    \end{tabular}
}

\newcommand{\Addition}[3]{\Operation{#1}{#2}{#3}{+}}

\newcommand{\Subtraction}[3]{\Operation{#1}{#2}{#3}{-}}

\newcommand{\Multiplication}[3]{\Operation{#1}{#2}{#3}{$\times$}}

\newcommand{\Register}[2]{\Number{#1:#2}}

\newcommand{\Mantiss}{m}
\newcommand{\Order}{p}
\newcommand{\Char}{c}

\newcommand{\MantissOf}[1]{\Mantiss_{#1}}
\newcommand{\OrderOf}[1]{\Order_{#1}}
\newcommand{\CharOf}[1]{\Char_{#1}}

\newcommand{\FloatExpression}[2]{\MantissOf{#1}\cdot {#2}^{\OrderOf{#1}}}

\newenvironment{Solve}[1]%
    {\begin{proof}[Решение]#1}
    {\end{proof}}
    
    

\title[Сложение в ПЗ]{Сложение в формате с плавающей точкой}

\newcounter{TaskSimpleCtr}
\setcounter{TaskSimpleCtr}{1}
\newcommand{\TaskSimpleNumber}{ \arabic{TaskSimpleCtr}) \addtocounter{TaskSimpleCtr}{1} }

%вставка изображений из metapost (post script)
\DeclareGraphicsRule{*}{mps}{*}{}

\begin{document}

\mode<article>{\maketitle\tableofcontents}
\frame<presentation>{\titlepage}
\begin{frame}<presentation>
    \frametitle{Содержание}
    \tableofcontents
\end{frame}



\begin{frame}
    \frametitle{Формат с плавающей точкой}
    
    \[
        X = \FloatExpression{X}{2},
    \]
    где $\MantissOf{X}$ --- нормализованная мантисса числа $X$, $\OrderOf{X}$ --- порядок числа $X$, подобранный так, чтобы $\MantissOf{X}$ была нормализованной.
    \begin{block}{Правила нормализации $X\neq 0$}
        Мантисса $\MantissOf{X}$ получается из двоичного представления $X$ переносом точки в такую позицию, чтобы целая часть была равна нулю, а в старшем разряде дробной части была единица:
        \[(0.\underbrace{1xxx\cdots xxx}_{\text{\tiny{разряды мантиссы}}})_2\]
        порядок $\OrderOf{X}$ определяет на сколько разрядов нужно передвинуть запятую в мантиссе, чтобы получить исходное число.
    \end{block}
\end{frame}


\section{Порядок}


\subsection{Примеры представления}


\begin{frame}
    \frametitle{Формат для примеров}

    \[
        \FloatMyHex{X}{XXXXXXXXX}{X}{XXXXX}
    \]
    \begin{itemize}
        \item Разряды нормализованной мантиссы в прямом коде хранятся в разрядах $[15:6]$.
        \item Порядок в прямом коде хранится в разрядах $[5:0]$.
    \end{itemize}
\end{frame}


\begin{frame}
    \frametitle{Целое, без потерь}
    \[
        -9=(-1001)_2=(-.1001)\cdot 2^{4}
    \]
    
    \[
        \FloatMyHex{1}{100100000}{0}{00100}
    \]
    
    \[
        (-.100100000)_2\cdot 2^{4}=-9
    \]
\end{frame}


\begin{frame}
    \frametitle{Целое, с потерями}
    \[
        999=(1111100111)_2=(.1111100111)\cdot 2^{10}
    \]
    
    \[
        \FloatMyHex{0}{111110011}{0}{01010}
    \]
    
    \[
        (.111110011)_2\cdot 2^{10}=\fbox{998}
    \]
\end{frame}


\begin{frame}
    \frametitle{Дробное, без потерь}
    \[
        0.171875=(0.001011)_2=(.1011)\cdot 2^{-2}
    \]
    
    \[
        \FloatMyHex{0}{101100000}{1}{00010}
    \]
    
    \[
        (.101100000)_2\cdot 2^{-2}=0,171875
    \]
\end{frame}

\begin{frame}
    \frametitle{Дробное, с потерями}
    \[
        -0.2=(-0.[0011])_2=(-.[1100])\cdot 2^{-2}
    \]
    
    \[
        \FloatMyHex{1}{110011001}{1}{00010}
    \]
    
    \[
        (-.110011001)_2\cdot 2^{-2}=\fbox{-0.19970703125}
    \]
\end{frame}

\begin{frame}
    \frametitle{Ноль}
    
    \[
        \FloatMyHex{0}{000000000}{0}{00000}
    \]
\end{frame}


\subsection{Правила сложения}


\begin{frame}
    \frametitle{Правила сложения}

    \[X+Y=\FloatExpression{X}{2}+\FloatExpression{Y}{2}\]
        
    \begin{enumerate}
        \item Порядки чисел выравниваются до большего, мантисса числа с меньшим порядком сдвигается вправо на модуль разности порядков.
        \[
            \begin{cases}
                (\OrderOf{X}-\OrderOf{Y}) \ge 0, & \MantissOf{Y}'\gets(\MantissOf{Y}\gg |\OrderOf{X}-\OrderOf{Y}|), \MantissOf{X}'\gets\MantissOf{X},\\
                (\OrderOf{X}-\OrderOf{Y}) < 0,   & \MantissOf{X}'\gets(\MantissOf{X}\gg |\OrderOf{X}-\OrderOf{Y}|), \MantissOf{Y}'\gets\MantissOf{Y}.
            \end{cases}
        \]
        \item Получившиеся мантиссы складываются $\MantissOf{R}=\MantissOf{X}'+\MantissOf{Y}'$. При этом порядок результата: $\OrderOf{R}=\max(\OrderOf{X}, \OrderOf{Y})$.
        
        \item Выполняется нормализация результата, если он получился не нормализованым.
    \end{enumerate}
\end{frame}


\subsection{Примеры сложения}

\begin{frame}
    \frametitle{13+29}

    \begin{center}
        \FloatMyHex{0}{110100000}{0}{00100} +
        \FloatMyHex{0}{111010000}{0}{00101}
    \end{center}
    
    \resizebox{!}{.75\height}{
        \begin{tabular}{r|r|l}
            \multicolumn{1}{c|}{$\Mantiss$}
                                        & \multicolumn{1}{|c|}{$\Order$}
                                                                & \multicolumn{1}{|c}{прим.} \\ \hline\hline
            \Number{0,110100000}        & \Number{0,00100}      & $X=13$, ПК\\ 
            \Number{0,111010000}        & \Number{0,00101}      & $Y=29$, ПК\\ \hline
                                        & \Addition{00,00100}
                                                   {11,11011}
                                                   {11,11111}   & $\OrderOf{X}-\OrderOf{Y} < 0$, в МДК, денормализуется $\MantissOf{X}$ \\ \hline
            \Number{0,011010000}        & \Number{0,00101}      & $X'$, денормализованное \\ \hline
            \Addition{00,011010000}     
                     {00,111010000}     
                     {01,010100000}     &                       & $\MantissOf{R} = \MantissOf{X}'+\MantissOf{Y}'$, в МДК, ПРС!\\ \hline
            \NumberHi{01}{,010100000}   & \Number{0,00101}      & Нормализовать! $\MantissOf{R}\gets \MantissOf{R}\gg 1$; 
                                                                                 $\OrderOf{R}\gets\OrderOf{R}+1$ \\
            \Number{0,101010000}        & \Number{0,00110}      & Рез-т!
        \end{tabular}
    }
    \[
        \FloatMyHex{0}{101010000}{0}{00110}
    \]
\end{frame}

\begin{frame}
    \frametitle{-17+14}

    \begin{center}
        \FloatMyHex{1}{100010000}{0}{00101} +
        \FloatMyHex{0}{111000000}{0}{00100} 
    \end{center}
    
    \resizebox{!}{.6\height}{
        \begin{tabular}{r|r|l}
            \multicolumn{1}{c|}{$\Mantiss$}
                                        & \multicolumn{1}{|c|}{$\Order$}
                                                                & \multicolumn{1}{|c}{прим.} \\ \hline\hline
            \Number{1,100010000}        & \Number{0,00101}      & $X=-17$, ПК\\ 
            \Number{0,111000000}        & \Number{0,00100}      & $Y=14$,  ПК\\ \hline
                                        & \Addition{00,00101}
                                                   {11,11100}
                                                   {00,00001}   & $\OrderOf{X}-\OrderOf{Y} \ge 0$, МДК, денормализуется $\MantissOf{Y}$ \\ \hline
            \Number{0,011100000}        & \Number{0,00101}      & $Y'$, денормализованное \\ \hline
            \Addition{11,011110000} 
                     {00,011100000} 
                     {11,111010000}     &                      & $\MantissOf{R} = \MantissOf{X}'+\MantissOf{Y}'$, МДК\\ \hline
            \NumberHi{11}{,111010000}   & \Number{0,00101} & Получить модуль мантиссы для представления в ПК! \\ \hline
            \NumberHi{00}{,000110000}   & \Number{0,00101} & Нормализовать модуль! $\MantissOf{R}\gets \MantissOf{R}\ll 1$; 
                                                                                   $\OrderOf{R}\gets\OrderOf{R}-1$ \\ 
            \NumberHi{00}{,001100000}   & \Number{0,00100} & Нормализовать модуль! $\MantissOf{R}\gets \MantissOf{R}\ll 1$; 
                                                                                   $\OrderOf{R}\gets\OrderOf{R}-1$ \\
            \NumberHi{00}{,011000000}   & \Number{0,00011} & Нормализовать модуль! $\MantissOf{R}\gets \MantissOf{R}\ll 1$; 
                                                                                   $\OrderOf{R}\gets\OrderOf{R}-1$ \\ \hline
            \Number{1,110000000}        & \Number{0,00010} & Рез-т!
        \end{tabular}
    }
    \[
        \FloatMyHex{1}{110000000}{0}{00010}
    \]
\end{frame}
    

\begin{frame}
    \frametitle{-2+{-2}}

    \begin{center}
        \FloatMyHex{1}{100000000}{0}{00010} +
        \FloatMyHex{1}{100000000}{0}{00010} 
    \end{center}
    
    \resizebox{!}{.75\height}{
        \begin{tabular}{r|r|l}
            \multicolumn{1}{c|}{$\Mantiss$}
                                      & \multicolumn{1}{|c|}{$\Order$}
                                                              & \multicolumn{1}{|c}{прим.} \\ \hline\hline
            \Number{1,100000000}      & \Number{0,00010}      & $X=-2$, ПК\\ 
            \Number{1,100000000}      & \Number{0,00010}      & $Y=-2$, ПК\\ \hline
                                      & \Addition{00,00010}
                                                 {11,11110}
                                                 {00,00000}   & $\OrderOf{X}-\OrderOf{Y} = 0$, порядки одинаковы \\ \hline
            \Addition{11,100000000}
                     {11,100000000}
                     {11,000000000}   &                       & $\MantissOf{R} = \MantissOf{X}'+\MantissOf{Y}'$, МДК\\ \hline
            \NumberHi{11}{,000000000} & \Number{0,00010}      & Получить модуль мантиссы для представления в ПК! \\ \hline
            \NumberHi{01}{,000000000} & \Number{0,00010}      & Нормализовать модуль! $\MantissOf{R}\gets \MantissOf{R}\gg 1$; 
                                                                                      $\OrderOf{R}\gets\OrderOf{R}+1$ \\ \hline
            \Number{1,100000000}      & \Number{0,00011}      & Рез-т!
        \end{tabular}
    }
    \[
        \FloatMyHex{1}{100000000}{0}{00011}
    \]
\end{frame}

\begin{frame}
    \frametitle{ПРС (переполнение разрядной сетки)}

    \begin{center}
        \FloatMyHex{0}{111100000}{0}{11111} +
        \FloatMyHex{0}{110000000}{0}{11101} 
    \end{center}
    
    \resizebox{!}{.75\height}{
        \begin{tabular}{r|r|l}
            \multicolumn{1}{c|}{$\Mantiss$}
                                      & \multicolumn{1}{|c|}{$\Order$}
                                                              & \multicolumn{1}{|c}{прим.} \\ \hline\hline
            \Number{0,111100000}      & \Number{0,11111}      & $X$, ПК\\ 
            \Number{0,110000000}      & \Number{0,11101}      & $Y$, ПК\\ \hline
                                      & \Addition{00,11111}
                                                 {11,00011}
                                                 {00,00010}   & $\OrderOf{X}-\OrderOf{Y} \ge 0$, МДК, денормализуется $\MantissOf{Y}$ \\ \hline
            \Number{0,001100000}      & \Number{0,11111}      & $Y'$, денормализованное\\ \hline
            \Addition{00,111100000}   
                     {00,001100000}   
                     {01,001000000}   &                       & $\MantissOf{R} = \MantissOf{X}'+\MantissOf{Y}'$, МДК, ПРС мантиссы!\\ \hline
            \NumberHi{01}{,001000000} & \Number{0,11111} & Нормализовать! $\MantissOf{R}\gets \MantissOf{R}\gg 1$; 
                                                                          $\OrderOf{R}\gets\OrderOf{R}+1$ \\ \hline
            \NumberHi{00}{,100100000} & \Number{?,?????} & ПРС порядка --- настоящий ПРС в формате с ПЗ!
        \end{tabular}
    }
    
    \begin{center}
        Генерация ошибки вычислений!
    \end{center}
\end{frame}

\begin{frame}
    \frametitle{ПМР (потеря младщих разрядов)}

    \begin{center}
        \FloatMyHex{0}{100001000}{1}{11110} +
        \FloatMyHex{1}{111100000}{1}{11111} 
    \end{center}
    
    \resizebox{!}{.75\height}{
        \begin{tabular}{r|r|l}
            \multicolumn{1}{c|}{$\Mantiss$}
                                    & \multicolumn{1}{|c|}{$\Order$}
                                                            & \multicolumn{1}{|c}{прим.} \\ \hline\hline
            \Number{0,100001000}    & \Number{1,11110}      & $X$, ПК\\ 
            \Number{1,111100000}    & \Number{1,11111}      & $Y$, ПК\\ \hline
                                    & \Addition{11,00010}
                                               {00,11111}
                                               {00,00001}   & $\OrderOf{X}-\OrderOf{Y} \ge 0$, МДК, денормализуется $\MantissOf{Y}$ \\ \hline
            \Number{1,011110000}    & \Number{1,11110}      & $Y'$, денормализованное \\ \hline
            \Addition{00,100001000}
                     {11,100010000}
                     {00,000011000}   &                     & $\MantissOf{R} = \MantissOf{X}'+\MantissOf{Y}'$, МДК \\ \hline
            \NumberHi{00}{,000011000} & \Number{1,11110} & Нормализовать! $\MantissOf{R}\gets \MantissOf{R}\ll 1$; 
                                                                          $\OrderOf{R}\gets\OrderOf{R}-1$ \\
            \NumberHi{00}{,000110000} & \Number{1,11111} & Нормализовать! $\MantissOf{R}\gets \MantissOf{R}\ll 1$; 
                                                                          $\OrderOf{R}\gets\OrderOf{R}-1$ \\
            \NumberHi{00}{,001100000} & \Number{?,?????} & $\OrderOf{R}$ за пределом представления \emph{отрицательных} чисел в ПК! \\
        \end{tabular}
    }
    
    \[
        \FloatMyHex{0}{000000000}{0}{00000}
    \]
\end{frame}

\begin{frame}
    \frametitle{$X+Y=X, Y\neq 0$ ?}

    \begin{center}
        \FloatMyHex{0}{111000000}{0}{11000} +
        \FloatMyHex{0}{110000000}{0}{01110} 
    \end{center}
    
    \resizebox{!}{.75\height}{
        \begin{tabular}{r|r|l}
            \multicolumn{1}{c|}{$\Mantiss$}
                                     & \multicolumn{1}{|c|}{$\Order$}
                                                           & \multicolumn{1}{|c}{прим.} \\ \hline\hline
            \Number{0,111000000}     & \Number{0,11000}    & $X=7\cdot 2^{21}$, ПК\\ 
            \Number{0,110000000}     & \Number{0,01110}    & $Y=3\cdot 2^{12}$, ПК\\ \hline
                                     & \Addition{00,11000}
                                                {11,10010}
                                                {00,01010} & $|\OrderOf{X}-\OrderOf{Y}| \ge 9$, МДК, $\MantissOf{Y}$ денормализуется в 0 
            \\ \hline
            \Number{0,000000000} & \Number{0,11000} & $Y'=0$?
        \end{tabular}
    }
    
    \begin{center}
        \FloatMyHex{0}{111000000}{0}{11000} 
    \end{center}
\end{frame}

    
\section{Характеристика}


\begin{frame}
    \frametitle{Характеристика}

    \[
        X = \FloatExpression{X}{2}.
    \]
    
    Диапазон представления порядка $\OrderOf{X}$ в $n$-разрядной сетке будет\footnote{Если использовать дополнительный код}:
    \[
        \OrderOf{X}\in[-2^{n-1}, +(2^{n-1}-1)]
    \]
    
    Характеристика получается из порядка прибавлением фиксированной поправки $\Delta$, такой, что левая граница представления обращается в ноль. Таким образом, 
    \begin{block}{}
        характеристика $\CharOf{X}$ --- всегда положительное число.
        \begin{equation}
            \CharOf{X} = \OrderOf{X} + \Delta,
            \label{add:fpt:char}
        \end{equation}
        где\footnote{Опять же только в случае использования дополнительного кода} $\Delta=+2^{n-1}$, а $\CharOf{X}\in[0,2^n-1]$.
    \end{block}
\end{frame}

\begin{frame}
    \frametitle{Свойства $n$-разрядной характеристики}

    \begin{itemize}
        \item Характеристика --- положительное число.
        \item Разность характеристик равна разности порядков.
        \item Если в процессе нормализации (или денормализации) порядок увеличивается (или уменьшается), то то же самое происходит и с характеристикой.
        \item Если для работы с характеристиками использовать ДК или МДК, о ПРС при нормализации легко судить по знаковому разряду: он не должен быть 1.
        \item Если использвется поправка $\Delta=2^{n-1}$, то характеристика получается из дополнительного кода порядка инверсией знакового разряда.
    \end{itemize}
\end{frame}


\subsection{Примеры представления}


\begin{frame}
    \frametitle{Формат для примеров}

    \[
        \FloatMyCharHex{X}{XXXXXXXXX}{XXXXXX}
    \]
    \begin{itemize}
        \item Разряды нормализованной мантиссы в прямом коде хранятся в разрядах $[15:6]$.
        \item Характеристика хранится в разрядах $[5:0]$.
        \item $\Delta=2^5=32=(100000)_2$
    \end{itemize}
\end{frame}

\begin{frame}
    \frametitle{Целое, без потерь}
    \framesubtitle{$\Delta=2^5=32$}
    \[
        -9=(-1001)_2=(-.1001)\cdot 2^{4}
    \]
    
    \[
        \FloatMyCharHex{1}{100100000}{100100}
    \]
    
    \[
        (-.100100000)_2\cdot 2^{36-\Delta}=-9
    \]
\end{frame}

\begin{frame}
    \frametitle{Целое, с потерями}
    \framesubtitle{$\Delta=2^5=32$}
    \[
        999=(1111100111)_2=(.1111100111)\cdot 2^{10}
    \]
    
    \[
        \FloatMyCharHex{0}{111110011}{101010}
    \]
    
    \[
        (.111110011)_2\cdot 2^{42-\Delta}=\fbox{998}
    \]
\end{frame}

\begin{frame}
    \frametitle{Дробное, без потерь}
    \framesubtitle{$\Delta=2^5=32$}
    \[
        0.171875=(0.001011)_2=(.1011)\cdot 2^{-2}
    \]
    
    \[
        \FloatMyCharHex{0}{101100000}{011110}
    \]
    
    \[
        (.101100000)_2\cdot 2^{30-\Delta}=0,171875
    \]
\end{frame}

\begin{frame}
    \frametitle{Дробное, с потерями}
    \framesubtitle{$\Delta=2^5=32$}
    \[
        -0.2=(-0.[0011])_2=(-.[1100])\cdot 2^{-2}
    \]
    
    \[
        \FloatMyCharHex{1}{110011001}{011110}
    \]
    
    \[
        (-.110011001)_2\cdot 2^{30-\Delta}=\fbox{-0.19970703125}
    \]
\end{frame}

\begin{frame}
    \frametitle{Ноль}
    
    \[
        \FloatMyCharHex{0}{000000000}{000000}
    \]
\end{frame}


\subsection{Правила сложения}


\begin{frame}
    \frametitle{Правила}
    \framesubtitle{Исходя из формулы \eqref{add:fpt:char}, почти не меняются}

    \[
        X+Y=
            \MantissOf{X}\cdot 2^{\CharOf{X}-\Delta}+
            \MantissOf{Y}\cdot 2^{\CharOf{Y}-\Delta}
    \]
        
    \begin{enumerate}
        \item Характеристики чисел выравниваются до большей, мантисса числа с меньшей характеристикой сдвигается вправо на модуль разности характеристик.
        \[
            \begin{cases}
                (\CharOf{X}-\CharOf{Y}) \ge 0, & \MantissOf{Y}'\gets(\MantissOf{Y}\gg |\CharOf{X}-\CharOf{Y}|), \MantissOf{X}'\gets\MantissOf{X},\\
                (\CharOf{X}-\CharOf{Y}) < 0,   & \MantissOf{X}'\gets(\MantissOf{X}\gg |\CharOf{X}-\CharOf{Y}|), \MantissOf{Y}'\gets\MantissOf{Y}.
            \end{cases}
        \]
        
        \item Получившиеся мантиссы складываются $\MantissOf{R}=\MantissOf{X}'+\MantissOf{Y}'$. При этом характеристика результата: $\CharOf{R}=\max(\CharOf{X}, \CharOf{Y})$.
        
        \item Выполняется нормализация результата, если он получился не нормализованым.
    \end{enumerate}
\end{frame}


\subsection{Примеры сложения}


\begin{frame}
    \frametitle{13+57}

    \begin{center}
        \FloatMyCharHex{0}{110100000}{100100} +
        \FloatMyCharHex{0}{111001000}{100110}
    \end{center}
    
    \resizebox{!}{.75\height}{
        \begin{tabular}{r|r|l}
            \multicolumn{1}{c|}{$\Mantiss$}
                                        & \multicolumn{1}{|c|}{$\Char$}
                                                               & \multicolumn{1}{|c}{прим.} \\ \hline\hline
            \Number{0,110100000}        & \Number{100100}      & $X=13$\\ 
            \Number{0,111001000}        & \Number{100110}      & $Y=57$\\ \hline
                                        & \Addition{0,100100}
                                                   {1,011010}
                                                   {1,111110}  & $\CharOf{X}-\CharOf{Y} < 0$, ДК, денормализуется $\MantissOf{X}$ 
            \\ \hline
            \Number{0,001101000}        & \Number{0,100110}    & $X'$, денормализованное \\ \hline
            \Addition{00,001101000}     
                     {00,111001000}     
                     {01,000110000}     &                      & $\MantissOf{R} = \MantissOf{X}'+\MantissOf{Y}'$, в МДК, ПРС!\\ \hline
            \NumberHi{01}{,000110000}   & \Number{0,100111}    & Нормализовать! $\MantissOf{R}\gets \MantissOf{R}\gg 1$; 
                                                                                $\CharOf{R}\gets\CharOf{R}+1$ 
                                                                                \\ \hline
            \Number{0,100011000}        & \Number{100111}      & Рез-т!
        \end{tabular}
    }
    \[
        \FloatMyCharHex{0}{100011000}{100111}
    \]
\end{frame}

\begin{frame}
    \frametitle{-17+14}

    \begin{center}
        \FloatMyCharHex{1}{100010000}{100101} +
        \FloatMyCharHex{0}{111000000}{100100} 
    \end{center}
    
    \resizebox{!}{.6\height}{
        \begin{tabular}{r|r|l}
            \multicolumn{1}{c|}{$\Mantiss$}
                                        & \multicolumn{1}{|c|}{$\Char$}
                                                              & \multicolumn{1}{|c}{прим.} \\ \hline\hline
            \Number{1,100010000}        & \Number{100101}     & $X=-17$\\ 
            \Number{0,111000000}        & \Number{100100}     & $Y=14$\\ \hline
                                        & \Addition{0,100101} 
                                                   {1,011100} 
                                                   {0,000001} & $\CharOf{X}-\CharOf{Y} \ge 0$, ДК, денормализуется $\MantissOf{Y}$ 
            \\ \hline
            \Number{0,011100000}        & \Number{0,100101} & $Y'$, денормализованное \\ \hline
            \Addition{11,011110000} 
                     {00,011100000} 
                     {11,111010000}     &                   & $\MantissOf{R} = \MantissOf{X}'+\MantissOf{Y}'$, МДК\\ \hline
            \NumberHi{11}{,111010000}   & \Number{0,100101} & Получить модуль мантиссы для представления в ПК! \\ \hline
            \NumberHi{00}{,000110000}   & \Number{0,100101} & Нормализовать модуль! $\MantissOf{R}\gets \MantissOf{R}\ll 1$; 
                                                                                    $\CharOf{R}\gets\CharOf{R}-1$ \\ 
            \NumberHi{00}{,001100000}   & \Number{0,100100} & Нормализовать модуль! $\MantissOf{R}\gets \MantissOf{R}\ll 1$; 
                                                                                    $\CharOf{R}\gets\CharOf{R}-1$ \\
            \NumberHi{00}{,011000000}   & \Number{0,100011} & Нормализовать модуль! $\MantissOf{R}\gets \MantissOf{R}\ll 1$; 
                                                                                    $\CharOf{R}\gets\CharOf{R}-1$ 
            \\ \hline
            \Number{1,110000000}        & \Number{100010} & Рез-т!
        \end{tabular}
    }
    \[
        \FloatMyCharHex{1}{110000000}{100010}
    \]
\end{frame}

\begin{frame}
    \frametitle{ПРС (переполнение разрядной сетки)}

    \begin{center}
        \FloatMyCharHex{0}{111100000}{111111} +
        \FloatMyCharHex{0}{110000000}{111101} 
    \end{center}
    
    \resizebox{!}{.75\height}{
        \begin{tabular}{r|r|l}
            \multicolumn{1}{c|}{$\Mantiss$}
                                      & \multicolumn{1}{|c|}{$\Char$}
                                                             & \multicolumn{1}{|c}{прим.} \\ \hline\hline
            \Number{0,111100000}      & \Number{111111}      & $X$\\ 
            \Number{0,110000000}      & \Number{111101}      & $Y$\\ \hline
                                      & \Addition{0,111111}
                                                 {1,000011}
                                                 {0,000010}  & $\CharOf{X}-\CharOf{Y} \ge 0$, ДК, денормализуется $\MantissOf{Y}$ \\ \hline
            \Number{0,001100000}      & \Number{0,111111}    & $Y'$, денормализованное\\ \hline
            \Addition{00,111100000}   
                     {00,001100000}   
                     {01,001000000}   &                       & $\MantissOf{R} = \MantissOf{X}'+\MantissOf{Y}'$, МДК, ПРС мантиссы!\\ \hline
            \NumberHi{01}{,001000000} & \Number{0,111111}     & Нормализовать! $\MantissOf{R}\gets \MantissOf{R}\gg 1$; 
                                                                               $\CharOf{R}\gets\CharOf{R}+1$ \\ \hline
            \NumberHi{00}{,100100000} & \NumberHi{1}{,000000} & $\CharOf{R} < 0$, выход за правую границу представления --- ПРС!
        \end{tabular}
    }
    
    \begin{center}
        Генерация ошибки вычислений!
    \end{center}
\end{frame}

\begin{frame}
    \frametitle{ПМР (потеря младщих разрядов)}

    \begin{center}
        \FloatMyCharHex{0}{100001000}{000010} +
        \FloatMyCharHex{1}{111100000}{000001} 
    \end{center}
    
    \resizebox{!}{.7\height}{
        \begin{tabular}{r|r|l}
            \multicolumn{1}{c|}{$\Mantiss$}
                                      & \multicolumn{1}{|c|}{$\Char$}
                                                            & \multicolumn{1}{|c}{прим.} \\ \hline\hline
            \Number{0,100001000}      & \Number{000010}     & $X$\\ 
            \Number{1,111100000}      & \Number{000001}     & $Y$\\ \hline
                                      & \Addition{0,000010}
                                                 {1,111111}
                                                 {0,000001} & $\CharOf{X}-\CharOf{Y} \ge 0$, МДК, денормализуется $\MantissOf{Y}$ \\ \hline
            \Number{1,011110000}      & \Number{0,000010}   & $Y'$, денормализованное \\ \hline
            \Addition{00,100001000}
                     {11,100010000}
                     {00,000011000}   &                     & $\MantissOf{R} = \MantissOf{X}'+\MantissOf{Y}'$, МДК \\ \hline
            \NumberHi{00}{,000011000} & \Number{0,000010} & Нормализовать! $\MantissOf{R}\gets \MantissOf{R}\ll 1$; 
                                                                           $\CharOf{R}\gets\CharOf{R}-1$ \\
            \NumberHi{00}{,000110000} & \Number{0,000001} & Нормализовать! $\MantissOf{R}\gets \MantissOf{R}\ll 1$; 
                                                                           $\CharOf{R}\gets\CharOf{R}-1$ \\
            \NumberHi{00}{,001100000} & \Number{0,000000} & Нормализовать! $\MantissOf{R}\gets \MantissOf{R}\ll 1$; 
                                                                           $\CharOf{R}\gets\CharOf{R}-1$ \\
            \NumberHi{00}{,011000000} & \NumberHi{1}{,111111} & $\CharOf{R} < 0$, выход за левую границу представления --- ПМР! \\
        \end{tabular}
    }
    
    \[
        \FloatMyCharHex{0}{000000000}{000000}
    \]
\end{frame}


\appendix


\section{Задания на практику}


\subsection{Проходное}

\begin{frame}
    \frametitle{\TaskSimpleNumber}
    
    Придумать правила определения ПРС/ПМР при работе 
    \begin{itemize}
        \item с порядками в прямом коде;
        \item с характеристиками.
    \end{itemize}
\end{frame}

\begin{frame}
    \frametitle{\TaskSimpleNumber}

    Разработать собственный 10-разрядный формат\footnote{Преподавателю: обязательно проследить, чтобы были использованы и порядки и характеристики} и сложить в нем числа:

    \begin{enumerate}
        \item $9$ и $-11$;
        \item $10$ и $7$;
        \item $0.625$ и $0.75$;
        \item $-0.625$ и $0.375$.
    \end{enumerate}
\end{frame}

\subsection{Мегамозг}

\begin{frame}
    \frametitle{\TaskSimpleNumber}
    
    Придумать пример из последовательности трех чисел, сумма которых зависимсит от порядка суммирования. 
\end{frame}

\section{Самообучение}

\begin{frame}
    \frametitle{Советы самоучке}
    
    Стандарт на формат с плавающей точкой IEEE 754.
\end{frame}

\begin{frame}[allowframebreaks]{Библиография}
    \bibliographystyle{gost780u}
    \bibliography{./../../../bibliobase}
\end{frame}

\end{document}