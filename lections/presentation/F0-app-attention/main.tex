%include part: see main.beamer.tex and main.article.tex
%include common packages and settings
\usepackage{etex} %эта магическая херь избавляет от переполнения регистров TeX а!!!

\mode<article>{\usepackage{fullpage}}
\mode<presentation>{
    \usetheme{Madrid} %%Boadilla,Madrid,AnnArbor,CambridgeUS,Malmoe,Singapore,Berlin
    \useoutertheme{shadow}
} 

\usepackage[utf8]{inputenc}
\usepackage[russian]{babel}
\usepackage{indentfirst}
\usepackage{graphicx}

\usepackage{amsmath}
\usepackage{amsfonts}
\usepackage{amsthm}
\usepackage{algorithm}
\usepackage{algorithmic}

\usepackage[all]{xy}

\date{Лекция по дисциплине <<информатика>>\\(\today)}
\author[М.~М.~Шихов]{Михаил Шихов \\ \texttt{\underline{m.m.shihov@gmail.com}}}

%для рисования графов пакетом xy-pic
\entrymodifiers={++[o][F-]}

%для псевдокода алгоритмов (algorithm,algorithmic)
\renewcommand{\algorithmicrequire}{\textbf{Вход:}}
\renewcommand{\algorithmicensure}{\textbf{Выход:}}
\renewcommand{\algorithmiccomment}[1]{// #1}
\floatname{algorithm}{Псевдокод}

%%определённые мной команды логической разметки
\newcommand{\Signs}[2]{\fbox{{\small{\underbar{#1}}}{#2}}}
\newcommand{\Sign}[1]{\fbox{#1}}
\newcommand{\DC}[1]{\text{ДК}(#1)}
\newcommand{\OC}[1]{\text{ОК}(#1)}
\newcommand{\PC}[1]{\text{ПК}(#1)}

\newcommand{\MyProc}{\text{$\mathbf{R}_8$}}

\newboolean{IsNeedAnswer}
\setboolean{IsNeedAnswer}{false} %true/false

\newcommand{\Machine}[1]{\texttt{#1}}
\newcommand{\Opcode}[1]{\texttt{\bf{#1}}}
\newcommand{\Operand}[1]{\texttt{#1}}
\newcommand{\CmdOneAddr}[2]{\text{\Opcode{#1} \Operand{#2}}}
\newcommand{\CmdTwoAddr}[3]{\text{\Opcode{#1} \Operand{#2}, \Operand{#3}}}
\newcommand{\CmdThreeAddr}[4]{\text{\Opcode{#1} {\Operand{#2}}, \Operand{#3}, \Operand{#4}}}

\newcommand{\ProofAnswer}[1]{
    \ifthenelse{\boolean{IsNeedAnswer}}{
        \begin{proof}[Ответ]
            #1
        \end{proof}
    }{}
}

\newcommand{\LabeledAnswer}[1]{
    \ifthenelse{\boolean{IsNeedAnswer}}{
        \emph{Отв.}: #1 \qed
    }{}
}

\newcommand{\PlainAnswer}[1]{
    \ifthenelse{\boolean{IsNeedAnswer}}{#1}{}
}

\newcommand{\UnsignedAny}[2]{\text{\upshape
    \begin{tabular}{lr}
        \tiny{#1} & \tiny{0}\\ 
        \hline
        \multicolumn{2}{|c|}{\texttt{#2}} \\ 
        \hline
    \end{tabular}
}}

\newcommand{\UnsignedByte}[1]{\UnsignedAny{7}{#1}}

\newcommand{\UnsignedTwoBytes}[1]{\UnsignedAny{15}{#1}}

\newcommand{\FixedAny}[4]{\text{\upshape
    \begin{tabular}{clr}
        \tiny{#1}  &\tiny{#2} & \tiny{0}\\ 
        \hline
        \multicolumn{1}{|c|}{\texttt{#3}} & \multicolumn{2}{|c|}{\texttt{#4}} \\ 
        \hline
    \end{tabular}
}}

\newcommand{\FixedByte}[2]{\FixedAny{7}{6}{#1}{#2}}

\newcommand{\FixedTwoBytes}[2]{\FixedAny{15}{14}{#1}{#2}}

\newcommand{\SignedAny}[4]{\text{\upshape
    \begin{tabular}{clr}
        \tiny{#1}  &\tiny{#2} & \tiny{0}\\ 
        \hline
        \multicolumn{1}{|c|}{\Machine{#3}} & \multicolumn{2}{|c|}{\Machine{#4}} \\ 
        \hline
    \end{tabular}
}}

\newcommand{\SignedNibble}[2]{\SignedAny{3}{2}{#1}{#2}}

\newcommand{\SignedByte}[2]{\SignedAny{7}{6}{#1}{#2}}

\newcommand{\SignedTwoBytes}[2]{\SignedAny{15}{14}{#1}{#2}}

\newcommand{\FloatMyHex}[4]{\text{\upshape
    \begin{tabular}{clrclr}
        \tiny{15}  &\tiny{14} & \tiny{6} & \tiny{5} & \tiny{4} & \tiny{0}\\ 
        \hline
        \multicolumn{1}{|c|}{\texttt{#1}} 
            & \multicolumn{2}{|c|}{\texttt{#2}} 
                & \multicolumn{1}{|c|}{\texttt{#3}} 
                    & \multicolumn{2}{|c|}{\texttt{#4}} \\ 
        \hline
    \end{tabular}
}}

\newcommand{\FloatMyCharHex}[3]{\text{\upshape
    \begin{tabular}{clrlr}
        \tiny{15}  &\tiny{14} & \tiny{6} & \tiny{5} & \tiny{0}\\ 
        \hline
        \multicolumn{1}{|c|}{\texttt{#1}} 
            & \multicolumn{2}{|c|}{\texttt{#2}} 
                & \multicolumn{2}{|c|}{\texttt{#3}} \\ 
        \hline
    \end{tabular}
}}

\newcommand{\FloatMyDcMantCharHex}[2]{\text{\upshape
    \begin{tabular}{lrlr}
        \tiny{15} & \tiny{6} & \tiny{5} & \tiny{0}\\ 
        \hline
        \multicolumn{2}{|c|}{\texttt{#1}} 
            & \multicolumn{2}{|c|}{\texttt{#2}} \\ 
        \hline
    \end{tabular}
}}

\newcommand{\FloatESShort}[3]{\text{\upshape
    \begin{tabular}{clrlr}
        \tiny{31}  &\tiny{30} & \tiny{24} & \tiny{23} & \tiny{0}\\ 
        \hline
        \multicolumn{1}{|c|}{\texttt{#1}} 
            & \multicolumn{2}{|c|}{\texttt{#2}} 
                & \multicolumn{2}{|c|}{\texttt{#3}} 
                    \\ 
        \hline
    \end{tabular}
}}

\newcommand{\FloatPCShort}[3]{\text{\upshape
    \begin{tabular}{clrlr}
        \tiny{31}  &\tiny{30} & \tiny{23} & \tiny{22} & \tiny{0}\\ 
        \hline
        \multicolumn{1}{|c|}{\texttt{#1}} 
            & \multicolumn{2}{|c|}{\texttt{#2}} 
                & \multicolumn{2}{|c|}{\texttt{#3}} 
                    \\ 
        \hline
    \end{tabular}
}}

\newcommand{\FloatMyOrderX}[4]{\text{\upshape
    \begin{tabular}{clrclr}
        \tiny{9}  &\tiny{8} & \tiny{4} & \tiny{3} & \tiny{2} & \tiny{0}\\ 
        \hline
        \multicolumn{1}{|c|}{\texttt{#1}} 
            & \multicolumn{2}{|c|}{\texttt{#2}} 
                & \multicolumn{1}{|c|}{\texttt{#3}} 
                    & \multicolumn{2}{|c|}{\texttt{#4}} \\ 
        \hline
    \end{tabular}
}}

\newcommand{\FloatMyCharX}[3]{\text{\upshape
    \begin{tabular}{clrlr}
        \tiny{9}  &\tiny{8} & \tiny{4} & \tiny{3} & \tiny{0}\\ 
        \hline
        \multicolumn{1}{|c|}{\texttt{#1}} 
            & \multicolumn{2}{|c|}{\texttt{#2}} 
                & \multicolumn{2}{|c|}{\texttt{#3}} \\ 
        \hline
    \end{tabular}
}}

\newcommand{\FloatMyDcOrderX}[3]{\text{\upshape
    \begin{tabular}{lrclr}
        \tiny{9} & \tiny{4} & \tiny{3} & \tiny{2} & \tiny{0}\\ 
        \hline
        \multicolumn{2}{|c|}{\texttt{#1}} 
            & \multicolumn{1}{|c|}{\texttt{#2}} 
                & \multicolumn{2}{|c|}{\texttt{#3}} \\ 
        \hline
    \end{tabular}
}}

\newcommand{\FloatMyDcCharX}[2]{\text{\upshape
    \begin{tabular}{lrlr}
        \tiny{9} & \tiny{4} & \tiny{3} & \tiny{0}\\ 
        \hline
        \multicolumn{2}{|c|}{\texttt{#1}} 
            & \multicolumn{2}{|c|}{\texttt{#2}} \\ 
        \hline
    \end{tabular}
}}


%--- СПЕЦИФИЧНЫЕ ДЛЯ УМНОЖЕНИЯ КОМАНДЫ ---------------------------------------------------------------------------------------------


\newcommand{\Number}[1]{
    \texttt{#1}
}

\newcommand{\NumberHi}[2]{
    \underline{\underline{\texttt{#1}}}\texttt{#2}
}

\newcommand{\NumberMid}[3]{
    \texttt{#1}\underline{\underline{\texttt{#2}}}\texttt{#3}
}

\newcommand{\NumberLo}[2]{
    \texttt{#1}\underline{\underline{\texttt{#2}}}
}

\newcommand{\Stack}[2]{
    \begin{tabular}[t]{@{}r@{}}
        {#1}\\ \hline
        {#2}\\ 
    \end{tabular}
}

\newcommand{\StackThree}[3]{
    \begin{tabular}[t]{@{}r@{}}
        {#1}\\ \hline
        {#2}\\ \hline
        {#3}\\
    \end{tabular}
}

\newcommand{\StackFour}[4]{
    \begin{tabular}[t]{@{}r@{}}
        {#1}\\ \hline
        {#2}\\ \hline
        {#3}\\ \hline
        {#4}\\
    \end{tabular}
}

\newcommand{\Operation}[4]{
    \begin{tabular}[t]{@{}r@{}}
        \texttt{#4}
        \begin{tabular}{@{}r@{}}
            \Number{#1}\\
            \Number{#2}\\ \hline
        \end{tabular} \\ 
        \Number{#3}\\
    \end{tabular}
}

\newcommand{\Addition}[3]{\Operation{#1}{#2}{#3}{+}}

\newcommand{\Subtraction}[3]{\Operation{#1}{#2}{#3}{-}}

\newcommand{\Register}[2]{\Number{#1:#2}}

\newcommand{\Mantiss}{m}
\newcommand{\Order}{p}
\newcommand{\Char}{c}

\newcommand{\MantissOf}[1]{\Mantiss_{#1}}
\newcommand{\OrderOf}[1]{\Order_{#1}}
\newcommand{\CharOf}[1]{\Char_{#1}}

\newcommand{\FloatExpression}[2]{\MantissOf{#1}\cdot {#2}^{\OrderOf{#1}}}

\newcommand{\DivAnswer}[2]{(\texttt{$#1$ rem $#2$})}

\newenvironment{Solve}[1]%
    {\begin{proof}[Решение]#1}
    {\end{proof}}
    
    
%определённые мной команды логической разметки
\newcommand{\Signs}[2]{\fbox{{\small{\underbar{#1}}}{#2}}}
\newcommand{\Sign}[1]{\fbox{#1}}
\newcommand{\DC}[1]{\text{ДК}(#1)}
\newcommand{\OC}[1]{\text{ОК}(#1)}
\newcommand{\PC}[1]{\text{ПК}(#1)}

\newcommand{\MyProc}{\text{$\mathbf{R}_8$}}

\newboolean{IsNeedAnswer}
\setboolean{IsNeedAnswer}{false} %true/false

\newcommand{\Machine}[1]{\texttt{#1}}
\newcommand{\Opcode}[1]{\texttt{\bf{#1}}}
\newcommand{\Operand}[1]{\texttt{#1}}
\newcommand{\CmdOneAddr}[2]{\text{\Opcode{#1} \Operand{#2}}}
\newcommand{\CmdTwoAddr}[3]{\text{\Opcode{#1} \Operand{#2}, \Operand{#3}}}
\newcommand{\CmdThreeAddr}[4]{\text{\Opcode{#1} {\Operand{#2}}, \Operand{#3}, \Operand{#4}}}

\newcommand{\ProofAnswer}[1]{
    \ifthenelse{\boolean{IsNeedAnswer}}{
        \begin{proof}[Ответ]
            #1
        \end{proof}
    }{}
}

\newcommand{\LabeledAnswer}[1]{
    \ifthenelse{\boolean{IsNeedAnswer}}{
        \emph{Отв.}: #1 \qed
    }{}
}

\newcommand{\PlainAnswer}[1]{
    \ifthenelse{\boolean{IsNeedAnswer}}{#1}{}
}

\newcommand{\UnsignedAny}[2]{\text{\upshape
    \begin{tabular}{lr}
        \tiny{#1} & \tiny{0}\\ 
        \hline
        \multicolumn{2}{|c|}{\texttt{#2}} \\ 
        \hline
    \end{tabular}
}}

\newcommand{\UnsignedByte}[1]{\UnsignedAny{7}{#1}}

\newcommand{\UnsignedTwoBytes}[1]{\UnsignedAny{15}{#1}}

\newcommand{\FixedAny}[4]{\text{\upshape
    \begin{tabular}{clr}
        \tiny{#1}  &\tiny{#2} & \tiny{0}\\ 
        \hline
        \multicolumn{1}{|c|}{\texttt{#3}} & \multicolumn{2}{|c|}{\texttt{#4}} \\ 
        \hline
    \end{tabular}
}}

\newcommand{\FixedByte}[2]{\FixedAny{7}{6}{#1}{#2}}

\newcommand{\FixedTwoBytes}[2]{\FixedAny{15}{14}{#1}{#2}}

\newcommand{\SignedAny}[4]{\text{\upshape
    \begin{tabular}{clr}
        \tiny{#1}  &\tiny{#2} & \tiny{0}\\ 
        \hline
        \multicolumn{1}{|c|}{\Machine{#3}} & \multicolumn{2}{|c|}{\Machine{#4}} \\ 
        \hline
    \end{tabular}
}}

\newcommand{\SignedNibble}[2]{\SignedAny{3}{2}{#1}{#2}}

\newcommand{\SignedByte}[2]{\SignedAny{7}{6}{#1}{#2}}

\newcommand{\SignedTwoBytes}[2]{\SignedAny{15}{14}{#1}{#2}}

\newcommand{\FloatMyHex}[4]{\text{\upshape
    \begin{tabular}{clrclr}
        \tiny{15}  &\tiny{14} & \tiny{6} & \tiny{5} & \tiny{4} & \tiny{0}\\ 
        \hline
        \multicolumn{1}{|c|}{\texttt{#1}} 
            & \multicolumn{2}{|c|}{\texttt{#2}} 
                & \multicolumn{1}{|c|}{\texttt{#3}} 
                    & \multicolumn{2}{|c|}{\texttt{#4}} \\ 
        \hline
    \end{tabular}
}}

\newcommand{\FloatMyCharHex}[3]{\text{\upshape
    \begin{tabular}{clrlr}
        \tiny{15}  &\tiny{14} & \tiny{6} & \tiny{5} & \tiny{0}\\ 
        \hline
        \multicolumn{1}{|c|}{\texttt{#1}} 
            & \multicolumn{2}{|c|}{\texttt{#2}} 
                & \multicolumn{2}{|c|}{\texttt{#3}} \\ 
        \hline
    \end{tabular}
}}

\newcommand{\FloatMyDcMantCharHex}[2]{\text{\upshape
    \begin{tabular}{lrlr}
        \tiny{15} & \tiny{6} & \tiny{5} & \tiny{0}\\ 
        \hline
        \multicolumn{2}{|c|}{\texttt{#1}} 
            & \multicolumn{2}{|c|}{\texttt{#2}} \\ 
        \hline
    \end{tabular}
}}

\newcommand{\FloatESShort}[3]{\text{\upshape
    \begin{tabular}{clrlr}
        \tiny{31}  &\tiny{30} & \tiny{24} & \tiny{23} & \tiny{0}\\ 
        \hline
        \multicolumn{1}{|c|}{\texttt{#1}} 
            & \multicolumn{2}{|c|}{\texttt{#2}} 
                & \multicolumn{2}{|c|}{\texttt{#3}} 
                    \\ 
        \hline
    \end{tabular}
}}

\newcommand{\FloatPCShort}[3]{\text{\upshape
    \begin{tabular}{clrlr}
        \tiny{31}  &\tiny{30} & \tiny{23} & \tiny{22} & \tiny{0}\\ 
        \hline
        \multicolumn{1}{|c|}{\texttt{#1}} 
            & \multicolumn{2}{|c|}{\texttt{#2}} 
                & \multicolumn{2}{|c|}{\texttt{#3}} 
                    \\ 
        \hline
    \end{tabular}
}}

\newcommand{\FloatMyOrderX}[4]{\text{\upshape
    \begin{tabular}{clrclr}
        \tiny{9}  &\tiny{8} & \tiny{4} & \tiny{3} & \tiny{2} & \tiny{0}\\ 
        \hline
        \multicolumn{1}{|c|}{\texttt{#1}} 
            & \multicolumn{2}{|c|}{\texttt{#2}} 
                & \multicolumn{1}{|c|}{\texttt{#3}} 
                    & \multicolumn{2}{|c|}{\texttt{#4}} \\ 
        \hline
    \end{tabular}
}}

\newcommand{\FloatMyCharX}[3]{\text{\upshape
    \begin{tabular}{clrlr}
        \tiny{9}  &\tiny{8} & \tiny{4} & \tiny{3} & \tiny{0}\\ 
        \hline
        \multicolumn{1}{|c|}{\texttt{#1}} 
            & \multicolumn{2}{|c|}{\texttt{#2}} 
                & \multicolumn{2}{|c|}{\texttt{#3}} \\ 
        \hline
    \end{tabular}
}}

\newcommand{\FloatMyDcOrderX}[3]{\text{\upshape
    \begin{tabular}{lrclr}
        \tiny{9} & \tiny{4} & \tiny{3} & \tiny{2} & \tiny{0}\\ 
        \hline
        \multicolumn{2}{|c|}{\texttt{#1}} 
            & \multicolumn{1}{|c|}{\texttt{#2}} 
                & \multicolumn{2}{|c|}{\texttt{#3}} \\ 
        \hline
    \end{tabular}
}}

\newcommand{\FloatMyDcCharX}[2]{\text{\upshape
    \begin{tabular}{lrlr}
        \tiny{9} & \tiny{4} & \tiny{3} & \tiny{0}\\ 
        \hline
        \multicolumn{2}{|c|}{\texttt{#1}} 
            & \multicolumn{2}{|c|}{\texttt{#2}} \\ 
        \hline
    \end{tabular}
}}


%--- СПЕЦИФИЧНЫЕ ДЛЯ УМНОЖЕНИЯ КОМАНДЫ ---------------------------------------------------------------------------------------------


\newcommand{\Number}[1]{
    \texttt{#1}
}

\newcommand{\NumberHi}[2]{
    \underline{\underline{\texttt{#1}}}\texttt{#2}
}

\newcommand{\NumberMid}[3]{
    \texttt{#1}\underline{\underline{\texttt{#2}}}\texttt{#3}
}

\newcommand{\NumberLo}[2]{
    \texttt{#1}\underline{\underline{\texttt{#2}}}
}

\newcommand{\Stack}[2]{
    \begin{tabular}[t]{@{}r@{}}
        {#1}\\ \hline
        {#2}\\ 
    \end{tabular}
}

\newcommand{\StackThree}[3]{
    \begin{tabular}[t]{@{}r@{}}
        {#1}\\ \hline
        {#2}\\ \hline
        {#3}\\
    \end{tabular}
}

\newcommand{\StackFour}[4]{
    \begin{tabular}[t]{@{}r@{}}
        {#1}\\ \hline
        {#2}\\ \hline
        {#3}\\ \hline
        {#4}\\
    \end{tabular}
}

\newcommand{\Operation}[4]{
    \begin{tabular}[t]{@{}r@{}}
        \texttt{#4}
        \begin{tabular}{@{}r@{}}
            \Number{#1}\\
            \Number{#2}\\ \hline
        \end{tabular} \\ 
        \Number{#3}\\
    \end{tabular}
}

\newcommand{\Addition}[3]{\Operation{#1}{#2}{#3}{+}}

\newcommand{\Subtraction}[3]{\Operation{#1}{#2}{#3}{-}}

\newcommand{\Register}[2]{\Number{#1:#2}}

\newcommand{\Mantiss}{m}
\newcommand{\Order}{p}
\newcommand{\Char}{c}

\newcommand{\MantissOf}[1]{\Mantiss_{#1}}
\newcommand{\OrderOf}[1]{\Order_{#1}}
\newcommand{\CharOf}[1]{\Char_{#1}}

\newcommand{\FloatExpression}[2]{\MantissOf{#1}\cdot {#2}^{\OrderOf{#1}}}

\newcommand{\DivAnswer}[2]{(\texttt{$#1$ rem $#2$})}

\newenvironment{Solve}[1]%
    {\begin{proof}[Решение]#1}
    {\end{proof}}
    
    

\date{Научно-практическая конференция\\"Семья и школа №27", г.~Киров\\(\today)}
\author[А.~Усольцева]{Анастасия Усольцева}

\title[Привниматическое оружие]{Опыт создания и использования привниматического оружия}


\begin{document}

%титул и содержание статьи
\mode<article>{\maketitle}

%титул и содержание презентации
\frame<presentation>{\titlepage}


\begin{frame}
    \frametitle{Об авторах}
    
    \alert{Автор и докладчик}: Усольцева Настя, дипломированный хулиган, кандидат привнематических наук, почётный член академии нытиков.
    
    \alert{Техническая и моральная поддержка}:
    \begin{itemize}
        \item папа Миша;
        \item мама Оля;
        \item сестра Полина;
        \item кошки (2 шт.);
        \item \alert{КОТ};
        \item собака.
    \end{itemize}
    
\end{frame}

\begin{frame}
    \frametitle{Аннотация}
    
    В работе дается определение и рассматриваются виды \alert{привниматического оружия} (далее будем говорить просто: оружие). Приводятся \alert{особенности и правила} использования такого оружия. Анализируются \alert{существующие модели}. Приводится пример разработки и использования модели вниматомета \alert{БЗДЕНЬК-2} и дается прогноз направлений развития привниматического оружия.
\end{frame}

Работа не может считаться научной, если в ней нет непонятных слов. Задача настоящего ученого --- использовать только непонятные слова, потому что тогда он кажется умнеее. Придумать новое непонятное слово --- признак настоящего мастера.
    
\begin{frame}
    \frametitle{Введение и обоснование актуальности}
    
    \begin{block}{Определение}
        \alert{Привниматическое} оружие --- это оружие, 
        \begin{itemize}
            \item \alert{ПРИ}влекающее 
            \item \alert{ВНИ}мание 
            \item авто\alert{МАТИЧЕСКИ}.
        \end{itemize}
    \end{block}
    
    \alert{Внимание} --- это духовный ресурс, которого на всех не хватает. Внимательных людей мало и за внимание людей невнимательных, люди непримечательные издавна воюют между собой. 
    
    А какая война без оружия?
\end{frame}
    
Невнимательных и нечутких людей становится все больше. Виной тому множество причин. Но основная --- столь ценное внимание рассеивается на неживые объекты, такие как телевизоры, мобильные телефоны и компьютеры.
    
Поэтому разработка мощного привниматического оружия --- это одна из актуальнейших задач настоящего времени.

\section{Изучение}

\begin{frame}
    \frametitle{Характеристики оружия}
    
    Привниматическое оружие должно:
    \begin{itemize}
        \item Не причинять физический вред человеку, внимания которого нужно добиться. \alert{безвредно} \mode<article>{(Ведь нужно внимание, а не труп. Труп обычно совершенно невнимателен.)}
        
        \item Максимально раздражать. \alert{обидно} \mode<article>{(Обида --- это тоже внимание. На войне как на войне.)} 
        
        \item Давать возможность скрытного использования.\alert{скрытно} \mode<article>{(Или хотя бы возможность сказать "Ой, я случайно...")} 
        
        \item Не затруднять движений. \alert{носимо} \mode<article>{(Иногда от внимания некоторых приходится спасаться бегством.)}
        
        \item Иметь низкую стоимость зарядов. \alert{дешево} \mode<article>{(Тратить на патроны больше средств, чем получишь внимания - бессмысленно.)} 
        
        \item Быть простым в использовании. \alert{просто} \mode<article>{(Иначе придется не купаться во внимании других, а уделять все свое внимание оружию.)} 
        
    \end{itemize}
\end{frame}

\begin{frame}
    \frametitle{Оценка характеристик оружия}

    \begin{columns}
        \column{.40\textwidth}
            Оценить каждую характеристику можно по пятибалльной шкале и свести в таблицу. Например, оценка оружия \alert{"букет цветов"} приводится в правой части слайда.
            
            
        \column{.50\textwidth}
            \begin{center}
                \begin{tabular}{c|c}
                    \hline\hline
                    Характеристика   & Оценка\\ \hline\hline
                    безвредно        & 5 \\
                    обидно           & 1 \\
                    скрытно          & 2 \\
                    носимо           & 3 \\
                    дешево           & 2 \\
                    просто           & 2 \\ \hline
                \end{tabular}
            \end{center}
    \end{columns}    
\end{frame}

\begin{frame}
    %резиночки
    \begin{columns}
        \column{.70\textwidth}
            духовые трубочки, комочки бумаги, бумажные рогаточки, резиночки для плетения!
            
        \column{.30\textwidth}
            \begin{center}
                \begin{tabular}{c|c}
                    \hline
                    безвредно        & 5 \\
                    обидно           & 1 \\
                    скрытно          & 2 \\
                    носимо           & 3 \\
                    дешево           & 2 \\
                    просто           & 2 \\ \hline
                \end{tabular}
            \end{center}
    \end{columns}    
\end{frame}

\section{Создание и использование}

\begin{frame}
    \frametitle{БЗДЕНЬК-2}
    \framesubtitle{Внешний вид}
    
\end{frame}

\begin{frame}
    \frametitle{БЗДЕНЬК-2}
    \framesubtitle{Характеристики}
    
    \begin{columns}
        \column{.70\textwidth}
            %фото бзденька
        \column{.30\textwidth}
            \begin{center}
                \begin{tabular}{c|c}
                    \hline
                    безвредно        & 4 \\
                    обидно           & 5 \\
                    скрытно          & 4 \\
                    носимо           & 4 \\
                    дешево           & 3 \\
                    просто           & 5 \\ \hline
                \end{tabular}
            \end{center}
    \end{columns}    
\end{frame}

\begin{frame}
    \frametitle{БЗДЕНЬК-2}
    \framesubtitle{Устройство. Механика}
    %TODO: механика
\end{frame}

\begin{frame}
    \frametitle{БЗДЕНЬК-2}
    \framesubtitle{Устройство. Электроника}
    %TODO: электроника
\end{frame}

\begin{frame}
    \frametitle{БЗДЕНЬК-2}
    \framesubtitle{Устройство. Программирование}
    %TODO: программирование
\end{frame}

\begin{frame}
    \frametitle{БЗДЕНЬК-2}
    \framesubtitle{Бюджет проекта}
    %TODO:
\end{frame}

\begin{frame}
    \frametitle{БЗДЕНЬК-2}
    \framesubtitle{Режимы работы}
    
    \begin{itemize}
        \item вялый;
        \item импульсный;
        \item шквал.
    \end{itemize}    
\end{frame}

\begin{frame}
    \frametitle{БЗДЕНЬК-2}
    \framesubtitle{Испытание на котах}
    %TODO: фото сбившихся и испуганных котов на коробке наверху
\end{frame}

\begin{frame}
    \frametitle{БЗДЕНЬК-2}
    \framesubtitle{Испытание на собаке}
    %TODO: фото спящей собаки
\end{frame}

\begin{frame}
    \frametitle{БЗДЕНЬК-2}
    \framesubtitle{Испытание на сестре}
    %TODO: фото улыбающейся Насти с подбитым глазом и "выбитым" зубом
\end{frame}

\begin{frame}
    \frametitle{БЗДЕНЬК-2}
    \framesubtitle{Испытание на родителях}
    %TODO: фото ремня, 
    Испытание на родителях решено не проводить\ldots
\end{frame}


\section{Оценки и выводы}

\begin{frame}
    \frametitle{Вывод}
    
    Главный вывод звучит странно\ldots 
    
    \begin{block}{Главный вывод}
        Привнематическое оружие \alert{не должно} быть направлено на того, чье внимание нужно привлечь. 
    \end{block}
    
    Лучше использовать его вместе, в дружеской игре --- тогда внимание будет \alert{взаимным} и \alert{добрым}.
\end{frame}

\begin{frame}
    \frametitle{Заключение}
    
    Обобщив полученный опыт, можно с уверенностью утверждать, что:
    
    \begin{block}{Заключительный вывод}
        Самое мощное и \alert{единственное настоящее} привниматическое оружие --- это умный, сильный, увлекающийся, любопытный, а главное --- \alert{внимательный к окружающему миру} человек.
    \end{block}
\end{frame}
    
\begin{frame}
    \frametitle{Перспективы и направления развития}
    
    \begin{block}{Перспективы}
        \begin{center}
            Внимание каждому!!!
        \end{center}
    \end{block}
    
    Направления\footnote{В \alert{одном} человеке могут сочетаться \alert{многие} направления, что повышает \alert{убойную силу}} развития \alert{настоящего} привнематического оружия: 
    \begin{itemize}
        \item искусство (музыка, танец, рисование, лепка, рукоделие,\ldots);
        \item наука (физика, математика, химия, биология, психология,\ldots);
        \item спорт (гимнастика, атлетика, плавание, боевые искусства,\ldots);
        \item помощь (людям, животным, растениям, чистоте улиц,\ldots);
        \item \ldots
    \end{itemize}
\end{frame}


\appendix

\begin{frame}
    \frametitle{Список использованной литературы}

    \begin{itemize}
        \item foo \cite{bib:cotelnikov}
    \end{itemize}
\end{frame}

\begin{frame}[allowframebreaks]{Библиография}
    \bibliographystyle{plain}
    \bibliography{./../../../bibliobase}
\end{frame}

\end{document}